%!TEX TS-program = xelatex
%!TEX encoding = UTF-8 Unicode
\documentclass[12pt]{memoir}
\usepackage[
  xetex,bookmarksnumbered=true,pdfborder=0,hyperfootnotes=true
]{hyperref}
\usepackage{xltxtra}
\usepackage{polyglossia}
\usepackage{bidi}
\usepackage{xunicode}
\usepackage{graphicx}

\usepackage{fontspec}
\usepackage{color}

\setdefaultlanguage{german}
\setmainfont{DejaVu Serif}

\hypersetup{
  pdfauthor = {Hassan},
  pdftitle = {Autobiographie von Hassan},
  pdfsubject = {},
  pdfkeywords = {Islam Religion Allah Gott Glaube Skeptizismus Leben},
}

\definecolor{darkblue}{rgb}{0,0,0.5}

% Transliterated "hamza".
\def\´{ʾ} % ˀ
\let \hamza=\´ % Needed in \footnotetext.

% Transliterated "`ayin".
\def\`{ʿ} % ˁ
\let \ayin=\` % Needed in \footnotetext.

% Aliases for subscript and superscript:
\let \Sub=\textsubscript
\let \Sup=\textsuperscript

% Hyphenation rules:
\hyphenation{Mau-la-na Za-kar-ya Kan-dha-la-wi Shu-\`ay-bi}

% Corrective comments.
\newcommand{\cmt}[2]{#1} % #2 comments on #1.
\newcommand{\cor}[2]{#2} % #2 corrects #1.

% Use Scheherazade for Arabic. Use \ar{...} for inline Arabic text.
\newfontfamily{\arabicfont}[Script=Arabic,Scale=1.7]{Scheherazade}
\newcommand{\ar}[1]{\RL{\arabicfont#1}}

% Macros for the word Qur'an.
\def \Quran{Qur\-\´ān} % Used when followed by a punctuation mark.

% Breakable forward slash.
\def\/{\discretionary{/}{}{/}}

% For hyphenation: Trick TeX into thinking al\–XYZ are two separate words.
\def\–{-\hskip0pt}
\def\al{al-\hskip0pt} % Version with "al-".

% Prints symbols for dividing paragraphs.
\def\pardivider{\centerline{***}} % \ar{۞۞۞} ❧❦

% Qur'an reference numbers. TODO: link to web page.
\newcommand{\QRef}[1]{{\color{darkblue}#1}}

% "Nota bene" macro.
\newcommand{\NB}[1]{\emph{\small NB: #1}}

% Image macro. #1 Optional params. #2 File name. #3 Optional caption.
\newcommand{\img}[3]{\begin{center}%
\includegraphics[#1]{#2}\\{\small\em#3}%
\end{center}}
\graphicspath{{books/imgs/}}

% Paragraph spacing:
\setlength{\parskip}{1ex plus 0.5ex minus 0.2ex}

\pagestyle{plain}

\title{Eine Autobiographie}
\author{\emph{von Hassan}}

%%%%%%%%%%%%%%%%%%%%%%%%%%%%%%% Document begin %%%%%%%%%%%%%%%%%%%%%%%%%%%%%%%%
\begin{document}
\frontmatter

% Title:
\maketitle
\thispagestyle{empty}
\cleardoublepage

% Table of Contents:
\setcounter{page}{1}
\tableofcontents

% Preface:
\chapter{Vorwort}

% \begin{comment}
Dies ist meine Geschichte. Ich habe sie als Selbsttherapie geschrieben,
außerdem habe ich sie geschrieben, um zu versuchen mein Leben zu verstehen.
Um zu verstehen warum ich bestimmte Wege gegangen bin,
und warum ich zu bestimmten Schlussfolgerungen gelangt bin.
Ich hoffe auch, dass es andere interessiert und aufklärt.
Ich habe mich auf die frühen und späten Jahre konzentriert.
Ich habe keine echten Namen benutzt, außer in wenigen Einzelfällen.
% \end{comment}

\img{scale=0.1}{Tunis_Hassan.jpg}{}

Mein Vater war Ägypter und meine Mutter Engländerin
und ich habe 8 Brüder und Schwestern.
Ich wurde als Muslim geboren, aber ich begann erst zu praktizieren
als ich zwanzig und sehr fromm und engagiert wurde.
Für die nächsten zwanzig Jahre diente der Islam als Anleitung
für jeden Aspekt meines Lebens.
Ich absolvierte ein B.A.-Studium in Arabistik und Islamwissenschaften
an der Schule für Orientalistik und afrikanistische Studien,
wo ich Präsident der „Islamischen Gesellschaft“ wurde.
Nachdem ich von der Universität abgegangen bin,
wurde ich zum Leiter einer Da\`wah-Gruppe im Norden Londons
und gab ein islamisches Magazin mit dem Namen
„The Clarion“ („Der Weckruf“) heraus.
Ich schrieb vier Bücher für muslimische Kinder
und verbrachte fünfzehn Jahre als Lehrer an der ‚Islamia School‘ –
jene, die von Yusuf Islam (Cat Stevens) gegründet wurde.
Doch zeitnah zu meinem 48.\ Geburtstag wusste ich,
dass ich nicht mehr länger an den Islam glaubte.
Hättet ihr mir vor ein paar Jahren erzählt,
dass mir so etwas passieren würde,
so hätte ich euch niemals geglaubt und hätte außerdem hinzugefügt,
dass kein ‚wahrer‘ Muslim jemals den Islam ablehnen würde,
nachdem er die Süße des Iman (Glaubens) gekostet hat.
Niemand, der in die Schönheit des \Quran\ eingetaucht war
und seine Weisheit verstand, könnte jemals leugnen,
dass es das Wort Gottes ist. Hätte jemand so etwas behauptet,
so hätte ich ihm instinktiv misstraut
und gegen seine Motive Verdacht geschöpft.
Doch es ist erstaunlich wie sich Wahrnehmungen ändern können und wie Dinge,
die ich früher als unvorstellbar erachtete,
jetzt vollkommen vernünftig erscheinen.
Natürlich erfolgte dieser Wandel nicht von heute auf morgen.
Es begann vor ein paar Jahren,
als ich anfing meine Ansichten zu hinterfragen,
die ich seit Langem als selbstverständlich ansah,
und als ich anfing den Islam im neuen Lichte zu betrachten.

Immer mehr Zweifel schlichen sich nach und nach ein.
Anfangs versuchte ich sie zu unterdrücken und reagierte auf Kritik
gegen den Islam mit Verleugnung, Zorn und Vorwürfen.
Ich leugnete, dass etwas nicht stimmte,
war Kritik gegenüber überempfindlich und beschuldigte den Westen
provokant zu sein und Probleme zu verursachen.
Als ich schließlich akzeptierte, dass Muslime Verantwortung
für ihre Probleme tragen mussten,
konnte ich dennoch nicht akzeptieren,
dass der Islam selbst verantwortlich zu machen war.
Die Art und Weise, in der der Islam interpretiert wurde,
war das Problem. Ich begann für eine Neuinterpretation
und für die Reform von traditionellen Ansichten zu argumentieren,
aber anstatt mein Gewissen zu erleichtern,
hebte dies nur die Zwecklosigkeit
und Unaufrichtigkeit dieser Ansichten hervor.
Schlussendlich versuchte ich mir selbst einzureden, dass,
obwohl es meinem rationalen Verstand schwer fiel
bestimmte Dinge im Islam zu glauben,
es Erklärungen jenseits meiner Verständnisfähigkeit geben muss,
und dass ‚Gott es am besten weiß‘.
Die sicherste und weiseste Wahl war an ‚Allahs Seil festzuhalten‘.
Ich dachte, ich hatte nichts zu verlieren und alles zu gewinnen,
indem ich ein Gläubiger blieb.
Also lebte ich den Alltag eines ‚guten‘ Muslims, in der Hoffnung,
dass mein Glaube zurückkehren würde.
Doch dieser Schein machte mich nur depressiv
und ich verlor jegliche Motivation.
Das Problem ist, dass man sich das Glauben nicht aussuchen kann.
Entweder man glaubt oder nicht, und wenn es einen Gott gibt,
dann wäre das letzte, was er von mir wollen würde,
dass ich zum Schein etwas glaube, was ich nicht glauben konnte.
Es war eine riesige Erleichterung als ich mir endlich eingestand,
dass ich an den Islam nicht mehr glaubte.

Doch die Tatsache, dass ich nicht mehr an den Islam glaube,
bedeutet nicht, dass ich auf einmal zum Islamhasser geworden bin.
Ich weiß, dass der Islam dem Leben einer Unzahl von Menschen sehr viel Hilfe,
Trost und gute Werte bringt.
Ich weiß, dass die meisten Muslime gute und anständige Leute sind.
Wie könnte ich jemals Muslime hassen, wenn meine Familie doch muslimisch ist?
Wenn ich mit meinen älteren Kindern darüber spreche, was ich denke,
so erzähle ich ihnen, dass sie für sich selbst herausfinden müssen,
was sie glauben, und wenn sie sich als Muslime glücklich fühlen,
dann soll es so sein.
Ich empfinde es sicherlich nicht als Notwendigkeit,
meine eigenen Ansichten bezüglich Gott und Religion weiterzugeben –
etwas, wozu ich mich verpflichtet fühlte, als ich Muslim war.

Während ich nicht daran glaube, anderen zu sagen was sie glauben sollen,
denke ich, dass man den Mut haben sollte die Ansichten ehrlich zu prüfen,
die essentiell für das eigene Leben sind und die eigenen Taten leiten.
Wenn man wirklich mit ihnen zufrieden ist,
dann sollte man sie vollständig mit Herz und Verstand annehmen,
aber wenn sie der genauen Prüfung nicht standhalten,
dann sollte man sie verwerfen.
Das Leben ist zu kurz, um es von Ansichten diktieren zu lassen,
an die man nicht aufrichtig glaubt.

\hfill Copyright © 2008–2010 by Hassan

\hfill Copyright © 2008–2010 Übersetzung von
\href{mailto:alsana.aziz@gmail.com}{Aziz}

\mainmatter

% Chapters:

%Chapter 1:
\chapter{Hast du den \Quran\ gelesen?}

\img{scale=0.7}{Baby_Hassan.jpg}{}

Jede Nacht, bevor wir zu Bett gingen, sagten wir die \emph{Al-Fatiha} auf –
das kurze erste Kapitel des \Quran.
Ich fühlte, dass Gott über mich wachte.
Mein Vater, öfter jedoch meine Mutter, würde an der Türe stehen,
während ich und meine zwei Brüder unsere Arme verschränkten
und aufrecht im Bett saßen.
Wir sagten es blitzschnell auf Arabisch und Englisch auf,
dann Kopf aufs Kissen und Licht aus.
Manchmal fügte ich mein spezielles Gebet hinzu:
„Lieber Gott, bitte mach, dass alles gut wird.“
Ich war zu müde, um die Einzelheiten aufzuzählen.
Gott verstand schon.

Nur bei besonderen Anlässen mussten wir
die offiziellen täglichen Gebete verrichten,
w.z.B. beim Besuch vom Großvater,
wo wir als Wohltat für ihn eine große Schau der Frömmigkeit abziehen mussten.
Nach dem Gebet zeigten mir meine ältesten Schwestern, Kamelia und Susi,
wie man Du\`a (Bittgebet) macht, um Gott um etwas, das ich wollte, zu bitten.

\begin{quote}
„Halte deine Hände auf und lass keine Lücken offen!“ sagte Kamelia.

„Warum muss ich meine Hände so halten?“ fragte ich.

„Damit du die Geschenke auffangen kannst, die Gott dir schickt!“
antwortete Kamelia.
\end{quote}

Ich akzeptierte diese Antwort kritiklos,
obwohl ich keine Geschenke sehen konnte.
Ich begann mit der Art und Weise vertraut zu werden,
in der Gott die Dinge handhabte.
Wenn ich Gott um ein Fahrrad bat, wusste ich,
dass ich es nicht gleich bekommen würde,
oder es vielleicht überhaupt nicht bekommen würde.
Es war nicht Gottes Schuld, sondern meine,
denn ich war auf irgendeine Art ungezogen und verdiente es nicht.
Wenn ich etwas unbedingt wollte,
würde ich mir große Mühe geben so lange wie möglich artig zu sein.
Ich tat dies, als ich Gott darum bat,
meine Mamma und meinen Papa dazu zu bringen,
mit dem Streiten aufzuhören.
Als sie nicht aufhörten, wusste ich, dass ich nicht artig genug war.

\img{scale=1}{Hassan_Boy.jpg}{}

Als ich älter wurde, rückte der Islam mehr und mehr in den Hintergrund
und im Alter von zwölf, während des langen heißen Sommers im Jahre 1971,
war mein einziges Interesse so wie Marc Bolan zu sein.
Mit meinem Lockenkopf, meiner Samtjacke und meinen Plateaustiefeln,
dachte ich auch, dass ich aussah wie er.
Ein bisschen Glitter auf meinen Wangen und ich war weg,
schmollend und tänzelnd vor dem Spiegel zu ‚Hot Love‘ und ‚Get it On‘.
Ich zeichnete Bilder von ihm und Mickey Finn und hängte sie an die Wand
im Schlafzimmer auf, das ich mit meinen zwei Brüdern teilte.
Unser Schlafzimmer war so sehr zugedeckt mit Zeichnungen
und Bildern von Männern mit Make-Up und hautengen Trikots,
dass mein Vater überzeugt war, dass wir langsam schwul wurden.

\begin{quote}
„Was um alles in der Welt haben all diese Homosexuellen
an euren Wänden zu suchen?“

„Das sind keine Homosexuellen.“

„Natürlich sind sie das. Schau mal seine Augen an,“ sagte er,
und zeigte auf ein Bild von Dave Hill an der Wand,
dem Gitarrist von ‚\href{http://de.wikipedia.org/wiki/Slade}{Slade}‘.

Ich betrachtete seine Augen genau, und versuchte festzustellen
was genau meinem Vater so deutlich den Eindruck „HOMOSEXUELL“ vermittelte.

„Alle tragen Glitzer. Es ist nur Show!“

„Ich möchte, dass ihr sie herunternimmt. Sofort!“
\end{quote}

Wir taten, wie uns geheißen wurde. Man diskutierte nicht mit meinem Vater.

\img{scale=0.8}{Hassans_Father.jpg}
{Mein Vater Aziz (ganz rechts),
während des Krieges gegen Israel im Jahre 1948.}

Mein Vater, geboren als Sohn einer wohlhabenden, ägyptischen Familie,
sprach selten über das Land und die Kultur, aus der er gekommen war.
Er war auch nicht besonders religiös.
Lange Zeit wusste ich nicht, dass er aus Ägypten kam.
Als er zum ersten Mal nach England kam,
nannte er sich \emph{Jean Pierre} und erzählte jedem, dass er ein Franzose ist.
Meine Mutter, Mary Magson, war die Tochter von Horace Magson,
eine Buchhalterin aus Darlington.
Ihre Familie war methodistisch, aber ähnlich wie mein Vater,
war auch sie nicht religiös.
Die beiden begegneten sich in Paris im Jahre 1951.
Er studierte an der Universität von Lausanne, während sie in der ATS war,
wo sie Rettungswagen während des Krieges fuhr.
Nachdem sie geheiratet hatten, zogen sie nach Ägypten, um dort zu leben,
aber infolge der ägyptischen Revolution von 1952 und der Sueskrise,
holte sie die britische Regierung zurück,
zusammen mit anderen anglo-ägyptischen Familien.
Sie wurden in eine alte Tudorhütte umgesiedelt,
mitten in Suffolk in Coddenham, wo ich am 12.\ Mai 1959 geboren wurde,
als fünftes Kind in einer achtköpfigen Familie.

\img{scale=0.7}{Hassans_Mother.jpg}
{Meine Mutter Mary Magson (stehend) in der ATS im 2.\ WK.}

Als ich 6 Jahre alt war, zogen wir nach Finchley in Nord-London um.
Das war der Ort, wo ich den Großteil
meiner Kindheit und Jugendjahre verbrachte.
Während der 60er Jahre war es ein sehr bürgerlicher, weißer Bezirk,
und bald wurde mir bewusst, dass wir nicht ganz hierher passten.
Ich hasste es jemandem meinen Namen zu sagen.
Es war immer die selbe Reaktion;
peinliche Stille und verwirrte oder spöttische Blicke.
Mein Klassenlehrer erstellte einmal eine Liste der Religionszugehörigkeit.
Als er meinen Namen aufrufte, zögerte ich lange,
was natürlich nur mehr Aufmerksamkeit auf mich zog.
Schließlich flüsterte ich „Muslim“.

\begin{quote}
„Was?“ sagte der Lehrer.

„Muslim, Herr Lehrer.“
\end{quote}

Einige Kinder kicherten.
Ein Junge bestand darauf, dass ‚Muslim‘ überhaupt keine echte Religion sei,
und dass ich es ausgedacht hätte.
Meine Geschwister und ich wurden zum Ziel von bigotten Personen und Raufbolden.
Die Beharrlichkeit meines Vaters,
dass wir von Schulversammlungen ausgenommen werden mussten,
machte die Dinge nicht besser.
Ich weiß nicht genau, warum er uns abmeldete.
Ich könnte es besser verstehen, wenn er religiös gewesen wäre,
aber das einzige Ergebnis davon war, dass es jedem bestätigte,
dass wir ‚anders‘ waren und auch so behandelt werden sollten.
Anstatt dem Gleichnis in der Geschichte von ‚The Little Red Hen‘
(Die kleine rote Henne) zuzuhören oder ‚All Things Bright and Beautiful‘
(Alle heiteren und schönen Dinge) zu singen – mein Vater dachte,
dass dies meinen Verstand mit dem Christentum infizieren würde –
wurde ich den Freuden eines pornografischen Romans ausgesetzt,
welches der ältere Junge neben mir las, als wir in dem Raum saßen,
der jenen zugewiesen wurde, die von der Versammlung abgemeldet waren.

Mein Vater war voller Paradoxien und Widersprüche.
In manchen Bereichen war er sehr liberal
und in anderen Bereichen ultra-konservativ.
Seine Einstellung zu Frauen war sehr altmodisch und er erwartete,
dass meine Mutter zu Hause blieb, kochte, putzte und für seine Kinder sorgte,
und er wurde gewalttätig, wenn er dachte,
dass sie nicht all seinen Befehlen gehorchte.
Nachdem sie eine lange Zeit gelitten hatte, klagte meine Mutter auf Scheidung.
Der Rechtsstreit um das Sorgerecht war erbittert und hielt lange an.

\img{scale=0.3}{Hassan_and_Diane.jpg}
{Ich \& Diane (1978)}

Im Alter von 18 Jahren begann ich ein Soziologiestudium
am Polytechnikum in Nord-London,
das ein Hort von Marxisten und militanten trotzkistischen Studenten war.
Ich trat der „Broad Left“ (Breite Linke Partei) bei,
eine der zwei sozialistischen Hauptparteien im Nord-Londoner Polytechnikum,
und wurde als Karikaturist für das Magazin „Fuse“ eingestellt.
Unser Hauptrivale am Polytechnikum war „The Socialist Workers Party“
(Die Sozialistische Arbeiterpartei) und ich zeichnete
viele bissige Karikaturen über die Führerschaft,
verhöhnte die lächerlichen Dinge, die sie sagten,
und verhöhnte ihren Kleidergeschmack und ihre dürftige Körperpflege.
Meine Karikaturen wurden einstimmig hochgelobt, bis ich eine zeichnete,
die meine eigene Partei – die Breite Linke – kritisierte.
Der Herausgeber fühlte sich so sehr beleidigt,
dass ich auf der Stelle gefeuert wurde.
Ich verstieß gegen eine der grundlegendsten Regeln der Politik:
Ordne dich der Parteilinie unter.
Aber ich war nie jemand, der sich irgendeiner Parteilinie unterordnete,
und meine Erfahrung mit linker Studentenpolitik verstimmte mich nur dagegen.
Ich war viel mehr daran interessiert,
wie ich Diane zu einem Rendezvous einladen könnte.
Diane war eine Soziologiestudentin im ersten Jahr, genau wie ich,
und wir wurden fast augenblicklich ziemlich gute Freunde.
Sie hatte einen quirligen und fröhlichen Charakter,
mit dem man sich leicht gut verstehen konnte.
Sie liebte Fußball, hatte Freude an guten Debatten
und wir teilten einen ähnlichen Musikgeschmack.

Nach ein paar Wochen meiner schüchternen und stümperhaften Versuche
etwas anderes außer nur „Hallo“ zu sagen, lud sie mich ein.

\begin{quote}
„Hast du Lust heute Nacht zum Stapleton zu gehen?
Die haben eine Live-Band gebucht.“

„Klar! Um welche Zeit?“

„Komm zu mir um acht Uhr.“
\end{quote}

Ich war dort um sieben, in meinen besten zerrissenen Jeans,
und duftete nach indischem Patschuli.

Es war nur ein Ausflug zur Kneipe,
aber es war der Beginn meiner ersten Beziehung.
Wir wurden zu regelmäßigen Gefährten
und begannen eine Wohnung mit einander zu teilen.

Am Ende des ersten Jahres wurden wir den langweiligen Vorträgen über
Auguste Comte, Max Weber und Karzl Marx überdrüssig.
Wir beide erhofften uns, dass ein Soziologiestudium
eine kreative Erfahrung sein würde,
und vielleicht Antworten auf die großen Fragen des Lebens liefern würde.
Wir fanden enttäuschend heraus, dass es bloß eine Übung im Wiederkäuen
einer Auswahl von langweiligen Büchern war.
Wir entschieden uns das Studium abzubrechen und nach Wales zu trampen.
Darüber hinaus hatten wir keinen anderen Plan.
Nachdem wir mehrere Tage lang in Aberystwyth
psychoaktive Pilze gepflückt hatten,
machten wir uns auf den Weg zum Haus ihrer Eltern in Warrington,
um zu entscheiden was wir als nächstes vorhatten.

Diane hatte mal als Schwester im Calderstones-Hospital für körperlich
und geistig Behinderte gearbeitet, also schlug sie vor,
dass wir beide dort einen Job bekommen sollten.
Calderstones befand sich in einem malerischen Lancashire-Dorf
am Fuße von Pendle Hill, ein Hügel,
der berüchtigt für die Verbrennung von Hexen im Mittelalter war.
Ich dachte es wäre eine wunderbare Idee
und wir mieteten einen kleinen Wohnwagen auf dem Feld eines Bauers,
das sich am Höhenrücken von Pendle Hill befand.
Nur zu Fuß konnte der Wohnwagen über einen sehr langen
und steilen Weg erreicht werden,
der im Dorf begann und sich in die Nebel von Pendle Hill hineinschlängelte.
Der Wohnwagen stand in einer abgeschiedenen Ecke,
umgeben von einem Mist bedeckten Morast,
wo der Wind und der Regen nie aufhörten.
Ich liebte es, aber Diane war nicht so sehr begeistert,
besonders nach der Nacht, in der ich einen Geist sah.
Ich wurde mitten in der Nacht durch das schlagende Geräusch
der Wohnwagentür aufgeweckt.
In meiner schläfrigen Benommenheit schaute ich auf zur Eingangstür
und erblickte was so ähnlich aussah wie ein junges Mädchen –
wahrscheinlich im frühen Jugendalter –
das am Fußende des Bettes stand.
Sie sah so aus, als ob sie Lumpen anhatte,
mit einem traurigen Lächeln im Gesicht.
Sie lehnte ihren Kopf leicht gegen die Trennwand des Wohnwagens
und starrte mich an.
Vielleicht war es die Tochter des Bauers, dachte ich,
und ich konnte nicht verstehen warum sie zu dieser späten Stunde
im eiskalten Winterregen außer Haus war.
Bevor ich meinen Kopf heben und etwas sagen konnte,
erschien hinter ihr ein Mann mit einem Dreispitz.
Er rannte durch den Wohnwagen und mich hindurch,
und verschwand wieder auf der anderen Seite.
Ich sprang nun auf und rufte Dianes Namen, um sie aufzuwecken.
Das Mädchen war weg und alles war still, abgesehen vom Wind.
Diane war verständlicherweise total erschrocken,
und obwohl ich ihr versicherte, dass es nur ein Traum war,
ausgelöst durch die örtlichen Geschichten von Hexenverbrennungen
und außerdem durch den Konsum von großen Mengen an Dope und Alkohol,
wollte sie dennoch nicht mehr im Wohnwagen bleiben.
Wir zogen später in eine kleine Wohnung in Blackburn um.

Es war nun das Jahr 1979 und ich hatte eine Gruppe
von engen Freunden und ein gutes Sozialleben.
Doch im Inneren fühlte ich mich ruhelos und unsicher über meine Identität.
Dies wurde ab und an durch die negative Reaktion
von ein paar bigotten Menschen unterstrichen,
die deutlich machten, dass jemand mit einem lustigen Namen
und einer fremden Religion, obwohl ich nicht praktizierend war,
niemals als Engländer akzeptiert werden würde.
Eine Stationsschwester des Krankensaals, in dem ich arbeitete,
machte es sich zur Gewohnheit mich mit „Dreckiger Araber!“ zu rufen.
Sie fing an, es ab dem Moment zu sagen, an dem sie meinen Namen erfuhr.
Sie schrieb auf mein Zeugnis, dass ich „dreckig, faul und ineffizient“ sei,
etwas das ich nur viel später herausfand,
als eine sympathische Stationsschwester mir das Zeugnis zeigte.
Nichts was sie sagte stimmte.
Sie mochte mich nicht, einfach nur weil mein Name Hassan war.
Es gab keinen anderen Grund.
Die Ironie war, dass ich mich bis zu diesem Zeitpunkt
nie als Araber gesehen hatte.

Dann geschahen eine Reihe von Ereignissen,
die das Überdenken meiner Ansichten über die Kultur und die Religion anregten,
die ich mit großer Mühe versuchte zu meiden.
Das erste war die Islamische Revolution im Iran.
Ich erinnere mich an die Fernseh-Berichterstattung
über die laufenden Straßenschlachten in Teheran,
zwischen den einfachen Leuten und der schwer bewaffneten Garde des Schahs.
Ich fand die Bilder inspirierend:
trotzende Zivilisten, die sich gegen die Macht eines Tyranns erhoben.
Der revolutionäre Geist aus meiner Studienzeit war in mir geblieben,
und ich sah es instinktiv als Kampf eines Volkes gegen die mächtige Elite an,
aber mir war auch bewusst welche Rolle die Religion spielte – eine Religion,
zu der ich eine Verbindung fühlte – wenn auch nur eine schwache.

Ich wurde mit einem weiteren Beispiel der Kraft der Religion konfrontiert,
als mein enger Freund John Shackleton von einem Campingausflug zurückkam,
um zu verkünden, dass er nun ein wiedergeborener Christ sei.
Es war ein enormer Schock,
da er sich immer so verächtlich über Religion äußerte.
Jetzt weigerte er sich mit mir zur Kneipe um die Ecke zu gehen
oder Musik anzuhören – außer es war Musik über Jesus.
Er war auf nervige Weise frohlockend wegen seiner Religion,
und versuchte ständig mich und Diane zu konvertieren.

\begin{quote}
„Jesus liebt dich und möchte dir verzeihen,“ sagte er.

„Aber ich habe nichts verbrochen,“ sagte ich.

„Wir sind alle Sünder.
Jesus kann dich wieder zu dem machen, wie Gott dich haben wollte.“

„Warum? Hat Gott beim ersten Mal etwa einen Fehler gemacht?“
\end{quote}

Das ständige Bombardement mit religiösem Eifer zwang mich dazu,
Stellung zum Christentum und Religion im Allgemeinen zu beziehen,
etwas worüber ich zuvor nicht lange nachgedacht hatte.
Je mehr John über Dinge wie die Trinität,
die Erbsünde und das Erlösungswerk sprach, desto mehr verstand ich,
dass dies Konzepte waren, an die ich niemals glauben könnte.
Der Gedanke, dass Gott „Drei in Einem“ ist,
und dass der Mensch in Sünde geboren wird,
oder dass Sünden vergeben werden können,
nicht wegen irgendeiner verdienstvollen Tat einer Person,
sondern weil jemand an ein Stück Holz genagelt wurde –
all das stand im Widerspruch mit meinem Verstand und meinem Gerechtigkeitssinn.
John trat einer kleinen,
wenige Kilometer entfernten Gemeinschaft von wiedergebohrenen Christen bei,
aber er besuchte uns weiterhin und betete,
dass wir mit dem Heiligen Geist erfüllt werden mögen.

Ich fühlte mich unberührt von der Idee eines Heiligen Geistes
oder irgendeiner übernatürlichen Kraft.
Ich ging immer davon aus, dass es im Universum mit rechten Dingen zuging,
und hatte nie selbst eine mystische Erfahrung.
Doch im Einklang mit dem religiösen Thema in dieser Reihe von Ereignissen
würde mir bald etwas sehr Mystisches passieren.
Ich campte mit Diane am Musikfestival von Deeply Vale in Lancashire,
und wir unternahmen einen Spaziergang an einer Talseite hinauf.
Als wir an der Spitze standen, hörte ich einen Klang,
den ich nie erwartet hätte.

\begin{quote}
„Kannst du das hören?“ fragte ich.

„Was? Ich höre gar nichts.“

„Es ist der muslimische Gebetsruf… Hör zu!“ sagte ich.
\end{quote}

Ich konnte deutlich diese Worte hören:
\emph{„Asch-hadu an la ilahe illah, asch-hadu anna Muhammad Rasul-allah!“}
(Ich bezeuge, dass es keinen Gott außer Gott gibt, ich bezeuge,
dass Muhammad der Prophet Gottes ist!)
Sie ertönten mit einer ungewohnten Klarheit
über dem Trubel unten beim Festival.
Diane hörte überhaupt nichts.
Ich konnte nicht verstehen warum.
Weder konnte ich verstehen, warum jemand bei einem Musikfestival
den muslimischen Gebetsruf rezitierte.
Ich suchte nach einer vernünftigen Erklärung, aber konnte keine finden,
die mich überzeugte.
Ich fing an zu denken, dass vielleicht doch übernatürliche Dinge passierten,
und dass dies mein persönlicher Weckruf zum Glauben sein könnte.

Ein paar Monate später sah ich Cat Stevens im Fernsehen,
als er ein Abschiedskonzert gab.
Er war Muslim geworden und gab das Musikgeschäft auf,
wandte Ruhm und Reichtum den Rücken zu,
um sein Leben seinem neugewonnen Glauben zu widmen.
Die Öffentlichkeit war verblüfft.
Warum würde jemand, der alles im Leben besaß, es wegen der Religion wegwerfen?
Religion war nicht cool; es war für naive, unsichere Menschen.
Ein Teil von mir hatte die selbe Reaktion.
Aber ein anderer Teil von mir war stolz und inspiriert.
Wenn so eine kreative und angesehene Person wie Cat Stevens
etwas Wertvolles im Islam sah,
vielleicht sollte ich es selbst viel ernster nehmen,
als jemand, der mit islamischem Erbe zur Welt kam.

Die letzte Episode in dieser Folge von Ereignissen war,
als mein Vater unerwarteterweise vor meiner Tür aufkreuzte,
und mich einlud mit ihm für ein paar Wochen nach Ägypten zu gehen.
Ich hatte meinen Vater nicht sehr viel zu Gesicht bekommen,
seitdem ich sein Haus verlies.
Er war nun ins Dorf von Ramsbottom umgezogen,
wo er als Direktor einer örtlichen Schule eingestellt wurde.
Da ich ziemlich in der Nähe wohnte, nahm ich Diane mit, um ihn zu treffen.
Er empfand eine sofortige Abneigung gegen Diane.
Sie war eine unabhängige, willensstarke, junge Frau,
die sich nicht scheute zu sagen, was sie dachte.
Das gefiel mir an ihr, meinem Vater aber nicht.
Er mochte keine Frauen, die widersprachen;
es war gegen die natürliche Ordnung der Dinge.
Frauen sollten eigentlich das tun, was Männer ihnen befahlen –
mein Vater dachte, dass Gott sie in dieser Art erschuf und mein Vater wollte,
dass es auch so bleibt.
Er fing an zwischen uns auf seine übliche Art und Weise Zwietracht zu säen,
indem er negative Randbemerkungen, Andeutungen und abfällige Witze machte,
und behauptete, dass sie gegenüber ihm unhöflich war, und ihr sogar vorwarf,
dass sie aus seinem Haus stahl.
Ich bin mir nicht sicher, ob sein Angebot einer Reise nach Ägypten
ein Teil dieses Bestrebens war.
Er hatte dort wichtige Geschäfte zu erledigen,
bezüglich der Erbschaft seines Vaters.
Aber ich vermute, dass er dies auch für eine wunderbare Gelegenheit hielt,
um mich von Diane wegzukriegen.
Ich sagte zu mir selbst, dass es unsere Beziehung nicht beeinflussen würde,
und dass ein Pauschalurlaub nach Ägypten ein Angebot war,
das ich nicht ablehnen konnte.
Ich denke im Unterbewusstsein fühlte ich mich auch gelangweilt und ruhelos.
Der Ort, zu dem ich und Diane umgezogen waren,
ein Neubau am Stadtrand von Blackburn, war langweilig und seelenlos.
Unser Leben füllte sich mit Routinen und unsere Beziehung
verfiel in den alltäglichen Trott.
Ich war begierig auf einen Szenenwechsel – zumindest für zwei Wochen.

Bevor wir zum Flughafen fuhren,
hielt mein Vater beim Haus meiner Schwester Susi in London an,
um sich zu verabschieden.

\begin{quote}
„Was willst du in Ägypten unternehmen, Hassan?“ fragte sie.

„Ich habe darüber nachgedacht, nach Möglichkeit Ägyptologie
an der Amerikanischen Universität von Cairo zu studieren.“

„Inscha-Allah,“ (So Gott will) ermunterte Susi.

„Ich verstehe nicht, warum ich das sagen sollte, Susi,“ antwortete ich barsch.
„Entweder werde ich Ägyptologie studieren, oder ich werde es nicht.
Was hat Gott damit zu tun?“

„Nichts geschieht ohne den Willen Gottes.“

„Macht er, dass Mörder andere Menschen töten?“

„Er lässt es zu, weil er uns einen freien Willen gibt,“ erwiderte Susi.

„Na dann sehe ich nicht ein, warum Gott sich einmischen sollte,
wenn ich Ägyptologie studieren wollte!“

„Wenn du meinst, Hassan,“ erwiderte Susi müde.
Ich dachte darüber nach, was ich eben gesagt hatte,
als wir zum Flughafen fuhren;
es war arrogant und ich bedauerte, es gesagt zu haben.
\end{quote}

Es war ein langer Flug mit einem zweistündigen Zwischenaufenthalt
in einer trostlosen osteuropäischen Stadt,
in der es anscheinend nichts außer graue Betonmauern
und Polizisten mit Schlagstöcken gab.
Ich war erleichtert, als wir endlich in Ägypten landeten.
Sobald ich aus dem Flugzeug stieg und in die heiße,
feuchte Atmosphäre des Kairoer Flughafen gelangte, erkannte ich,
dass ich in einem anderen Universum war.
Der Duft von Weihrauch wanderte durch ein kunstvolles Gitterfenster.
Eseln beladen mit Gemüse bahnten sich den Weg
durch ein Orchestra von Autogehupe;
Straßenverkäufer priesen lautstark ihre Waren mit Sirenentönen an,
die mich hochschrecken ließen;
Männer beteten auf dem Gehsteig in Pyjamas;
und Frauen warfen Kübel voller Schalen von Balkonen herunter.
Es war ein verrücktes, chaotisches Flickwerk von Gerüchen,
Geräuschen und Farben und es war wie ein großer Kulturschock für mich.
Doch trotz der Fremdartigkeit fühlte ich mich bald wie zu Hause.
Zum ersten Mal in meinem Leben musste ich meine Herkunft nicht verstecken
oder über sie beschämt sein.
Überdies bewunderte und respektierte jeder beide Hälften
meiner kulturellen Herkunft.

\img{scale=0.3}{Hassan_Cousins.jpg}
{Ich \& zwei Cousins in Ägypten (1981)}

Wir übernachteten im Haus meines Onkels in Kairo.
Die Ägypter trugen westliche Kleidung,
schauten synchronisierte Hollywoodfilme an
und besaßen viele der modernen Annehmlichkeiten, die man aus England kannte.
Doch während ich auf dem französischen Einrichtungsstück –
eine Replika aus dem 18.\ Jahrhundert – saß,
begann der Gebetsruf aus dem Lautsprecher draußen auf der Straße zu heulen.
Dies löste eine Welle von Gebetsrufen aus,
die sich langsam über die Dächer in Kairo und in die weite Ferne ausbreiteten.
Selbst im Fernsehen unterbrach man Clark Gable mitten im Satz,
als ein Hinweis auf Arabisch kam, der das Gebet ankündigte.
Als alle für das Gebet aufstanden,
und ich am Tisch alleine sitzen gelassen wurde,
fühlte ich mich etwas unangenehm.

Nach dem Gebet kam meine Cousine Nihal,
die meinem Tantchen in der Küche dabei geholfen hatte Essen zu braten,
mit einem dampfenden Gericht ins Zimmer herein.

\begin{quote}
„Magst du ‚Beetles‘?“

„Ähm… Ich habe nie welche probiert!“ sagte ich und fühlte mich etwas mulmig.

„Ich liebe sie viel zu sehr!“
Sie legte den Teller auf den Tisch.
„Vor allem mag ich Paul; er ist viel zu süß!“

„Ach so…“ sagte ich mit einem Seufzer der Erleichterung.
„Ja, ich mag sie, aber sie haben sich vor ein paar Jahren aufgelöst, weißt du!“
Ägypter lieben alles Englische und wissen eine Menge über England,
jedoch scheinen ihre Informationen etwa ein Jahrzehnt alt zu sein.

„Aufgelöst?“

„Sie spielen nicht mehr zusammen.“

„Ach? Warum?“

„Na ja, nach einer Weile geschieht das mit Bands…“

„Georgie der Beste!“ unterbrach Hamdy und gab mir die Daumen hoch.
„Manchester United! Gut.“

„Na ja, eigentlich bin ich Fan der Spurs.“

„Sopurs? Was ist Sopurs?“

„Tottenham Hotspur – das ist ein Fußballteam.“

Nihal zeigte mir ein Bild in einer ägyptischen Zeitung von Ayatollah Khomeini,
wo er ein kleines Mädchen umarmt.
„Ohhh wie süß, er ist so ein guter Mann!“

„Die Leute scheinen ihn zu lieben.“

„Er sagt, dass es keinen Unterschied zwischen den Sunniten
und den Schiiten gibt.
Er sagt, wir sind alle Muslime und sollten vereint sein.“
\end{quote}

Nihal war eine sehr willensstarke, unabhängige Frau, die ihre Freiheit,
das zu tun was sie beliebte, für selbstverständlich hielt.
Sie trug kein Kopftuch und hatte sehr westliche Gewohnheiten und Vorlieben.
Dennoch schien sie sich dabei völlig wohl zu fühlen sich mit traditionellen,
orthodoxen Ansichten zu identifizieren – etwas, womit die meisten Ägypter,
die ich kennenlernte, überhaupt keine Probleme hatten,
obwohl sie relativ verwestlicht waren.

\begin{quote}
„Iss, Hassan!“ sagte Tantchen Ola, während sie sich neben mich setzte.
„Wir haben für dich englisches Essen zubereitet: Fish and Chips!“

„Sprichst du deine Gebete, Hassan?“ fragte mein Onkel.

„Um ehrlich zu sein: nein, das tue ich nicht.“

„Oh, du musst beten! Der Prophet Mohammed sagte,
dass ‚das Gebet der Schlüssel zum Paradies ist‘.“

„Ich bin mir nicht sicher, ob ich an all das glaube.
Ich meine, warum braucht Gott unsere Gebete?“

„Gott braucht unsere Gebete nicht. Aber wir haben das Bedürfnis zu beten.
Um Dank zu sagen, und um seine Hilfe zu ersuchen.“

„Ich verstehe immer noch nicht,
warum wir Dank sagen oder in Gebeten um Hilfe bitten müssen.“

„Hast du den \Quran\ gelesen, Hassan?“

„Ein wenig.“
\end{quote}

Onkel Fouad nahm ein Buch aus dem Regal.

\begin{quote}
„Hier ist eine englische Übersetzung für dich.
Ich möchte, dass du mir versprichst, es zu lesen.“
\end{quote}

Ich sträubte mich etwas zu versprechen, das ich nicht tun wollte,
aber da ich ein Gast in seinem Haus war, konnte ich kaum ablehnen.
Ich dachte, ich könnte ein paar Seiten lesen,
und es dann höflich zur Seite legen.

\begin{quote}
„Danke. Okay, mach ich.“

„Insha-Allah,“ ermunterte Onkel Fouad.

„Insha-Allah,“ erwiderte ich.
\end{quote}

Am nächsten Tag gingen mein Vater und mein Onkel hinaus,
und ließen mich zu Hause mit Tantchen Ola zurück.
Also nahm ich den \Quran\ in die Hand, wie ich versprach, und begann zu lesen.
Zu meiner Überraschung konnte ich es nicht weglegen.
Der \Quran\ ist kein Buch wie jedes andere.
Es befolgt überhaupt keine der Grundsätze der normalen Prosa.
Es hat keinen klaren Anfang und kein klares Ende.
Es gibt keine lineare Handlung und keine saubere Lösung.
Es scheint von einer Schilderung sehr plötzlich zur anderen zu springen.
Selbst sein Stil ändert sich ohne Vorwarnung von einer stetigen Erzählung
zu einer temporeichen Reimprosa.
Dennoch fand ich es seltsamerweise unwiderstehlich.

\begin{quote}
„Elif Lam Mim.“
\end{quote}

Ich sah auf zu Tantchen Ola, die still sitzend eine Zigarette rauchte,
während sie ein Magazin las,
das schöne Frauen beim Stolzieren über einen Laufsteg zeigte.

\begin{quote}
„Was bedeuted ‚Elif Lam Mim‘?“

„Niemand weiß es.“ Sie lächelte.
„Manche Kapitel im \Quran\ beginnen mit Buchstaben des Alphabets.
Gelehrte haben versucht sie zu erklären.
Aber niemand weiß es genau.“

„Meinst du, es ist ein Mysterium?“

„Ja.“
\end{quote}

Mir gefielen Mysterien.

\begin{quote}
\itshape
„Allah ist das Licht der Himmel und der Erde.
Das Gleichnis Seines Lichts ist wie eine Nische,
worin sich eine Lampe befindet.
Die Lampe ist in einem Glas.
Das Glas ist gleichsam ein glitzernder Stern –
angezündet von einem gesegneten Baum, einem Ölbaum,
weder vom Osten noch vom Westen, dessen Öl beinah leuchten würde,
auch wenn das Feuer es nicht berührte.
Licht über Licht. …“
(\QRef{24:35})

„Und wenn Meine Diener dich nach Mir fragen (sprich):
‚Ich bin nahe. Ich antworte dem Gebet des Bittenden, wenn er zu Mir betet. …‘“
(\QRef{2:186})

„Wahrlich, Wir erschufen den Menschen, und Wir wissen alles,
was sein Fleisch ihm zuflüstert; denn Wir sind ihm näher als die Halsader.“
(\QRef{50:16})

„Allahs ist der Osten und der Westen; wohin immer ihr also euch wendet,
dort ist Allahs Angesicht. Wahrlich, Allah ist freigebig, allwissend.“
(\QRef{2:115})

„Und weise deine Wange nicht verächtlich den Menschen
und wandle nicht hochmütig auf Erden;
denn Allah liebt keine eingebildeten Prahler.“
(\QRef{31:18})

„Und (gedenket der Zeit) da Wir einen Bund schlossen mit den Kindern Israels:
‚Ihr sollt nichts anbeten denn Allah;
und Güte (erzeigen) den Eltern und den Verwandten und den Waisen und den Armen;
und redet Gutes zu den Menschen
und verrichtet das Gebet und zahlet die Zakat.‘ …“
(\QRef{2:83})

„O ihr Menschen, Wir haben euch von Mann und Weib erschaffen
und euch zu Völkern und Stämmen gemacht, dass ihr einander kennen möchtet.
Wahrlich, der Angesehenste von euch ist vor Allah der,
der unter euch der Gerechteste ist. …“
(\QRef{49:13})
\end{quote}

Ich fing an zu weinen.
Ich fühlte mich dumm und versuchte meine Tränen vor Tantchen Ola zu verstecken.
Aber ich konnte nicht aufhören.
Ich fühlte eine seltsame Kraft in meinem tiefsten Inneren.
Gewiss, es war die sanfte und liebevolle Gegenwart Gottes, die zu mir sprach.
Es war so, als ob ein Schleier von meinem Gesicht fiel
und ich die Wahrheit entdeckte, nach der ich von Kindheit an suchte.
Es war eine zutiefst spirituelle und emotionale Zeit für mich.
Ich konnte kaum den \Quran\ erkennen,
so wie er heute von den Medien zitiert wird,
mit seinen heftigen und schlimmen Passagen.
Nie sah ich solche Verse zu dieser Zeit.
Nicht dass sie da gewesen wären,
aber sie sprachen mich nicht auf wörtliche Weise an.

Ich unternahm eine Fahrt zur Amerikanischen Universität,
aber der Leiter der Abteilung für Ägyptologie war zu der Zeit nicht anwesend.
Ich ging nicht noch einmal, denn ich hatte das Interesse verloren.
Die meiste Zeit meiner zwei Wochen in Ägypten verbrachte ich damit,
nur den \Quran\ zu lesen, und ich machte außerdem gelegentlich Besuche,
um andere Mitglieder meiner neu entdeckten Großfamilie kennenzulernen.
Auch dort wandten sich die Gespräche ständig der Religion zu.

\begin{quote}
„Ein Freund von mir sagt,
dass ich nur durch den Glauben an Jesus erlöst werden kann,
weil er für unsere Sünden gestorben ist.“

„Der Islam sagt das Gegenteil,“ sagte Magdi. „Der \Quran\ sagt:

‚Wer den rechten Weg befolgt, der befolgt ihn nur zu seinem eignen Heil;
und wer irregeht, der geht irre allein zu seinem eignen Schaden.
Und keine Lasttragende trägt die Last einer andern.‘
(\QRef{17:15})“

„Der Islam ist die Religion unserer
‚\href{http://de.wikipedia.org/wiki/Fitra}{Fitra}‘ (Natur);
es ist in völligem Einklang mit unserem natürlichen Instinkt.“

„Warum kann es dann nicht jeder einsehen?“

„Der Prophet sagte:
\emph{‚Die Menschen sind im Schlaf und nur wenn sie sterben, wachen sie auf.‘}
Das ist die Natur dieser Welt, Hassan.
Wenn alles klar und einfach wäre, dann gäbe es keine Prüfung.“
\end{quote}

Magdi gab mir ein Buch von Hadithen (Aussprüche des Propheten Mohammed),
das ich von der ersten bis zur letzten Seite las.
Ein Hadith im Besonderen berührte mich tief:

\begin{quote}
„(Gott sagt) Ich bin wie Mein Diener über Mich denkt.
Ich bin mit ihm, wenn er Meiner gedenkt.
Wenn er sich Mir um eine Handspanne nähert,
nähere Ich Mich ihm um eine Armlänge.
Wenn er sich Mir um eine Armlänge nähert,
nähere Ich Mich ihm um zwei Armlängen.
Wenn er zu Mir gehend kommt, komme Ich ihm eilend entgegen.“
(Bukhari 8/171)
\end{quote}

Als es Zeit wurde nach England zurückzukehren,
wollte ich Ägypten nicht verlassen und schwor bald wieder zurück zu sein.
Es war ein unglaubliches Erlebnis und ich fühlte ein Gefühl des Überschwangs –
so als ob sich eine magische Tür geöffnet hätte.
Ich hatte endlich Gott gefunden.
Ich kehrte zurück nach England voller Eifer und Entschlossenheit,
um mehr über meinen wiederentdeckten Glauben zu erfahren.
Diane war schockiert über meine plötzliche Konvertierung und war sich sicher,
dass es bloß eine vorübergehende Modeerscheinung war.
Sie gab mir nach und hoffte, dass ich wieder zur Vernunft kommen würde.
Aber das einzige worüber ich reden konnte, war der Islam.
Ich erklärte, dass der Islam Sex vor der Ehe verbietet und bestand darauf,
dass wir aufhören zusammen zu schlafen.
Ich hörte auf zu trinken und zu rauchen, und fing an regelmäßig zu beten.
Ich belehrte Diane über den Tag des Jüngsten Gerichts und Himmel und Hölle
und die Aussprüche des Propheten Mohammed,
und versuchte sie verzweifelt vom Islam zu überzeugen.
Schließlich haben wir beide erkannt, dass wir unsere Zeit verschwendeten.
Diane hatte kein Interesse daran Muslimin zu werden
und ich machte auch keine Phase durch.
Unsere Beziehung war zu Ende.
Ich zog aus unserer Wohnung aus,
und ließ unter anderem meine Plattensammlung und meine Gemälde zurück.
Ich hatte keine Notwendigkeit
für götzendienerische Dinge, die mich vom Pfad Allahs ablenkten.


%Chapter 2:
\chapter{Der Pfad Allahs}

\img{scale=1}{Hassan_1980.jpg}{}

Kurz nach meiner Rückkehr von Ägypten war ich entschlossen,
mich in das Studium des Islam und der arabischen Sprache zu vertiefen.

Meine eigene religiöse Erweckung schien
mit einer allgemeinen Erweckung zum Islam in der Welt um mich herum
in den späten 70er und frühen 80er Jahren zusammenzufallen.
Ich weiß nicht, ob dieser Anstieg des islamischen Bewusstseins
von Ereignissen zu dieser Zeit ausgelöst wurde,
aber viele Muslime der zweiten Generation in Großbritannien fingen an
ihre Geburtsreligion wiederzuentdecken,
nachdem sie eine lange Zeit nur dem Namen nach Muslim waren.
Ein Hauch von Spannung und Dynamik herrschte
in der muslimischen Gemeinschaft, besonders in London.
Studienkreise und familiäre Versammlungen entstanden
in Wohnzimmern oder Gemeindezentren.
Jeden Freitag schien der Schaukasten in der Eingangshalle
der Moschee in Regent’s Park immer mehr voll mit kleinen Kärtchen zu werden,
die neue Ereignisse ankündigten.
Selbst das Grüßen mit ‚Salam‘ nach dem Gebet führte für gewöhnlich
zu einer Einladung zum Dhikr (Gedenken an Gott)
oder zum Halaqa (islamischer Gesprächskreis).
Als ich bei solchen Versammlungen teilnahm,
war ich erstaunt über die Vielfalt der Teilnehmer:
Asiaten, Europäer, Afrikaner, Türken, Kurden, Araber, Malaysier und Iraner.
Sie kamen aus allen Gesellschaftsschichten:
Beamte, Studenten, Busfahrer, Ärzte und Parkplatzwächter.
Es gab keine Barrieren zwischen Rassen, Klassen und Nationalitäten.
Muslim zu sein, war das einzige was zählte und es gewährte
sofortige Mitgliederschaft in der Ummah (Gemeinschaft).
Das Ziel der islamischen Treffen war, über den Islam zu lernen,
aber ebenso wichtig war die soziale Seite –
das Kennenlernen anderer Muslime in der Umgebung
und der Aufbau eines Gefühls von Brüderlichkeit.
Nach und nach brachten diese Treffen andere Treffen hervor,
die den Bedürfnissen bestimmter Gegenden,
ethnischen Gruppen oder Interessen gerecht wurden.

Die erste Versammlung, die ich im Jahre 1980 zu besuchen begann,
war „Die Islamische Gesellschaft der Gläubigen“,
welche abwechselnd in den Wohnzimmern von Bruder Schafiq und Bruder Azim
abgehalten wurden – beide waren Asiaten aus Ost-Afrika
und die Ältesten der örtlichen Gemeinschaft.
Die meisten Besucher waren gebürtige Muslime, jedoch keine praktizierenden,
und so gab es einen gewissen Entdeckergeist.
Wir fingen an, die Grundlagen der Glaubensvorstellungen
und Praktiken des Islam durchzugehen.
In jenen Tagen gab es keine Sektiererei oder andere Uneinigkeiten,
die später hervortraten, und es gab kein dogmatisches Beharren
auf diesem oder jenem Aspekt des islamischen Glaubens.
Die Versammlungen waren tolerant und aufgeschlossen.
Außerdem hatte ich Freude an diesen Treffen, wegen all dem wunderbaren Essen,
das wir am Ende hatten. Na\`ima – Schafiqs Ehefrau –
würde uns marokkanischen Kuskus, libanesisches gefülltes Gemüse
oder türkische Köfte servieren, während Kaira – Azims Ehefrau –
uns Biryani, Curry und Samosas vorlegen würde.
Ich liebte das Gefühl der Zugehörigkeit und der Identität,
das ich dadurch erhielt.
Es war so, wie ein Teil einer riesigen Familie zu sein,
die eine besondere Verbundenheit gemein hatte.

Eines der Dinge, das bei den Versammlungen
der „Islamischen Gesellschaft der Gläubigen“ und bei Gesprächen
mit fast jedem Muslim herauskam,
war die Wichtigkeit Arabisch zu lernen – die Sprache des \Quran.
Ich wusste es war ein essentieller Schritt,
um den Islam wirklich zu verstehen.
Ich hatte auch das Gefühl, dass es meine Pflicht war Arabisch zu lernen,
da es ein Teil meiner Herkunft und Identität war.
Ich schrieb mich für ein Studium in Arabistik und Islamwissenschaften
an der Schule für Orientalistik und afrikanistische Studien (SOAS) ein.
Mein Studienleiter war David Cowan, der Autor von „Modern Literary Arabic“
(Dichterisches Arabisch in der Moderne), ein älterer aber rüstiger Schotte,
der in seinem Jugendalter zum Islam übertrat.
Zu dieser Zeit waren auch Dr.\ Wansbrough und Dr.\ Cook bei SOAS –
die Salman Rushdies ihrer Zeit.
Beide hatten vor kurzem Bücher veröffentlicht,
die die Behauptungen der göttlichen Urheberschaft des \Quran\ untergruben.
Wansbrough behauptete, dass der \Quran\ ein Produkt verschiedener Quellen sei,
verbessert und perfektioniert durch die mündliche Erzählung nach Mohammed,
und Dr.\ Cook, zusammen mit Patricia Crone, behauptete,
dass der Islam als eine Variante des Judentums begann
und sich nur viel später zu einer getrennten Religion entwickelte.
Keines der beiden Bücher erzeugte einen bemerkenswerten Aufschrei.
Heute würde solche Kritik gegen den Islam zweifellos
weltweite Demonstrationen auslösen,
doch die muslimische Gemeinschaft in den späten 70ern und frühen 80ern
war noch nicht politisiert.
Die radikalen Gruppen, die heute existieren,
hatten erst begonnen in Großbritannien ihren Einfluss zu verbreiten.
Als ein junger Muslim,
der alles über den Islam begierig in sich aufnehmen wollte,
fand ich die Atmosphäre bei SOAS belebend,
und ich nahm an jeder außerschulischen Vorlesung und Debatte teil.
Auch konnte ich nicht aufhören alles Mögliche zu lesen,
das den Islam zum Thema hatte,
und so wurde die SOAS-Bibliothek mein zweites Zuhause.
Ich blieb und studierte dort bis spät in die Nacht,
und wurde regelmäßig vom Personal darum gebeten zu gehen,
damit es zusperren konnten.

Mitten in meinem Bestreben mehr über den Islam zu lernen,
war ich auch damit beschäftigt die Botschaft an andere zu verbreiten,
im Besonderen an die Mitglieder meiner eigenen Familie,
die nicht praktizierten.
Ich hatte ein brennendes Verlangen danach ihnen zu lehren was ich
entdeckt hatte, um sie vor dem Höllenfeuer zu retten.
Meine zwei ältesten Schwestern waren bereits fromme Musliminnen,
aber der Rest meiner Familie war es nicht.
Es war meine islamische Pflicht ihnen Da\`wa zu geben,
was wortwörtlich ‚Einladung‘ bedeutet.
Ich hatte kurz zuvor den Film „Der Gesandte Gottes“
über das Leben des Propheten Mohammed gesehen.
Es fängt dramatisch mit drei maskierten Reitern an,
die durch die Wüste gallopieren bis sie an den Höfen der Herrscher
von Byzanz, Persien und Ägypten ankommen, um ihnen Briefe zu überbringen,
die sie zum Islam einluden.
Dies ist der Brief an Heraklius, der byzantinische Herrscher:

\begin{quote}
\em
Im Namen Gottes, des Allbarmherzigen.
Hiermit fordere ich Dich auf, in den Islam einzutreten.
Wenn Du den Islam annimmst, dann bist Du erlöst
und Gott gibt Dir im Diesseits und im Jenseits zweifache Belohnung.
Wenn Du das nicht tust, dann wirst du die Konsequenzen tragen müssen.

Siegel, Prophet Mohammed
\end{quote}

„OK,“ dachte ich mir, „wenn das die Art des Propheten war,
dann werde ich es genauso tun.“

\begin{quote}
\em
Liebe Evette,

Ich lade dich zum Islam ein.
Wenn Du den Islam annimmst,
dann bist Du erlöst und Gott gibt Dir im Diesseits
und im Jenseits zweifache Belohnung.
Wenn Du das nicht tust, dann wirst du die Konsequenzen tragen müssen.

Alles Liebe,
Hassan
\end{quote}

Ich weiß nicht, wie Evette auf diesen Brief reagiert hat,
denn sie hat nie etwas gesagt.
Ich entschied mich dazu meine Taktik zu ändern
und mich auf Einzelgespräche zu konzentrieren.
Ich fing mit meinen jüngeren Geschwistern an.
Ich habe sehr viel Zeit damit verbracht,
Verse aus dem \Quran\ zu lesen und deren Bedeutung zu diskutieren.

\img{scale=0.4}{Hassan_in_Regents_Park_Mosque.jpg}
{Die Moschee im Regent’s Park}

Eines Tages war ich zu Besuch bei Bruder Schafiq,
als er vorschlug ihn zu einem „Tablighi Jamaat“ zu begleiten.
Tablighi Jamaat war eine Bewegung,
die in Indien von Muhammad Kandhalawi gegründet wurde,
der die Muslime zurück auf den Pfad des reinen Islam bringen wollte.
Sie halten sich an eine Sorte des sunnitischen Islam namens Deoband,
die von großen Teilen Südost-Asiens befolgt wird,
und die eine enorme Anhängerschaft in Pakistan und Indien hat.
Während den 80ern wurde die Gruppe sehr populär in Großbritannien.

\begin{quote}
„Was ist die Tablighi Jamaat?“ fragte ich.

„Es ist eine Versammlung, wo sie Vorträge über den Islam halten.
Möchtest du nicht kommen?“

„Wo ist das?“

„Dewsbury.“

„Wo ist Dewsbury?“

„In der Nähe von Leeds.“

„Leeds? Das ist ja meilenweit entfernt!“

„Ach, es wird nicht lange dauern. Komm schon –
du wirst eine Menge lernen und daraus wirklich Nutzen ziehen.“
\end{quote}

Bevor ich mich versah,
befand ich mich auf einer langen Autoreise nach Yorkshire.
Nachdem wir ein paar Mal angehalten haben, um ein paar Brüder abzuholen,
kamen wir in Dewsbury an – einer kleinen Stadt umgeben vom Yorkshire-Moor,
mit viktorianisch altertümlichen Reihenhäusern.
Es sah überhaupt nicht nach einem Ort aus,
an dem sich Muslime versammeln würden, um über den Islam zu diskutieren.
Doch als wir um eine Ecke abbogen,
sahen wir uns einer riesigen Moschee am Ende der Straße gegenübergestellt.
Was noch weniger passte war der Anblick von kleinen asiatischen Kindern
in pakistanisch traditionellen Kleidern,
die den ganzen Weg entlang zur Moschee
mit Spielsachen auf den Pflastersteinen spielten,
während Frauen in Kopftüchern in Urdu plaudernd
an halb geöffneten Türen standen, die zum Bürgersteig führten.
Wäre da nicht der graue, nördliche Himmel über uns gewesen,
so hätte ich es womöglich für die Innenstadt von Karachi gehalten.

Die Moschee in Dewsbury war nach wie vor nicht fertig.
Zementsäcke und Gipsplatten lagen gestapelt an Porenbetonsteinen,
wobei die Hauptnutzfläche durch unfertige Säulen gestört wurde,
aus denen Drahtgitter herausragten.
Die Moschee war gefüllt und es liefen mehrere Vorträge.
Die meisten Vorträge waren in Urdu,
aber ich wurde zu einer Versammlung für Englischsprachige mitgenommen.
Der Maulana, der diese Versammlung leitete, saß auf einem erhöhten Podest
und sprach in gebrochenem Englisch zu einer gemischten Versammlung
von Arabern, Afrikanern, Malaysiern und Asiaten –
von denen ich annahm, dass sie nicht Urdu verstanden.
Auf einer Seite von mir saß ein dicker,
afrikanischer Mann in einem farbenfrohen Kaftan mit einem zierenden Hut.
Er lächelte und schielte auf mich durch seine Brille,
welche derart dicke Linsen hatte, dass seine Augen doppelt so groß schienen.
Während der Rede starrte er mich weiterhin an.
Ich sah kurz zu ihm hinüber – runzelte leicht die Stirn,
in der Hoffnung er würde meinen Ärger bemerken,
doch er grinste nur und starrte mich noch unaufhörlicher an.

\begin{quote}
„Bist du Muslim?“ wagte er schließlich zu fragen.

„Ja,“ antwortete ich mit gedämpfter Stimme, um nicht die Rede zu stören.

„Maschallah! (Was Gott will!) Wie lange bist du schon Muslim?“
\end{quote}

Ich zögerte. Ich war mir nicht sicher, was ich sagen sollte.
Eigentlich wurde ich als Muslim geboren.

\begin{quote}
„Nun ja, ich praktiziere seit einem Jahr.“

„Weißt du wie man betet?“

„Ja.“
\end{quote}

Er dachte, ich wäre ein Konvertit zum Islam.
Bald stellte ich fest, dass britische oder westliche Konvertiten
eine besondere Aufmerksamkeit von Muslimen auf sich zogen.
Es war teilweise ein Beschützerinstinkt die Uneingeweihten zu leiten,
und um neuen Muslimen dabei zu helfen die Grundlagen zu verstehen.
Doch ein weiterer Grund war das Gefühl der Unterlegenheit,
die das Zeitalter des europäischen Kolonialismus bei den Muslimen zurückließ.
Im Unterbewusstsein erachteten immer noch viele Muslime
ihre weißen europäischen Kolonisatoren als überlegene Wesen.
Die Tatsache, dass ein weißer Mann oder eine weiße Frau
zum Islam konvertiert war, war eine besondere Art von Beweis,
dass der Islam in der Tat richtig ist.
Solche Konvertiten erhielten ziemlich viel Aufmerksamkeit.
Dieser Einstellung begegnete ich sehr oft und ich musste mir auch Vorträge
über die grundlegendsten und offensichtlichsten Aspekte des Islam anhören.
Es fühlte sich äußerst herablassend an.

\begin{quote}
„Mein Vater ist aus Ägypten; meine Mutter ist Britin.“
Das war ein Satz, den ich oft verwenden würde.

„Ist deine Mutter Muslimin?“

„Ähm... nun ja... nicht wirklich.“

„Oh, du musst versuchen sie zu einer Muslimin zu machen.“

„Ja... inschallah...“

„Es ist deine Pflicht deine Familie vor dem Höllenfeuer zu retten!“

„Ich hoffe es macht dir nichts aus, wenn wir später darüber sprechen, Bruder.
Ich möchte den Vortrag anhören.“
\end{quote}

Der Maulana sprach über den Tod und den Tag des Jüngsten Gerichts.

\begin{quote}
„Der Prophet sagte: ‚Wenn ein Toter zu Grabe getragen wird,
dann werden ihm drei Dinge folgen.
Zwei von ihnen werden nach der Beerdigung zurückkehren,
aber eines wird bei ihm bleiben.
Das sind seine Verwandten, sein Reichtum, und seine Taten.‘“
Er pausierte.

„Seine Verwandten werden zurückkehren und sein Vermögen wird verteilt werden!“

Er machte seine Augen weit auf: „Doch seine Taten werden bei ihm bleiben.“

„Ja, meine verehrten Brüder, unsere Taten sind von größter Bedeutung.
Nur unsere Taten werden uns bis zu unserem Grabe begleiten.
Sie werden bei uns bleiben, während wir auf das Gericht warten,
und sie werden an unserer Seite stehen, wenn wir befragt werden.
Sie werden dann entweder für uns oder gegen uns sprechen.
Dann werden wir die Wichtigkeit selbst der kleinsten Taten erkennen.
Doch es wird zu spät sein zurückzukehren, um die Dinge wieder gutzumachen.
Keine zweite Chance!“
\end{quote}

Der Amir fing an die Tugenden des Tabligh
(die Verbreitung der Botschaft des Islam) zu preisen.

\begin{quote}
„Was ist wichtiger als Zeit auf dem Pfad Allahs zu verbringen?
Lieben wir Allah und seinen Propheten wirklich mehr als uns selbst?
Oder hängen wir zu sehr an den Annehmlichkeiten
und den irdischen Freuden dieser Welt?
Sind wir so schwach in unserem Glauben,
dass wir es uns nicht leisten können nicht einmal ein bisschen Zeit
für den Dienst gegenüber unserem Schöpfer zu nehmen?
Alles, was du tun musst, ist Niyyat (Absicht) zu machen,
für ein paar Wochen oder ein Monat oder ein Jahr!“

„Was meint er mit ‚Niyyat für ein Jahr machen‘?“
flüsterte ich zum jungen, malaysischen Studenten neben mir.

„Er meint, du musst die Absicht fassen
mit einer ‚Jamaat‘ (Gruppe) mitzufahren, um andere zum Islam einzuladen.“

„Was?“
\end{quote}

Nacheinander um mich herum standen sie auf, um zu gehen.
Die meisten hatten schon ihren Rucksack bereit.
Manche griffen sich sogar Reisepässe und Flugtickets.

\begin{quote}
„Ich kann nirgendwo hingehen! Ich muss nach Hause.“
\end{quote}

Doch der malaysische Student war bereits aufgestanden,
um sich einer Gruppe anzuschließen, die auf dem Weg nach Newcastle war.

Ich unterdrückte den Drang zur Tür hinauszustürmen,
und versuchte stattdessen eine Ausrede auszudenken.
Aber nach all dem Gerede über den Tag des Jüngsten Gerichts
und die Wichtigkeit selbst der kleinsten Taten,
hörte sich jede Ausrede in etwa so an:
\emph{„Es ist OK, es macht mir nichts aus in der Hölle zu schmoren.“}
Ich stand auf und sah mich nach Shafiq um,
aber als ich ihn sichtete, lächelte und winkte er nur.

\begin{quote}
„Wir sehen uns wieder in zwei Wochen, Hassan!“ rief er,
bevor er mit einer großen Gruppe wegging.
\end{quote}

Der Amir bemerkte meinen verwirrten Ausdruck.

\begin{quote}
„Hast du Absicht gefasst, Bruder?“

„Meine Familie weiß nicht wo ich bin.“

„Du kannst sie anrufen.“

„Ich muss zurück nach Hause. Ich habe Dinge zu erledigen!“

„Ist es nicht wichtiger Zeit auf dem Pfad Allahs zu verbringen?“

„Nun ja... ich denke schon... aber was werde ich essen?“

„Für Verpflegung wird gesorgt!“

„Wo werde ich schlafen?“

„Du wirst in der Moschee schlafen!“

„Was ist mit meiner Mutter?“

„Sie wird schon zurechtkommen!“

„Was ist mit...“ Mir kam kurz der Gedanke, mein Bein zu packen,
vor Schmerzen zusammenzuzucken und über eine alte Kriegswunde zu klagen.

„Fass einfach Absicht, und Allah wird dir Tür und Tor öffnen!“
\end{quote}

Meine Absicht war nach Hause zu gehen, aber Gottes Absicht war offenbar anders,
und so wurde ich einer Delegation zugewiesen, die auf dem Weg nach Leeds war.
Wir gingen der Reihe nach aus der Moschee und stiegen in einen Kleinbus ein,
der auf uns draußen wartete.

Ich war noch nie in Leeds und tröstete mich selbst mit dem Gedanken,
dass ich zumindest einen neuen Ort besuchen würde.
Aber abgesehen von der Moschee, sah ich sehr wenig von Leeds.
In den nächsten vierzehn Tagen aßen, schliefen und beteten wir in der Moschee.
Mir wurde ein Schlafsack und eine schlecht passende Shalwar Kamiz
aus Baumwolle gegeben (langes Hemd mit Schlabberhose.)
Zur Essenszeit wurden lange Papierrollen ausgebreitet
und serviert wurden Curry mit Chapatis.
Es wurde abwechselnd von den Mitgliedern der Gruppe gekocht,
allerdings ließen sie mich nur den Abwasch machen.

Uns wurde gesagt, dass wir Politik und religiöse Streitthemen
nicht diskutieren sollen.
Uns wurde sogar davon abgeraten die Bedeutung des \Quran\ zu diskutieren.
Einmal saß ich zusammen mit einem Bruder und ich versuchte ihm einige
der arabischen Wörter zu erklären, doch dann sagte man mir damit aufzuhören.

\begin{quote}
„Du solltest keinen Tafsir geben (Kommentar zum \Quran),
außer wenn du ein Gelehrter bist,“ sagte der Bruder.
„Lerne einfach die Suren auswendig, die uns der Maulana gelehrt hat.“

„Aber dann würden wir Wörter rezitieren ohne sie zu verstehen!“
beschwerte ich mich.

„Das ist egal. Du wirst von der Rezitation einen Nutzen ziehen!“
\end{quote}

Mir fiel es schwer einzusehen, wie man davon Nutzen ziehen konnte,
Wörter zu rezitieren ohne ihre Bedeutung zu verstehen.
Doch bald lernte ich, dass die Handlung an sich
als eine Art Gottesdienst gesehen wird,
der jenen Segen bringt, die ihn verrichten.
Daher ist das Auswendiglernen des \Quran\ ohne es zu verstehen
sehr weit verbreitet unter den Muslimen, insbesondere bei jenen,
die aus Ländern kommen, in denen nicht Arabisch gesprochen wird.
(Selbst jene aus arabischsprechenden Ländern haben Schwierigkeiten
die archaische Sprache des \Quran\ zu verstehen.)
Die Mehrheit jener, die mit mir in der Leeds-Moschee waren,
waren pakistanischer oder indischer Herkunft,
und obwohl sie sehr lange Passagen, manche sogar den ganzen \Quran,
auswendig gelernt hatten, konnten nur wenige die Worte verstehen.
Den \Quran\ auswendig zu lernen ist nicht so schwierig wie es erscheinen mag.
Die Sprache des \Quran\ hat einen poetischen Rhythmus
mit sich wiederholenden Phrasen und Mustern.
Zum Beispiel beginnen Sätze mit „Qul!“ (Sprich!)
oder sie enden mit zwei Attributen von Gott.
Bestimmte Geschichten über frühere Propheten oder Parabeln
über Gläubige und Ungläubige werden überall im \Quran\ wiederholt aufgegriffen,
so dass ein Teil, den man  bereits auswendig gelernt hat,
in einer ähnlichen Form an anderer Stelle vorkommen wird.
Kindern wird der \Quran\ auch von klein auf gelehrt –
in manchen Fällen sobald sie Laute nachahmen können –
und Familien veranstalten eine Feier namens Khatam (Beendigung),
wenn der Sohn oder die Tochter den ganzen \Quran\ gelesen hat.
Ich konnte die letzte Dschuz\´ auswendig (1/30 des \Quran),
welche die letzten Suren (Kapiteln) am Ende des \Quran\ beinhaltet,
und diese werden am häufigsten in den täglichen Gebeten verwendet.
Ich lernte auch viele weitere wichtigen Suren und Verse auswendig,
w.z.B. die letzten drei Verse von Al-Baqara und Al-Kahf und „Der Thronvers“
und Suren wie „Yasin“ und „Al-Rahman“.
Glücklicherweise half mir das Studium an der SOAS dabei
die Bedeutung des Textes zu lernen, den ich mir einprägte.

Es gab ein besonderes Buch, welches wir verstehen sollten
und wozu wir ermuntert wurden.
Es hieß „Die Lehren des Islam“ von Maulana Zakariya al\–Kandhalawi;
ein dicker Wälzer mit schlecht bedruckten Seiten aus billigem Papier
und gebunden in einem kitschig-roten Plastikeinband.
Jedem von uns wurde ein ureigenes Exemplar gegeben, das wir studieren sollten,
wenn wir nicht dabei waren Gebete zu verrichten oder Reden anzuhören.
Dieses Buch war die Quelle von vielen Vorträgen,
die ich in der Dewsbury-Mosche und in den folgenden Tagen in Leeds hörte.
Es erzählte Geschichten über den Propheten und seine Gefährten
und es zitierte Passagen aus dem \Quran.
Die Betonung lag darauf die allerhöchste Stufe der Pietät,
Abstinenz und Gottesfurcht zu erreichen.
Anfangs fand ich viele dieser Geschichten seltsam
und sah keinen Zusammenhang zur Gesellschaft um mich herum.

Eine der Geschichten begann mit der Überschrift
\emph{„Der Prophet tadelt seine Gefährten fürs Lachen.“}
Es stand geschrieben:

\begin{quote}
„Der Prophet (Friede sei mit ihm) kam einmal zur Moschee für das Gebet,
wo er ein paar Leute beim Lachen und Kichern bemerkte.
Er sagte: ‚Wenn ihr eurem Tod gedenken würdet,
so würde ich euch so nicht sehen.
Gedenket oft eurem Tod. Kein Tag vergeht, an dem das Grab nicht ruft:
„Ich bin eine Wüste,“ „Ich bin ein Ort des Staubs,“
„Ich bin ein Ort der Insekten.“
Wenn ein Gläubiger in das Grab gelegt wird, so sagt es:
„Willkommen. Es ist gut, dass du in mich gekommen bist.
Von all den Leuten, die auf der Erde wandeln, mag ich dich am meisten.
Da du nun in mich gekommen bist, wirst du sehen,
wie ich dich unterhalten werde.“
Es weitet sich dann aus, so weit der Bewohner sehen kann.
Eine Tür aus dem Paradies wird für ihn im Grab geöffnet
und durch diese Tür bekommt er die frische und duftende Luft des Paradieses.
Aber wenn ein böser Mensch in das Grab gelegt wird, sagt es:
„Keine Willkommensgrüße an dich. Dass du in mir bist,
ist sehr schlecht für dich. Von all den Leuten, die auf der Erde wandeln,
missfällst du mir am meisten.
Da du mir nun übergeben wurdest,
wirst du sehen wie ich mit dir umgehen werde!“
Es schließt ihn dann so sehr ein,
bis dass die Rippen auf der einen Seite
in die Rippen auf der anderen Seite eindringen.
Nicht weniger als 70 Schlangen werden auf ihn gesetzt,
die ihn bis zum Tag der Auferstehung beißen werden.‘“
\end{quote}

Der Autor erklärt den Zweck des Buches wie folgt:

\begin{quote}
„[Damit] muslimische Mütter, wenn sie in der Nacht zu Bett gehen,
ihren Kindern statt Mythen und Fabeln zu erzählen,
ihnen solche wirklichen und wahren Geschichten des goldenen Zeitalters
des Islam erzählen mögen, so dass in ihnen ein islamischer Geist der Liebe
und Achtung für die Sahaba entstehen kann,
um dadurch ihren Glauben zu verbessern,
und damit es ein nützlicher Ersatz
für die jetzigen Geschichtenbücher sein möge.“
\end{quote}

Je mehr ich las, desto mehr passte ich mich der Mentalität
von Arabien im siebenten Jahrhundert an.
Nach einer mehrtätigen Einschließung in der Moschee
und einer konstanten Routine von Vorträgen,
Gebeten und Lesungen aus diesem Buch,
war ich so sehr in die Ereignisse der frühen Muslime eingetaucht,
dass ich anfing mich von der Welt um mich herum entfremdet zu fühlen,
die ich nun als sündhaft erachtete.
Doch trotz ihrer Sündhaftigkeit,
ordnete der Amir regelmäßig Expeditionen zu genau dieser Welt.
Jeden Tag wählte er eine Gruppe aus drei Leuten aus,
die eine Adresse auf einer Liste besuchen mussten,
welche von Brüdern zuvor gesammelt wurde.
Diese Adressen waren von ortsansässigen Muslimen,
von denen aus welchen Gründen auch immer gedacht wurde,
dass sie „Da\`wa“ nötig hatten – mit anderen Worten:
sie mussten zurück in den Schoß des Islam gebracht werden.
Ich fühlte mich ein wenig beklommen,
als ich schließlich ausgewählt wurde bei einer Gruppe dabei zu sein.

Der Amir gab uns sehr spezifische Regeln für unser Benehmen draußen.
Wir hatten unsere Blicke abzuwenden, von all den haram (verbotenen) Dingen.
Wir hatten unsere Augen auf den Boden ein paar Yards vor uns zu senken,
und ein kleines Gebet für uns selbst zu wiederholen.
Das machte es ein wenig schwierig Wegbeschreibungen zu folgen
und verkehrsreiche Straßen zu überqueren.
Schließlich kamen wir unangemeldet bei dem Haus an.
Ein kleiner, glattrasierter, junge Asiate ließ uns herein
und bot uns etwas zum Essen und Trinken an,
was wir höflich abgelehnt haben, so wie zuvor belehrt –
im Falle, dass der Verdammte verbotene Zutaten beim Kochen verwendet hatte,
wie Gelatine, Fleisch, das nicht halal war, oder Alkohol.
Wir setzten uns behutsam auf den Rand seines Sofas hin, während unser Führer –
für jede Gruppe wurde immer ein Führer bestimmt –
die Vorträge über den Tod und den Tag des Jüngsten Gerichts
und andere bewegende Geschichten aus dem großen, roten Buch wiederholte.
Wir flehten den irregangenen Bruder an zur Moschee zu kommen,
um unseren Amir anzuhören.
Schließlich stimmte er zu sich uns beim Nachmittagsgebet anzuschließen.
Auftrag ausgeführt. Wir machten uns vorsichtig zurück auf den Weg.

Ich begann in eine obsessive Mentalität abzugleiten.
Selbst die kleinsten Rituale und Praktiken des Propheten
nahmen eine außerordentliche und übertriebene Wichtigkeit an –
die befolgt werden mussten, sofern ich die Feuer der Hölle meiden wollte –
während alle anderen Angelegenheiten des Lebens irrevelant schienen.

\begin{quote}
„Wir sollten niemals denken,
dass wir auf irgendeine Weise fortgeschrittener sind als der Prophet es war,“
sagte der Amir.
„Alles, was er getan hat, ist das beste Beispiel für uns.
Selbst wenn wir mit dem Auto oder mit dem Flugzeug reisen,
sollten wir daran denken,
dass das Reisen mit dem Kamel oder dem Esel besser ist,
weil das die Art war, in der der Prophet es getan hat.“
\end{quote}

Die Nachahmung des Propheten beinhaltete auch die Art
wie wir unsere Zähne putzten,
und unser Amir hielt einen Vortrag über die Wichtigkeit
einen Miswak zu verwenden – ein Zweig von einer Baumart,
die man im Nahen Osten findet.
Er erzählte von einem Ausspruch des Propheten:

\begin{quote}
„Wäre da nicht meine Angst meinem Volk eine Erschwernis aufzuerlegen,
dann hätte ich angeordnet,
dass der Miswak vor jedem Gebet benutzt werden muss.“
\end{quote}

Er holte anschließend einen Miswak hervor und demonstrierte
wie man es benutzen sollte.
Er entfernte etwas Rinde an der Spitze
und kaute an ihr bis sie ausgefranst war.
Dann rieb er es über seine Zähne von der einen Seite auf die andere.
Als er fertig war, erzählte er eine Geschichte aus der Zeit Omars,
dem zweiten Kalif des Islam.

\begin{quote}
„Während der Eroberung von Ägypten,
hatte die muslimische Armee enorme Schwierigkeiten den Feind zu besiegen.
Als Omar davon hörte, sagt er, dass es bestimmt wegen einer Missetat sein muss,
die sie begangen haben.
Also fragten sich die muslimischen Kämpfer,
ob sie nicht irgendeine religiöse Pflicht vernachlässigten,
doch sie stellten fest, dass das nicht der Fall war.
Dann fragten sie sich, ob sie nicht eine Sunna
(Praktiken des Propheten) vernachlässigten, und sie erkannten,
dass sie vergessen hatten ihre Zähne mit dem Miswak zu putzen.
Also versammelten sie sich und begannen den Miswak zu benutzen.
Als die Feinde das sahen, dachten sie die Muslime würden sich
darauf vorbereiten über sie herzufallen, und flüchteten deswegen.“
\end{quote}

Jede Sekunde meines Alltags wurde nun von dieser
oder jener Sunna kontrolliert und festgelegt.
Mir wurde gelehrt, mich beim Gang zur Toilette
auf eine bestimmte Weise zu reinigen und ein Gebet zu äußern,
wenn ich die Toilette betrat und verließ.
Der Islam regelte auch die Art wie ich schlief.
In meiner zweiten Nacht wurde ich vom Amir getadelt,
weil ich auf die falsche Art geschlafen habe.
Er erklärte, dass ein Muslim niemals mit seinen Füßen
in Richtung Mekka zeigend schlafen sollte,
man sollte mit dem Gesicht dorthin schauend schlafen.
Ich war mir nicht ganz sicher, ob er meinte,
dass nur mein Kopf in Richtung Mekka zeigen sollte,
oder ob ich buchstäblich mein Gesicht nach Mekka richten sollte.
Sicherheitshalber richtete ich mein Gesicht in Richtung Mekka
und hinderte mich selbst daran mich von Seite zu Seite drehen,
so wie ich es normalerweise tat.
Das machte das Schlafen sehr unangenehm und schwierig.

Ich machte mir mehr und mehr Sorgen darüber,
dass solch ein hohes Niveau auf Form und Detail Acht zu geben,
außerhalb der abgeschirmten Umgebung der Moschee nicht aufrechterhaltbar wäre,
und ich machte mir Sorgen über meine Erlösung,
falls ich es nicht aufrechterhalten können sollte.
Doch es war schwierig diese Sorgen in einer Atmosphäre zu äußern,
in der die Gruppenmentalität jegliches Scheitern stark missbilligte,
wenn es um die Erfüllung der festgelegten Normen ging.
Der Maulana schien darauf stolz zu sein,
wie hart und schwierig es war den Islam ordentlich zu praktizieren
und sprach davon, wie der Prophet Mohammed gesagt hatte:

\begin{quote}
„Es wird eine Zeit kommen, in der jemand,
der standhaft in seiner Religion ist, so ähnlich sein wird wie jemand,
der eine glühende Kohle in den Händen hält.“
\end{quote}

Der Scheich erklärte, dass wenn man ein guter Muslim
in der heutigen Zeit sein will,
es so ist wie wenn man eine rotglühende Kohle umklammert.
Man möchte es instinktiv wegwerfen,
doch man muss sich gegen den Instinkt wehren und es festhalten,
wenn man das Paradies erreichen will und die Hölle meiden möchte.
Ein ‚wahrer‘ Muslim muss auf die Annehmlichkeiten
und die Freuden dieser Welt verzichten und Entbehrung
und Ungemach auf sich nehmen,
wenn er die Annehmlichkeiten und Freuden des Paradieses gewinnen möchte.
Er muss erwarten von Nicht-Muslimen als eigenartig
und seltsam angesehen zu werden, und das Gespött der Gesellschaft ertragen,
so wie der Prophet einmal sagte:

\begin{quote}
„Der Islam fing als etwas Fremdes an,
und es wird wieder zu dieser Fremdheit zurückkehren,
so wie es am Anfang war.
Gibt den Fremden also frohe Botschaft.“
Jemand fragte: „Wer sind die Fremden?“
Der Prophet erwiderte:
„Jene, die sich dem Islam zuliebe von ihren Leuten trennen.“
\end{quote}

Obwohl ich nur zwei Wochen lang in der Moschee in Leeds war,
schien es viel länger gewesen zu sein, und als wir zurück nach Dewsbury kamen,
fühlte ich mich desorientiert und hatte Furcht davor
zur ‚realen Welt‘ zurückzukehren, mit seinen bösen Versuchungen,
bereit um mich von Gott wegzulocken.
Ich schaute mich nach Shafiq um, aber er war nicht zurückgekommen.
Es war Ramadan und wir hatten den ganzen Tag lang gefastet.
Bei Sonnenuntergang brachen alle ihr Fasten zusammen in der Dewsbury\–Moschee.
Ich war am Verhungern, und das Curry
und die Chapatis hatten noch nie so gut geschmeckt.

Sobald wir mit dem Essen fertig waren, fingen wieder die Vorträge an.
Ich entschied mich kurz hinauszugehen, um etwas frische Luft zu holen.
Als ich hinauskam, sah ich, dass draußen um die fünfzehn bis zwanzig Männer
in einer langen Schlange bei der Hinterwand der Moschee standen.
Sie waren alle hinausgeschlichen, um in Ruhe eine Zigarette zu rauchen.
Das war die erste Zigarette nach einem langen Fastentag
und vielen war wegen dem plötzlichen Nikotinrausch schwindelig,
weswegen sie ihre ‚Salams‘ nur lallten,
als sie mich verlegen mit ihren glasigen Augen anblickten.
Obwohl ich vor nicht allzu langer Zeit selbst ein Raucher war,
hatte ich nun das Gefühl, dass es sehr ‚unislamisch‘ ist
und ich missbilligte es sehr.
Die Tatsache, dass viele von ihnen lange Bärte
und große Turbane hatten steigerte meine Empöhrung.
Doch zur selben Zeit gab es einen Teil in mir,
der anscheinend Trost in der Tatsache fand,
dass es praktizierende Muslime gab, die nicht völlig perfekt waren.
Ich fühlte mich besser bei dem Gedanken,
dass ich womöglich selbst kein ‚perfekter Muslim‘ war.

Ich war erfreut, endlich Shafiq am nächsten Tag zu sehen.
Ich wollte wissen, ob er ähnliche Erfahrungen durchgemacht hat wie ich,
und ob er sich auch besorgt fühlte, über die Dinge, die er gehört hatte.
Doch er sah gelassen aus und war nicht beunruhigt durch seine Erfahrungen.
Er erzählte mir wie wunderbar es gewesen ist
und er konnte es kaum erwarten auf sein nächstes Jamaat zu gehen.
Ich war erleichtert, als ich zu Hause ankam.
Ich fühlte mich etwas nervös.
Die Dinge sahen anders aus.
Meine Prioritäten hatten sich geändert.
Ich war um dieses Leben weniger besorgt
und konzentrierte mich viel mehr auf das nächste Leben.
Ich ließ mir einen Bart wachsen, trug einen langen,
weißen Jilbab und eine Mütze.
Nicht nur verrichtete ich die Pflichtgebete,
die zusätzlichen Gebete verrichtete ich noch dazu.
Ich fing an jeden Montag zu fasten und putzte meine Zähne
– selbstverständlich – mit einem Miswak.
Ich entschloss mich dazu dieses höhere Bewusstsein der Gottesfurcht
nicht zu vergessen, welches ich in der Moschee erfahren hatte.
Meine Freunde und meine Familie waren zunächst ein bisschen überrascht
vom noch mehr tiefreligiösen Hassan,
und mit Geduld ertrugen sie meine etwas obsessive Beachtung
jedes winzigen Details im Islam.
Doch bald merkte ich, dass mein Zurücksein in der ‚realen Welt‘
mir eine ausgeglichenere Sichtweise gab,
und das Gefühl der Besorgnis und Furcht wurde langsam schwächer,
da ich erkannte, dass die Besessenheit der Jamaatu Tableegh mit den Regeln
und Ritualen eine verzerrte Auffassung des Islam darstellte.
I habe zu schätzen gelernt, was der \Quran\ lehrte, und zwar,
dass „Gott keinem Menschen mehr auferlegt, als er zu tragen vermag,“
und dass die Anforderungen dieser Welt mit den Anforderungen
der nächsten Welt nicht in Konflikt stehen mussten.
Ich begann ein intellektuelleres und tieferes Verständnis
des Islam anzustreben, als das, was die Jamaatu-Tableegh anzubieten hatte.


%Chapter 3:
\chapter{Der Pfad des Sufi}

\img{scale=0.7}{Members_SOAS_Islamic_Society.jpg}
{Mitglieder der „SOAS Islamic Society“, October 1980.}

Ich trat der „SOAS Students Union Islamic Society“ (Studentenvereinigung) bei,
sobald ich bei der Londoner Universität
im September Jahre 1980 anfing zu studieren,
und wurde zum Präsident im Jahre 1981;
ein Posten den ich bis 1984 inne hatte.
Während meiner Präsidentschaft eröffnete ich einen islamischen Bücherstand,
organisierte Reden und Debatten und führte Filme vor,
bereitete einen Raum für das tägliche Gebet vor
und erhielt die Erlaubnis eine der Hörsäle für Freitagsgebete zu verwenden.
Am Anfang teilten wir unter uns Studenten die Aufgabe die Predigt abzuhalten,
doch dies erwies sich als schwierig,
da die meisten entweder zu beschäftigt waren um eine Rede vorzubereiten
oder sich nicht qualifiziert genug fühlten um es zu tun.
Schließlich entschieden wir uns einen Redner von draußen einzuladen
und sprachen Dr.\ Kalim Siddiqui an –
der Direktor von „The Muslim Institute“ (Das muslimische Institut),
gleich ums Eck in der Endsleigh-Straße.
Dr.\ Siddiqui war besessen von der islamischen Revolution im Iran
und er sah es als seine Mission an eine ähnlich revolutionäre Ideologie
unter den britischen Muslimen zu verbreiten.
Später gründete er „The Muslim Parliament of Great Britain“
(Das muslimische Parlament Großbritanniens) mit dem Ziel
einen „nicht\–territorialen, islamischen Staat“ in GB zu erschaffen.

Anfangs waren unsere Freitagsgebete,
die von Dr.\ Siddiqui und seinem Stellvertreter geführt wurden,
sehr gut besucht.
Die Reden waren immer höchst politisch und drehten sich
um die Geschichte des europäischen Kolonialismus in muslimischen Ländern,
und sie betonten wie dies die Ursache für die Rückständigkeit,
den Armut und den islamischen Werteverfall war,
was die Muslime im Moment ertragen mussten.
Ein großer Teil der Rhetorik war auch auf Amerika gerichtet,
„Der große Satan“, der – wie behauptet wurde –
auf dem Weg war eine neue Form des ökonomischen
und sozialen Kolonialismus zu erschaffen, der viel unheimlicher war,
weil es Marionettenregime benutzte,
um Muslime versklavt und unwissend über ihr islamisches Erbe zu halten.
So interessant dies für junge Studenten auch war,
von denen viele begierig waren sich einer Sache anzunehmen,
so war jedoch das Thema von Gott und Spiritualität auffallend abwesend,
und viele von uns waren es müde Woche für Woche dasselbe anzuhören.
Nach einer Weile fing die Anzahl der Besucher weniger zu werden,
und da die Verantwortung die Stühle wegzuräumen,
Zetteln auszuteilen und alles wieder zurück an ihren Ort zu legen
zu einer lästigen Routinearbeit wurde,
die mein enger Freund Hussein und ich alleine tun mussten,
entschied ich die Freitagsgebete nicht mehr auf dem Kampus abzuhalten
und empfahl den Studenten eine nahegelegene Moschee zu besuchen.

Es gab zwei Moscheen, die nahe genug lagen, so dass man sie besuchen
und für die Nachmittagsvorlesungen wieder rechtzeitig zurück sein konnte
– jedoch musste man das Mittagessen auslassen.
„Die Moschee des Lichts“ nahe der Euston Station,
war eine kleine Räumlichkeit im Keller einer Edward’schen Terrasse,
die schon mal bessere Tage gesehen hatte.
Die meisten in der Gemeinde kamen aus der örtlichen,
bangladeschischen Gemeinschaft,
von denen viele für die British Rail oder als Parkwächter arbeiteten.
Sie würden direkt von der Arbeit zur Moschee rasen und –
immer noch in Uniform gekleidet – Taschentücher auf ihre Köpfe legen
und sich in die engen Plätze hineinquetschten, was dazu führte,
dass die Menge bis zur Straße hinaus überschwappte.
Die einzigen Anzeichen dafür, dass der Raum eine Moschee war,
waren die Beläge auf dem harten Boden und ein kleines, kaputtes Bücherregal,
das sich vollbeladen mit Koranausgaben vorlehnte.
Die zweite Moschee war in den neuen,
vornehmen Büroräumen der „Muslim World League“ (Islamische Weltliga)
in Tottenham Court Road.
Finanziert durch Saudi Arabien, um den Islam zu verbreiten,
war es voll von gut bezahlten Administratoren,
die Anträge auf Förderungen bearbeiteten.
Der Gebetsraum war ein eleganter, mit Teppich belegter Saal,
gut gefüllt mit teuer gebundenen Büchern.

Anfangs besuchte ich die „Moschee des Lichts“,
da mir die Atmosphäre dort authentischer vorkam,
aber dann fing der berühmte Sufi-Prediger Scheich Nasim
jeden Freitag Predigten bei der Islamischen Weltliga abzuhalten.
Ich hatte etwas von den großen Sufi-Dichtern und Autoren gelesen
und fühlte mich von ihrer universalistischen
und mystischen Herangehensweise angezogen,
daher war ich begierig darauf Scheich Nasim zuzuhören.
Geboren im Jahre 1922 in der Stadt Larnaca in Zypern,
reiste er zuerst nach Istanbul, um Chemieingenieurwesen zu studieren,
und dann nach Damaskus im Jahre 1945,
um als Schüler von Scheich Abdullah ad\–Daghestani zu studieren,
der Mitglied des Naqschabandi-Sufi-Ordens war.
Von dort aus reiste er um die Welt, vor allem in West-Europa,
wo er eine Anhängerschaft erlangte, und zur Zeit in Peckham, Süd-London,
seinen Wohnsitz hatte.
Sie sprachen nur Gutes über ihn und erzählten mir,
dass er ‚besonderes‘ Wissen über viele Dinge hatte,
unter anderem wusste er über die Ankunft des Mahdi –
‚der Rechtgeleitete‘, der von den Hadithen prophezeit wird.
Es kommt mir jetzt seltsam vor, dass die Saudis,
die den Sufismus nicht guthießen,
ihm erlaubten bei der Islamischen Weltliga zu predigen,
aber der Direktor zur jenen Zeit war ein freundlicher,
aufgeschlossener Saudi, der viele verschiedene Ansichten akzeptierte,
und man sollte bedenken, dass sich in den frühen 80ern
unter den Muslimen in England noch keine scharfe,
ideologische Trennung entwickelt hatte,
die später noch kommen und sie einnehmen würde.

Scheich Nasim kam an und hielt seine Predigt.
Er trug einen riesigen, grünen Turban und einen wallenden, grünen Umhang.
Zur Seite standen ihm Murids – Schüler seines spirituellen Pfades –
die auf die selbe Weise gekleidet waren.
Ich war sofort von seiner Präsenz und seinem Charisma beeindruckt.
Er hatte eine langsame und wohlüberlegte Art zu sprechen, eine Art,
in der er bei Wörtern pausierte, wenn er sie betonen wollte,
und wenn er lächelte – was er oft tat – sah er aus wie dein liebster Opa,
der kurz davor war etwas Süßes hinter seinem Ohr hervorzuzaubern.
Sein starker türkischer Akzent machte die reizende Qualität
seiner Rede nur noch besser.
Ich genoss seinen Vortrag sehr und wollte persönlich mit ihm
über seine Vorhersagungen über den Mahdi sprechen,
aber es gab nie genug Zeit, da ich immer zurück zur Universität eilen musste.
Doch als mich einer seiner Anhänger
zu einer kleinen Versammlung in Peckham einlud,
nahm ich diese einmalige Gelegenheit sofort an.

Die Moschee in Peckham war eine umfunktionierte Wohnung,
die sich über einem Lebensmittelladen befand,
und der äußere Anschein verriet nichts über die Andachten,
die drinnen stattfinden würden.
Mehr als die hälfte der Anwesenden waren weiße Europäer,
eine komische Mischung aus Berufstätigen,
Ex\–Hippies, Exzentrikern und Schulabbrechern.
Die meisten, mit denen ich gesprochen habe,
schienen sehr narzisstisch zu sein und einen messianischen Komplex zu haben,
was gar nicht mit der Vorstellung des Sufismus zusammenpasste,
die ich durch das Lesen von Rumi und Ibn Arabi bekommen habe.
Ich setzte mich neben einem jungen, britischen Konvertiten nieder,
der aussah wie ein Gangster aus 1920 –
glatte, schwarze Haare, Nadelstreifhose und dunkle Sonnenbrille.
Da der Raum nur schwach beleuchtet war, nahm ich an,
dass er kaum etwas durch seine Brille sehen konnte.

\begin{quote}
„Assalamu-Alaykum, mein Name ist Hassan.“

„Mein Name ist Sulayman.“
Er senkte seine Brille.
„Aber du kannst mich Slim nennen.“

„Bist du regelmäßig hier in der Gesellschaft von Scheich Nasim?“
\end{quote}

Ohne ersichtlichen Grund antwortete er auf Arabisch.

\begin{quote}
“Al-Hamdulilah, kuntu asooru Schaykh Nasim al-halaqa li mudda atiqa.”

(„Gelobt sei Allah, ich besuche die Gesellschaft von Scheich Nasim
schon seit einer Ewigkeit!“)

„Oh,“ nickte ich.

„Wa min ayna anta?“ (Woher kommst du?) fuhr er in Arabisch fort.

„Ana min Finchley fi schimal London,“
(Ich komme aus Finchley in Nord-London)
sagte ich und versuchte in Stimmung zu kommen.
\end{quote}

Scheich Nasim kam herein,
gefolgt von seinen Schülern in grünen Turbanen und Mäntel.
Als er Platz nahm, stürzte eine mollige,
deutsche Dame auf ihn zu und küsste seine Füße.
Einige andere folgten und verbeugten sich entweder vor seinen Füßen
oder küssten seine Hände.
Slim ermunterte mich dazu mit ihm gemeinsam den Scheich zu grüßen.

\begin{quote}
„Assalamu alaykum,“ sagte ich steif,
während ich ihm meine Hand zum Händeschütteln reichte.

“Wa alaykum assalam wa rahmatullah,” antwortete er,
meine Hand nehmend und lächelnd.
\end{quote}

Als wir alle wieder Platz genommen hatten,
begann Scheich Nasim den Zikr (Gottesgedenken) anzuführen –
das gemeinsame Singen der Namen Gottes oder von Gebeten.
Er fing an die Wörter „Allahu, Allah Haqq“
(Gott! Gott ist Wahrheit) zu wiederholen.
Alle machten sofort mit und fingen an
sich sanft im Rhythmus der Wörter hin und her zu bewegen.
Nach einer scheinbar sehr langen Weile signalisierte Scheich Nasim
einen Wechsel der Wörter und des Tempos,
indem er „Allahu, Allah Hayy“ (Gott! Gott ist das Leben) wiederholte.
Als Scheich Nasim noch einen Wechsel einleitete,
nahm das Energieniveau so sehr zu,
dass der gesamte Raum mit dem Rhythmus zu vibrieren schien.

\begin{quote}
„Allah Hayy, Ya Qayyum.“ (Gott ist das Leben, Oh der Wachsame!)
\end{quote}

Die meisten hielten ihre Augen geschlossen
und schienen in einem Zustand der tiefen Konzentration zu sein,
während sie sich zusammen hin und her bewegten.
Ich konnte dem pochenden Rhythmus nicht widerstehen
und begann vor und zurück zu schaukeln,
während ich ständig „Allah Hayy, Ya Qayyum“ wiederholte.
Der Raum war verschwommen und sah wie ein Schwarzweiß-Negativfilm aus;
alles um mich herum fing an zu verschwinden.
Es war fast wie eine psychedelische Erfahrung
und ich war vollkommen versunken in diesem Moment,
als ich mich selbst hin und herwarf ohne jegliche Hemmung und Verlegenheit.
Leider hieß das auch,
dass ich das abrupte Ende der Geschehnisse nicht bemerkte,
und sehr zu meiner Beschämung fuhr ich ein paar Sekunden fort
mit dem Singen und dem Schaukeln,
nachdem alle anderen aufgehört hatten.

Scheich Nasim rezitierte ein paar Gebete und fing mit seiner Predigt an.
Er sprach mit einem schweren Akzent
und sprach viele englische Wörter falsch aus.

\begin{quote}
„Allah liebt alle seine Die-hner.
Allah lieben alles!
Jeden Menschen, jede Kreatur, jede Pflanze, jeden Stein.
Allah hasst nicht.
Wenn Allah hasst etwas, es kann nicht existieren.“
\end{quote}

Er würde jemanden aus dem Zuhörerkreis aussuchen und in die Augen schauen,
so als ob er zu jemand besonderen sprechen würde.

\begin{quote}
„Allahs Liebe ist nicht tierische Liebe, wir sehen in dieser Dunya (Welt).
Es ist Liebe, die sich nie ändert.
Es ist Liebe, die nie sterben.
Unser Ziel ist höhere Liebe zu erreichen und in Liebesozeane einzutauchen.“
\end{quote}

Ich war mir nicht sicher was er mit ‚Liebesozeanen‘ meinte,
aber es hörte sich wunderbar an.

\begin{quote}
„Nur wenn Die-hner anbetet Allah, kann er aufwecken Liebesozeane.
Wir leben in Zeit von Hass und Leid.
Die meisten Menschen kennen nur körperliche Liebe
und werden darum unglücklich und miserabel.
Ohne aufwecken von Liebesozeane, wir können nie zufrieden sein.“
\end{quote}

Ich war beeindruckt, sowohl von seiner Botschaft als auch von seinem Auftreten,
aber fühlte mich unwohl wegen der übertriebenen Verehrung,
mit der seine Anhänger ihn überschütteten.
Ich entschied mich, ihn darüber zu befragen,
während der Frage- und Antwortrunde, die nachher kam.

\begin{quote}
„Ist es islamisch, Leuten zu erlauben sich vor deine Füße niederzuwerfen?
Der Prophet Mohammed ließ die Leute nicht zu seinen Füßen fallen, oder?“
\end{quote}

Meine Frage wurde mit wütendem Gemurmel und Buhrufen begegnet,
was mir die Röte ins Gesicht trieb und meine Augen wässrig machte.

\begin{quote}
„Ich sage nicht sie sollen das tun,
aber wenn sie wünschen für Liebe und Respekt, ich akzeptiere.
Vergiss nicht, Eltern von Prophet Yusuf warfen sich nieder vor ihm.“
\end{quote}

Ich wollte ihm noch mehr Fragen stellen,
vor allem über seine Vorhersage über den Mahdi,
aber andere wollten auch unbedingt sprechen.
Ein junges Paar präsentierte sich –
beide Konvertiten, einer Brite und die andere Asiatin.

\begin{quote}
„Wir würden gerne ihre Heiligkeit bitten,
den Segen für unsere Heirat zu geben.“
\end{quote}

Nachdem er nach ihren Namen und ein bisschen über ihren Hintergrund fragte,
gab er seine Zustimmung.
Er legte seine Hände auf ihre Stirn und sprach ein Gebet.
Andere traten schnell mit Fragen hervor.

\begin{quote}
„Ich hatte einen Traum, Scheich, in dem ich auf der Spitze eines Berges war,
und da war ein Licht über mir und es fing an Blumen zu regnen.“

„Das ist ein guter Traum mein Kind, du bekommst göttlichen Segen.“

„Meine Schwester ist krank, Scheich. Bitte bete für ihre Genesung.“
\end{quote}

Er hielt seine Hände hoch und betete, und alle machten es ihm nach.

Schließlich stand er auf und bewegte sich zur Treppe,
immer noch umgeben von Bittstellern.
Das war meine letzte Chance, bevor er verschwand.
Ich zwängte mich vorwärts durch die Menge.

\begin{quote}
„Scheich Nasim, darf ich dich über den Mahdi befragen.
Du sagtest er wird kommen?“
\end{quote}

Er begann die Stufen hinaufzugehen, gefolgt von seinen grünen Bodyguards.

\begin{quote}
„Er ist hier!“

„In diesem Raum?“ Ich folgte ihm die Stiegen hinauf.

„Nein, im Hidschas.“ (Die Gegend rundherum Mekka und Medina.)

„Weiß irgendjemand, wer er ist?“

„Er hat sich noch niemandem gezeigt.“

„Woher weißt du es dann?“

„Mein Scheich sagen mir.“

„Ist das der Scheich, der tot ist?“
\end{quote}

Scheich Nasim glaubte, dass er in Kontakt mit einem Scheich war,
der im Jahre 1940 gestorben ist.

\begin{quote}
„Kommt darauf an, was du meinen mit tot?“

„In Sinne von nicht mehr am Leben sein.“
\end{quote}

Scheich Nasims Bodyguards hatten genug von meinen Fragen
und blockierten aggressiv meinen Zugang,
indem sie mich die Treppen hinunterdrängten.
Der Scheich wurde rasch hinauseskortiert und verschwand.

Ich war ein bisschen enttäuscht von dem,
was ich bei Scheich Nasisms Runde erlebt habe.
Nicht so sehr von Scheich Nasim selbst, aber mehr von der Art und Weise
wie seine Anhänger vor ihm gekrochen sind.
Es schien nichts weiter als ein Personenkult zu sein.
Ich war außerdem äußerst skeptisch gegenüber seinen Behauptungen
besonderes Wissen von einem Scheich zu bekommen.
Nachdem Scheich Nasim sich mit einigen seiner Murids
nach oben zurückgezogen hatte,
fing ich an zum Ausgang zu gehen,
wobei sich mir ein großgewachsener Engländer näherte.

\begin{quote}
„Bist du auf dem Pfad, Bruder?“

„Meinst du, ob ich ein Sufi bin?
Nein, nicht wirklich.
Ich mag den Sufismus und möchte mehr erfahren,
weshalb ich heute hierher gekommen bin.“

„Um mehr zu lernen, musst du den Weg gehen.“

„Das Problem ist, dass ich ein paar Dinge entmutigend finde,
w.z.B. das Füße-Küssen und dieses besondere Wissen von einem toten Scheich.“

„Um uns folgen zu können, darfst du nicht urteilen und nichts widersprechen.
Das ist die Weise, in der der Wissenssuchende sich dem Lehrer annähern muss,
genauso wie Khidr Moses erklärte nichts zu hinterfragen,
falls er lernen wolle.“

„Nun ja, Moses ist eine Sache,
aber es scheint hier ein gefährliches Potential dafür zu geben,
dass die Blinden die Blinden führen, meinst du nicht auch?“

„Deshalb musst du auch dem wahren Scheich folgen,
damit du ihm vollkommen vertrauen kannst.“

„Mir fällt es schwer jemandem in diesen Dingen völlig zu vertrauen.“

„Dann ist das der Ursprung deiner Probleme, Bruder.“

„Weshalb vertraust du Scheich Nasim so sehr?“

„In jedem Zeitalter gibt es einen auserwählten Vertreter (Khalifah) Gottes.
In unserer Zeit ist es Scheich Nasim. Er ist der perfekte Heilige.“

„Was ist dein Beweis?“

„Jene, die ihm folgen haben den Beweis. Scheich Nasim weiß Dinge,
von denen normale Leute keine Kenntniss haben können.
Er hat dies bei vielen Gelegenheit unter Beweis gestellt.
Zum Beispiel habe ich selbst bezeugt wie er die exakte Zeit voraussagte,
in der einer seiner Anhänger sterben würde,
und es passierte genau so wie er es behauptete.“

„Ich zweifle nicht deine Worte an,
aber es könnte viele rationale Erklärungen dafür geben.“

„Ein weiterer Beweis für dich, mein verehrter Bruder, ist,
dass er die Fähigkeit hat mit jedem seiner Murids
zu beliebiger Zeit zusammen sein kann.
Er kann mit einem von ihnen an einem Ort und mit einem anderen
an einem anderen Ort zusammen sein.“

„Tut mir leid, aber mir fällt es sehr schwer das zu glauben.“

„Es ist die Arroganz in deinem Nafs (Ego), das dich vom Glauben abhält.
Du musst aufhören zu widerstehen, loslassen und dein Herz öffnen.“
\end{quote}

Unsere Unterhaltung erinnerte mich an eine Passage aus Alice im Wunderland:

\begin{quote}
\itshape
“Das kann ich nicht glauben!”, sagte Alice.

“Nein?”, sagte die Königin mitleidig.
“Versuch es noch einmal: Tief Luft holen, Augen zu.”

Alice lachte. “Ich brauche es gar nicht zu versuchen,” sagte sie.
“Etwas Unmögliches kann man nicht glauben.”

“Du wirst darin eben noch nicht die rechte Übung haben,” sagte die Königin.
“In deinem Alter habe ich täglich eine halbe Stunde darauf verwendet.
Zuzeiten habe ich vor dem Frühstück
bereits bis zu sechs unmögliche Dinge geglaubt.”
\end{quote}

Scheich Nasim war Gegenstand vieler absonderlicher Behauptungen
und wurde mit Plänen in Verbindung gebracht,
die an den Haaren herbeigezogen waren.
Eine der vielleicht bizarrsten war das „Mond-Tempel-Projekt“,
gedacht als ein „wahres Zeichen für rationale Menschen.“
Der Gründer des Projekts, ein russischer Muslim namens Asadula, sagte:
„Es gibt eine göttliche Vorherbestimmung in der Tatsache,
dass das Symbol des Islam und die Minaretten
auf unseren Moscheen Raumschiffen ähneln.“
Asadula erhielt die Berufung einen Mondtempel zu bauen,
während er für seinen Bruder, der operiert wurde, betete:

\begin{quote}
\itshape
„Und in einem Augenblick, oh Wunder,
verlor ich jeglichen Sinn für die Realität.
Es war so, als ob ein sehr kraftvoller Lichtstrahl in mein Auge traf,
mit seiner strahlenden Helligkeit.
Ich schloss meine Augen, um das blendende Licht zu blockieren,
doch es half nichts –
der himmlische Strahl drang weiter durch bis zum Kern meines Wesens.
Es fühlte sich so an, als ob alle Organe meines Körpers durchdrungen
und auf diesem mysteriösen Faden eines noch nie dagewesenen Lichtes
ausgestreckt wurden.
Und als ob wir eine Unterhaltung von vor langer Zeit fortfuhren –
was meinen ganzen Leib erschauerte –
sagte eine Stimme zu mir:

‚Und solche Torturen werden vorübergehen,
doch dein Schicksal ist unbezwinglich!
Die für deinen Bruder vorgesehene ansehnliche Versorgung
ist nocht nicht am Ende.
Er wird genesen.
Und er wird an deiner Seite stehen, wenn ihr schuftet,
um einen Tempel auf dem Mond zu bauen.
Das wird keine einfache Reise sein.
Dein Weg, Asadula, ist gewunden und steinig.
Und du wirst einen Tempel auf dem Mond errichten
als Symbol menschlichen Glaubens.
Und der fünchfache Ezān (Gebetsruf) zur Erlösung
wird durch das ganze Universum hörbar sein.‘“
\end{quote}

Der Sufismus wird im Westen bewundert, weil es zum Trend passt,
dass die Religion moderat, liberal
und tolerant zu verschiedenen Lebensweisen sein muss.
Ein Sufi-Scheich, den ich einlud um bei SOAS eine Rede zu halten,
erklärte, dass der Sufismus der Gipfel aller Religionen ist
und man sowohl christlichen, jüdischen und hinduistischen Sufis
als auch muslimischen Sufis begegnen wird.
Sufi-Scheichs wurden in den 60er und 70er Jahren
oft mit der gleichen Ehrfurcht angesehen wie die Gurus des Hinduismus,
und manch eine Hippie-Anhängerschaft würde letztendlich dem Sufismus
in der einen oder anderen Form folgen.
Doch ich fand Scheich Nasims Gesellschaft
und die Sufi\–Kreise, die ich später besuchte, nicht überzeugend.
Meine Erfahrungen stimmten mich sehr skeptisch
gegenüber den Behauptungen solcher Scheichs.
Es schien bestenfalls harmlose Exzentrizität
und im schlimmsten Fall gefährliche Selbsttäuschung zu sein.
Natürlich gab ich dem Sufismus nicht als Ganzes die Schuld
wegen den Schwächen bestimmter Sufi-Scheichs oder deren Anhängern,
jedoch nahm es mir die Lust einem Sufi\–Orden beizutreten.
Nichtsdestotrotz fühlte ich mich hingezogen
zum spirituellen und metaphorischen Verständnis des Islam,
welches der Sufismus lehrte,
und obwohl ich mich nicht einen Sufi nennen konnte,
entwickelte ich eine Tendenz
zu einer weniger wortgetreuen Interpretation des \Quran.
Einem Glauben, dass hinter den Worten ein tieferer Sinn versteckt lag,
der nicht sofort ersichtlich war.
Ich genoss auch Schriften, Gedichte und Parabeln von Sufis.
Unter diesen Gleichnissen befanden sich die heiteren Geschichten des Joha
(auch bekannt als Mullah Nasreddin.)
Sie werden überall in der muslimischen Welt erzählt,
und sind ein Teil einer mündlichen Tradition, die Jahrhunderte alt ist.
Es sind humorvolle Geschichten, die außerdem auch eine Weisheit vermitteln.
Eine meiner vielleicht Liebsten ist jene,
die sich über den Sufismus selbst lustig macht:

\begin{quote}
\itshape
Joha spatzierte im Basar mit einer großen Gruppe von Anhängern.
Was auch immer Joha tat, machten ihm seine Anhänger nach.
Nach ein paar Schritten würde Joha immer stillstehen
und seine Hände in der Höhe schütteln,
seine Füße berühren, aufspringen und „Hu Hu Hu!“ rufen.
Also würden seine Anhänger ebenfalls stillstehen
und genau dasselbe tun.

Einer der Händler, der Joha kannte, fragte ihn leise:

„Was tust du, mein alter Freund?
Warum ahmen dich diese Leute nach?“

„Ich bin jetzt ein Sufi-Scheich,“ erwiderte Joha.
„Dies sind meine Murids (spirituelle Suchende).
Ich helfe ihnen die Erleuchtung zu erlangen!“

„Woher weißt du, wann sie die Erleuchtung erlangt haben?“

„Das ist sehr einfach! Jeden Morgen zähle ich sie.
Jene, die nicht mehr da sind, haben die Erleuchtung erlangt!“
\end{quote}


%Chapter 4:
\chapter{Die Zunahme des Aktivismus}

\img{scale=0.7}{Assassination_of_Sadat.jpg}{}

„Hassan! Hassan! Sieh dir das an!“
rief mein Vater aus dem Wohnzimmer.
Ich eilte die Stiegen hinunter und sah, wie er mitten
im Zimmer stand und auf den Fernseher starrte.
Die Szene sah irgendwie nach einer Art Militärparade aus,
doch irgendetwas schien völlig daneben gegangen zu sein.
Es war chaotisch, Leute schrien auf Arabisch herum
und ich konnte das Feuern der automatischen Waffen hören.
Im Zentrum stand ein Podium und die Sessel lagen verstreut herum.
Soldaten, die hinter einem Armee-Laster standen,
feuerten auf das Zentrum des Podiums.
Ein Soldat rannte ganz nahe zum Podium,
hob seine Waffe an und fing an auf die Sessel zu schießen.
Es war der Oktober 1981, und das war das Attentat auf den Präsidenten Sadat
von einer gänzlich unbekannten Gruppe namens „Islamischer Jihad“ –
unter den Mitgliedern war Aiman Zawahiri.
Wenn ich jemals Zweifel hatte an der politischen Natur des Islam,
so würden diese Zweifel bald zerstreut werden.
Muslime waren zur jenen Zeit selten nicht im TV zu sehen –
von der Geiselnahme von Teheran zur Bombardierung der US-Botschaft in Beirut.

Es war meine spirituelle Suche nach der Wahrheit,
die mich zum Islam gebracht hatte.
Ich fühlte mich von den mystischen Passagen des \Quran\ angezogen,
und vom Versprechen innerer Weisheit und Erleuchtung.
Doch sobald ich ein Muslim wurde, war es wie selbstverständlich,
dass ich die politische Einstellung anderer Muslime
zu Themen wie Palästina, Kaschmir, Afghanistan
und später Bosnien, Tschetschenien und Irak unterstützen sollte.
Diese Annahme wurde nicht nur von Muslimen,
sondern auch von Nicht\–Muslimen gemacht.
Sobald jemand wusste, dass ich ein Muslim bin,
würde ich Bemerkungen wie folgende zu hören bekommen:

\begin{quote}
„War es nicht dumm von den Amerikanern, Helikopter zu schicken,
um zu versuchen die Geiseln im Iran zu retten?“

„Ich denke die Teilung von Zypern in zwei Staaten ist die einzige Lösung,
meinst du nicht auch?“

„Hast du gehört was Präsident Asad den Muslimen in Hama angetan hat?“
\end{quote}

Zu der Zeit wusste ich sehr wenig über diese Themen,
aber bald erkannte ich, dass von mir Expertenwissen erwartet wurde,
einfach aufgrund der Tatsache, dass ich ein Muslim bin.
Nur wenige Tage, nachdem ich aus Ägypten im Jahre 1979 zurückgekehrt war,
saß ich im Haus meines Schwagers mit meinem Vater
und einem jungen arabischen Studenten.

\begin{quote}
“Woher kommst du?” fragte ich.

“Palästina.”

“Wo ist Palästina?”
\end{quote}

Ich hatte das Wort Palästina oft gehört.
Als ich sehr klein war, zeigte mir mein Vater eine Narbe, und sagte:
„Genau da haben mich die Juden in Palästina getroffen.“
Es war eines der wenigen Male, wo er über seine Vergangenheit gesprochen hat.
Ich dachte die Narbe war ziemlich cool,
aber hatte keine Ahnung was er mit ‚Palästina‘ meinte.
Sowohl mein Vater als auch der junge Palästinenser
waren entsetzt über meine Unwissenheit.
Nachdem sie mir eine kurze Geschichtslektion gegeben hatten,
erzählte mein Vater wieder stolz von der Geschichte
wie er in 1948 verletzt wurde,
als er mit der Muslimbruderschaft gegen die ‚Juden‘ kämpfte.
Ich fühlte mich beschämt wegen meiner Unkenntnis
über ein solch wichtiges Thema,
und nahm mir deswegen vor, mich in diese Sache einzulesen.

Im Gegensatz zu Christentum sieht der Islam
keine Trennung zwischen Politik und Religion vor.
Es gibt kein „Gebt dem Kaiser, was des Kaisers ist,
und Gott, was Gottes ist.“
Der Prophet Mohammed war sowohl ein militärischer und politischer Führer
als auch ein spiritueller Führer,
und er machte deutlich, dass der Islam für jeden Teil des Lebens gilt,
im öffentlichen sowie im privaten Raum.
Muslime müssen als ein Körper betrachtet werden; so wie er sagte:
„Wenn ein Teil des Körpers Schmerz erleidet, dann leidet der ganze Körper.“
Darum dachte ich, dass meine Hingabe zum Islam auch eine Hingabe
zu meinen muslimischen Brüdern und Schwestern in der ganzen Welt bedeuten müsste.
Ich fing an ein starkes Interesse für die Weltpolitik zu entwickeln
und las über Konflikte und Länder, über die ich bisher nichts wusste.
Die zwei größten Streitthemen zu dieser Zeit waren
die russische Invasion von Afghanistan und natürlich Palästina.

Die Not der Palästinenser im Jahre 1982 wurde hervorgehoben,
als unbewaffnete palästinensische Männer, Frauen und Kinder
in Sabra und Schatila von einer christlichen Miliz massakriert wurden,
während die Camps von der israelischen Armee umgeben waren.
Ich erinnere mich an die Bilder ganzer Familien,
die tot in den engen Straßen lagen,
und wie ihre Körper durch die heiße Sonne aufgebläht waren,
und ihre Hände immer noch an den Ausweispapieren festhielten,
die sie in ihrer Verzweiflung gezeigt hatten.
Die Bilder erzeugten eine enorme Wut in mir.
Ich empfand auch eine Menge Frustration,
weil die muslimischen Führungskräfte nichts unternahmen, um zu helfen.
Der russische Einmarsch in Afghanistan löste ganz andere Emotionen in mir aus.
Der Kampf der Mudschahidin gegen die Gewalt einer Supermacht war inspirierend
und es bestätigte in mir den Glauben,
dass die Muslime nur durch die Rückkehr zum Islam
die erlittenen Ungerechtigkeiten vergelten können.

Ein weiterer Faktor, der dazu beitrug die Muslime in GB
zur jenen Zeit zu politisieren,
war der Zufluss von Muslimen aus anderen Ländern, insbesondere Arabern.
Manche waren Vertriebene und Dissidenten,
die eine internationale Agenda mit sich brachten,
doch die meisten waren Studenten oder schlecht bezahlte Arbeiter,
die hierher für ein besseres Leben kamen.
Viele waren keine praktizierenden Muslime, als sie ankamen.
Gefühle der Isolation und Entfremdung brachte sie zu den Moscheen
und dann zu örtlichen Studienkreisen, wo viele bald sehr religiös wurden.
Ich dachte schon immer, dass es ironisch ist, dass Muslime,
die in ihrem Heimatland nicht religiös waren,
erst dann religiös wurden,
nachdem sie zu einem nicht-muslimischen Land ausgewandert sind.
Es scheint ein enger Zusammenhang zu bestehen,
zwischen dem Verlust der Identität,
als Folge der Entwurzelung aus der eigenen Umgebung,
und dem Zugehörigkeitsgefühl und Selbstbewusstsein, das die Religion bietet.

Insbesondere die Araber entdeckten bald,
dass andere Muslime in England zu ihnen aufschauten.
Eine Rassenhierarchie hat schon immer unter den Muslimen existiert –
mit den Arabern an der Spitze,
trotz der Tatsache, dass sie nur etwa 12\% der Muslime weltweit ausmachen.
Der Hauptgrund, warum zu ihnen aufgeschaut wird, ist,
dass sie den Schlüssel zum Verständnis der Worte Gottes besitzen:
die arabische Sprache.
Muslime glauben, dass der \Quran\ das buchstäbliche Wort Gottes ist
und nicht übersetzt werden kann.
Sobald es übersetzt wird, ist es nicht mehr länger das Wort Gottes,
sondern nur noch die Interpretation einer Person.
Jemand, der kein Arabisch kann, wird immer im Nachteil
gegenüber einem Arabisch Sprechenden sein,
wenn es um Streitigkeiten über die Bedeutung des \Quran\ geht.
Wenn nichts mehr hilft, kann der Arabisch Sprechende einfach behaupten,
dass der andere die wahre Bedeutung der Worte Gottes nicht verstehen kann.
Ein weiterer Grund für diese Hierarchie ist, dass der Prophet selbst ein Araber war.
Dies hat eine religiöse Bedeutung,
da die Nachahmung des Propheten Mohammed ein wichtiger Aspekt des Islam ist.
Mohammed spiegelt die arabische Kultur wieder, in der er lebte,
und daher haben arabische Bräuche, Kleidung und Ernährungsgewohnheiten
die islamischen Bräuche,
Kleidung und Ernährungsgewohnheiten stark geprägt.
Viele traditionelle Rechtsschulen des Islam legen auch fest,
dass der Kalif (Führer der Muslime) ein Araber
vom Stamm der Quraisch sein muss, dem Stamm des Propheten.

Viele Araber, die nach England kamen,
genossen das Gefühl der Selbstwichtigkeit,
das sie durch ihre neuen Rollen als islamische ‚Experten‘ erhielten.
Sie tauschten ihren westlichen Lebensstil
für einen Jilbab, eine weiße Kappe und den Titel eines Scheichs ein.
Der Wandel von verwestlichten, jungen Männern zu religiösen Führern
wird von Abu Hamza gut veranschaulicht.
Er kam in England im Jahre 1979 an –
im selben Jahr, in dem ich praktizierender Muslim wurde –
und das einzige, das er zu dieser Zeit praktizierte,
waren seine Anmachsprüche im Nachtclub Soho, wo er als Türsteher arbeitete.
Es ist schwierig sich ein Bild von Abu Hamza vorzustellen,
das anders ist, als jenes vom einäugigen Wasserspeier mit Hakenhand –
so beliebt bei der Boulevardpresse.
Doch laut seiner englischen Ex-Frau war er ein attraktiver,
romantischer und zärtlicher junger Mann, dem viele Frauen hinterherjagten.
Im Jahre 1984 aber, nach dem Besuch von radikalen Predigten
bei einer örtlichen Moschee,
wurde er zunehmend zutiefst religiös.
Er reiste nach Afghanistan, um im Jihad gegen die Russen zu kämpfen,
wo er eine Hand und das Licht des linken Auges verlor.
Dann kam er zurück nach England und fing an als Scheich Abu Hamza zu predigen.
Abu Hamza war ein eindrucksvolles Beispiel für diesen Wandel,
aber dieses Muster erlebte ich sehr häufig.
Ein gemeinsamer Faktor bei allen war,
dass sie eine simplistische, wortgetreue Sichtweise des Islam annahmen,
die ihnen erlaubte den \Quran\ unkritisch zu erklären,
ohne sich auf jahrhundertealte,
traditionelle islamische Gelehrsamkeit zu beziehen,
in der keiner von ihnen eine formale Ausbildung hatte.

Die Mehrheit der heimischen, britischen Muslime
waren Asiaten aus Süd\–Ost\–Asien oder Ost\–Afrika.
Ihre Eltern waren in den 60ern und 70ern hierher gekommen,
und sahen sich trotzdem als Pakistanis oder Inder statt Briten.
Ihre Kinder, andererseits,
fühlten eine geringere Verbindung zu den Ländern ihrer Eltern,
da sie hier in England aufwuchsen.
Sie waren daran interessiert, mehr über einen Islam zu lernen,
der frei von kulturellen Einflüssen ist,
um es in ihren Alltag zu integrieren,
und sie sahen eher die Religion als die Nationalität,
als Grundlage ihrer Identität an.
Aber sie waren abgeschnitten von den islamischen Bräuchen
und Gemeinden des indischen Subkontinents.
Ihre Eltern konnten ihnen nicht viel Hilfe in islamischen Themen bieten.
Da sie als Arbeitsmigranten in den Westen kamen,
war ihr Glaube an den Islam entweder nur ein Lippenbekenntnis
oder ihr Wissen war gering und vermengt mit ihrer lokalen Kultur.
Als Folge wurde die jüngere Generation zunehmend
von den neueren arabischen Einwanderern beeinflusst.
Diese Kombination erzeugte die richtigen Zutaten für
die Etablierung von wortgetreuen und militanten Bewegungen in England.
Der egalitäre Charakter des Buchstabenglaubens war attraktiv für die Jungen,
ebenso wie das Gefühl der Rebellion,
das von der militanten Ideologie angeboten wurde.
In Wahrheit hatten diese Gruppen mehr gemein
mit solchen revolutionär politischen Parteien,
wie ich sie an der Universität erlebt hatte,
als mit den traditionellen, islamischen Bewegungen.

Trotzdem waren die meisten islamischen Treffen in den frühen 80ern
aufgeschlossen und vor allem tolerant.
‚The East Finchley Da\`wa Society‘ (die Ost\–Finchley\–Da\`wa\–Gesellschaft)
– von der ich ein Mitglied wurde – war ein typisches Beispiel.
Es war ein Klub speziell für junge Muslime,
die die Grundlagen des Islam lernen wollten.
Die Treffen waren gemischt und informell
und wir rotierten wöchentlich die Führerschaft unter den Mitgliedern.
Die unterschiedlichsten Redner waren eingeladen,
darunter waren der Scheich Darsh,
eine Delegation von der ‚Federation of Student Islamic Societies‘,
ein Redner über alternative Medizin, ein Experte für Yoga,
der Modernist Dr.\ Essawi, und der Bruder Yusuf Islam.
Einmal luden wir einige amerikanische Evangelikale
zu unserer Versammlung ein, um sie teilhaben zu lassen.
Sie verbrachten den ganzen Abend mit dem Versuch,
uns alle zum Christentum zu bekehren.
Etwas, das jeder dort mit guter Laune und vorzüglicher Gastlichkeit hinnahm.

\begin{quote}
„Ich möchte beeeten mit dir, oh Hassaan!“ sagte Tom,
ein großer und breitschultriger Mann aus Tennessee,
als er seine Hand auf meinen Schenkel legte und ihn liebevoll drückte.

„Okay,“ sagte ich, und versuchte nicht auf seine Hand hinunterzusehen.

Die Gruppe der Amerikaner neigte dann ihre Köpfe, während Tom betete.

\emph{„Im Namen Jeeesu danken wir dir Herr,
für die Gelegenheit deine Liebe
mit unseren geehrten moooslemischen Freunden zu teilen.
Wir erbitten im mächtigen Namen Jeeesu,
dass du ihre Herzen öffnen mögest, um dein Wort zu empfangen.
Mögest du dein Angesicht über sie leuchten,
und ihnen zeigen, dass deine Liebe größer ist als alles,
das wir uns vorstellen können.
Im Namen Jeeesu beten wir für diese Dinge.“}

\emph{„Ah-men!“} sagten sie alle gleichzeitig.
\end{quote}

Tom führte dann die Diskussion und erklärte uns die ‚Fehler‘
im islamischen Verständnis über Jesus und das Christentum,
und bat uns alle dringend Jesus in unser Leben zu lassen,
so dass wir errettet werden mögen.
Danach hatten wir gemeinsam Tee und Jammie Dodgers (Gebäck.)

Die Da\`wa-Gesellschaft gab ihr eigenes Magazin
mit dem Namen „The Clarion“ (Der Weckruf) heraus,
das ich selbst bearbeitete, und das Artikel
über den Islam und aktuelle Themen beinhaltete.
Manchmal waren diese ernst, oft aber heiter und humorvoll,
wie z.B. ein Parodierezept von Benazir Bhutto –
damalige Präsidentin Pakistans – namens „Political Hot Pot“
(Politisches Fondue mit Brühe,)
und ein Maulana, der Ratschläge dafür gab,
wie man das Schnarchen seines Mannes stoppen kann,
und ein Schein\–Interview mit Hafiz al-Assad dem Präsidenten Syriens,
Vater des jetzigen Präsidenten.
Der Interviewer fragt:

\begin{quote}
\itshape
„Sie sind schon seit vielen Jahren der Herrscher ihres Landes.
Was ist das Geheimnis ihres Erfolgs?“

„Es ist ziemlich einfach.
Ich halte keine Wahlen ab.
Sie verursachen zu viele Probleme, wie z.B. die Möglichkeit,
dass jemand anderer gewinnt.
Also habe ich Wahlen verboten und lasse jegliche Opposition foltern und töten.“

„Folter und Mord scheinen entscheidende Faktoren
ihrer Politik zu sein, nicht wahr?“

„Oh ja, ich finde die beste Art, mit Menschen umzugehen, ist, sie zu töten.
Einige meiner zuverlässigsten Verbündeten sind tote Leute.“

„Aber ihre Leibwächter sind nicht tot?“

„Nein, nein… natürlich nicht,
immerhin wären sie nicht besonders nützlich, wenn sie tot wären, nicht wahr?
Ich habe nur ihre Gehirne entfernen,
ihre Zungen herausziehen und ihre Augen aufspießen lassen.“

„Sie haben ihnen das Gehör gelassen?“

„Ich bin nicht der herzlose Kapitalist, Zionist, Kommunist, rechter,
linker, westlicher, östlicher – und alles was je existiert hat –
Schlächter, als den mich die Presse darstellen will.
Außerdem möchte ich ja, dass sie meine Befehle hören.“
\end{quote}

Wir haben auch sportliche Aktivitäten,
Camping-Ausflüge und Exkursionen organisiert.
Bei einem Ausflug nach Exeter –
um örtliche Muslime dort zu treffen –
machten wir eine Pause in einem nahge gelegenen Park
und die Männer fingen bald an Teams zu bilden, um Fußball zu spielen.
Da es uns an Spielern fehlte, versuchten wir unser bestes,
um einige der älteren Mitglieder unserer Gruppe zum Spielen anzuregen.
Unter anderen war Scheich Hamid in unserem Team,
einer der Imame in der Regent’s Park-Moschee zur jenen Zeit.
Er war eine riesige Gestalt mit einem großen, runden Bauch,
und war in seinem vollen Freitagspredigt-Gewand mit Turban gekleidet.

\begin{quote}
„Komm schon Scheich!“ sagte ich.
„Sport zu machen ist ein Teil davon, ein guter Muslim zu sein!“ neckte ich.

Zu unserer Überraschung stand Scheich Hamid mutig auf
und begann zu uns zu kommen.

„Er ist auf unserer Seite!“
sagte ich und packte Scheich Hamid am Arm.*

„Das ist nicht fair, du hast einen göttlichen Vorteil,“ kicherte Khalid.

„Die können ihn haben,“ nuschelte Ishfaq;
„Ich denke nicht, dass er eine große Hilfe sein wird.“
\end{quote}

Der Scheich nahm das Spiel jedoch ernst
und setzte sein Gewicht ein, um alle wie kleine Kegel umzuhauen,
bevor er dann den Ball in die Richtung stoßte, in die er gerade schaute.
Alle lachten so sehr, dass wir kaum noch richtig spielen konnten.
Nach dem Spiel, als wir auf dem Rückweg waren,
und Scheich Hamid seinen Turban
und seine Robe über seinem Arm trug und der Schweiß
über sein pausbäckiges Gesicht runterrollte,
lächelte er und sagte:

\begin{quote}
„Wann ist das nächste Spiel?“

„Na ja, wann immer es ist, Scheich,“ sagte Khalid.
„Das nächste Mal bist du auf meiner Seite!“
\end{quote}

Doch bis Mitte der 80er Jahre hatte die Da\`wa\–Gesellschaft
bereits begonnen sich zu ändern.
Ein Strom von kompromisslosen und engstirnigen Doktrinen
begann sich in die Treffen einzuschleichen,
so wie eine versteckte Infektion.
Oft war es eine beiläufige Bemerkung, die jemand ‚irgendwo‘ gehört hatte:

\begin{quote}
„Du darfst nicht ‚Salam‘ zu Nicht\–Muslimen sagen.“

„Es ist verboten deinen Bart zu schneiden.“

„Du darfst keine Gesichter malen.“
\end{quote}

Diese Bemerkungen wurden unhinterfragt weitergegeben,
weil die meisten nicht über das Wissen oder die Expertise verfügten,
um sie anzufechten.
Die Treffen wurden immer mehr getrennt, als diejenigen,
die eine bestimmte Doktrin aufdrängten, darauf bestanden,
dass nur sie den ‚wahren‘ Islam hatten,
und dass alle anderen falsch lagen.
Wir sahen wie neue Mitglieder die Da\`wa\–Gesellschaft besuchten,
jeder von ihnen mit ihrer eigenen Agenda.

\begin{quote}
„Irgendwelche Ankündigungen?“ fragte ich am Ende einer Versammlung.

„Ja, die ‚Islamic Association of North London‘ wird nächsten Sonntag
eine Veranstaltung im Gemeindesaal abhalten,
um den Geburtstag des Propheten Mohammed zu feiern.
Alle sind eingeladen.“

„Das ist Bid\`a!“ warf ein junger Mann ein,
der zum ersten Mal bei der Versammlung erschienen war.
„Wir dürfen nicht den Geburtstag des Propheten feiern. Es ist haram!“
\end{quote}

Nach einer kurzen Stille machten wir mit anderen Ankündigungen weiter.

Er nannte sich selbst Abu Zubayr, obwohl das nicht sein echter Name war.
Er reiste regelmäßig den ganzen Weg von East-London hierher,
zusammen mit seinen Freunden, um uns Lektionen über die ‚korrekten‘
islamischen Glaubensinhalte zu geben.

\begin{quote}
„Der Islam ist perfekt und vollkommen.
Es kann in keinster Weise verändert werden.
Alles Neue ist ‘Bid\`a’ (Neuerung) und wird dich ins Höllenfeuer bringen.“
\end{quote}

Er sprach langsam und präzise,
so als ob er sich an einen gut einstudierten Text hielt.

\begin{quote}
„Der Prophet Mohammed – Friede sei mit ihm – sagte:
‚Jede Bid\`a ist eine Irreführung und jede Irreführung ist im Höllenfeuer.‘“
Er pausierte um Wasser zu trinken,
und er nahm drei Schlückchen genau so wie der Prophet.

„Heutzutage geben Muslime üblen Abirrungen allerart nach.
Sie sind irregeleitet und verdorben.
Brüder und Schwestern!
Wir müssen zum reinen Islam zurückkehren.
Kehrt zurück zum Islam, so wie er vom Propheten
und seinen rechtschaffenen Gefährten praktiziert wurde –
wenn wir Erfolg in diesem und im nächsten Leben haben wollen.“
\end{quote}

Im Laufe der nächsten Wochen haben wir gelernt,
dass noch mehr Dinge verboten waren –
wie z.B. das Erzählen von Witzen, die nicht buchstäblich wahr sind,
da sie als eine Form der Lüge betrachtet wurden.
Dies bedeutete, dass die meisten Witze verboten waren,
da es nur wenige gibt, in denen keine imaginären Situationen vorkommen.
Das Anhören von Musik war ebenfalls verboten –
abgesehen vom Spielen einer Trommel aus Tierhaut während der Eid-Feste.
Abu Zubayr erzählte einer neuen Konvertitin,
dass es ihr nicht erlaubt sei am Weihnachtsessen mit ihren Eltern teilzunehmen,
weil sie dadurch Schirk begehen würde (Götzendienst) –
die größte Sünde in den Augen Gottes.
Abu Zubayr und seine Gefährten vertraten
eine Minderheit bei unseren Versammlungen,
doch die meisten von uns hatten nicht die Fähigkeit
oder die Kraft, um seine Angriffe abzuwehren.

\begin{quote}
„Wir müssen stets gottesandächtig sein,“
sagte ich während einer Rede, die ich vorbereitet hatte.
„Wir sollten daran denken, dass Gott immer mit uns ist.
Er ist überall!“

„Das ist Kufr (Unglaube)!“ warf Abu Zubayr ein.

„Was denn?“

„Zu sagen, dass Allah überall ist!“

„Warum?“

„Weil uns Allah im \Quran\ gesagt hat,
dass Er auf dem Thron über den sieben Himmeln ist:
\emph{‚Der Gnadenreiche, Der Sich auf den Thron niederließ.‘}“

„Das ist aber nur eine Redewendung.
Gott übersteigt unser Verständnis.“

„Indem du sagst, dass Gott überall ist, deutest du an,
dass Gott auch in der Toilette ist –
wa \`audu billah (Ich suche Zuflucht bei Allah)!“
\end{quote}

Abu Zubayr und seine Freunde waren Salafis – auch bekannt als Wahhabiten –
die eine buchstabenverliebte und puritanische
Variante des Islam unterstützten,
welche die Religion von den Dingen zu reinigen sucht,
die als Erfindungen, Aberglauben und Irrlehren betrachtet werden.
Während der 80er Jahre finanzierte Saudi\–Arabien
die Ausbreitung der salafistischen Lehren
und subventionierte salafistische Bücher,
welche muslimische Buchläden im ganzen Land überschwemmten.
Sie richteten auch Büros wie die ‚Muslim World League‘
(Islamische Weltliga) ein,
in Tottenham Court Road,
die finanzielle Hilfe gaben für islamische Organisationen,
Moscheen, Schulen, Studenten und Privatpersonen
die bereit waren, ihre Ansichten anzunehmen.
Da Muslime keine andere Quelle hatten, zu der sie sich wenden konnten
als sie am meisten Hilfe brauchten,
waren die meisten, wenn nicht alle,
bereit die festgelegten Bedingungen zu akzeptieren,
und sie dachten, dass es kein ernstes Problem sei.
Ich selbst bewarb mich bei der Islamischen Weltliga
für ein Stipendium, um Arabisch und den Islam in Ägypten zu studieren.
Mir wurde ein Platz angeboten,
nicht in Ägypten, wo sie die Lehre als Deviation ansahen,
sondern in Saudi\–Arabien an der Universität von Medina.
Hier wurden die Lehrpläne sorgfältig vorbereitet
und von Salafi\–Lehrern gelehrt,
um sicherzustellen, dass alle Schüler den ‚wahren‘ Islam lernen würden.
Bevor ich meine Entscheidung treffen musste,
hatte ich zum Glück einen Platz bei SOAS angeboten bekommen,
und konnte daher das andere Angebot ablehnen,
doch mein Bruder Lutfi entschied sich dafür ein ähnliches Angebot anzunehmen,
und studierte an der Universität von Medina für mehrere Jahre.
Er kehrte mit einem sehr düsteren Bild von der saudischen Bildung
und vielen schockierende Geschichten zurück,
wie z.B. als er in der Moschee des Propheten in Medinah einschlief,
nur um von der Mutawwa (Religionspolizei)
aufgeweckt und mit Stöcken geschlagen zu werden.
Die Mutawwa wurde durch das „Komitee für die Verbreitung der Tugend
und die Verhinderung des Lasters“ berufen, um sicherzustellen,
dass jeder das islamische Gesetz einhält.
Dies beinhaltete die Durchsetzung von islamischer Kleidung,
Gebetszeiten, Speisegesetzen,
die Festnahme von Jungen und Mädchen,
die miteinander in Gesellschaft erwischt wurden,
und die Beschlagnahmung von unislamischen Gegenständen
wie z.B. westlicher Musik und Filme.
Sie verbieten auch den Götzendienst,
was lose definiert ist und offenbar auch
das Einschlafen in der Moschee des Propheten einschließt.
Im Jahre 2002 haben sie Schulmädchen an der Flucht
aus einem brennenden Gebäude in Mekka gehindert,
weil die Mädchen nicht richtig verschleiert waren.
Fünfzehn Mädchen starben und viele weitere waren als Folge verletzt.
Die Mutawwa haben für die Durchsetzung der Schari\`a
sogar durch moderne Technik Hilfe angeworben,
indem sie eine Webseite gestartet haben,
wo man die Behörden über unislamische Aktivitäten anonym informieren kann.
Die Salafis waren nicht die einzige militante Gruppe
auf dem Vormarsch während dieser der Zeit.
Hizb ut\–Tahrir (wörtlich: die Partei der Befreiung)
war ihr Hauptkonkurrent bei der Eroberung der Herzen und Gedanken
der jungen britischen Muslime.
Einer ihrer prominenten Mitglieder zu der Zeit, Farid Kasim,
wurde zum regelmäßigen Besucher von Versammlungen der Da\`wa\–Gesellschaft.
Er war besessen von der Idee eines „islamischen Staates“ (Kalifat.)
Er hörte sich unsere Gespräche an, aber nicht um zu lernen oder beizutragen,
sondern um sie an sich zu reißen und über das Kalifat zu reden.

\begin{quote}
„Demokratie ist gänzlich gegen den Islam.
Ein vom Kalifat geleiteter islamischer Staat
ist die einzige akzeptable Form einer islamischen Regierung:
‚Wer nicht nach dem richtet, was Allah hinabgesandt hat –
das sind die Ungläubigen.‘ (\QRef{5:44})“

„Doch es gibt viele Prinzipien in der Demokratie,
die vollkommen islamisch sind,“ erwiderte ich.
„Das Prinzip der Schura (Beratung,) zum Beispiel,
ist das Fundament der Demokratie!“

„Demokratie ist völlig inkompatibel mit dem Islam.
Demokratie bedeutet, dass Menschen das Recht haben Gesetze zu bestimmen.
Es gibt dem Menschen das, was ausschließlich dem Schöpfer gebührt.
Es ist die Pflicht eines jeden Muslims die Demokratie abzulehnen.“

„Muslimische Länder sind nicht viel besser, oder sind sie das?“

„Das kommt daher, weil sie das Kalifat und die Schari\`a aufgegeben haben.“

„Aber du kannst den Menschen nicht einfach
einen islamischen Staat aufzwingen.“

„Wenn du wartest bist die Leute bereit sind,
dann wirst du niemals einen islamischen Staat haben.
Wir müssen zuerst die Regierung ändern,
dann werden sich auch die Menschen ändern.“

„Geht es im Islam nicht darum, den Menschen zu verwandeln?“

„Im Islam geht es darum, das Gesetz Gottes auf Erden durchzusetzen.“
\end{quote}

Farid war ein sehr streitlustiger junger Mann
mit einer unglaublichen Energie und voller Tatendrang,
was kennzeichnend für die Partei war,
bei deren Gründung in England er geholfen hatte,
mit Omar Bakri und anderen.
In der Tat waren er und Omar Bakri
sogar für die Hizb ut\–Tahrir viel zu radikal,
und sie gingen weg um eine noch militantere Gruppe
namens Al\–Muhadschirun zu gründen,
die berühmt waren, unter anderem für die „Magnificent 19“\–Konferenz,
in der die Selbstmordattentäter gelobt wurden,
die für die Angriffe des 11.\ September verwantwortlich waren.

...

%Chapter 5:
\chapter{Die Islamia-Schule}

...

%Chapter 6:
\chapter{Scheich Faisal}

...

%Chapter 7:
\chapter{Dienstag Nachmittag}

\img{scale=0.4}{911_Second_Plane.jpg}{}

...


%Chapter 8:
\chapter{Offenbarung \& Vernunft}

\img{scale=0.6}{Man_Reading_Koran.jpg}{}

...

%Chapter 9:
\chapter{Bombenanschläge in London}

\img{scale=1.2}{London_Bombings.jpg}{}

...

%Chapter 10:
\chapter{Das Unreformierbare reformieren}

\img{scale=0.4}{Quranic_Script.jpg}{}

...

%Chapter 11:
\chapter{Religion}

\img{scale=0.6}{Neasden_Temple.jpg}{}

...

%Chapter 12:
\chapter{Die Büchse der Pandora}

\img{scale=0.5}{Pandoras_Box.jpg}{}

...

\backmatter

\chapter{Index}
...

\end{document}
