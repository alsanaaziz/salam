%!TEX TS-program = xelatex
%!TEX encoding = UTF-8 Unicode
\documentclass[12pt]{memoir}
\usepackage[
  xetex,bookmarksnumbered=true,pdfborder=0,hyperfootnotes=true
]{hyperref}
\usepackage{xltxtra}
\usepackage{polyglossia}
\usepackage{bidi}
\usepackage{xunicode}
\usepackage{graphicx}

\usepackage{fontspec}
\usepackage{color}

\setdefaultlanguage{german}
\setmainfont{DejaVu Serif}

\hypersetup{
  pdfauthor = {Hassan},
  pdftitle = {Autobiographie von Hassan},
  pdfsubject = {},
  pdfkeywords = {Islam Religion Allah Gott Glaube Skeptizismus Leben},
}

\definecolor{darkblue}{rgb}{0,0,0.5}

% Transliterated "hamza".
\def\´{ʾ} % ˀ
\let \hamza=\´ % Needed in \footnotetext.

% Transliterated "`ayin".
\def\`{ʿ} % ˁ
\let \ayin=\` % Needed in \footnotetext.

% Aliases for subscript and superscript:
\let \Sub=\textsubscript
\let \Sup=\textsuperscript

% Hyphenation rules:
\hyphenation{Mau-la-na Za-kar-ya Kan-dha-la-wi Shu-\`ay-bi}

% Corrective comments.
\newcommand{\cmt}[2]{#1} % #2 comments on #1.
\newcommand{\cor}[2]{#2} % #2 corrects #1.

% Use Scheherazade for Arabic. Use \ar{...} for inline Arabic text.
\newfontfamily{\arabicfont}[Script=Arabic,Scale=1.7]{Scheherazade}
\newcommand{\ar}[1]{\RL{\arabicfont#1}}

% Macros for the word Qur'an.
\def \Quran{Qur\-\´ān} % Used when followed by a punctuation mark.

% Breakable forward slash.
\def\/{\discretionary{/}{}{/}}

% For hyphenation: Trick TeX into thinking al\–XYZ are two separate words.
\def\–{-\hskip0pt}
\def\al{al-\hskip0pt} % Version with "al-".

% Prints symbols for dividing paragraphs.
\def\pardivider{\centerline{***}} % \ar{۞۞۞} ❧❦

% Qur'an reference numbers. TODO: link to web page.
\newcommand{\QRef}[1]{{\color{darkblue}#1}}

% "Nota bene" macro.
\newcommand{\NB}[1]{\emph{\small NB: #1}}

% Image macro. #1 Optional params. #2 File name. #3 Optional caption.
\newcommand{\img}[3]{\begin{center}%
\includegraphics[#1]{#2}\\{\small\em#3}%
\end{center}}
\graphicspath{{books/imgs/}}

% Paragraph spacing:
\setlength{\parskip}{1ex plus 0.5ex minus 0.2ex}

\pagestyle{plain}

\title{Eine Autobiographie}
\author{\emph{von Hassan}}

%%%%%%%%%%%%%%%%%%%%%%%%%%%%%%% Document begin %%%%%%%%%%%%%%%%%%%%%%%%%%%%%%%%
\begin{document}
\frontmatter

% Title:
\maketitle
\thispagestyle{empty}
\cleardoublepage

% Table of Contents:
\setcounter{page}{1}
\tableofcontents

% Preface:
\chapter{Vorwort}

% \begin{comment}
Dies ist meine Geschichte. Ich habe sie als Selbsttherapie geschrieben,
außerdem habe ich sie geschrieben, um zu versuchen mein Leben zu verstehen.
Um zu verstehen warum ich bestimmte Wege gegangen bin,
und warum ich zu bestimmten Schlussfolgerungen gelangt bin.
Ich hoffe auch, dass es andere interessiert und aufklärt.
Ich habe mich auf die frühen und späten Jahre konzentriert.
Ich habe keine echten Namen benutzt, außer in wenigen Einzelfällen.
% \end{comment}

\img{scale=0.1}{Tunis_Hassan.jpg}{}

Mein Vater war Ägypter und meine Mutter Engländerin
und ich habe 8 Brüder und Schwestern.
Ich wurde als Muslim geboren, aber ich begann erst zu praktizieren
als ich zwanzig und sehr fromm und engagiert wurde.
Für die nächsten zwanzig Jahre diente der Islam als Anleitung
für jeden Aspekt meines Lebens.
Ich absolvierte ein B.A.-Studium in Arabistik und Islamwissenschaften
an der Schule für Orientalistik und afrikanistische Studien,
wo ich Präsident der „Islamischen Gesellschaft“ wurde.
Nachdem ich von der Universität abgegangen bin,
wurde ich zum Leiter einer Da\`wah-Gruppe im Norden Londons
und gab ein islamisches Magazin mit dem Namen
„The Clarion“ („Der Weckruf“) heraus.
Ich schrieb vier Bücher für muslimische Kinder
und verbrachte fünfzehn Jahre als Lehrer an der ‚Islamia School‘ –
jene, die von Yusuf Islam (Cat Stevens) gegründet wurde.
Doch zeitnah zu meinem 48.\ Geburtstag wusste ich,
dass ich nicht mehr länger an den Islam glaubte.
Hättet ihr mir vor ein paar Jahren erzählt,
dass mir so etwas passieren würde,
so hätte ich euch niemals geglaubt und hätte außerdem hinzugefügt,
dass kein ‚wahrer‘ Muslim jemals den Islam ablehnen würde,
nachdem er die Süße des Iman (Glaubens) gekostet hat.
Niemand, der in die Schönheit des Koran eingetaucht war
und seine Weisheit verstand, könnte jemals leugnen,
dass es das Wort Gottes ist. Hätte jemand so etwas behauptet,
so hätte ich ihm instinktiv misstraut
und gegen seine Motive Verdacht geschöpft.
Doch es ist erstaunlich wie sich Wahrnehmungen ändern können und wie Dinge,
die ich früher als unvorstellbar erachtete,
jetzt vollkommen vernünftig erscheinen.
Natürlich erfolgte dieser Wandel nicht von heute auf morgen.
Es begann vor ein paar Jahren,
als ich anfing meine Ansichten zu hinterfragen,
die ich seit Langem als selbstverständlich ansah,
und als ich anfing den Islam im neuen Lichte zu betrachten.

Immer mehr Zweifel schlichen sich nach und nach ein.
Anfangs versuchte ich sie zu unterdrücken und reagierte auf Kritik
gegen den Islam mit Verleugnung, Zorn und Vorwürfen.
Ich leugnete, dass etwas nicht stimmte,
war Kritik gegenüber überempfindlich und beschuldigte den Westen
provokant zu sein und Probleme zu verursachen.
Als ich schließlich akzeptierte, dass Muslime Verantwortung
für ihre Probleme tragen mussten,
konnte ich dennoch nicht akzeptieren,
dass der Islam selbst verantwortlich zu machen war.
Die Art und Weise, in der der Islam interpretiert wurde,
war das Problem. Ich begann für eine Neuinterpretation
und für die Reform von traditionellen Ansichten zu argumentieren,
aber anstatt mein Gewissen zu erleichtern,
hebte dies nur die Zwecklosigkeit
und Unaufrichtigkeit dieser Ansichten hervor.
Schlussendlich versuchte ich mir selbst einzureden, dass,
obwohl es meinem rationalen Verstand schwer fiel
bestimmte Dinge im Islam zu glauben,
es Erklärungen jenseits meiner Verständnisfähigkeit geben muss,
und dass ‚Gott es am besten weiß‘.
Die sicherste und weiseste Wahl war an ‚Allahs Seil festzuhalten‘.
Ich dachte, ich hatte nichts zu verlieren und alles zu gewinnen,
indem ich ein Gläubiger blieb.
Also lebte ich den Alltag eines ‚guten‘ Muslims, in der Hoffnung,
dass mein Glaube zurückkehren würde.
Doch dieser Schein machte mich nur depressiv
und ich verlor jegliche Motivation.
Das Problem ist, dass man sich das Glauben nicht aussuchen kann.
Entweder man glaubt oder nicht, und wenn es einen Gott gibt,
dann wäre das letzte, was er von mir wollen würde,
dass ich zum Schein etwas glaube, was ich nicht glauben konnte.
Es war eine riesige Erleichterung als ich mir endlich eingestand,
dass ich an den Islam nicht mehr glaubte.

Doch die Tatsache, dass ich nicht mehr an den Islam glaube,
bedeutet nicht, dass ich auf einmal zum Islamhasser geworden bin.
Ich weiß, dass der Islam dem Leben einer Unzahl von Menschen sehr viel Hilfe,
Trost und gute Werte bringt.
Ich weiß, dass die meisten Muslime gute und anständige Leute sind.
Wie könnte ich jemals Muslime hassen, wenn meine Familie doch muslimisch ist?
Wenn ich mit meinen älteren Kindern darüber spreche, was ich denke,
so erzähle ich ihnen, dass sie für sich selbst herausfinden müssen,
was sie glauben, und wenn sie sich als Muslime glücklich fühlen,
dann soll es so sein.
Ich empfinde es sicherlich nicht als Notwendigkeit,
meine eigenen Ansichten bezüglich Gott und Religion weiterzugeben –
etwas, wozu ich mich verpflichtet fühlte, als ich Muslim war.

Während ich nicht daran glaube, anderen zu sagen was sie glauben sollen,
denke ich, dass man den Mut haben sollte die Ansichten ehrlich zu prüfen,
die essentiell für das eigene Leben sind und die eigenen Taten leiten.
Wenn man wirklich mit ihnen zufrieden ist,
dann sollte man sie vollständig mit Herz und Verstand annehmen,
aber wenn sie der genauen Prüfung nicht standhalten,
dann sollte man sie verwerfen.
Das Leben ist zu kurz, um es von Ansichten diktieren zu lassen,
an die man nicht aufrichtig glaubt.

\hfill Copyright © 2008–2010 by Hassan

\hfill Copyright © 2008–2010 Übersetzung von Aziz

\mainmatter

% Chapters:

%Chapter 1:
\chapter{Hast du den \Quran\ gelesen?}

\img{scale=0.7}{Baby_Hassan.jpg}{}

Jede Nacht, bevor wir zu Bett gingen, sagten wir die \emph{Al-Fatiha} auf –
das kurze erste Kapitel des \Quran.
Ich fühlte, dass Gott über mich wachte.
Mein Vater, öfter jedoch meine Mutter, würde an der Türe stehen,
während ich und meine zwei Brüder unsere Arme verschränkten
und aufrecht im Bett saßen.
Wir sagten es blitzschnell auf Arabisch und Englisch auf,
dann Kopf aufs Kissen und Licht aus.
Manchmal fügte ich mein spezielles Gebet hinzu:
„Lieber Gott, bitte mach, dass alles gut wird.“
Ich war zu müde, um die Einzelheiten aufzuzählen.
Gott verstand schon.

Nur bei besonderen Anlässen mussten wir
die offiziellen täglichen Gebete verrichten,
w.z.B. beim Besuch vom Großvater,
wo wir als Wohltat für ihn eine große Schau der Frömmigkeit abziehen mussten.
Nach dem Gebet zeigten mir meine ältesten Schwestern, Kamelia und Susi,
wie man Du\`a (Bittgebet) macht, um Gott um etwas, das ich wollte, zu bitten.

\begin{quote}
„Halte deine Hände auf und lass keine Lücken offen!“ sagte Kamelia.

„Warum muss ich meine Hände so halten?“ fragte ich.

„Damit du die Geschenke auffangen kannst, die Gott dir schickt!“
antwortete Kamelia.
\end{quote}

Ich akzeptierte diese Antwort kritiklos,
obwohl ich keine Geschenke sehen konnte.
Ich begann mit der Art und Weise vertraut zu werden,
in der Gott die Dinge handhabte.
Wenn ich Gott um ein Fahrrad bat, wusste ich,
dass ich es nicht gleich bekommen würde,
oder es vielleicht überhaupt nicht bekommen würde.
Es war nicht Gottes Schuld, sondern meine,
denn ich war auf irgendeine Art ungezogen und verdiente es nicht.
Wenn ich etwas unbedingt wollte,
würde ich mir große Mühe geben so lange wie möglich artig zu sein.
Ich tat dies, als ich Gott darum bat,
meine Mamma und meinen Papa dazu zu bringen,
mit dem Streiten aufzuhören.
Als sie nicht aufhörten, wusste ich, dass ich nicht artig genug war.

\img{scale=1}{Hassan_Boy.jpg}{}

Als ich älter wurde, rückte der Islam mehr und mehr in den Hintergrund
und im Alter von zwölf, während des langen heißen Sommers im Jahre 1971,
war mein einziges Interesse so wie Marc Bolan zu sein.
Mit meinem Lockenkopf, meiner Samtjacke und meinen Plateaustiefeln,
dachte ich auch, dass ich aussah wie er.
Ein bisschen Glitter auf meinen Wangen und ich war weg,
schmollend und tänzelnd vor dem Spiegel zu ‚Hot Love‘ und ‚Get it On‘.
Ich zeichnete Bilder von ihm und Mickey Finn und hängte sie an die Wand
im Schlafzimmer auf, das ich mit meinen zwei Brüdern teilte.
Unser Schlafzimmer war so sehr zugedeckt mit Zeichnungen
und Bildern von Männern mit Make-Up und hautengen Trikots,
dass mein Vater überzeugt war, dass wir langsam schwul wurden.

\begin{quote}
„Was um alles in der Welt haben all diese Homosexuellen
an euren Wänden zu suchen?“

„Das sind keine Homosexuellen.“

„Natürlich sind sie das. Schau mal seine Augen an,“ sagte er,
und zeigte auf ein Bild von Dave Hill an der Wand,
dem Gitarrist von ‚\href{http://de.wikipedia.org/wiki/Slade}{Slade}‘.

Ich betrachtete seine Augen genau, und versuchte festzustellen
was genau meinem Vater so deutlich den Eindruck „HOMOSEXUELL“ vermittelte.

„Alle tragen Glitzer. Es ist nur Show!“

„Ich möchte, dass ihr sie herunternimmt. Sofort!“
\end{quote}

Wir taten, wie uns geheißen wurde. Man diskutierte nicht mit meinem Vater.

\img{scale=0.8}{Hassans_Father.jpg}
{Mein Vater Aziz (ganz rechts),
während des Krieges gegen Israel im Jahre 1948.}

Mein Vater, geboren als Sohn einer wohlhabenden, ägyptischen Familie,
sprach selten über das Land und die Kultur, aus der er gekommen war.
Er war auch nicht besonders religiös.
Lange Zeit wusste ich nicht, dass er aus Ägypten kam.
Als er zum ersten Mal nach England kam,
nannte er sich \emph{Jean Pierre} und erzählte jedem, dass er ein Franzose ist.
Meine Mutter, Mary Magson, war die Tochter von Horace Magson,
eine Buchhalterin aus Darlington.
Ihre Familie war methodistisch, aber ähnlich wie mein Vater,
war auch sie nicht religiös.
Die beiden begegneten sich in Paris im Jahre 1951.
Er studierte an der Universität von Lausanne, während sie in der ATS war,
wo sie Rettungswagen während des Krieges fuhr.
Nachdem sie geheiratet hatten, zogen sie nach Ägypten, um dort zu leben,
aber infolge der ägyptischen Revolution von 1952 und der Sueskrise,
holte sie die britische Regierung zurück,
zusammen mit anderen anglo-ägyptischen Familien.
Sie wurden in eine alte Tudorhütte umgesiedelt,
mitten in Suffolk in Coddenham, wo ich am 12.\ Mai 1959 geboren wurde,
als fünftes Kind in einer achtköpfigen Familie.

\img{scale=0.7}{Hassans_Mother.jpg}
{Meine Mutter Mary Magson (stehend) in der ATS im 2.\ WK.}

Als ich 6 Jahre alt war, zogen wir nach Finchley in Nord-London um.
Das war der Ort, wo ich den Großteil
meiner Kindheit und Jugendjahre verbrachte.
Während der 60er Jahre war es ein sehr bürgerlicher, weißer Bezirk,
und bald wurde mir bewusst, dass wir nicht ganz hierher passten.
Ich hasste es jemandem meinen Namen zu sagen.
Es war immer die selbe Reaktion;
peinliche Stille und verwirrte oder spöttische Blicke.
Mein Klassenlehrer erstellte einmal eine Liste der Religionszugehörigkeit.
Als er meinen Namen aufrufte, zögerte ich lange,
was natürlich nur mehr Aufmerksamkeit auf mich zog.
Schließlich flüsterte ich „Muslim“.

\begin{quote}
„Was?“ sagte der Lehrer.

„Muslim, Herr Lehrer.“
\end{quote}

Einige Kinder kicherten.
Ein Junge bestand darauf, dass ‚Muslim‘ überhaupt keine echte Religion sei,
und dass ich es ausgedacht hätte.
Meine Geschwister und ich wurden zum Ziel von bigotten Personen und Raufbolden.
Die Beharrlichkeit meines Vaters,
dass wir von Schulversammlungen ausgenommen werden mussten,
machte die Dinge nicht besser.
Ich weiß nicht genau, warum er uns abmeldete.
Ich könnte es besser verstehen, wenn er religiös gewesen wäre,
aber das einzige Ergebnis davon war, dass es jedem bestätigte,
dass wir ‚anders‘ waren und auch so behandelt werden sollten.
Anstatt dem Gleichnis in der Geschichte von ‚The Little Red Hen‘
(Die kleine rote Henne) zuzuhören oder ‚All Things Bright and Beautiful‘
(Alle heiteren und schönen Dinge) zu singen – mein Vater dachte,
dass dies meinen Verstand mit dem Christentum infizieren würde –
wurde ich den Freuden eines pornografischen Romans ausgesetzt,
welches der ältere Junge neben mir las, als wir in dem Raum saßen,
der jenen zugewiesen wurde, die von der Versammlung abgemeldet waren.

Mein Vater war voller Paradoxien und Widersprüche.
In manchen Bereichen war er sehr liberal
und in anderen Bereichen ultra-konservativ.
Seine Einstellung zu Frauen war sehr altmodisch und er erwartete,
dass meine Mutter zu Hause blieb, kochte, putzte und für seine Kinder sorgte,
und er wurde gewalttätig, wenn er dachte,
dass sie nicht all seinen Befehlen gehorchte.
Nachdem sie eine lange Zeit gelitten hatte, klagte meine Mutter auf Scheidung.
Der Rechtsstreit um das Sorgerecht war erbittert und hielt lange an.

\img{scale=0.3}{Hassan_and_Diane.jpg}
{Ich \& Diane (1978)}

Im Alter von 18 Jahren begann ich ein Soziologiestudium
am Polytechnikum in Nord-London,
das ein Hort von Marxisten und militanten trotzkistischen Studenten war.
Ich trat der „Broad Left“ (Breite Linke Partei) bei,
eine der zwei sozialistischen Hauptparteien im Nord-Londoner Polytechnikum,
und wurde als Karikaturist für das Magazin „Fuse“ eingestellt.
Unser Hauptrivale am Polytechnikum war „The Socialist Workers Party“
(Die Sozialistische Arbeiterpartei) und ich zeichnete
viele bissige Karikaturen über die Führerschaft,
verhöhnte die lächerlichen Dinge, die sie sagten,
und verhöhnte ihren Kleidergeschmack und ihre dürftige Körperpflege.
Meine Karikaturen wurden einstimmig hochgelobt, bis ich eine zeichnete,
die meine eigene Partei – die Breite Linke – kritisierte.
Der Herausgeber fühlte sich so sehr beleidigt,
dass ich auf der Stelle gefeuert wurde.
Ich verstieß gegen eine der grundlegendsten Regeln der Politik:
Ordne dich der Parteilinie unter.
Aber ich war nie jemand, der sich irgendeiner Parteilinie unterordnete,
und meine Erfahrung mit linker Studentenpolitik verstimmte mich nur dagegen.
Ich war viel mehr daran interessiert,
wie ich Diane zu einem Rendezvous einladen könnte.
Diane war eine Soziologiestudentin im ersten Jahr, genau wie ich,
und wir wurden fast augenblicklich ziemlich gute Freunde.
Sie hatte einen quirligen und fröhlichen Charakter,
mit dem man sich leicht gut verstehen konnte.
Sie liebte Fußball, hatte Freude an guten Debatten
und wir teilten einen ähnlichen Musikgeschmack.

Nach ein paar Wochen meiner schüchternen und stümperhaften Versuche
etwas anderes außer nur „Hallo“ zu sagen, lud sie mich ein.

\begin{quote}
„Hast du Lust heute Nacht zum Stapleton zu gehen?
Die haben eine Live-Band gebucht.“

„Klar! Um welche Zeit?“

„Komm zu mir um acht Uhr.“
\end{quote}

Ich war dort um sieben, in meinen besten zerrissenen Jeans,
und duftete nach indischem Patschuli.

Es war nur ein Ausflug zur Kneipe,
aber es war der Beginn meiner ersten Beziehung.
Wir wurden zu regelmäßigen Gefährten
und begannen eine Wohnung mit einander zu teilen.

Am Ende des ersten Jahres wurden wir den langweiligen Vorträgen über
Auguste Comte, Max Weber und Karzl Marx überdrüssig.
Wir beide erhofften uns, dass ein Soziologiestudium
eine kreative Erfahrung sein würde,
und vielleicht Antworten auf die großen Fragen des Lebens liefern würde.
Wir fanden enttäuschend heraus, dass es bloß eine Übung im Wiederkäuen
einer Auswahl von langweiligen Büchern war.
Wir entschieden uns das Studium abzubrechen und nach Wales zu trampen.
Darüber hinaus hatten wir keinen anderen Plan.
Nachdem wir mehrere Tage lang in Aberystwyth
psychoaktive Pilze gepflückt hatten,
machten wir uns auf den Weg zum Haus ihrer Eltern in Warrington,
um zu entscheiden was wir als nächstes vorhatten.

Diane hatte mal als Schwester im Calderstones-Hospital für körperlich
und geistig Behinderte gearbeitet, also schlug sie vor,
dass wir beide dort einen Job bekommen sollten.
Calderstones befand sich in einem malerischen Lancashire-Dorf
am Fuße von Pendle Hill, ein Hügel,
der berüchtigt für die Verbrennung von Hexen im Mittelalter war.
Ich dachte es wäre eine wunderbare Idee
und wir mieteten einen kleinen Wohnwagen auf dem Feld eines Bauers,
das sich am Höhenrücken von Pendle Hill befand.
Nur zu Fuß konnte der Wohnwagen über einen sehr langen
und steilen Weg erreicht werden,
der im Dorf begann und sich in die Nebel von Pendle Hill hineinschlängelte.
Der Wohnwagen stand in einer abgeschiedenen Ecke,
umgeben von einem Mist bedeckten Morast,
wo der Wind und der Regen nie aufhörten.
Ich liebte es, aber Diane war nicht so sehr begeistert,
besonders nach der Nacht, in der ich einen Geist sah.
Ich wurde mitten in der Nacht durch das schlagende Geräusch
der Wohnwagentür aufgeweckt.
In meiner schläfrigen Benommenheit schaute ich auf zur Eingangstür
und erblickte was so ähnlich aussah wie ein junges Mädchen –
wahrscheinlich im frühen Jugendalter –
das am Fußende des Bettes stand.
Sie sah so aus, als ob sie Lumpen anhatte,
mit einem traurigen Lächeln im Gesicht.
Sie lehnte ihren Kopf leicht gegen die Trennwand des Wohnwagens
und starrte mich an.
Vielleicht war es die Tochter des Bauers, dachte ich,
und ich konnte nicht verstehen warum sie zu dieser späten Stunde
im eiskalten Winterregen außer Haus war.
Bevor ich meinen Kopf heben und etwas sagen konnte,
erschien hinter ihr ein Mann mit einem Dreispitz.
Er rannte durch den Wohnwagen und mich hindurch,
und verschwand wieder auf der anderen Seite.
Ich sprang nun auf und rufte Dianes Namen, um sie aufzuwecken.
Das Mädchen war weg und alles war still, abgesehen vom Wind.
Diane war verständlicherweise total erschrocken,
und obwohl ich ihr versicherte, dass es nur ein Traum war,
ausgelöst durch die örtlichen Geschichten von Hexenverbrennungen
und außerdem durch den Konsum von großen Mengen an Dope und Alkohol,
wollte sie dennoch nicht mehr im Wohnwagen bleiben.
Wir zogen später in eine kleine Wohnung in Blackburn um.

Es war nun das Jahr 1979 und ich hatte eine Gruppe
von engen Freunden und ein gutes Sozialleben.
Doch im Inneren fühlte ich mich ruhelos und unsicher über meine Identität.
Dies wurde ab und an durch die negative Reaktion
von ein paar bigotten Menschen unterstrichen,
die deutlich machten, dass jemand mit einem lustigen Namen
und einer fremden Religion, obwohl ich nicht praktizierend war,
niemals als Engländer akzeptiert werden würde.
Eine Stationsschwester des Krankensaals, in dem ich arbeitete,
machte es sich zur Gewohnheit mich mit „Dreckiger Araber!“ zu rufen.
Sie fing an, es ab dem Moment zu sagen, an dem sie meinen Namen erfuhr.
Sie schrieb auf mein Zeugnis, dass ich „dreckig, faul und ineffizient“ sei,
etwas das ich nur viel später herausfand,
als eine sympathische Stationsschwester mir das Zeugnis zeigte.
Nichts was sie sagte stimmte.
Sie mochte mich nicht, einfach nur weil mein Name Hassan war.
Es gab keinen anderen Grund.
Die Ironie war, dass ich mich bis zu diesem Zeitpunkt
nie als Araber gesehen hatte.

Dann geschahen eine Reihe von Ereignissen,
die das Überdenken meiner Ansichten über die Kultur und die Religion anregten,
die ich mit großer Mühe versuchte zu meiden.
Das erste war die Islamische Revolution im Iran.
Ich erinnere mich an die Fernseh-Berichterstattung
über die laufenden Straßenschlachten in Teheran,
zwischen den einfachen Leuten und der schwer bewaffneten Garde des Schahs.
Ich fand die Bilder inspirierend:
trotzende Zivilisten, die sich gegen die Macht eines Tyranns erhoben.
Der revolutionäre Geist aus meiner Studienzeit war in mir geblieben,
und ich sah es instinktiv als Kampf eines Volkes gegen die mächtige Elite an,
aber mir war auch bewusst welche Rolle die Religion spielte – eine Religion,
zu der ich eine Verbindung fühlte – wenn auch nur eine schwache.

Ich wurde mit einem weiteren Beispiel der Kraft der Religion konfrontiert,
als mein enger Freund John Shackleton von einem Campingausflug zurückkam,
um zu verkünden, dass er nun ein wiedergeborener Christ sei.
Es war ein enormer Schock,
da er sich immer so verächtlich über Religion äußerte.
Jetzt weigerte er sich mit mir zur Kneipe um die Ecke zu gehen
oder Musik anzuhören – außer es war Musik über Jesus.
Er war auf nervige Weise frohlockend wegen seiner Religion,
und versuchte ständig mich und Diane zu konvertieren.

\begin{quote}
„Jesus liebt dich und möchte dir verzeihen,“ sagte er.

„Aber ich habe nichts verbrochen,“ sagte ich.

„Wir sind alle Sünder.
Jesus kann dich wieder zu dem machen, wie Gott dich haben wollte.“

„Warum? Hat Gott beim ersten Mal etwa einen Fehler gemacht?“
\end{quote}

Das ständige Bombardement mit religiösem Eifer zwang mich dazu,
Stellung zum Christentum und Religion im Allgemeinen zu beziehen,
etwas worüber ich zuvor nicht lange nachgedacht hatte.
Je mehr John über Dinge wie die Trinität,
die Erbsünde und das Erlösungswerk sprach, desto mehr verstand ich,
dass dies Konzepte waren, an die ich niemals glauben könnte.
Der Gedanke, dass Gott „Drei in Einem“ ist,
und dass der Mensch in Sünde geboren wird,
oder dass Sünden vergeben werden können,
nicht wegen irgendeiner verdienstvollen Tat einer Person,
sondern weil jemand an ein Stück Holz genagelt wurde –
all das stand im Widerspruch mit meinem Verstand und meinem Gerechtigkeitssinn.
John trat einer kleinen,
wenige Kilometer entfernten Gemeinschaft von wiedergebohrenen Christen bei,
aber er besuchte uns weiterhin und betete,
dass wir mit dem Heiligen Geist erfüllt werden mögen.

Ich fühlte mich unberührt von der Idee eines Heiligen Geistes
oder irgendeiner übernatürlichen Kraft.
Ich ging immer davon aus, dass es im Universum mit rechten Dingen zuging,
und hatte nie selbst eine mystische Erfahrung.
Doch im Einklang mit dem religiösen Thema in dieser Reihe von Ereignissen
würde mir bald etwas sehr Mystisches passieren.
Ich campte mit Diane am Musikfestival von Deeply Vale in Lancashire,
und wir unternahmen einen Spaziergang an einer Talseite hinauf.
Als wir an der Spitze standen, hörte ich einen Klang,
den ich nie erwartet hätte.

\begin{quote}
„Kannst du das hören?“ fragte ich.

„Was? Ich höre gar nichts.“

„Es ist der muslimische Gebetsruf… Hör zu!“ sagte ich.
\end{quote}

Ich konnte deutlich diese Worte hören:
\emph{„Asch-hadu an la ilahe illah, asch-hadu anna Muhammad Rasul-allah!“}
(Ich bezeuge, dass es keinen Gott außer Gott gibt, ich bezeuge,
dass Muhammad der Prophet Gottes ist!)
Sie ertönten mit einer ungewohnten Klarheit
über dem Trubel unten beim Festival.
Diane hörte überhaupt nichts.
Ich konnte nicht verstehen warum.
Weder konnte ich verstehen, warum jemand bei einem Musikfestival
den muslimischen Gebetsruf rezitierte.
Ich suchte nach einer vernünftigen Erklärung, aber konnte keine finden,
die mich überzeugte.
Ich fing an zu denken, dass vielleicht doch übernatürliche Dinge passierten,
und dass dies mein persönlicher Weckruf zum Glauben sein könnte.

Ein paar Monate später sah ich Cat Stevens im Fernsehen,
als er ein Abschiedskonzert gab.
Er war Muslim geworden und gab das Musikgeschäft auf,
wandte Ruhm und Reichtum den Rücken zu,
um sein Leben seinem neugewonnen Glauben zu widmen.
Die Öffentlichkeit war verblüfft.
Warum würde jemand, der alles im Leben besaß, es wegen der Religion wegwerfen?
Religion war nicht cool; es war für naive, unsichere Menschen.
Ein Teil von mir hatte die selbe Reaktion.
Aber ein anderer Teil von mir war stolz und inspiriert.
Wenn so eine kreative und angesehene Person wie Cat Stevens
etwas Wertvolles im Islam sah,
vielleicht sollte ich es selbst viel ernster nehmen,
als jemand, der mit islamischem Erbe zur Welt kam.

Die letzte Episode in dieser Folge von Ereignissen war,
als mein Vater unerwarteterweise vor meiner Tür aufkreuzte,
und mich einlud mit ihm für ein paar Wochen nach Ägypten zu gehen.
Ich hatte meinen Vater nicht sehr viel zu Gesicht bekommen,
seitdem ich sein Haus verlies.
Er war nun ins Dorf von Ramsbottom umgezogen,
wo er als Direktor einer örtlichen Schule eingestellt wurde.
Da ich ziemlich in der Nähe wohnte, nahm ich Diane mit, um ihn zu treffen.
Er empfand eine sofortige Abneigung gegen Diane.
Sie war eine unabhängige, willensstarke, junge Frau,
die sich nicht scheute zu sagen, was sie dachte.
Das gefiel mir an ihr, meinem Vater aber nicht.
Er mochte keine Frauen, die widersprachen;
es war gegen die natürliche Ordnung der Dinge.
Frauen sollten eigentlich das tun, was Männer ihnen befahlen –
mein Vater dachte, dass Gott sie in dieser Art erschuf und mein Vater wollte,
dass es auch so bleibt.
Er fing an zwischen uns auf seine übliche Art und Weise Zwietracht zu säen,
indem er negative Randbemerkungen, Andeutungen und abfällige Witze machte,
und behauptete, dass sie gegenüber ihm unhöflich war, und ihr sogar vorwarf,
dass sie aus seinem Haus stahl.
Ich bin mir nicht sicher, ob sein Angebot einer Reise nach Ägypten
ein Teil dieses Bestrebens war.
Er hatte dort wichtige Geschäfte zu erledigen,
bezüglich der Erbschaft seines Vaters.
Aber ich vermute, dass er dies auch für eine wunderbare Gelegenheit hielt,
um mich von Diane wegzukriegen.
Ich sagte zu mir selbst, dass es unsere Beziehung nicht beeinflussen würde,
und dass ein Pauschalurlaub nach Ägypten ein Angebot war,
das ich nicht ablehnen konnte.
Ich denke im Unterbewusstsein fühlte ich mich auch gelangweilt und ruhelos.
Der Ort, zu dem ich und Diane umgezogen waren,
ein Neubau am Stadtrand von Blackburn, war langweilig und seelenlos.
Unser Leben füllte sich mit Routinen und unsere Beziehung
verfiel in den alltäglichen Trott.
Ich war begierig auf einen Szenenwechsel – zumindest für zwei Wochen.

Bevor wir zum Flughafen fuhren,
hielt mein Vater beim Haus meiner Schwester Susi in London an,
um sich zu verabschieden.

\begin{quote}
„Was willst du in Ägypten unternehmen, Hassan?“ fragte sie.

„Ich habe darüber nachgedacht, nach Möglichkeit Ägyptologie
an der Amerikanischen Universität von Cairo zu studieren.“

„Inscha-Allah,“ (So Gott will) ermunterte Susi.

„Ich verstehe nicht, warum ich das sagen sollte, Susi,“ antwortete ich barsch.
„Entweder werde ich Ägyptologie studieren, oder ich werde es nicht.
Was hat Gott damit zu tun?“

„Nichts geschieht ohne den Willen Gottes.“

„Macht er, dass Mörder andere Menschen töten?“

„Er lässt es zu, weil er uns einen freien Willen gibt,“ erwiderte Susi.

„Na dann sehe ich nicht ein, warum Gott sich einmischen sollte,
wenn ich Ägyptologie studieren wollte!“

„Wenn du meinst, Hassan,“ erwiderte Susi müde.
Ich dachte darüber nach, was ich eben gesagt hatte,
als wir zum Flughafen fuhren;
es war arrogant und ich bedauerte, es gesagt zu haben.
\end{quote}

Es war ein langer Flug mit einem zweistündigen Zwischenaufenthalt
in einer trostlosen osteuropäischen Stadt,
in der es anscheinend nichts außer graue Betonmauern
und Polizisten mit Schlagstöcken gab.
Ich war erleichtert, als wir endlich in Ägypten landeten.
Sobald ich aus dem Flugzeug stieg und in die heiße,
feuchte Atmosphäre des Kairoer Flughafen gelangte, erkannte ich,
dass ich in einem anderen Universum war.
Der Duft von Weihrauch wanderte durch ein kunstvolles Gitterfenster.
Eseln beladen mit Gemüse bahnten sich den Weg
durch ein Orchestra von Autogehupe;
Straßenverkäufer priesen lautstark ihre Waren mit Sirenentönen an,
die mich hochschrecken ließen;
Männer beteten auf dem Gehsteig in Pyjamas;
und Frauen warfen Kübel voller Schalen von Balkonen herunter.
Es war ein verrücktes, chaotisches Flickwerk von Gerüchen,
Geräuschen und Farben und es war wie ein großer Kulturschock für mich.
Doch trotz der Fremdartigkeit fühlte ich mich bald wie zu Hause.
Zum ersten Mal in meinem Leben musste ich meine Herkunft nicht verstecken
oder über sie beschämt sein.
Überdies bewunderte und respektierte jeder beide Hälften
meiner kulturellen Herkunft.

\img{scale=0.3}{Hassan_Cousins.jpg}
{Ich \& zwei Cousins in Ägypten (1981)}

Wir übernachteten im Haus meines Onkels in Kairo.
Die Ägypter trugen westliche Kleidung,
schauten synchronisierte Hollywoodfilme an
und besaßen viele der modernen Annehmlichkeiten, die man aus England kannte.
Doch während ich auf dem französischen Einrichtungsstück –
eine Replika aus dem 18.\ Jahrhundert – saß,
begann der Gebetsruf aus dem Lautsprecher draußen auf der Straße zu heulen.
Dies löste eine Welle von Gebetsrufen aus,
die sich langsam über die Dächer in Kairo und in die weite Ferne ausbreiteten.
Selbst im Fernsehen unterbrach man Clark Gable mitten im Satz,
als ein Hinweis auf Arabisch kam, der das Gebet ankündigte.
Als alle für das Gebet aufstanden,
und ich am Tisch alleine sitzen gelassen wurde,
fühlte ich mich etwas unangenehm.

Nach dem Gebet kam meine Cousine Nihal,
die meinem Tantchen in der Küche dabei geholfen hatte Essen zu braten,
mit einem dampfenden Gericht ins Zimmer herein.

\begin{quote}
„Magst du ‚Beetles‘?“

„Ähm… Ich habe nie welche probiert!“ sagte ich und fühlte mich etwas mulmig.

„Ich liebe sie viel zu sehr!“
Sie legte den Teller auf den Tisch.
„Vor allem mag ich Paul; er ist viel zu süß!“

„Ach so…“ sagte ich mit einem Seufzer der Erleichterung.
„Ja, ich mag sie, aber sie haben sich vor ein paar Jahren aufgelöst, weißt du!“
Ägypter lieben alles Englische und wissen eine Menge über England,
jedoch scheinen ihre Informationen etwa ein Jahrzehnt alt zu sein.

„Aufgelöst?“

„Sie spielen nicht mehr zusammen.“

„Ach? Warum?“

„Na ja, nach einer Weile geschieht das mit Bands…“

„Georgie der Beste!“ unterbrach Hamdy und gab mir die Daumen hoch.
„Manchester United! Gut.“

„Na ja, eigentlich bin ich Fan der Spurs.“

„Sopurs? Was ist Sopurs?“

„Tottenham Hotspur – das ist ein Fußballteam.“

Nihal zeigte mir ein Bild in einer ägyptischen Zeitung von Ayatollah Khomeini,
wo er ein kleines Mädchen umarmt. „Ohhh wie süß, er ist so ein guter Mann!“

„Die Leute scheinen ihn zu lieben.“

„Er sagt, dass es keinen Unterschied zwischen den Sunniten
und den Schiiten gibt.
Er sagt, wir sind alle Muslime und sollten vereint sein.“
\end{quote}

Nihal war eine sehr willensstarke, unabhängige Frau, die ihre Freiheit,
das zu tun was sie beliebte, für selbstverständlich hielt.
Sie trug kein Kopftuch und hatte sehr westliche Gewohnheiten und Vorlieben.
Dennoch schien sie sich dabei völlig wohl zu fühlen sich mit traditionellen,
orthodoxen Ansichten zu identifizieren – etwas, womit die meisten Ägypter,
die ich kennenlernte, überhaupt keine Probleme hatten,
obwohl sie relativ verwestlicht waren.

\begin{quote}
„Iss, Hassan!“ sagte Tantchen Ola, während sie sich neben mich setzte.
„Wir haben für dich englisches Essen zubereitet: Fish and Chips!“

„Sprichst du deine Gebete, Hassan?“ fragte mein Onkel.

„Um ehrlich zu sein: nein, das tue ich nicht.“

„Oh, du musst beten! Der Prophet Mohammed sagte,
dass ‚das Gebet der Schlüssel zum Paradies ist‘.“

„Ich bin mir nicht sicher, ob ich an all das glaube.
Ich meine, warum braucht Gott unsere Gebete?“

„Gott braucht unsere Gebete nicht. Aber wir haben das Bedürfnis zu beten.
Um Dank zu sagen, und um seine Hilfe zu ersuchen.“

„Ich verstehe immer noch nicht,
warum wir Dank sagen oder in Gebeten um Hilfe bitten müssen.“

„Hast du den \Quran\ gelesen, Hassan?“

„Ein wenig.“
\end{quote}

Onkel Fouad nahm ein Buch aus dem Regal.

\begin{quote}
„Hier ist eine englische Übersetzung für dich.
Ich möchte, dass du mir versprichst, es zu lesen.“
\end{quote}

Ich sträubte mich etwas zu versprechen, das ich nicht tun wollte,
aber da ich ein Gast in seinem Haus war, konnte ich kaum ablehnen.
Ich dachte, ich könnte ein paar Seiten lesen,
und es dann höflich zur Seite legen.

\begin{quote}
„Danke. Okay, mach ich.“

„Insha-Allah,“ ermunterte Onkel Fouad.

„Insha-Allah,“ erwiderte ich.
\end{quote}

Am nächsten Tag gingen mein Vater und mein Onkel hinaus,
und ließen mich zu Hause mit Tantchen Ola zurück.
Also nahm ich den \Quran\ in die Hand, wie ich versprach, und begann zu lesen.
Zu meiner Überraschung konnte ich es nicht weglegen.
Der \Quran\ ist kein Buch wie jedes andere.
Es befolgt überhaupt keine der Grundsätze der normalen Prosa.
Es hat keinen klaren Anfang und kein klares Ende.
Es gibt keine lineare Handlung und keine saubere Lösung.
Es scheint von einer Schilderung sehr plötzlich zur anderen zu springen.
Selbst sein Stil ändert sich ohne Vorwarnung von einer stetigen Erzählung
zu einer temporeichen Reimprosa.
Dennoch fand ich es seltsamerweise unwiderstehlich.

\begin{quote}
„Elif Lam Mim.“
\end{quote}

Ich sah auf zu Tantchen Ola, die still sitzend eine Zigarette rauchte,
während sie ein Magazin las,
das schöne Frauen beim Stolzieren über einen Laufsteg zeigte.

\begin{quote}
„Was bedeuted ‚Elif Lam Mim‘?“

„Niemand weiß es.“ Sie lächelte.
„Manche Kapitel im \Quran\ beginnen mit Buchstaben des Alphabets.
Gelehrte haben versucht sie zu erklären.
Aber niemand weiß es genau.“

„Meinst du, es ist ein Mysterium?“

„Ja.“
\end{quote}

Mir gefielen Mysterien.

\begin{quote}
\itshape
„Allah ist das Licht der Himmel und der Erde.
Das Gleichnis Seines Lichts ist wie eine Nische,
worin sich eine Lampe befindet.
Die Lampe ist in einem Glas.
Das Glas ist gleichsam ein glitzernder Stern –
angezündet von einem gesegneten Baum, einem Ölbaum,
weder vom Osten noch vom Westen, dessen Öl beinah leuchten würde,
auch wenn das Feuer es nicht berührte.
Licht über Licht. …“
(\QRef{24:35})

„Und wenn Meine Diener dich nach Mir fragen (sprich):
‚Ich bin nahe. Ich antworte dem Gebet des Bittenden, wenn er zu Mir betet. …‘“
(\QRef{2:186})

„Wahrlich, Wir erschufen den Menschen, und Wir wissen alles,
was sein Fleisch ihm zuflüstert; denn Wir sind ihm näher als die Halsader.“
(\QRef{50:16})

„Allahs ist der Osten und der Westen; wohin immer ihr also euch wendet,
dort ist Allahs Angesicht. Wahrlich, Allah ist freigebig, allwissend.“
(\QRef{2:115})

„Und weise deine Wange nicht verächtlich den Menschen
und wandle nicht hochmütig auf Erden;
denn Allah liebt keine eingebildeten Prahler.“
(\QRef{31:18})

„Und (gedenket der Zeit) da Wir einen Bund schlossen mit den Kindern Israels:
‚Ihr sollt nichts anbeten denn Allah;
und Güte (erzeigen) den Eltern und den Verwandten und den Waisen und den Armen;
und redet Gutes zu den Menschen
und verrichtet das Gebet und zahlet die Zakat.‘ …“
(\QRef{2:83})

„O ihr Menschen, Wir haben euch von Mann und Weib erschaffen
und euch zu Völkern und Stämmen gemacht, dass ihr einander kennen möchtet.
Wahrlich, der Angesehenste von euch ist vor Allah der,
der unter euch der Gerechteste ist. …“
(\QRef{49:13})
\end{quote}

Ich fing an zu weinen.
Ich fühlte mich dumm und versuchte meine Tränen vor Tantchen Ola zu verstecken.
Aber ich konnte nicht aufhören.
Ich fühlte eine seltsame Kraft in meinem tiefsten Inneren.
Gewiss, es war die sanfte und liebevolle Gegenwart Gottes, die zu mir sprach.
Es war so, als ob ein Schleier von meinem Gesicht fiel
und ich die Wahrheit entdeckte, nach der ich von Kindheit an suchte.
Es war eine zutiefst spirituelle und emotionale Zeit für mich.
Ich konnte kaum den \Quran\ erkennen,
so wie er heute von den Medien zitiert wird,
mit seinen heftigen und schlimmen Passagen.
Nie sah ich solche Verse zu dieser Zeit.
Nicht dass sie da gewesen wären,
aber sie sprachen mich nicht auf wörtliche Weise an.

Ich unternahm eine Fahrt zur Amerikanischen Universität,
aber der Leiter der Abteilung für Ägyptologie war zu der Zeit nicht anwesend.
Ich ging nicht noch einmal, denn ich hatte das Interesse verloren.
Die meiste Zeit meiner zwei Wochen in Ägypten verbrachte ich damit,
nur den \Quran\ zu lesen, und ich machte außerdem gelegentlich Besuche,
um andere Mitglieder meiner neu entdeckten Großfamilie kennenzulernen.
Auch dort wandten sich die Gespräche ständig der Religion zu.

\begin{quote}
„Ein Freund von mir sagt,
dass ich nur durch den Glauben an Jesus erlöst werden kann,
weil er für unsere Sünden gestorben ist.“

„Der Islam sagt das Gegenteil,“ sagte Magdi. „Der \Quran\ sagt:

‚Wer den rechten Weg befolgt, der befolgt ihn nur zu seinem eignen Heil;
und wer irregeht, der geht irre allein zu seinem eignen Schaden.
Und keine Lasttragende trägt die Last einer andern.‘
(\QRef{17:15})“

„Der Islam ist die Religion unserer
‚\href{http://de.wikipedia.org/wiki/Fitra}{Fitra}‘ (Natur);
es ist in völligem Einklang mit unserem natürlichen Instinkt.“

„Warum kann es dann nicht jeder einsehen?“

„Der Prophet sagte:
\emph{‚Die Menschen sind im Schlaf und nur wenn sie sterben, wachen sie auf.‘}
Das ist die Natur dieser Welt, Hassan.
Wenn alles klar und einfach wäre, dann gäbe es keine Prüfung.“
\end{quote}

Magdi gab mir ein Buch von Hadithen (Aussprüche des Propheten Mohammed),
das ich von der ersten bis zur letzten Seite las.
Ein Hadith im Besonderen berührte mich tief:

\begin{quote}
„(Gott sagt) Ich bin wie Mein Diener über Mich denkt.
Ich bin mit ihm, wenn er Meiner gedenkt.
Wenn er sich Mir um eine Handspanne nähert,
nähere Ich Mich ihm um eine Armlänge.
Wenn er sich Mir um eine Armlänge nähert,
nähere Ich Mich ihm um zwei Armlängen.
Wenn er zu Mir gehend kommt, komme Ich ihm eilend entgegen.“
(Bukhari 8/171)
\end{quote}

Als es Zeit wurde nach England zurückzukehren,
wollte ich Ägypten nicht verlassen und schwor bald wieder zurück zu sein.
Es war ein unglaubliches Erlebnis und ich fühlte ein Gefühl des Überschwangs –
so als ob sich eine magische Tür geöffnet hätte.
Ich hatte endlich Gott gefunden.
Ich kehrte zurück nach England voller Eifer und Entschlossenheit,
um mehr über meinen wiederentdeckten Glauben zu erfahren.
Diane war schockiert über meine plötzliche Konvertierung und war sich sicher,
dass es bloß eine vorübergehende Modeerscheinung war.
Sie gab mir nach und hoffte, dass ich wieder zur Vernunft kommen würde.
Aber das einzige worüber ich reden konnte, war der Islam.
Ich erklärte, dass der Islam Sex vor der Ehe verbietet und bestand darauf,
dass wir aufhören zusammen zu schlafen.
Ich hörte auf zu trinken und zu rauchen, und fing an regelmäßig zu beten.
Ich belehrte Diane über den Tag des Jüngsten Gerichts und Himmel und Hölle
und die Aussprüche des Propheten Mohammed,
und versuchte sie verzweifelt vom Islam zu überzeugen.
Schließlich haben wir beide erkannt, dass wir unsere Zeit verschwendeten.
Diane hatte kein Interesse daran Muslimin zu werden
und ich machte auch keine Phase durch.
Unsere Beziehung war zu Ende.
Ich zog aus unserer Wohnung aus,
und ließ unter anderem meine Plattensammlung und meine Gemälde zurück.
Ich hatte keine Notwendigkeit
für götzendienerische Dinge, die mich vom Pfad Allahs ablenkten.


%Chapter 2:
\chapter{Der Pfad Allahs}

\img{scale=1}{Hassan_1980.jpg}{}

...

%Chapter 3:
\chapter{Der Pfad des Sufi}

\img{scale=0.7}{Members_SOAS_Islamic_Society.jpg}
{Mitglieder der SOAS Islamic Society, Oktober 1980.}

...

%Chapter 4:
\chapter{Die Zunahme des Aktivismus}

\img{scale=0.7}{Assassination_of_Sadat.jpg}{}

...

%Chapter 5:
\chapter{Die Islamia-Schule}

...

%Chapter 6:
\chapter{Scheich Faisal}

...

%Chapter 7:
\chapter{Dienstag Nachmittag}

\img{scale=0.4}{911_Second_Plane.jpg}{}

...


%Chapter 8:
\chapter{Offenbarung \& Vernunft}

\img{scale=0.6}{Man_Reading_Koran.jpg}{}

...

%Chapter 9:
\chapter{Bombenanschläge in London}

\img{scale=1.2}{London_Bombings.jpg}{}

...

%Chapter 10:
\chapter{Das Unreformierbare reformieren}

\img{scale=0.4}{Quranic_Script.jpg}{}

...

%Chapter 11:
\chapter{Religion}

\img{scale=0.6}{Neasden_Temple.jpg}{}

...

%Chapter 12:
\chapter{Die Büchse der Pandora}

\img{scale=0.5}{Pandoras_Box.jpg}{}

...

\backmatter

\chapter{Index}
...

\end{document}
