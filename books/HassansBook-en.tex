%!TEX TS-program = xelatex
%!TEX encoding = UTF-8 Unicode
\documentclass[12pt]{memoir}
\usepackage[
  xetex,bookmarksnumbered=true,pdfborder=0,hyperfootnotes=true
]{hyperref}
\usepackage{xltxtra}
\usepackage{bidi}
\usepackage{xunicode}

\usepackage{fontspec}
\usepackage{color}
\usepackage[UKenglish]{babel}

\setmainfont{DejaVu Serif}

\hypersetup{
  pdfauthor = {Hassan},
  pdftitle = {Autobiography by Hassan},
  pdfsubject = {},
  pdfkeywords = {Islam Religion Allah God Faith Scepticism Life},
}

\definecolor{darkblue}{rgb}{0,0,0.62}

% Transliterated "hamza".
\def\´{ʾ} % ˀ
\let \hamza=\´ % Needed in \footnotetext.

% Transliterated "`ayin".
\def\`{ʿ} % ˁ
\let \ayin=\` % Needed in \footnotetext.

% Aliases for subscript and superscript:
\let \Sub=\textsubscript
\let \Sup=\textsuperscript

% Hyphenation rules:
% \hyphenation{}
\hyphenation{}

% Corrective comments.
\newcommand{\cmt}[2]{#1} % #1 is the actual text. #2 is the comment.

% Use Scheherazade for Arabic. Use \ar{...} for inline Arabic text.
\newfontfamily{\arabicfont}[Script=Arabic,Scale=1.7]{Scheherazade}
\newcommand{\ar}[1]{\RL{\arabicfont#1}}

% Macros for the word Qur'an.
\def \Quran{Qur\-\´ān} % Used when followed by a punctuation mark.

% Breakable forward slash.
\def\/{\discretionary{/}{}{/}}

% For hyphenation: Trick TeX into thinking al\–XYZ are two separate words.
\def\–{-\hskip0pt}
\def\al{al-\hskip0pt} % Version with "al-".

% Prints symbols for dividing paragraphs.
\def\pardivider{\centerline{***}} % \ar{۞۞۞} ❧❦

% Qur'an reference numbers. TODO: link to web page.
\newcommand{\QRef}[1]{{\color{darkblue}#1}}

% "Nota bene" macro.
\newcommand{\NB}[1]{\emph{\small NB: #1}}

% Paragraph spacing:
\setlength{\parskip}{1ex plus 0.5ex minus 0.2ex}

\pagestyle{plain}

\title{Autobiography}
\author{\emph{by Hassan}}

%%%%%%%%%%%%%%%%%%%%%%%%%%%%%%% Document begin %%%%%%%%%%%%%%%%%%%%%%%%%%%%%%%%
\begin{document}
\frontmatter

% Title:
\maketitle
\thispagestyle{empty}
\cleardoublepage

% Table of Contents:
\setcounter{page}{1}
\tableofcontents

% Preface:
\chapter{Preface}

% \picture{Tunis_Hassan.jpg}

My father was Egyptian and my mother English and I have 8 brothers and sisters.
I was born Muslim but didn’t start practicing until I was twenty
when I became very devout and committed.
For the next twenty eight years Islam guided every aspect of my life.
I completed a BA in Arabic and Islamic Studies at the School of Oriental
and African Studies where I became President of the Islamic Society.
After leaving university I became Amir of a Da\`wah group
in North London and edited an Islamic magazine called ‘The Clarion’.
I wrote four books for Muslim children and spent fifteen years as a teacher
at Islamia School, the one founded by Yusuf Islam (Cat Stevens).
But shortly before my 48th birthday I knew I no longer believed in Islam.
If you had told me a few years ago that such a thing would happen to me,
I would never have believed it and added that no ‘true’ Muslim
would ever reject Islam after having tasted the sweetness of Iman (faith).
No one who has immersed himself in the beauty of the \Quran\
and appreciated its wisdom could ever deny that it is the word of God.
If anyone had claimed such a thing I would have instinctively
doubted him and been suspicious of his motives.
But it is amazing how perceptions can change
and things I once thought unimaginable now seem perfectly reasonable.
Of course this change didn’t happen overnight.
It began a few years ago when I started to question the beliefs
I had for so long taken for granted
and started to look at Islam in a new light.

Little by little doubts began creeping in.
At first I tried to suppress them
and reacted to criticism of Islam with denial, anger and blame.
I denied there was anything wrong, felt hyper-sensitive
to criticism and blamed the West for provoking and creating problems.
When I did eventually accept that Muslims had to take responsibility
for the problems we faced,
I still couldn’t accept that Islam itself was to blame.
It was the way Islam was being interpreted that was the problem.
I started arguing for a reinterpretation and reform of traditional views,
but instead of easing my conscience,
this only highlighted the futility and dishonesty of such views.
Finally I tried to tell myself that although my rational mind
found it difficult to believe certain things in Islam,
there must be explanations beyond my capacity to understand
and that ‘God knows best’.
The safest and wisest option was to “Hold fast to the rope of Allah”.
I reckoned I had nothing to lose and everything to gain by remaining a believer
and so I went through the motions of being a ‘good’ Muslim,
in the hope that my faith would return.
But this pretence only made me depressed and lose all motivation.
The problem is that one cannot choose to believe.
Either one does or not and if there is a God,
the last thing he would have wanted me to do
was to pretend to believe in something that I didn’t.
It was quite a relief when I finally admitted to myself
that I no longer believed in Islam.

However the fact that I no longer believe in Islam
doesn’t mean I have suddenly turned into a hater of Islam.
I know that Islam brings a great deal of guidance,
comfort and worthy values to the lives of millions of people.
I know that most Muslims are good and decent people.
How could I possibly hate Muslims when my family is Muslim?
When I speak to my older children about what I think,
I tell them they must find out for themselves what they believe
and if they feel happy being Muslims then that is what they should be.
I certainly don’t feel the need to pass on my own beliefs
concerning God and religion to them –
something I felt it was my duty to do when I was a Muslim.

While I do not believe in telling anyone what they should believe,
I do think one should have the courage to honestly examine
the beliefs that are central to one’s life and guide one’s actions.
If one is truly satisfied with them,
then they should be fully embraced with one’s heart and mind,
but if they do not stand up to close scrutiny,
then they should be discarded.
Life is too short to allow it to be dictated
by beliefs one does not truly believe.

\hfill Copyright © 2008–2010 by Hassan

\mainmatter

% Chapters:

%Chapter 1:
\chapter{Have you read the \Quran?}

% [[File:Baby_Hassan.jpg|250px|center]]

Every night, before going to bed, we recited \emph{al\–Fatiha},
the short first chapter of the \Quran.
I felt God was watching over me.
My father, or more usually, my mother, would stand at the door,
while my two brothers and I folded our arms and sat up in bed.
We recited it at lightening speed in Arabic and English,
then heads down, lights out.
Sometimes I also added my special prayer quietly to myself;
\emph{“Please God, make everything alright”}
I was too tired to go through the details.
God understood.

The only times we had to say the formal daily prayers was on special occasions,
like a visit from Granddad,
when we had to make a big show of religious piety for his benefit.
After praying, my eldest sisters, Kamelia and Suzy showed me
how to make Du\`a (supplication) to ask God for anything I wanted.

\begin{quote}
“Hold your hands out and don’t leave any gaps!” Kamelia said.

“Why do I have to put my hands like that?” I asked.

“So that you can catch the presents God is sending you!” Kamelia answered.
\end{quote}

I accepted that reply without question
even though I could not see any presents.
I was becoming familiar with God’s way of doing things.
If I asked God for a bicycle, I knew it would not come immediately
and perhaps may never come at all.
It wasn’t God’s fault, but because I had been naughty in some way
and didn’t deserve it.
When I wanted something very badly I would try very hard
to be good for as long as possible.
I did that when I asked God to make mummy and daddy stop fighting.
When they didn’t stop I knew it was because I wasn’t good enough.

% [[File:Hassan_Boy.jpg|200px|center]]

As I got older, Islam receded further and further into the background
and by the age of twelve, during the long hot summer of 1971,
my only concern was to be like Marc Bolan.
With my curly hair, velvet jacket and platform boots,
I thought I looked like him too.
A dab of glitter on my cheeks and I was away,
pouting and prancing in front of the mirror to ‘Hot Love’ and ‘Get it On’.
I drew pictures of him and Mickey Finn
and stuck them on the wall of the bedroom I shared with my two brothers.
Our bedroom became so covered in drawings and pictures of men in make-up
and tight-fitting leotards from ‘Oh Boy!’ ‘Pop Swap’ and ‘Jackie’
that my father was convinced we were becoming queer.

\begin{quote}
“What on earth are all these homosexuals doing on your walls?”

“They’re not homosexuals.”

“Of course they are; just look at his eyes,”
he said, pointing to a picture of Dave Hill, the guitarist from Slade.
\end{quote}

I studied his eyes carefully, trying to determine exactly what it was
that transmitted “HOMOSEXUAL” so plainly to my dad.

\begin{quote}
“They all wear glitter; it’s just for show!”

“I want you to take them down. Now!”
\end{quote}

We did as we were told.
One didn’t argue with my father.

% [[File:Hassans_Father.jpg|450px|center]]
% My father, Aziz (far right) during the war against Israel 1948.

Born to a wealthy Egyptian family,
my father rarely talked about the country
and culture he came from and wasn’t very religious.
For a long time I never knew he was from Egypt.
When he first came to England he called himself \emph{Jean Pierre}
and told everyone he was French.
My mother, Mary Magson, was the daughter of Horace Magson,
an accountant from Darlington.
Her family were Methodists, but like my father, she wasn’t religious.
The two of them met in Paris in 1951.
He was studying at the University of Lausanne,
while she was in the ATS where she’d driven ambulances during the War.
After getting married they went to live in Egypt,
but as a result of the 1952 revolution and the Suez Crisis,
the British Government evacuated them
along with other Anglo-Egyptian families.
They were resettled in an old Tudor cottage in the middle
of the Suffolk village of Coddenham,
where I was born on the 12th of May, 1959,
fifth child in a family of eight.

% [[File:Hassans_Mother.jpg|250px|center]]
% My mother, Mary Magson (standing) in the ATS during WW2.

We moved to Finchley in North London when I was 6 years old
and that is where I spent most of my childhood and teenage years.
During the 1960s it was a very white middle-class area
and I soon became aware that we didn’t quite fit in.
I hated telling someone my name.
It was always the same reaction,
an embarrassing silence and a confused or mocking expression.
My class teacher once made a list of religious affiliations.
When he called my name out I hesitated for a long time,
which of course only attracted more attention.
Eventually I whispered Muslim.

\begin{quote}
“What?” said my teacher.

“Muslim, sir.”
\end{quote}

Some children sniggered.
One boy insisted that ‘Muslim’ wasn’t a real religion at all
and I had made it up.
My siblings and I became a target for bigots and bullies.
This wasn’t helped by my father’s insistence
that we were to be withdrawn from school assemblies.
I don’t really know why he wanted to withdraw us.
I could better understand it if he had been religious,
but the only effect it had was to confirm to everyone
that we were ‘different’ and should be treated as such.
Instead of listening to the parable of ‘The Little Red Hen’
or singing ‘All Things Bright and Beautiful’,
which my father thought would infect my mind with Christianity,
I was exposed to the delights of a pornographic novel
which the older boy next to me was reading
as we sat in the room assigned for those who opted out of assembly.

My father was full of paradoxes and contradictions.
In some ways he was very liberal-minded and in other ways ultra\–conservative.
His attitude towards women was very old fashioned
and he expected my mother to stay at home, cooking,
cleaning and looking after his children
and he became violent if he felt she wasn’t obeying his every command.
After suffering for many years my mother eventually sued for divorce.
The court battle over custody of the children was long and bitter.

% [[File:Hassan_and_Diane.jpg|500px|center|Me & Diane 1978]]
% Me & Diane 1978

At the age of 18 I started a degree in Sociology
at the Polytechnic of North London,
a hotbed of Marxists and Trotskyite student militants.
I joined the Broad Left, one of the two main Socialist parties
at North London Polytechnic and was appointed cartoonist
for the college magazine, “Fuse”.
Our main rival at the Polytechnic were The Socialist Workers Party
and I drew many scathing caricatures of the leadership,
lampooning the ridiculous things they said,
their lack of dress sense and poor personal hygiene.
My cartoons were unanimously praised until I drew one
criticising my own party – the Broad Left.
The Editor took such offence that I was sacked on the spot.
I had contravened the most basic rule of politics – stick to the party line.
But I had never been one for sticking to any party line
and my experience with left-wing student politics
only served to alienate me from it.
I became far more interested in how to ask Diane out on a date.
Diane was a first year sociology student,
like myself and we had become fairly good friends almost immediately.
She had a bubbly and cheerful character that was easy to get on with.
She loved football, enjoyed a good debate
and we shared a similar taste in music.

After a few weeks of my shy and bumbling attempts
to say something other than “Hi”, she asked me out.

\begin{quote}
“Fancy going to the Stapleton tonight? They’ve got a live band playing.”

“Yeah! What time?”

“Come over about eight.”
\end{quote}

I was there at seven, in my best ripped jeans and smelling of Patchouli.

It was only a trip to the pub, but it was the start of my first relationship.
We became constant companions and started sharing a flat together.

At the end of the first year we’d grown tired of boring lectures
on Auguste Comte, Max Weber and Karl Marx.
Both of us had been hoping that studying Sociology
would be a creative experience,
perhaps providing clues to the great questions of life.
We were disappointed to find it was merely an exercise
in regurgitating a selection of tedious books.
We decided to drop-out and hitch-hike to Wales.
We didn’t have much of a plan beyond that.
After several days of picking psychedelic mushrooms in Aberystwyth,
we made our way to her parent’s house in Warrington to decide what to do next.

Diane had once worked as a nurse in Calderstones Hospital
for the physically \& mentally handicapped
and she suggested we both get jobs there.
Calderstones was set in a picturesque Lancashire village at the foot of,
Pendle Hill, infamous for its association
with burning witches during the Middle Ages.
I thought it was a terrific idea and we rented a small caravan
in a farmer’s field along the ridge of Pendle Hill itself.
The caravan could only be reached by walking up a very long and steep road
that began in the village and wound up into the mists of Pendle Hill.
The caravan was in an isolated corner,
surrounded by manure covered boggy fields,
where the wind and the rain never ceased.
I loved it, but Diane was not so keen, particularly after the night I saw a ghost.
I was woken in the middle of the night by the sound of the caravan door banging.
I looked up from my sleepy stupor towards the entrance
and saw what looked like a young girl – perhaps in her early teens –
standing at the foot of my bed.
She looked as if she was wearing rags and had a sad smile on her face.
She leant her head slightly onto the partition wall of the caravan
and stared at me.
I thought perhaps it was the farmer’s daughter,
and I couldn’t understand why she would be out at this late hour,
in freezing winter rain.
Before I could lift my head up and say something a man in a three cornered hat
appeared from behind her and ran through the caravan,
past me and disappeared out the other side.
I jumped up now and was calling out for Diane to wake up.
The girl was gone and everything was quiet, apart from the wind.
Diane was understandably freaked out and although I re-assured her
that it was just a dream triggered by the local stories of witch burnings
we’d heard along with the large amount of dope and alcohol we’d consumed,
she didn’t want to stay in the caravan anymore.
We moved out later and into a little flat in Blackburn.

It was now 1979, and I had close group of friends and a good social life.
But under the surface I felt restless and unsure of my identity.
This was highlighted now and again by the negative reaction of a few bigots,
who made it clear that someone with a funny name and foreign religion,
even though I wasn’t practicing, would never be accepted as English.
A Charge Nurse, on a ward I worked on, took to calling me “Dirty Arab!”
She started saying it from the moment she discovered my name.
She wrote on my report that I was “Dirty, lazy and inefficient”
something I only discovered much later
when a sympathetic charge nurse showed me the report.
None of what she said was true.
She didn’t like me simply because my name was Hassan.
There was no other reason.
The irony was that, up until that point,
I had never considered myself to be Arab.

Then a series of events caused me to re-think my views
about the culture and religion I had tried hard to avoid.
The first was the Islamic revolution in Iran.
I remember watching the television coverage of running battles
on the streets of Tehran, between the ordinary people
and the heavily armed guard of the Shah.
I found the images inspiring:
defiant civilians rising up against the might of a tyrant.
The revolutionary spirit of my student days had remained with me
and I instinctively saw it as a people’s struggle against a powerful elite,
but I was also aware the role religion was playing –
a religion that I felt some connection to – albeit a weak one.

I was confronted with another example of the power of religion
when my close friend John Shackleton returned from a camping trip
to announce he was now a born-again Christian.
It was an enormous shock, as he had always been so scornful of religion.
Now he refused to come down to the pub with me or listen to music –
apart from music about Jesus.
He became annoyingly exultant about his faith
and constantly attempted to convert me and Diane.

\begin{quote}
“Jesus loves you and wants to forgive you,” he said.

“But I haven’t done anything,” I said.

“We are all sinners. Jesus can restore you to how God wanted you to be.”

“Why? Did God make a mistake the first time round?”
\end{quote}

The constant onslaught of religious fervour forced me
to take a position on Christianity and religion in general,
something I had not given a lot of thought to before.
The more John explained things such as the Trinity, Original Sin,
and Atonement, the more I knew these were concepts I could never believe in.
The idea that God is “Three persons in One” and that man is born sinful
because of the sin of another or that sins can be forgiven,
not because of any meritorious act by the person themselves,
but because someone else was nailed to a plank of wood,
all conflicted with reason and my sense of justice.
John joined a small community of born-again Christians a few miles away,
but continued to visit us, praying we would be filled with the Holy Spirit.

I was unmoved by the idea of a Holy Spirit or any other supernatural power.
I had always believed that this world was ‘what you see is what you get’
and had certainly never experienced anything mystical myself.
But in keeping with the religious theme of this chain of events,
something mystical was about to happen to me.
I had been camping at the Deeply Vale music festival in Lancashire with Diane,
when we went for a walk up one side of the valley.
As we stood at the top, I heard a sound I never expected to hear.

\begin{quote}
“Can you hear that?” I asked.

“What? I can’t hear anything.”

“It’s the Muslim call to prayer… Listen!” I said.
\end{quote}

I could clearly hear the words
\emph{“Ash-hadu an la ilaha illah, Ass-hadu anna Muhammad rasoolalllah!”}
(I bear witness there is no god but God, I bear witness
that Muhammad is the prophet of God!)
They resonated with a strange clarity above the hubbub of the festival below.
Diane didn’t hear anything.
I couldn’t understand why.
Nor could I understand why someone was reciting
the Muslim call to prayer at a music festival.
I tried to think of a rational explanation,
but couldn’t come up with one that convinced me.
I began to feel that perhaps supernatural things did happen
and that this may be my own personal wake-up call to faith.

A few months later I saw Cat Steven’s on television,
giving a farewell concert.
He had become a Muslim and was resigning from the music business,
turning his back on fame and money
in order to devote his life to his new-found faith.
The public were mystified. Why would someone
who had it all want to throw it away for religion?
Religion wasn’t for cool people; it was for simple-minded, insecure people.
Part of me also had the same reaction.
But part of me felt inspired and proud.
If such a creative and respected person like Cat Stevens
saw something worthwhile in Islam, perhaps I,
as someone born with an Islamic heritage,
should take it more seriously myself.

The final episode in this series of events was
when my father turned up unexpectedly on my doorstep
to invite me to go with him to Egypt for a couple of weeks.
I had not seen a great deal of my father since leaving his house.
He had now moved to the village of Ramsbottom,
where he had been appointed Head Teacher of a residential school.
As I was living fairly near by, I took Diane to meet him.
He took an instant dislike to Diane.
She was an independent, strong-minded young woman
who wasn’t afraid of saying what she thought.
I liked that about her, but my father did not.
He didn’t like women who answered back;
it was against the natural order of things.
Women were supposed to do what men told them –
that’s the way my father believed God created them
and that’s the way my father wanted it to stay.
He set about trying to cause trouble between us
in his usual way of oblique remarks, insinuations, snide jokes,
claiming she was rude to him,
and even accusing her of stealing from his house.
I can’t be certain of whether his offer of a trip
to Egypt was part of that effort.
He had important business there regarding his father’s inheritance.
But I suspect he also thought it a wonderful opportunity
to get me away from Diane.
I told myself it wouldn’t affect our relationship and that
an all-expenses-paid holiday to Egypt was an offer I couldn’t refuse.
I think subconsciously I was also feeling bored and restless.
The area Diane and I had moved into on a newly built estate
on the outskirts of Blackburn was dull and soulless.
Our life had become full of routines and our relationship slipped into a rut.
I was eager for a change of scenery – at least for two weeks.

Before driving to the airport,
my father stopped by my sister Suzy’s house in London to say goodbye.

\begin{quote}
“What are you planning to do in Egypt, Hassan?” she asked.

“I was thinking about exploring the possibilities
of studying Egyptology at the American University in Cairo.”

“Insha-Allah,” (God Willing) prompted Suzy.

“I don’t see why I should say that, Suzy,” I replied curtly.
“Either I will study Egyptology, or I won’t.
What has God got to do with it?”

“Nothing happens unless God wills it.”

“Does he will murderers to kill people?”

“He allows it to happen because he gives us free will,” replied Suzy.

“Well then, I don’t see why God should interfere with me studying Egyptology?”

“If you say so Hassan,”
Suzy replied, wearily.
I thought about what I had just said as we drove to the airport;
it was arrogant and I regretted saying it.
\end{quote}

It was a long flight with a two hour stop-over
in a bleak East European city that seemed to be full of nothing
but grey concrete rectangles and policemen with batons.
I was relieved when we finally landed in Egypt.
The moment I stepped off the plane and into the hot,
moist atmosphere of Cairo airport,
I realized I was in another universe.
The scent of incense drifted through an intricate lattice window.
Donkeys laden with vegetables weaved their way
through an orchestra of blaring car horns;
street merchants announced their wares with a siren cry that made me jump;
men prayed on the pavement, wearing pyjamas;
and women threw buckets of peelings from balconies above.
It was a mad, chaotic patchwork quilt of smells,
noise and colour and came as quite a culture shock to me.
But despite its strangeness I soon felt at home.
For the first time I didn’t have to hide or be embarrassed about my origins.
Moreover, everyone admired and respected both halves of my cultural background.

% [[File:Hassan_Cousins.jpg|500px|center|Me & two cousins in Egypt, 1981]]
% Me & two cousins in Egypt, 1981

We stayed at my uncle’s house in Cairo.
The Egyptians wore western clothes,
watched dubbed Hollywood movies
and had many of the modern conveniences found in England.
But as I sat on the replica 18th Century French furniture,
a loudspeaker in the street outside began bellowing the call to prayer.
This triggered a wave of prayer calls
that slowly unfurled across the Cairo rooftops and into the distance.
Even on television, Clark Gable was cut off in mid flow,
as a sign came up in Arabic, announcing the evening prayer.
When everyone got up to pray,
I was left sitting alone at the table,
I felt a little uncomfortable.

After prayer my cousin Nihal,
who had been helping my aunty fry some food in the kitchen,
came in carrying a steaming dish.

\begin{quote}
“You like beetles?”

“Er… I’ve never had them!” I said, feeling a little queasy.

“I like them too much!”
She put the plate on the table.
“Especially I like Paul; he’s too cute!”

“Oh…” I said with a sigh of relief,
“Yeah I like them, but they split up a few years ago, you know!”
Egyptians loved everything British and knew a great deal about the UK,
though their information seemed to be about a decade old.

“Split up?”

“They don’t play together anymore.”

“Oh? Why?”

“Well bands do that after a while…”

“Georgie Best!” interrupted Hamdy, giving me the thumbs up.
“Manchester United! Good.”

“Well I support Spurs actually”

“Sopurs? What is Sopurs?”

“Tottenham Hotspur – they’re a football team.”
\end{quote}

Nihal showed me a picture in an Egyptian newspaper
of Ayatollah Khomeini hugging a little girl.

\begin{quote}
“Awww, he is such a good man!”

“The people seem to love him.”

“He says there is no difference between Sunni and Shi\`ah.
He says we are all Muslims and should be united.”
\end{quote}

Nihal was a very strong minded,
independent woman who took her freedom to do as she wanted for granted.
She didn’t wear a headscarf and had very western habits and tastes.
Yet she seemed completely comfortable about identifying with traditional,
orthodox views – something most Egyptians I met
seemed totally at ease with despite being relatively westernised.

\begin{quote}
“Eat, Hassan!” said Aunty Ola as she sat next to me.
“We made you English food: Fish and Chips!”

“Do you say your prayers, Hassan?” said my uncle.

“To be honest, no, I don’t.”

“Oh you must pray!
Prophet Muhammad said that ‘Prayer is the key to Paradise’.”

“I’m not sure I really believe in all that.
I mean why does God need us to pray?”

“God doesn’t need us to pray.
But we need to pray.
To give thanks and seek His help.”

“I still don’t see why we have to give thanks or ask for help through prayer.”

“Have you read the \Quran, Hassan?”

“A bit.”

Uncle Fouad took a book from the shelf.

“Here’s an English translation for you.
I want you to promise me you will read it.”
\end{quote}

I was reluctant to promise something I didn’t want to do,
but as I was a guest in his house I could hardly refuse.
I thought I could read a few pages then politely put it to one side.

\begin{quote}
“Thanks. OK, I will.”

“Insha-Allah,” prompted Uncle Fouad.

“Insha-Allah,” I replied.
\end{quote}

The next day my father and uncle had gone out,
leaving me at home with Aunty Ola.
So I picked up the \Quran, as promised, and began to read.
To my surprise I found I couldn’t put it down.
The \Quran\ is not like any ordinary book.
It doesn’t follow any of the conventions of standard prose.
It has no definite beginning or end.
There is no plot to follow and no neat resolution.
It seems to jump rather abruptly from one account to another.
Even its style changes with little warning,
from a steady narrative to fast paced rhyming prose.
Yet I found it strangely irresistible.

\begin{quote}
“Alif Lam Mim.”
\end{quote}

I looked up at Aunty Ola who was quietly sitting smoking a cigarette
as she read a magazine full of beautiful women strutting down a cat walk.

\begin{quote}
“What does Alif Lam Mim mean?”

“Nobody knows.” She smiled.
“Some chapters of the \Quran\ begin with letters of the alphabet.
Scholars have tried to explain them.
But nobody knows for sure.”

“You mean it’s a mystery?”

“Yes.”
\end{quote}

I liked mysteries.

\begin{quote}
\itshape
“God is the Light of the heavens and the earth.
The Parable of His Light is as if there were a Niche and within it a Lamp:
the Lamp enclosed in Glass: the glass like a brilliant star:
Lit from a blessed Tree, an Olive, neither of the East nor of the West,
whose oil is well-nigh luminous,
though fire scarce touched it: Light upon Light!” (\QRef{24:35})

“When my servants ask you about me, (say)
I am indeed close and answer the prayer of the one who calls on me.”
(\QRef{2:186})

“We created man and We know what his soul whispers to him.
We are nearer to him than his jugular vein.” (\QRef{50:16})

“Wherever you turn your face, there is God’s presence.
God is all-Pervading.” (\QRef{2:115})

“Do not turn away from men with pride, nor walk arrogantly through the earth,
but be moderate in your pace and lower your voice.” (\QRef{31:18})

“Treat with kindness your parents and kindred, and orphans and those in need;
speak fair to the people; be steadfast in prayer;
and practice regular charity.” (\QRef{2:83})

“We made you into nations and tribes so that you may get to know one other.
Indeed the best among you, in the sight of God, is the best in conduct.”
(\QRef{49:13})
\end{quote}

I began to cry.
I felt silly and I tried to hide my tears from Aunty Ola.
But I couldn’t stop.
I felt a strange force deep within.
I was certain it was the gentle and loving presence of God speaking to me.
It was as though a veil had been lifted and I had discovered the truth
I’d been searching for ever since I was a child.
It was a deeply spiritual and emotional time for me.
I can barely recognise the \Quran\ as quoted by the media today,
with its harsh and severe passages.
I never saw such verses at the time.
It’s not that they weren’t there,
but they never spoke to me in a literal way.

I did make a trip to the American University,
but the head of the Egyptology Department wasn’t in at the time,
and I didn’t go back.
I’d lost interest.
I spent most of my two weeks in Egypt just reading the \Quran,
with the occasional trip to meet other members
of my newly discovered extended family.
There, also, the conversations invariably turned to religion.

\begin{quote}
“A friend of mine says that only by believing in Jesus can I be saved,
because he died for our sins.”

“Islam says the opposite,” said Magdi.

“The \Quran\ says:

\emph{‘Whosoever follows the right path, benefits his own soul
and whosoever goes astray harms himself.
No soul shall bear the burden of another.’}
(\QRef{17:15})”

“Islam is the religion of our ‘Fitrah’ (inborn disposition);
it is in complete harmony with our natural instinct.”

“Then why can’t everyone see it?”

“The prophet said;
\emph{“Men are asleep and only when they die they awake.”}
That’s the nature of this world, Hassan.
If everything was clear and easy, then there would be no test.”
\end{quote}

Magdi gave me a book of Hadith (sayings of Prophet Muhammad)
which I read from cover to cover.
One hadith in particular touched me deeply:

\begin{quote}
“(God says) I am as my servant thinks of me.
I am with him when he remembers me.
If he comes to me a hand’s span, I come to him an arm’s length.
If he comes to me one arm’s length,
I draw near to him by two outstretched arms.
If he comes to me walking, I come to him running.” (Bukhari)
\end{quote}

When it came time to return to England I didn’t want to leave
and vowed I’d be back soon.
It had been an amazing experience and I felt a sense of excitement,
as though a magical door had been opened.
I had finally found God.
I came back to England full of zeal and determination
to learn more about my new-found faith.
Diane was shocked by my sudden conversion
and felt sure it was just a passing fad.
She humoured me, hoping I’d come to my senses.
But all I could talk about was Islam.
I insisted we stop sleeping together,
explaining that Islam forbids sex before marriage.
I stopped drinking and smoking, and I began to pray regularly.
I lectured Diane about the Day of Judgment and Heaven and Hell
and the sayings of Prophet Muhammad,
trying desperately to convince her of Islam.
Eventually we both realized we were wasting our time.
Diane was not interested in becoming a Muslim
and I wasn’t going through a fad.
Our relationship was over.
I moved out of our flat, leaving behind, amongst other things,
my record collection and paintings.
I had no need for idolatrous distractions from the path of Allah.


%Chapter 2:
\chapter{The Path of Allah}
%Chapter 3:
\chapter{The Sufi Path}
%Chapter 4:
\chapter{Growth of Activism}
%Chapter 5:
\chapter{Islamia School}
%Chapter 6:
\chapter{Sheikh Faisal}
%Chapter 7:
\chapter{Tuesday Afternoon}
%Chapter 8:
\chapter{Revelation \& Reason}
%Chapter 9:
\chapter{London Bombings}
%Chapter 10:
\chapter{Reforming the Unreformable}
%Chapter 11:
\chapter{Religion}
%Chapter 12:
\chapter{Pandora’s Box}


\backmatter

\chapter{Index}
TODO:

\end{document}
