%!TEX TS-program = xelatex
%!TEX encoding = UTF-8 Unicode
\documentclass[12pt]{memoir}
\usepackage[
  xetex,bookmarksnumbered=true,pdfborder=0,hyperfootnotes=true
]{hyperref}
\usepackage{xltxtra}
\usepackage{bidi}
\usepackage{xunicode}

\usepackage{fontspec}
\usepackage{color}
\usepackage[UKenglish]{babel}

\setmainfont{DejaVu Serif}

\hypersetup{
  pdfauthor = {Hassan},
  pdftitle = {Autobiography by Hassan},
  pdfsubject = {},
  pdfkeywords = {Islam Religion Allah God Faith Scepticism Life},
}

\definecolor{darkblue}{rgb}{0,0,0.62}

% Transliterated "hamza".
\def\´{ʾ} % ˀ
\let \hamza=\´ % Needed in \footnotetext.

% Transliterated "`ayin".
\def\`{ʿ} % ˁ
\let \ayin=\` % Needed in \footnotetext.

% Aliases for subscript and superscript:
\let \Sub=\textsubscript
\let \Sup=\textsuperscript

% Hyphenation rules:
% \hyphenation{}
\hyphenation{}

% Corrective comments.
\newcommand{\cmt}[2]{#1} % #1 is the actual text. #2 is the comment.

% Use Scheherazade for Arabic. Use \ar{...} for inline Arabic text.
\newfontfamily{\arabicfont}[Script=Arabic,Scale=1.7]{Scheherazade}
\newcommand{\ar}[1]{\RL{\arabicfont#1}}

% Macros for the word Qur'an.
\def \Quran{Qur\-\´ān} % Used when followed by a punctuation mark.

% Breakable forward slash.
\def\/{\discretionary{/}{}{/}}

% For hyphenation: Trick TeX into thinking al\–XYZ are two separate words.
\def\–{-\hskip0pt}
\def\al{al-\hskip0pt} % Version with "al-".

% Prints symbols for dividing paragraphs.
\def\pardivider{\centerline{***}} % \ar{۞۞۞} ❧❦

% Qur'an reference numbers. TODO: link to web page.
\newcommand{\QRef}[1]{{\color{darkblue}#1}}

% "Nota bene" macro.
\newcommand{\NB}[1]{\emph{\small NB: #1}}

% Paragraph spacing:
\setlength{\parskip}{1ex plus 0.5ex minus 0.2ex}

\pagestyle{plain}

\title{Autobiography}
\author{\emph{by Hassan}}

%%%%%%%%%%%%%%%%%%%%%%%%%%%%%%% Document begin %%%%%%%%%%%%%%%%%%%%%%%%%%%%%%%%
\begin{document}
\frontmatter

% Title:
\maketitle
\thispagestyle{empty}
\cleardoublepage

% Table of Contents:
\setcounter{page}{1}
\tableofcontents

% Preface:
\chapter{Preface}

% \picture{Tunis_Hassan.jpg}

My father was Egyptian and my mother English and I have 8 brothers and sisters.
I was born Muslim but didn’t start practicing until I was twenty
when I became very devout and committed.
For the next twenty eight years Islam guided every aspect of my life.
I completed a BA in Arabic and Islamic Studies at the School of Oriental
and African Studies where I became President of the Islamic Society.
After leaving university I became Amir of a Da\`wah group
in North London and edited an Islamic magazine called ‘The Clarion’.
I wrote four books for Muslim children and spent fifteen years as a teacher
at Islamia School, the one founded by Yusuf Islam (Cat Stevens).
But shortly before my 48th birthday I knew I no longer believed in Islam.
If you had told me a few years ago that such a thing would happen to me,
I would never have believed it and added that no ‘true’ Muslim
would ever reject Islam after having tasted the sweetness of Iman (faith).
No one who has immersed himself in the beauty of the \Quran\
and appreciated its wisdom could ever deny that it is the word of God.
If anyone had claimed such a thing I would have instinctively
doubted him and been suspicious of his motives.
But it is amazing how perceptions can change
and things I once thought unimaginable now seem perfectly reasonable.
Of course this change didn’t happen overnight.
It began a few years ago when I started to question the beliefs
I had for so long taken for granted
and started to look at Islam in a new light.

Little by little doubts began creeping in.
At first I tried to suppress them
and reacted to criticism of Islam with denial, anger and blame.
I denied there was anything wrong, felt hyper-sensitive
to criticism and blamed the West for provoking and creating problems.
When I did eventually accept that Muslims had to take responsibility
for the problems we faced,
I still couldn’t accept that Islam itself was to blame.
It was the way Islam was being interpreted that was the problem.
I started arguing for a reinterpretation and reform of traditional views,
but instead of easing my conscience,
this only highlighted the futility and dishonesty of such views.
Finally I tried to tell myself that although my rational mind
found it difficult to believe certain things in Islam,
there must be explanations beyond my capacity to understand
and that ‘God knows best’.
The safest and wisest option was to “Hold fast to the rope of Allah”.
I reckoned I had nothing to lose and everything to gain by remaining a believer
and so I went through the motions of being a ‘good’ Muslim,
in the hope that my faith would return.
But this pretence only made me depressed and lose all motivation.
The problem is that one cannot choose to believe.
Either one does or not and if there is a God,
the last thing he would have wanted me to do
was to pretend to believe in something that I didn’t.
It was quite a relief when I finally admitted to myself
that I no longer believed in Islam.

However the fact that I no longer believe in Islam
doesn’t mean I have suddenly turned into a hater of Islam.
I know that Islam brings a great deal of guidance,
comfort and worthy values to the lives of millions of people.
I know that most Muslims are good and decent people.
How could I possibly hate Muslims when my family is Muslim?
When I speak to my older children about what I think,
I tell them they must find out for themselves what they believe
and if they feel happy being Muslims then that is what they should be.
I certainly don’t feel the need to pass on my own beliefs
concerning God and religion to them –
something I felt it was my duty to do when I was a Muslim.

While I do not believe in telling anyone what they should believe,
I do think one should have the courage to honestly examine
the beliefs that are central to one’s life and guide one’s actions.
If one is truly satisfied with them,
then they should be fully embraced with one’s heart and mind,
but if they do not stand up to close scrutiny,
then they should be discarded.
Life is too short to allow it to be dictated
by beliefs one does not truly believe.

\hfill Copyright © 2008 - 2010 by Hassan

\mainmatter

% Chapters:

%Chapter 1:
\chapter{Have you read the \Quran?}
%Chapter 2:
\chapter{The Path of Allah}
%Chapter 3:
\chapter{The Sufi Path}
%Chapter 4:
\chapter{Growth of Activism}
%Chapter 5:
\chapter{Islamia School}
%Chapter 6:
\chapter{Sheikh Faisal}
%Chapter 7:
\chapter{Tuesday Afternoon}
%Chapter 8:
\chapter{Revelation \& Reason}
%Chapter 9:
\chapter{London Bombings}
%Chapter 10:
\chapter{Reforming the Unreformable}
%Chapter 11:
\chapter{Religion}
%Chapter 12:
\chapter{Pandora’s Box}


\backmatter

\chapter{Index}
TODO:

\end{document}
