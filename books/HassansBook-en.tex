%!TEX TS-program = xelatex
%!TEX encoding = UTF-8 Unicode
\documentclass[12pt]{memoir}
\usepackage[
  xetex,bookmarksnumbered=true,pdfborder=0,hyperfootnotes=true
]{hyperref}
\usepackage{xltxtra}
\usepackage{polyglossia}
\usepackage{bidi}
\usepackage{xunicode}
\usepackage{graphicx}

\usepackage{fontspec}
\usepackage{color}

\setdefaultlanguage[variant=british]{english}
\setmainfont{DejaVu Serif}

\hypersetup{
  pdfauthor = {Hassan},
  pdftitle = {Autobiography by Hassan},
  pdfsubject = {},
  pdfkeywords = {Islam Religion Allah God Faith Scepticism Life},
}

\definecolor{darkblue}{rgb}{0,0,0.5}

% Transliterated "hamza".
\def\´{ʾ} % ˀ
\let \hamza=\´ % Needed in \footnotetext.

% Transliterated "`ayin".
\def\`{ʿ} % ˁ
\let \ayin=\` % Needed in \footnotetext.

% Aliases for subscript and superscript:
\let \Sub=\textsubscript
\let \Sup=\textsuperscript

% Hyphenation rules:
% \hyphenation{}
\hyphenation{Mau-la-na Za-kar-ya Kan-dha-la-wi}
\hyphenation{Shu-\`ay-bee}

% Corrective comments.
\newcommand{\cmt}[2]{#1} % #2 comments on #1.
\newcommand{\cor}[2]{#2} % #2 corrects #1.

% Use Scheherazade for Arabic. Use \ar{...} for inline Arabic text.
\newfontfamily{\arabicfont}[Script=Arabic,Scale=1.7]{Scheherazade}
\newcommand{\ar}[1]{\RL{\arabicfont#1}}

% Macros for the word Qur'an.
\def \Quran{Qur\-\´ān} % Used when followed by a punctuation mark.

% Breakable forward slash.
\def\/{\discretionary{/}{}{/}}

% For hyphenation: Trick TeX into thinking al\–XYZ are two separate words.
\def\–{-\hskip0pt}
\def\al{al-\hskip0pt} % Version with "al-".

% Prints symbols for dividing paragraphs.
\def\pardivider{\centerline{***}} % \ar{۞۞۞} ❧❦

% Qur'an reference numbers. TODO: link to web page.
\newcommand{\QRef}[1]{{\color{darkblue}#1}}

% "Nota bene" macro.
\newcommand{\NB}[1]{\emph{\small NB: #1}}

% Image macro. #1 Optional params. #2 File name. #3 Optional caption.
\newcommand{\img}[3]{\begin{center}%
\includegraphics[#1]{#2}\\{\small\em#3}%
\end{center}}
\graphicspath{{books/imgs/}}

% Paragraph spacing:
\setlength{\parskip}{1ex plus 0.5ex minus 0.2ex}

\pagestyle{plain}

\title{An Autobiography}
\author{\emph{by Hassan}}

%%%%%%%%%%%%%%%%%%%%%%%%%%%%%%% Document begin %%%%%%%%%%%%%%%%%%%%%%%%%%%%%%%%
\begin{document}
\frontmatter

% Title:
\maketitle
\thispagestyle{empty}
\cleardoublepage

% Table of Contents:
\setcounter{page}{1}
\tableofcontents

% Preface:
\chapter{Preface}

\img{scale=0.1}{Tunis_Hassan.jpg}{}

My father was Egyptian and my mother English and I have 8 brothers and sisters.
I was born Muslim but didn’t start practicing until I was twenty
when I became very devout and committed.
For the next twenty eight years Islam guided every aspect of my life.
I completed a BA in Arabic and Islamic Studies at the School of Oriental
and African Studies where I became President of the Islamic Society.
After leaving university I became Amir of a Da\`wah group
in North London and edited an Islamic magazine called ‘The Clarion’.
I wrote four books for Muslim children and spent fifteen years as a teacher
at Islamia School, the one founded by Yusuf Islam (Cat Stevens).
But shortly before my 48\Sup{th} birthday I knew I no longer believed in Islam.
If you had told me a few years ago that such a thing would happen to me,
I would never have believed it and added that no ‘true’ Muslim
would ever reject Islam after having tasted the sweetness of Iman (faith).
No one who has immersed himself in the beauty of the \Quran\
and appreciated its wisdom could ever deny that it is the word of God.
If anyone had claimed such a thing I would have instinctively
doubted him and been suspicious of his motives.
But it is amazing how perceptions can change
and things I once thought unimaginable now seem perfectly reasonable.
Of course this change didn’t happen overnight.
It began a few years ago when I started to question the beliefs
I had for so long taken for granted
and started to look at Islam in a new light.

Little by little doubts began creeping in.
At first I tried to suppress them
and reacted to criticism of Islam with denial, anger and blame.
I denied there was anything wrong, felt hyper-sensitive
to criticism and blamed the West for provoking and creating problems.
When I did eventually accept that Muslims had to take responsibility
for the problems we faced,
I still couldn’t accept that Islam itself was to blame.
It was the way Islam was being interpreted that was the problem.
I started arguing for a reinterpretation and reform of traditional views,
but instead of easing my conscience,
this only highlighted the futility and dishonesty of such views.
Finally I tried to tell myself that although my rational mind
found it difficult to believe certain things in Islam,
there must be explanations beyond my capacity to understand
and that ‘God knows best’.
The safest and wisest option was to “Hold fast to the rope of Allah”.
I reckoned I had nothing to lose and everything to gain by remaining a believer
and so I went through the motions of being a ‘good’ Muslim,
in the hope that my faith would return.
But this pretence only made me depressed and lose all motivation.
The problem is that one cannot choose to believe.
Either one does or not and if there is a God,
the last thing he would have wanted me to do
was to pretend to believe in something that I didn’t.
It was quite a relief when I finally admitted to myself
that I no longer believed in Islam.

However the fact that I no longer believe in Islam
doesn’t mean I have suddenly turned into a hater of Islam.
I know that Islam brings a great deal of guidance,
comfort and worthy values to the lives of millions of people.
I know that most Muslims are good and decent people.
How could I possibly hate Muslims when my family is Muslim?
When I speak to my older children about what I think,
I tell them they must find out for themselves what they believe
and if they feel happy being Muslims then that is what they should be.
I certainly don’t feel the need to pass on my own beliefs
concerning God and religion to them —
something I felt it was my duty to do when I was a Muslim.

While I do not believe in telling anyone what they should believe,
I do think one should have the courage to honestly examine
the beliefs that are central to one’s life and guide one’s actions.
If one is truly satisfied with them,
then they should be fully embraced with one’s heart and mind,
but if they do not stand up to close scrutiny,
then they should be discarded.
Life is too short to allow it to be dictated
by beliefs one does not truly believe.

\hfill Copyright © 2008–2010 by Hassan

\mainmatter

% Chapters:

%Chapter 1:
\chapter{Have you read the \Quran?}

\img{scale=0.7}{Baby_Hassan.jpg}{}

Every night, before going to bed, we recited \emph{al\–Fatiha},
the short first chapter of the \Quran.
I felt God was watching over me.
My father, or more usually, my mother, would stand at the door,
while my two brothers and I folded our arms and sat up in bed.
We recited it at lightening speed in Arabic and English,
then heads down, lights out.
Sometimes I also added my special prayer quietly to myself;
\emph{“Please God, make everything alright”}
I was too tired to go through the details.
God understood.

The only times we had to say the formal daily prayers was on special occasions,
like a visit from Granddad,
when we had to make a big show of religious piety for his benefit.
After praying, my eldest sisters, Kamelia and Suzy showed me
how to make Du\`a (supplication) to ask God for anything I wanted.

\begin{quote}
“Hold your hands out and don’t leave any gaps!” Kamelia said.

“Why do I have to put my hands like that?” I asked.

“So that you can catch the presents God is sending you!” Kamelia answered.
\end{quote}

I accepted that reply without question
even though I could not see any presents.
I was becoming familiar with God’s way of doing things.
If I asked God for a bicycle, I knew it would not come immediately
and perhaps may never come at all.
It wasn’t God’s fault, but because I had been naughty in some way
and didn’t deserve it.
When I wanted something very badly I would try very hard
to be good for as long as possible.
I did that when I asked God to make mummy and daddy stop fighting.
When they didn’t stop I knew it was because I wasn’t good enough.

\img{scale=1}{Hassan_Boy.jpg}{}

As I got older, Islam receded further and further into the background
and by the age of twelve, during the long hot summer of 1971,
my only concern was to be like Marc Bolan.
With my curly hair, velvet jacket and platform boots,
I thought I looked like him too.
A dab of glitter on my cheeks and I was away,
pouting and prancing in front of the mirror to ‘Hot Love’ and ‘Get it On’.
I drew pictures of him and Mickey Finn
and stuck them on the wall of the bedroom I shared with my two brothers.
Our bedroom became so covered in drawings and pictures of men in make-up
and tight-fitting leotards from ‘Oh Boy!’ ‘Pop Swap’ and ‘Jackie’
that my father was convinced we were becoming queer.

\begin{quote}
“What on earth are all these homosexuals doing on your walls?”

“They’re not homosexuals.”

“Of course they are; just look at his eyes,”
he said, pointing to a picture of Dave Hill, the guitarist from Slade.
\end{quote}

I studied his eyes carefully, trying to determine exactly what it was
that transmitted “HOMOSEXUAL” so plainly to my dad.

\begin{quote}
“They all wear glitter; it’s just for show!”

“I want you to take them down. Now!”
\end{quote}

We did as we were told.
One didn’t argue with my father.

\img{scale=0.8}{Hassans_Father.jpg}
{My father, Aziz (far right) during the war against Israel 1948.}

Born to a wealthy Egyptian family,
my father rarely talked about the country
and culture he came from and wasn’t very religious.
For a long time I never knew he was from Egypt.
When he first came to England he called himself \emph{Jean Pierre}
and told everyone he was French.
My mother, Mary Magson, was the daughter of Horace Magson,
an accountant from Darlington.
Her family were Methodists, but like my father, she wasn’t religious.
The two of them met in Paris in 1951.
He was studying at the University of Lausanne,
while she was in the ATS where she’d driven ambulances during the War.
After getting married they went to live in Egypt,
but as a result of the 1952 revolution and the Suez Crisis,
the British Government evacuated them
along with other Anglo-Egyptian families.
They were resettled in an old Tudor cottage in the middle
of the Suffolk village of Coddenham,
where I was born on the 12\Sup{th} of May, 1959,
fifth child in a family of eight.

\img{scale=0.7}{Hassans_Mother.jpg}
{My mother, Mary Magson (standing) in the ATS during WW2.}

We moved to Finchley in North London when I was 6 years old
and that is where I spent most of my childhood and teenage years.
During the 1960s it was a very white middle-class area
and I soon became aware that we didn’t quite fit in.
I hated telling someone my name.
It was always the same reaction,
an embarrassing silence and a confused or mocking expression.
My class teacher once made a list of religious affiliations.
When he called my name out I hesitated for a long time,
which of course only attracted more attention.
Eventually I whispered Muslim.

\begin{quote}
“What?” said my teacher.

“Muslim, sir.”
\end{quote}

Some children sniggered.
One boy insisted that ‘Muslim’ wasn’t a real religion at all
and I had made it up.
My siblings and I became a target for bigots and bullies.
This wasn’t helped by my father’s insistence
that we were to be withdrawn from school assemblies.
I don’t really know why he wanted to withdraw us.
I could better understand it if he had been religious,
but the only effect it had was to confirm to everyone
that we were ‘different’ and should be treated as such.
Instead of listening to the parable of ‘The Little Red Hen’
or singing ‘All Things Bright and Beautiful’,
which my father thought would infect my mind with Christianity,
I was exposed to the delights of a pornographic novel
which the older boy next to me was reading
as we sat in the room assigned for those who opted out of assembly.

My father was full of paradoxes and contradictions.
In some ways he was very liberal-minded and in other ways ultra\–conservative.
His attitude towards women was very old fashioned
and he expected my mother to stay at home, cooking,
cleaning and looking after his children
and he became violent if he felt she wasn’t obeying his every command.
After suffering for many years my mother eventually sued for divorce.
The court battle over custody of the children was long and bitter.

\img{scale=0.3}{Hassan_and_Diane.jpg}
{Me \& Diane, 1978}

At the age of 18 I started a degree in Sociology
at the Polytechnic of North London,
a hotbed of Marxists and Trotskyite student militants.
I joined the Broad Left, one of the two main Socialist parties
at North London Polytechnic and was appointed cartoonist
for the college magazine, “Fuse”.
Our main rival at the Polytechnic were The Socialist Workers Party
and I drew many scathing caricatures of the leadership,
lampooning the ridiculous things they said,
their lack of dress sense and poor personal hygiene.
My cartoons were unanimously praised until I drew one
criticising my own party — the Broad Left.
The Editor took such offence that I was sacked on the spot.
I had contravened the most basic rule of politics — stick to the party line.
But I had never been one for sticking to any party line
and my experience with left-wing student politics
only served to alienate me from it.
I became far more interested in how to ask Diane out on a date.
Diane was a first year sociology student,
like myself and we had become fairly good friends almost immediately.
She had a bubbly and cheerful character that was easy to get on with.
She loved football, enjoyed a good debate
and we shared a similar taste in music.

After a few weeks of my shy and bumbling attempts
to say something other than “Hi”, she asked me out.

\begin{quote}
“Fancy going to the Stapleton tonight? They’ve got a live band playing.”

“Yeah! What time?”

“Come over about eight.”
\end{quote}

I was there at seven, in my best ripped jeans and smelling of Patchouli.

It was only a trip to the pub, but it was the start of my first relationship.
We became constant companions and started sharing a flat together.

At the end of the first year we’d grown tired of boring lectures
on Auguste Comte, Max Weber and Karl Marx.
Both of us had been hoping that studying Sociology
would be a creative experience,
perhaps providing clues to the great questions of life.
We were disappointed to find it was merely an exercise
in regurgitating a selection of tedious books.
We decided to drop-out and hitch-hike to Wales.
We didn’t have much of a plan beyond that.
After several days of picking psychedelic mushrooms in Aberystwyth,
we made our way to her parent’s house in Warrington to decide what to do next.

Diane had once worked as a nurse in Calderstones Hospital
for the physically \& mentally handicapped
and she suggested we both get jobs there.
Calderstones was set in a picturesque Lancashire village at the foot of,
Pendle Hill, infamous for its association
with burning witches during the Middle Ages.
I thought it was a terrific idea and we rented a small caravan
in a farmer’s field along the ridge of Pendle Hill itself.
The caravan could only be reached by walking up a very long and steep road
that began in the village and wound up into the mists of Pendle Hill.
The caravan was in an isolated corner,
surrounded by manure covered boggy fields,
where the wind and the rain never ceased.
I loved it, but Diane was not so keen,
particularly after the night I saw a ghost.
I was woken in the middle of the night
by the sound of the caravan door banging.
I looked up from my sleepy stupor towards the entrance
and saw what looked like a young girl — perhaps in her early teens —
standing at the foot of my bed.
She looked as if she was wearing rags and had a sad smile on her face.
She leant her head slightly onto the partition wall of the caravan
and stared at me.
I thought perhaps it was the farmer’s daughter,
and I couldn’t understand why she would be out at this late hour,
in freezing winter rain.
Before I could lift my head up and say something a man in a three cornered hat
appeared from behind her and ran through the caravan,
past me and disappeared out the other side.
I jumped up now and was calling out for Diane to wake up.
The girl was gone and everything was quiet, apart from the wind.
Diane was understandably freaked out and although I re-assured her
that it was just a dream triggered by the local stories of witch burnings
we’d heard along with the large amount of dope and alcohol we’d consumed,
she didn’t want to stay in the caravan anymore.
We moved out later and into a little flat in Blackburn.

It was now 1979, and I had close group of friends and a good social life.
But under the surface I felt restless and unsure of my identity.
This was highlighted now and again by the negative reaction of a few bigots,
who made it clear that someone with a funny name and foreign religion,
even though I wasn’t practicing, would never be accepted as English.
A Charge Nurse, on a ward I worked on, took to calling me “Dirty Arab!”
She started saying it from the moment she discovered my name.
She wrote on my report that I was “Dirty, lazy and inefficient”
something I only discovered much later
when a sympathetic charge nurse showed me the report.
None of what she said was true.
She didn’t like me simply because my name was Hassan.
There was no other reason.
The irony was that, up until that point,
I had never considered myself to be Arab.

Then a series of events caused me to re-think my views
about the culture and religion I had tried hard to avoid.
The first was the Islamic revolution in Iran.
I remember watching the television coverage of running battles
on the streets of Tehran, between the ordinary people
and the heavily armed guard of the Shah.
I found the images inspiring:
defiant civilians rising up against the might of a tyrant.
The revolutionary spirit of my student days had remained with me
and I instinctively saw it as a people’s struggle against a powerful elite,
but I was also aware the role religion was playing —
a religion that I felt some connection to — albeit a weak one.

I was confronted with another example of the power of religion
when my close friend John Shackleton returned from a camping trip
to announce he was now a born-again Christian.
It was an enormous shock, as he had always been so scornful of religion.
Now he refused to come down to the pub with me or listen to music —
apart from music about Jesus.
He became annoyingly exultant about his faith
and constantly attempted to convert me and Diane.

\begin{quote}
“Jesus loves you and wants to forgive you,” he said.

“But I haven’t done anything,” I said.

“We are all sinners. Jesus can restore you to how God wanted you to be.”

“Why? Did God make a mistake the first time round?”
\end{quote}

The constant onslaught of religious fervour forced me
to take a position on Christianity and religion in general,
something I had not given a lot of thought to before.
The more John explained things such as the Trinity, Original Sin,
and Atonement, the more I knew these were concepts I could never believe in.
The idea that God is “Three persons in One” and that man is born sinful
because of the sin of another or that sins can be forgiven,
not because of any meritorious act by the person themselves,
but because someone else was nailed to a plank of wood,
all conflicted with reason and my sense of justice.
John joined a small community of born-again Christians a few miles away,
but continued to visit us, praying we would be filled with the Holy Spirit.

I was unmoved by the idea of a Holy Spirit or any other supernatural power.
I had always believed that this world was ‘what you see is what you get’
and had certainly never experienced anything mystical myself.
But in keeping with the religious theme of this chain of events,
something mystical was about to happen to me.
I had been camping at the Deeply Vale music festival in Lancashire with Diane,
when we went for a walk up one side of the valley.
As we stood at the top, I heard a sound I never expected to hear.

\begin{quote}
“Can you hear that?” I asked.

“What? I can’t hear anything.”

“It’s the Muslim call to prayer… Listen!” I said.
\end{quote}

I could clearly hear the words
\emph{“Ash-hadu an la ilaha illah, Ass-hadu anna Muhammad rasoolalllah!”}
(I bear witness there is no god but God, I bear witness
that Muhammad is the prophet of God!)
They resonated with a strange clarity above the hubbub of the festival below.
Diane didn’t hear anything.
I couldn’t understand why.
Nor could I understand why someone was reciting
the Muslim call to prayer at a music festival.
I tried to think of a rational explanation,
but couldn’t come up with one that convinced me.
I began to feel that perhaps supernatural things did happen
and that this may be my own personal wake-up call to faith.

A few months later I saw Cat Steven’s on television,
giving a farewell concert.
He had become a Muslim and was resigning from the music business,
turning his back on fame and money
in order to devote his life to his new-found faith.
The public were mystified. Why would someone
who had it all want to throw it away for religion?
Religion wasn’t for cool people; it was for simple-minded, insecure people.
Part of me also had the same reaction.
But part of me felt inspired and proud.
If such a creative and respected person like Cat Stevens
saw something worthwhile in Islam, perhaps I,
as someone born with an Islamic heritage,
should take it more seriously myself.

The final episode in this series of events was
when my father turned up unexpectedly on my doorstep
to invite me to go with him to Egypt for a couple of weeks.
I had not seen a great deal of my father since leaving his house.
He had now moved to the village of Ramsbottom,
where he had been appointed Head Teacher of a residential school.
As I was living fairly near by, I took Diane to meet him.
He took an instant dislike to Diane.
She was an independent, strong-minded young woman
who wasn’t afraid of saying what she thought.
I liked that about her, but my father did not.
He didn’t like women who answered back;
it was against the natural order of things.
Women were supposed to do what men told them —
that’s the way my father believed God created them
and that’s the way my father wanted it to stay.
He set about trying to cause trouble between us
in his usual way of oblique remarks, insinuations, snide jokes,
claiming she was rude to him,
and even accusing her of stealing from his house.
I can’t be certain of whether his offer of a trip
to Egypt was part of that effort.
He had important business there regarding his father’s inheritance.
But I suspect he also thought it a wonderful opportunity
to get me away from Diane.
I told myself it wouldn’t affect our relationship and that
an all-expenses-paid holiday to Egypt was an offer I couldn’t refuse.
I think subconsciously I was also feeling bored and restless.
The area Diane and I had moved into on a newly built estate
on the outskirts of Blackburn was dull and soulless.
Our life had become full of routines and our relationship slipped into a rut.
I was eager for a change of scenery — at least for two weeks.

Before driving to the airport,
my father stopped by my sister Suzy’s house in London to say goodbye.

\begin{quote}
“What are you planning to do in Egypt, Hassan?” she asked.

“I was thinking about exploring the possibilities
of studying Egyptology at the American University in Cairo.”

“Insha-Allah,” (God Willing) prompted Suzy.

“I don’t see why I should say that, Suzy,” I replied curtly.
“Either I will study Egyptology, or I won’t.
What has God got to do with it?”

“Nothing happens unless God wills it.”

“Does he will murderers to kill people?”

“He allows it to happen because he gives us free will,” replied Suzy.

“Well then, I don’t see why God should interfere with me studying Egyptology?”

“If you say so Hassan,”
Suzy replied, wearily.
I thought about what I had just said as we drove to the airport;
it was arrogant and I regretted saying it.
\end{quote}

It was a long flight with a two hour stop-over
in a bleak East European city that seemed to be full of nothing
but grey concrete rectangles and policemen with batons.
I was relieved when we finally landed in Egypt.
The moment I stepped off the plane and into the hot,
moist atmosphere of Cairo airport,
I realized I was in another universe.
The scent of incense drifted through an intricate lattice window.
Donkeys laden with vegetables weaved their way
through an orchestra of blaring car horns;
street merchants announced their wares with a siren cry that made me jump;
men prayed on the pavement, wearing pyjamas;
and women threw buckets of peelings from balconies above.
It was a mad, chaotic patchwork quilt of smells,
noise and colour and came as quite a culture shock to me.
But despite its strangeness I soon felt at home.
For the first time I didn’t have to hide or be embarrassed about my origins.
Moreover, everyone admired and respected both halves of my cultural background.

\img{scale=0.3}{Hassan_Cousins.jpg}
{Me \& two cousins in Egypt, 1981}

We stayed at my uncle’s house in Cairo.
The Egyptians wore western clothes,
watched dubbed Hollywood movies
and had many of the modern conveniences found in England.
But as I sat on the replica 18\Sup{th} Century French furniture,
a loudspeaker in the street outside began bellowing the call to prayer.
This triggered a wave of prayer calls
that slowly unfurled across the Cairo rooftops and into the distance.
Even on television, Clark Gable was cut off in mid flow,
as a sign came up in Arabic, announcing the evening prayer.
When everyone got up to pray,
I was left sitting alone at the table,
I felt a little uncomfortable.

After prayer my cousin Nihal,
who had been helping my aunty fry some food in the kitchen,
came in carrying a steaming dish.

\begin{quote}
“You like beetles?”

“Er… I’ve never had them!” I said, feeling a little queasy.

“I like them too much!”
She put the plate on the table.
“Especially I like Paul; he’s too cute!”

“Oh…” I said with a sigh of relief,
“Yeah I like them, but they split up a few years ago, you know!”
Egyptians loved everything British and knew a great deal about the UK,
though their information seemed to be about a decade old.

“Split up?”

“They don’t play together anymore.”

“Oh? Why?”

“Well bands do that after a while…”

“Georgie Best!” interrupted Hamdy, giving me the thumbs up.
“Manchester United! Good.”

“Well I support Spurs actually”

“Sopurs? What is Sopurs?”

“Tottenham Hotspur — they’re a football team.”
\end{quote}

Nihal showed me a picture in an Egyptian newspaper
of Ayatollah Khomeini hugging a little girl.

\begin{quote}
“Awww, he is such a good man!”

“The people seem to love him.”

“He says there is no difference between Sunni and Shi\`ah.
He says we are all Muslims and should be united.”
\end{quote}

Nihal was a very strong minded,
independent woman who took her freedom to do as she wanted for granted.
She didn’t wear a headscarf and had very western habits and tastes.
Yet she seemed completely comfortable about identifying with traditional,
orthodox views — something most Egyptians I met
seemed totally at ease with despite being relatively westernised.

\begin{quote}
“Eat, Hassan!” said Aunty Ola as she sat next to me.
“We made you English food: Fish and Chips!”

“Do you say your prayers, Hassan?” said my uncle.

“To be honest, no, I don’t.”

“Oh you must pray!
Prophet Muhammad said that ‘Prayer is the key to Paradise’.”

“I’m not sure I really believe in all that.
I mean why does God need us to pray?”

“God doesn’t need us to pray.
But we need to pray.
To give thanks and seek His help.”

“I still don’t see why we have to give thanks or ask for help through prayer.”

“Have you read the \Quran, Hassan?”

“A bit.”
\end{quote}

Uncle Fouad took a book from the shelf.

\begin{quote}
“Here’s an English translation for you.
I want you to promise me you will read it.”
\end{quote}

I was reluctant to promise something I didn’t want to do,
but as I was a guest in his house I could hardly refuse.
I thought I could read a few pages then politely put it to one side.

\begin{quote}
“Thanks. OK, I will.”

“Insha-Allah,” prompted Uncle Fouad.

“Insha-Allah,” I replied.
\end{quote}

The next day my father and uncle had gone out,
leaving me at home with Aunty Ola.
So I picked up the \Quran, as promised, and began to read.
To my surprise I found I couldn’t put it down.
The \Quran\ is not like any ordinary book.
It doesn’t follow any of the conventions of standard prose.
It has no definite beginning or end.
There is no plot to follow and no neat resolution.
It seems to jump rather abruptly from one account to another.
Even its style changes with little warning,
from a steady narrative to fast paced rhyming prose.
Yet I found it strangely irresistible.

\begin{quote}
“Alif Lam Mim.”
\end{quote}

I looked up at Aunty Ola who was quietly sitting smoking a cigarette
as she read a magazine full of beautiful women strutting down a cat walk.

\begin{quote}
“What does Alif Lam Mim mean?”

“Nobody knows.” She smiled.
“Some chapters of the \Quran\ begin with letters of the alphabet.
Scholars have tried to explain them.
But nobody knows for sure.”

“You mean it’s a mystery?”

“Yes.”
\end{quote}

I liked mysteries.

\begin{quote}
\itshape
“God is the Light of the heavens and the earth.
The Parable of His Light is as if there were a Niche and within it a Lamp:
the Lamp enclosed in Glass: the glass like a brilliant star:
Lit from a blessed Tree, an Olive, neither of the East nor of the West,
whose oil is well-nigh luminous,
though fire scarce touched it: Light upon Light!” (\QRef{24:35})

“When my servants ask you about me, (say)
I am indeed close and answer the prayer of the one who calls on me.”
(\QRef{2:186})

“We created man and We know what his soul whispers to him.
We are nearer to him than his jugular vein.” (\QRef{50:16})

“Wherever you turn your face, there is God’s presence.
God is all-Pervading.” (\QRef{2:115})

“Do not turn away from men with pride, nor walk arrogantly through the earth,
but be moderate in your pace and lower your voice.” (\QRef{31:18})

“Treat with kindness your parents and kindred, and orphans and those in need;
speak fair to the people; be steadfast in prayer;
and practice regular charity.” (\QRef{2:83})

“We made you into nations and tribes so that you may get to know one other.
Indeed the best among you, in the sight of God, is the best in conduct.”
(\QRef{49:13})
\end{quote}

I began to cry.
I felt silly and I tried to hide my tears from Aunty Ola.
But I couldn’t stop.
I felt a strange force deep within.
I was certain it was the gentle and loving presence of God speaking to me.
It was as though a veil had been lifted and I had discovered the truth
I’d been searching for ever since I was a child.
It was a deeply spiritual and emotional time for me.
I can barely recognise the \Quran\ as quoted by the media today,
with its harsh and severe passages.
I never saw such verses at the time.
It’s not that they weren’t there,
but they never spoke to me in a literal way.

I did make a trip to the American University,
but the head of the Egyptology Department wasn’t in at the time,
and I didn’t go back.
I’d lost interest.
I spent most of my two weeks in Egypt just reading the \Quran,
with the occasional trip to meet other members
of my newly discovered extended family.
There, also, the conversations invariably turned to religion.

\begin{quote}
“A friend of mine says that only by believing in Jesus can I be saved,
because he died for our sins.”

“Islam says the opposite,” said Magdi.

“The \Quran\ says:

\emph{‘Whosoever follows the right path, benefits his own soul
and whosoever goes astray harms himself.
No soul shall bear the burden of another.’}
(\QRef{17:15})”

“Islam is the religion of our ‘Fitrah’ (inborn disposition);
it is in complete harmony with our natural instinct.”

“Then why can’t everyone see it?”

“The prophet said;
\emph{“Men are asleep and only when they die they awake.”}
That’s the nature of this world, Hassan.
If everything was clear and easy, then there would be no test.”
\end{quote}

Magdi gave me a book of Hadith (sayings of Prophet Muhammad)
which I read from cover to cover.
One hadith in particular touched me deeply:

\begin{quote}
“(God says) I am as my servant thinks of me.
I am with him when he remembers me.
If he comes to me a hand’s span, I come to him an arm’s length.
If he comes to me one arm’s length,
I draw near to him by two outstretched arms.
If he comes to me walking, I come to him running.” (Bukhari)
\end{quote}

When it came time to return to England I didn’t want to leave
and vowed I’d be back soon.
It had been an amazing experience and I felt a sense of excitement,
as though a magical door had been opened.
I had finally found God.
I came back to England full of zeal and determination
to learn more about my new-found faith.
Diane was shocked by my sudden conversion
and felt sure it was just a passing fad.
She humoured me, hoping I’d come to my senses.
But all I could talk about was Islam.
I insisted we stop sleeping together,
explaining that Islam forbids sex before marriage.
I stopped drinking and smoking, and I began to pray regularly.
I lectured Diane about the Day of Judgment and Heaven and Hell
and the sayings of Prophet Muhammad,
trying desperately to convince her of Islam.
Eventually we both realized we were wasting our time.
Diane was not interested in becoming a Muslim
and I wasn’t going through a fad.
Our relationship was over.
I moved out of our flat, leaving behind, amongst other things,
my record collection and paintings.
I had no need for idolatrous distractions from the path of Allah.


%Chapter 2:
\chapter{The Path of Allah}

\img{scale=1}{Hassan_1980.jpg}{}

Shortly after returning from Egypt,
I was determined to immerse myself in the study of Islam and Arabic.

The awakening of my own faith seemed to coincide
with a general Islamic awakening in the world around me
during the late 70s and early 80s.
I don’t know whether this rise in Islamic awareness was prompted by events
at the time but many second generation Muslims in the UK
were beginning to \cor{re-discover}{rediscover} the religion of their birth,
after a long period of having been Muslim only in name.
There was an air of excitement and dynamism about the Muslim community,
particularly in London.
Study circles and informal gatherings sprang up
in living rooms or community centres.
Each Friday, the display window in the foyer of Regent’s Park Mosque
seemed to get ever more packed with little cards announcing new events.
Even greeting someone with Salams after prayers usually prompted an invitation
to a Zikr (Remembrance of Allah) or a Halaqa (Islamic Discussion Circle).
When I took part in such meetings
I was struck by the diversity of those attending:
Asians, Europeans, Africans, Turks, Kurds, Arabs, Malaysians and Iranians.
They came from all walks of life: civil servants, students, bus drivers,
doctors and parking attendants.
There was no barrier of race, class or nationality.
Being a Muslim was the only thing that mattered
and it granted instant membership of the Ummah (community).
The aim of the Islamic meetings was to learn about Islam,
but equally important was the social side —
getting to know other Muslims in the area
and building up a sense of brotherhood.
Gradually these meetings spawned other meetings
catering to the needs of a particular locality, ethnic group or interest.

The first meeting I began attending in 1980 was
“The Islamic Society of the Faithful”,
held alternately in the living rooms of Brother Shafiq or Brother Azim,
both Asians from East Africa and elders of the local community.
Most of us who attended were born Muslims but had not been practising
and so there was a sense of discovery and learning.
We started going through the basics of what were the beliefs
and practises of Islam.
In those days there was none of \cor{the sectarian}{sectarianism}
or other divisions that became apparent later,
none of the dogmatic insistence on this or that point of Islamic faith.
The meetings were broad-minded and inclusive.
I also enjoyed these meetings because of the wonderful food
we had at the end of them.
Na\`eema, Shafiq’s wife, would serve us Moroccan \cor{cous cous}{couscous},
Lebanese stuffed vegetables or Turkish kofta, while Khaira, Azim’s wife,
would present us with \cor{biriyani}{biryani}, curry and samosas.
I loved the sense of belonging and identity this gave me.
It was like being part of a huge family that shared a special bond.

One thing that came out of the meetings of
“The Islamic Society of the Faithful”
and from discussions with almost every Muslim I met
was the importance of learning Arabic, the language of the \Quran.
I knew it was an essential step towards truly understanding Islam.
I also felt it was my duty to learn Arabic
since it was part of my heritage and identity.
I enrolled for a degree in Arabic and Islamic Studies
at the School of Oriental and African Studies (SOAS).
My tutor was David Cowan, the author of “Modern Literary Arabic”,
an elderly but sprightly Scotsman who converted to Islam in his youth.
Also at SOAS at the time were Dr.\ Wansbrough and Dr.\ Cook,
the Salman Rushdies of their day.
They had both recently published books that undermined
the \Quran’s claims to divine authorship,
Wansbrough asserting that the \Quran\ was a product of various sources,
improved and perfected through oral narration after Muhammad,
and Dr.\ Cook, along with Patricia Crone,
claiming that Islam began as a variant form of Judaism
and only much later developed into a separate religion.
Neither book engendered much of an outcry.
Today, such criticism of Islam would undoubtedly
spark worldwide demonstrations,
but the Muslim community in the late 70s and early 80s was not yet politicised.
The radical groups that exist today were only just getting started in the UK
and had yet to spread their influence.
As a young Muslim keen to soak up everything I could about Islam,
I found the atmosphere at SOAS invigorating,
and I attended every extra-curricular lecture and debate.
I also couldn’t stop reading anything
and everything that had Islam as its subject matter,
and the SOAS Library became my second home.
I stayed there studying until late into the night
and regularly had to be asked to leave by staff locking up.

In between my own efforts to learn more about Islam,
I was also busy spreading the word to others and, in particular,
to members of my own family who were not practicing.
I was motivated by an ardent desire to share what I had discovered
and to save them from hell-fire.
My two eldest sisters were already devout Muslims,
but the rest of my family was not.
It was my Islamic duty to give them ‘Da\`wa’,
which literally means ‘invitation’.
I had recently seen the film “The Message” about the life of Prophet Muhammad.
It begins dramatically, with three masked horsemen galloping across the desert
until they arrive at the courts of the rulers of Byzantium,
Persia and Egypt, to deliver letters inviting them to Islam.
This is the letter to Heraclius, the Byzantine Emperor:

\begin{quote}
\itshape
In the name of God, the Beneficent the Merciful.
I call you to Islam.
Accept Islam and you will be safe
and Allah will give you a double reward in this life and in the next.
If you do not, you will have to face the consequences.

Seal; Prophet Muhammad
\end{quote}

‘OK,’ I thought to myself, ‘if that’s how the Prophet did it,
then that’s how I’ll do it.’

\begin{quote}
\itshape
Dear Evette,

I call you to Islam. Accept Islam and you will be safe
and Allah will give you a double reward in this life and in the next.
If you do not, you will have to face the consequences.

Love, Hassan.
\end{quote}

I don’t know how Evette reacted to that letter, for she never said anything.
I decided to change tactics and focus on one-to-one discussions.
I started with my younger siblings.
I spent a great deal of time reading verses of the \Quran\
and discussing their meaning.

\img{scale=0.4}{Hassan_in_Regents_Park_Mosque.jpg}
{\cor{Regents}{Regent’s} Park Mosque}

One day I was visiting brother Shafiq,
when he suggested accompanying him on a “Tablighi Jamaat.”
Tablighi Jamaat was a movement that \cor{started}{was started}
in India by Muhammad Kandhalawi who wanted to bring Muslims
back to the path of pure Islam.
They adhere to the Deobandi brand of Sunni Islam that is followed
in large parts of South East Asia
and has a huge following in Pakistan and India.
During the 80s the group had become very popular in the UK.

\begin{quote}
“What’s Tablighi Jamaat?” I asked.

“It’s a gathering where they give talks about Islam. Why don’t you come?”

“Where is it?”

“Dewsbury.”

“Where’s Dewsbury?”

“Near Leeds.”

“Leeds? That’s miles away!”

“Oh, it won’t take long.
Come on — you’ll learn a lot and really benefit from it.”
\end{quote}

Before I knew what had happened I found myself
on a long road trip heading to Yorkshire.
After making a couple of stops to pick up some brothers,
we arrived in the little town of Dewsbury nestled in the Yorkshire moors,
with its quaint rows of Victorian terraced houses.
It didn’t seem at all like a place
where Muslims would be gathering to discuss Islam.
But as we turned a corner I was confronted
by a huge mosque at the end of the road.
Even more incongruous was the sight of little Asian children
in Pakistani traditional dress all along the street leading to the mosque,
playing with toys on the cobblestones,
while women in headscarves stood half emerged from doors
that opened on to the pavement, chatting away in Urdu.
If it wasn’t for the grey northern skies above us,
I might have mistaken it for downtown Karachi.

The mosque in Dewsbury was still unfinished and there were bags of cement
and sheets of plaster board stacked against breeze blocks,
while the main floor space was interrupted
by unfinished columns with wire mesh poking out.
The mosque was full and there were several lectures in progress.
Most of the lectures were in Urdu,
but I was taken to a gathering for English speakers.
The Maulana leading this gathering was sitting on a raised platform
and spoke in broken English to a mixed gathering of Arabs,
Africans, Malaysians and Asians — who I assumed did not understand Urdu.
On one side of me sat a chubby African man
in a colourful Kaftan and decorative hat.
He smiled and peered at me through his glasses
that had such thick lenses his eyes seemed to be twice their actual size.
Throughout the talk he continued to stare at me.
I glanced over at him — frowning slightly,
hoping he would see my irritation,
but he only grinned and stared even more relentlessly.

\begin{quote}
“Are you Muslim?” he finally ventured.

“Yes,” I replied in a hushed tone, trying not to disturb the talk.

“Mashallah! (Whatever God Wills!) How long have you been a Muslim?”

I hesitated. I wasn’t sure what to say. Technically I was born a Muslim.

“Well I’ve been practising for about a year.”

“Do you know how to pray?”

“Yes.”
\end{quote}

He thought I was a convert to Islam.
I soon discovered that English or Western converts
attract special attention from Muslims.
It was partly a protective instinct to guide the uninitiated,
to help new Muslims understand the basics.
But another reason was the sense of inferiority
that the age of European colonialism had left Muslims with.
Subconsciously many Muslims still regarded
their white European colonizers as superior beings.
The fact that a white man or woman had converted to Islam
was a special kind of proof that Islam was indeed true
and such converts received a great deal of attention.
I encountered this attitude often and found myself having to sit
and listen to lectures on the most basic and obvious aspects of Islam.
It felt extremely patronizing.

\begin{quote}
“My father is from Egypt; my mother is English.”
It was a sentence I would have to use often.

“Is your mother Muslim?”

“Er… well… not really.”

“Oh, you must try and make her Muslim.”

“Yes… inshallah…”

“It is your duty to save your family from Hell Fire!”

“I hope you don’t mind if we talk about this later, brother.
I want to listen to the talk.”
\end{quote}

The Maulana was talking about death and the Day of Judgment.

\begin{quote}
“The prophet said,
‘When a dead person is carried to his grave,
he is followed by three things.
Two of them will return after his burial but one will remain with him.
They are his relatives, his wealth, and his deeds.’”
He paused.

“His relatives will go back
and his wealth will be \cor{re-distributed}{redistributed}!”

He opened his eyes wide, “But his deeds will remain with him.”

“Yes my dear brothers, our deeds are of utmost importance.
It is only our deeds that will follow us to our grave.
They will stay with us as we await Judgment
and they will stand by us when we are questioned.
They will then either speak for us or against us.
It is then that we will realize the importance of even the smallest of actions.
But it will be too late to return and make amends.
No second chance!”
\end{quote}

The Amir began extolling the virtues of Tabligh (spreading the word of Islam.)

\begin{quote}
“What is more important than spending time in the path of Allah?
Do we truly love Allah and his Prophet more than our own selves?
Or are we too attached to the comforts and earthly pleasures of this world?
Are we so weak in our faith that we cannot even afford to give
even a little time in the service of our creator?
All you have to do is make Niyyat (intention)
for a couple of weeks or a month or a year!”

“What does he mean ‘make Niyyat for a year’?”
I whispered to the young Malaysian student beside me.

“He means you must make intention to go with a ‘Jamaat’ (group)
and invite others to Islam.”

“What?”
\end{quote}

One by one those around me got up to go.
Most had their rucksacks ready.
Some were even clutching passports and plane tickets.

\begin{quote}
“I can’t go anywhere! I have to go home.”
\end{quote}

But the Malaysian student had already got up
to join a group heading for Newcastle.

I suppressed an urge to make a dash for the door
and instead tried to think of an excuse.
But after all the talk of the day of judgment
and the importance of even the smallest of actions,
every excuse I thought of just sounded like;
\emph{‘It’s OK, I don’t mind burning in hell.’}
I stood up and looked around for Shafiq,
but when I spotted him he just smiled and waved.

\begin{quote}
“See you in two weeks, Hassan!” he shouted, before exiting with a large group.
\end{quote}

The Amir noticed me looking bewildered.

\begin{quote}
“Have you made intention brother?

“My family don’t know where I am.”

“You can phone them.”

“I have to go back home. I have things to do!”

“Isn’t spending time in the path of Allah more important?”

“Well… I suppose so… but how will I eat?”

“Food will be provided!”

“How will I sleep?”

“You will sleep in the Mosque!”

“What about my mum?”

“She will be fine!”

“What about…” The thought of gripping my leg,
wincing in pain and complaining about an old war wound,
crossed my mind briefly.

“Just make intention, and Allah will make a way for you!”
\end{quote}

My intention was to go home, but God’s intention was obviously different
and I found myself being directed to a delegation bound for Leeds.
We filed out of the mosque and boarded a mini bus waiting outside.

I had never been to Leeds and consoled myself with the thought
that at least I’d be visiting somewhere new.
But apart from the mosque I saw very little of Leeds.
For the next fourteen days we ate, slept and prayed in the mosque.
I was given a sleeping bag and ill\–fitting cotton Shalwar Kameez
(long shirt and baggy trousers.)
At meal times long rolls of paper were laid down
and curry and chapattis were provided,
cooked in rotation by the members of our group,
though they only gave me washing-up duty.

We were told not to discuss politics or areas of religious dispute.
We were even discouraged from discussing the meaning of the \Quran.
On one occasion I was sitting with a brother,
trying to explain some of the Arabic words to him, when I was told to stop.

\begin{quote}
“You shouldn’t give Tafseer (meaning of the \Quran) unless you are a scholar,”
said the brother.
“Just learn by heart the Suras the Maulana taught us.”

“But then we’ll be reciting words we don’t understand?” I complained.

“It doesn’t matter. You will benefit from your recitation!”
\end{quote}

I found it difficult to see how one could benefit from reciting words
without understanding their meaning,
but I soon learnt that the act itself was considered a form of worship
that would confer blessings upon those who engaged in it.
As a result memorization of the \Quran\ without understanding it
is very common amongst Muslims,
particularly those from non-Arabic speaking countries.
(Even those from Arabic speaking countries
have problems understanding the archaic language of the \Quran.)
The majority of those with me in the Leeds Mosque
were of Pakistani or Indian origin,
and although they had memorized huge passages
and in some cases the whole of the \Quran, few could understand a word.
Memorizing the \Quran\ is not as difficult as it may seem.
The language of the \Quran\ has a poetic rhythm
with repeated phrases and patterns,
such as beginning a sentence with “Qul!” (Say!)
or ending it with two adjectives of God.
Certain stories about past prophets or parables about believers
and unbelievers are \cor{re-visited}{revisited} throughout the \Quran,
so that a sequence one has already memorized
will occur in a similar form elsewhere.
Children are also taught to read the \Quran\ from a very young age —
as soon as they can imitate sounds in some cases —
and families hold a celebration called a Khatam (completion)
once their son or daughter has read the whole \Quran.
I memorized the last Juz’ (1/30\Sup{th} of the \Quran),
which are the short Suras (chapters) at the end of the \Quran\
and which are most commonly used in daily prayers.
I also learnt several other important suras and verses
such as the last three verses of al-Baqara and al-Kahf and
“The Verse of Throne” and Suras such as “Yaseen” and “Al-Rahman”.
Fortunately my studies at SOAS helped me
to learn the meaning of what I was memorizing.

There was one book we were encouraged to understand.
It was called “The Teachings of Islam”
by Maulana Zakarya \cor{Kandhlwi}{Kandhalawi},
a large volume badly printed on cheap paper and bound in a gaudy red plastic.
We were all given our very own copy and told to study it
when we were not engaged in prayers or listening to talks.
This book was the source of many of the lectures
I heard at the Dewsbury Mosque and over the coming days in Leeds.
It related stories of the Prophet and his companions
and quoted passages of the \Quran.
The emphasis was on reaching the utmost state of piety,
abstinence and fear of God.
At first I found many of the stories strange
and disconnected from the society around me.

One story began with the heading
\emph{“The Prophet Reprimands the \cor{companions}{Companions}
for Laughing.”}
It read:

% \cmt{}{quotes must alternate when nested -> “I am a wilderness”}
\begin{quote}
“Once the Prophet (peace be upon him) came to the Mosque for prayer
where he noticed some people laughing and giggling.
He remarked: ‘If you remembered your death I would not see you like this.
Remember your death often.
Not a single day passes when the grave does not call out:
‘I am a wilderness’ ‘I am a place of dust’ ‘I am a place of insects’.
When a believer is laid in the grave it says;
‘Welcome to you.
Very good of you to have come into me.
Of all the people walking on the earth I liked you the best.
Now that you have come into me, you will see how I entertain you.’
It then expands as far as the occupant can see.
A door from Paradise is opened for him in the grave
and through this door he gets the fresh and fragrant air of Paradise.
But when an evil man is laid in the grave it says;
‘No word of welcome for you.
Your coming into me is very bad for you.
Of all the persons walking on earth I dislike you the most.
Now that you have been made over to me, you will see how I treat you!’
It then closes upon him so much so that his ribs
of one side penetrate into the ribs of the other.
As many as 70 serpents are then set on him
to keep biting him till the day of resurrection.”
\end{quote}

The author explains the purpose of the book in the forward:

\begin{quote}
“Muslim mothers, while going to bed at night,
instead of telling myths and fables to their children,
may narrate to them such real and true tales of the golden age of Islam
that may create in them an Islamic spirit of love and esteem for the Sahabah
and thereby improve their faith and that it may be
a useful substitute for the current story books.”
\end{quote}

The more I read the more acclimatized I became
to the mindset of seventh century Arabia.
After several days of confinement in the mosque
and a constant routine of talks, prayers and readings from this book,
I was so immersed in the events of the early Muslims
that I began to feel estranged from what I now regarded
as the sinful world around me.
But in spite of it’s sinfulness,
the Amir regularly ordered expeditions to this very world.
Each day he chose a group of three people to visit an address
on a list collected by brothers beforehand.
These addresses were those of local Muslims
who for one reason or another were believed to be in need of “Da\`wa” —
in other words bringing back into the fold of Islam.
I was a little anxious when finally chosen to join a group.

The Amir gave us very specific rules regarding our conduct outside.
We were not to look around at all the Haram (forbidden) things around us.
We were to keep our eyes lowered at the ground,
a couple of yards ahead of us and repeat a little prayer to ourselves.
This made it a little difficult to follow the directions,
to negotiate busy streets and cross roads.
Eventually we turned up \cor{un-announced}{unannounced} at the house.
A short clean shaven young Asian let us in and offered us some food and drink,
which we politely turned down, as instructed beforehand,
in case this reprobate had used forbidden ingredients such as gelatine,
non-halal meat, or alcohol in his cooking.
We sat gingerly on the edge of his sofa, as our leader —
a leader of every group was always appointed —
repeated the talks about death and the Day of Judgment
and other stirring stories from the big red book.
We beseeched the stray brother to come to the mosque to listen to our Amir.
Eventually he agreed to join us for afternoon prayer.
Mission accomplished, we carefully made our way back.

I began slipping into a very obsessive mindset.
Even the most minute rituals and practices of the Prophet
took on extraordinary and exaggerated importance
that must be followed if I was to avoid the fires of Hell,
while all other matters of life seemed irrelevant.

\begin{quote}
“We should never think we are in any way more advanced
than the Prophet was,” said the Amir.
“Everything he did was the best example for us.
Even when we travel by car or plane we should remember
that travelling by camel or donkey is better,
because that is the way the Prophet did it.”
\end{quote}

Copying the Prophet included the way we brushed our teeth,
and our Amir gave us a talk about the importance of using the Miswak,
a twig from a type of tree found in the Middle East.
He related the saying of the Prophet;

\begin{quote}
“Was it not for my fear of imposing a difficulty on my nation
I would have ordered that the Miswak be used before every prayer.”
\end{quote}

He then produced a Miswak and demonstrated how it should be used,
stripping off the bark from the tip and chewing it until it was frayed.
He then rubbed it against his teeth from side to side.
When he had finished he told us a story from the time of Omar,
the second Caliph of Islam.

\begin{quote}
“During the conquest of Egypt,
the Muslim army was having great difficulty in defeating the enemy.
When Omar heard of this he said it must be
because of a deed they have committed.
So the Muslim fighters asked themselves
if they were neglecting any religious duty, but they found they were not.
Then they asked themselves if they had neglected any Sunnah
(practice of the Prophet) and they discovered
that they had forgotten to brush their teeth using the Miswak.
So they got together and started using the Miswak.
Once the enemy saw this,
they thought the Muslims were preparing to eat them alive and fled.”
\end{quote}

Every second of my day was now controlled and defined by this or that Sunnah.
When going to the toilet I was taught to clean myself in a certain way
and utter a prayer when entering and leaving the toilet.
Islam even regulated the way I slept,
and on my second night I was rebuked by the Amir for sleeping the wrong way.
He explained that a Muslim should never sleep with his feet pointing
towards Mecca but should always sleep facing it.
I wasn’t quite sure if he meant only that my head should be pointing
towards Mecca or whether I should be literally facing Mecca.
To be on the safe side I kept my face in the direction of Mecca
and prevented myself from turning side to side as I normally did,
which made it very uncomfortable and difficult to sleep.

I became increasingly concerned that such a high level of attention to form
and detail was not sustainable outside the sheltered environment of the mosque
and worried about my salvation if I was unable to maintain it.
But it was difficult to voice this concern in an atmosphere
where group mentality strongly disapproved of any failure
to live up to the standards set.
The Maulana seemed to take pride in how hard and difficult it was
to practise Islam properly and said that Prophet Muhammad had said:

\begin{quote}
“A time will come upon people wherein the one steadfast
to his religion will be like one holding a burning coal.”
\end{quote}

The sheikh explained that in this day and age to be a good Muslim
is like clasping hold of a red hot piece of coal.
One instinctively wants to throw it away,
but one must resist the instinct and grab it tightly
if one wants to achieve Paradise and avoid Hell.
A ‘true’ Muslim had to sacrifice the comforts
and pleasures of the world for austerity
and hardship if he was to gain the comforts and pleasures of paradise.
He must expect to be thought of as \cor{a}{} weird
and strange by non\–Muslims and suffer ridicule from the society around him,
as the prophet said:

\begin{quote}
“Islam began as something strange,
and it will revert to being strange as it was in the beginning,
so good tidings for the strangers.”
Someone asked, “Who are the strangers?”
The Prophet replied,
“The ones who break away from their people for the sake of Islam.”
\end{quote}

Although I had only been at the mosque in Leeds for two weeks,
it seemed much longer, and when we got back to Dewsbury I felt disoriented
and apprehensive about returning to ‘the real world’
with its evil temptations ready to entice me away from God.
I looked for Shafiq, but he hadn’t returned.
It was \cor{during}{} Ramadan, % Grammar?
and we had been fasting all day.
At sunset everyone broke their fast together in Dewsbury mosque.
I was starving, and the curry and chapattis never tasted so good.

As soon as we had finished eating the talks started up again.
I decided to nip outside for some fresh air.
When I got outside I saw there were about fifteen or twenty men
standing in a long line around the back wall of the mosque.
They had all sneaked outside to have a quiet cigarette.
This was their first cigarette after a long day of fasting
and many were dizzy from the sudden nicotine rush,
and as a result they slurred their ‘Salams’
as they glanced at me sheepishly with glazed eyes.
Although I had been a smoker myself not so long ago,
I now felt it was very ‘un-Islamic’ and I strongly disapproved.
The fact that many had long beards and large turbans increased my indignation.
But at the same time there was a part of me that seemed to take comfort
in the fact that there were practicing Muslims who were less than perfect.
It made me feel better about the possibility
that I might fall short of being a ‘perfect Muslim’ myself.

I was delighted to finally see Shafiq the next day.
I wanted to know if he had been through a similar experience as me
and whether he too felt worried about some of the things he had heard.
But he looked relaxed and \cor{un-phased}{unfazed} by his experiences
and told me how wonderful it had been
and how he could not wait to go out on his next Jamaat.
I was relieved to get back home.
I felt a little nervous.
Things looked different.
My priorities had shifted.
I was less concerned about this life and far more focused on the next life.
I grew my beard, wore a long white Jilbab and cap.
I not only prayed all the compulsory prayers,
but I prayed all the extra prayers, too.
I began fasting every Monday and, of course,
always brushed my teeth with a Miswak.
I was also determined to keep in my mind
that heightened state of fear of God that I had felt in the mosque.
My friends and family were at first a little surprised
by the even more devoutly religious Hassan,
and humoured my somewhat obsessive attention to every tiny detail of Islam.
However I soon found that being back in the ‘real world’
gave me a more balanced perspective
and the sense of anxiety and fear gradually subsided
as I realised that Jamaatu\cor{ }{\–}Tableegh’s obsession
with form and ritual was a distorted perception of Islam.
I came to appreciate what the \Quran\ said, that
“God does not task a soul beyond what it can bear,”
and that the needs of this world
and the needs of the next world did not have to be in conflict.
I began to seek a more sophisticated and deeper appreciation of Islam than
Jamaatu\–Tableegh offered.
% Jamaatu\–Tableegh: With or w/o '-'? Previously written w/o.


%Chapter 3:
\chapter{The Sufi Path}

\img{scale=0.7}{Members_SOAS_Islamic_Society.jpg}
{Members of SOAS Islamic Society, October 1980.}

I joined the SOAS Students Union Islamic Society as soon as I started
at London University in September 1980
and became it\cor{’}{}s president in 1981,
a post I held until 1984.
During my presidency I set up an Islamic bookstall,
organized talks and debates and showed films,
set up a room for daily prayers and got permission
to use one of the lecture rooms for Friday Prayers.
At first we shared the responsibility of giving the sermon among the students,
but this proved difficult,
as most were either too busy too prepare a talk or felt unqualified to do it.
Eventually we decided to invite a speaker from outside
and approached Dr.\ Kalim Siddiqui,
the Director of “The Muslim Institute” just down the road in Endsleigh Street.
Dr.\ Siddiqui was obsessed with the Islamic revolution in Iran
and was on a mission to promote a similar revolutionary ideology
amongst British Muslims.
He later went on to form “The Muslim Parliament of Great Britain”
with the aim of creating a “non-territorial Islamic state” in the UK.

At first our Friday prayers, led by Dr.\ Siddiqui and his understudy,
Dr.\ Ghayasuddin, were well attended.
The talks were always highly political
and revolved around the history of European colonialism in Muslim lands
and how this was the cause of the backwardness,
poverty and decline in Islamic values that Muslims were suffering at present.
A great deal of rhetoric was also directed at America,
‘The Great Satan’, which, it was claimed,
was embarked on a new form of economic
and social colonialism even more sinister,
since it used puppet regimes to keep Muslims enslaved
and ignorant of their Islamic heritage.
Interesting as this was to young students eager for a cause to take up,
the subject of God and spirituality seemed conspicuously absent,
and many of us began to tire of hearing the same thing,
week after week.
After a while the numbers attending started to go down,
and as the responsibility of clearing chairs,
laying sheets and putting everything back was becoming a chore
that my close friend Hussein and I had to do on our own,
I decided to end Friday prayers on the campus
and recommend students attend a local Mosque.

There were two mosques near enough to walk to
and be back in time for afternoon lectures — though it meant skipping lunch.
“The Mosque of Light” near Euston Station,
was a small room in the basement of an Edwardian terrace
that had seen better days.
Most of the congregation came from the Bangladeshi community in the area,
many of whom worked for British Rail or as traffic wardens.
They would rush to the mosque, straight from work and,
still in uniform, place handkerchiefs on their heads
and squeeze into the cramped space, spilling out onto the street.
The only signs of the room being a mosque were the sheets on the hard floor
and a small broken bookshelf that leaned over, top heavy with \Quran{}s.
The second mosque was in the plush new offices
of the “Muslim World League” in Tottenham Court Road.
Funded by Saudi Arabia to propagate Islam,
it was full of well paid administrators
who dealt with applications for grants.
The prayer room was a smart carpeted hall,
well stocked with expensively bound books.

At first I attended “The Mosque of Light”,
as I found the atmosphere there much more authentic,
but then the renowned Sufi preacher, Sheikh Nazim,
began giving sermons at The Muslim World League every other Friday.
I had been reading some of the great Sufi poets and writers
and was drawn to their universalistic and mystical approach,
so I was eager to listen to Sheikh Nazim.
Born in 1922 in the town of Larnaca in Cyprus,
he first moved to Istanbul to study Chemical Engineering at university,
and then to Damascus in 1945 to study under Shaykh Abdullah ad-Daghestani,
of the Naqshabandi Sufi Order.
From there he traveled around the world, particularly in Western Europe,
where he had built up a large following and was presently based in Peckham,
south London.
His followers spoke very highly of him
and told me he had ‘special’ knowledge about many things,
including the coming of the Mahdi —
‘the Rightly Guided One’ prophesized in hadith.
It seems odd now that the Saudis, who did not approve of Sufism,
allowed him to preach at the World League,
but the director at the time was an amiable,
broad minded Saudi who embraced many different views,
and it should be remembered that in the early 80s
there had not yet developed the sharp ideological divisions
amongst Muslims in Britain that were soon to occupy them.

Sheikh Nazim arrived to deliver his sermon, wearing a huge green turban
and a long flowing green cloak.
He was flanked by Murids, students of his spiritual path,
dressed in the same manner.
I was immediately struck by his presence and charisma.
He had a slow and deliberate way of speaking,
in which he paused on words he wished to emphasise,
and when he smiled, which was often, he looked like your favourite granddad,
about to make a sweet appear from behind his ear.
His strong Turkish accent only added to the endearing quality of his speech.
I enjoyed his talk greatly and wanted to speak to him in person
about his predictions of the Mahdi, but there was never time,
as I had to rush back to university.
However, when one of his followers invited me
to attend a small gathering in Peckham, I jumped at the chance.

The mosque in Peckham was a converted flat over a grocer’s shop
and gave no outward clue to the devotions inside.
More than half of those present were white Europeans,
an odd assortment of professionals, ex-hippies, eccentrics and drop-outs.
Many of those I spoke to seemed to be very narcissistic
with a messianic complex,
which jarred with the perception of Sufism
I had got from reading Rumi and Ibn Arabi.
I sat next to a young English convert who looked like a 1920s gangster —
slick back hair, pin striped suit and dark glasses.
As the room was only dimly lit,
I assumed he could see nothing through the glasses.

\begin{quote}
“Assalamu-\cor{Aalykum}{Alaykum}, my name is Hassan.”

“My name is Sulayman.” He lowered his glasses.
“But you can call me Slim.”

“Are you a regular here at Sheikh Nazim’s circles?”

For no apparent reason he replied in Arabic.

“Al-Hamdulilah, Kuntu azooru Sheikh Nazim al\–Halaqa li Mudda Atiqa.”
\end{quote}

It translated as “Praise be to God,
I was visiting Sheikh Nazim the Circle for an ancient time!”

\begin{quote}
“Oh.” I nodded.

“Wa min ayna anta?” (Where are you from?) he continued in Arabic.

“Ana min Finchley fi Shimal London,”
(I’m from Finchley in North London)
I said trying to get into the spirit of things.
\end{quote}

Sheikh Nazim came in followed by his students in green turbans and cloaks.
As he sat down, a plump German lady rushed up and kissed his feet.
Several others followed and either prostrated at his feet or kissed his hands.
Slim encouraged me to join him in greeting the Sheikh.

\begin{quote}
“Assalamu Alaykum,” I said stiffly, extending my hand for him to shake.

“Wa Alaykum Assalam wa rahmatullah,” he replied, taking my hand and smiling.
\end{quote}

Once we were all seated again,
Sheikh Nazim began leading the Zikr (Remembrance),
congregational chanting of God’s names or supplications.
He started repeating the words “Allahu, Allah Haqq” (God! God is Truth).
Everyone immediately joined in
and started gently swaying to the rhythm of the words.
After what seemed like a very long time Sheik Nazim signaled a change
in the words and the pace, by repeating

\begin{quote}
“Allahu, Allah Hayy” (God! God is Living).
\end{quote}

By the time Sheikh Nazim led another change,
the energy level had grown so great
that the whole room seemed to be vibrating to the rhythm.

\begin{quote}
“Allah Hayy, Ya Qayyum.” (God is Living, Oh the Awake!)
\end{quote}

Most had their eyes closed and appeared to be
in a state of intense concentration as they swayed together.
I couldn’t resist the pounding rhythm
and became carried away with rocking back
and forth as I repeated “Allah Hayy, Ya Qayyum” over and over again.
The room became hazy and looked like a black and white negative image;
everything around me was disappearing.
It was an almost psychedelic experience
and I was completely lost in the moment,
throwing myself backwards and forwards without any inhibition
or self\cor{ }{\–}consciousness.
Unfortunately it also meant that I was unaware of the abrupt end
to the proceedings and to my great embarrassment,
I continued chanting and swaying for a second or two
after everyone else had stopped.

Sheik Nazim recited some prayers and began his sermon.
He spoke in a heavy accent, mispronouncing many English words.

\begin{quote}
“Allah love all his serv-hents. Allah love everything!
Every human, every creature, every plant, every rock.
Allah does not hate. If Allah hate something it cannot exist.”
\end{quote}

He would choose someone in the audience to look at in the eye,
as though he were speaking to someone special.

\begin{quote}
“Allah’s love is not animal love we see in Dunya (this world).
It is love that never change. It is love that never die.
Our purpose is to reach higher love and immerse ourselves in Love Oceans.”
\end{quote}

I wasn’t sure what he meant by ‘Love Oceans’, but it sounded wonderful.

\begin{quote}
“Only when the serv-hent worship Allah can he wake Love Oceans.
We live in time of hate and misery.
Most human know only physical love and so become unhappy and miserable.
Without waking Love Oceans we can never content.”
\end{quote}

I was impressed by both his message and his manner,
but felt uncomfortable about the exaggerated reverence
his followers showered on him,
and decided to ask him about this,
during the question and answer session that followed.

\begin{quote}
“Is it Islamic to allow people to prostrate at your feet?
Prophet Muhammad didn’t have people prostrating at his feet, did he?”
\end{quote}

My question prompted angry murmurs and boos,
which made my face blush and eyes \cor{water}{watery}.

\begin{quote}
“I do not ask they do, but if they wish for love and respect, I accept.
Remember parents prophet Yusuf prostrate to him.”
\end{quote}

I wanted to ask him more questions,
particularly about his prediction about the Mahdi,
but others were keen to speak.
A young couple presented themselves —
both converts, one English and one Asian.

\begin{quote}
“We would like to ask your holiness for your blessing to get married.”
\end{quote}

After asking their names and a little about their background,
he gave them his approval,
placing his hand on their foreheads and saying a prayer.
Others quickly came forward with questions.

\begin{quote}
“I had a dream Sheikh that I was on top of a mountain,
there was a light above me and flowers began raining down.”

“This is good dream my child, you receiving Divine gifts.”

“My sister is sick, Sheikh. Please pray for her recovery.”
\end{quote}

He held his hands up and prayed, and everyone followed.

Eventually he got up and moved towards the staircase,
still surrounded by petitioners.
This was my final chance, before he disappeared.
I squeezed myself forward through the crowd.

\begin{quote}
“Sheikh Nazim, can I ask you about the Mahdi. You said he’s coming?”
\end{quote}

He started walking up the stairs followed by his green bodyguards.

\begin{quote}
“He is here!”

“In this room?” I followed him up the stairs.

“No, in Hijaz.” (The area around Makka and Medina.)

“Does anyone know who he is?”

“He has not exposed to anyone yet.”

“Then how do you know?”

“My Sheikh tell me.”

“Is that the Sheikh who’s dead?”
\end{quote}

Sheikh Nazim believed he was in contact with a Sheikh
who had died in the 1940s.

\begin{quote}
“It depend on what you mean dead?”

“In the meaning of not being alive.”
\end{quote}

Sheikh Nazim’s bodyguards had heard enough of my questions
and aggressively blocked my access, pushing me back down the stairs.
The Sheikh was swiftly escorted out of sight.

I was a little disappointed
with what I had witnessed at Sheikh Nazim’s circle.
Not so much with Sheikh Nazim himself,
but the way his followers fawned upon him.
It seemed little more than a cult of personality.
I was also extremely \cor{skeptical}{sceptical} of his claims
to special knowledge from a dead Sheikh.
After Sheikh Nazim had retired upstairs with some of his Murids,
I started to move towards the exit
when I was approached by a tall bearded Englishman.

\begin{quote}
“Are you on the path, brother?”

“Do you mean am I a Sufi? No, not really.
I like Sufism and want to learn more, which is why I came here today.”

“To learn more you must take the path.”

“The problem is I find some things a bit off-putting,
like kissing feet and special knowledge from a dead Sheikh.”

“In order to follow us you must not judge or object to anything.
This is how the Seeker of knowledge must approach his teacher,
just as Khidr told Moses not to question anything if he wished to learn.”

“Well, Moses is one thing, but it seems there is a dangerous potential here
for the blind to lead the blind, wouldn’t you say?”

“That’s why you must follow the true Sheikh,
so you can completely trust him.”

“I find it difficult to completely trust anyone in such matters.”

“Then that is the source of your problem, brother.”

“What makes you trust Sheikh Nazim so completely?”

“In every age there is one chosen representative (Khalifah) of God.
In our age it is Sheikh Nazim. He is the Perfect Saint.”

“What is your evidence?”

“Those who follow him have evidence.
Sheikh Nazim knows things that cannot be known by ordinary men.
He has proven this on many occasions.
For example I myself witnessed him predict the precise time
that one of his followers would die
and it happened exactly as he said it would.”

“I’m not doubting your word,
but there may be many rational explanations for that.”

“Yet another proof for you, my dear brother,
is that he has the ability to be with every one of his Murids
at every given moment.
He can be in one place with one and with another in a different place.”

“I’m sorry but I find that very hard to believe.”

“It’s the arrogance in your Nafs (Ego) that prevents you from believing.
You must stop resisting, let go and open your heart.”
\end{quote}

Our conversation reminded me of a passage in Alice in Wonderland:

\begin{quote}
\itshape
“I can’t believe that!” said Alice.

“Can’t you?” the queen said in a pitying tone.
“Try again, draw a long breath, and shut your eyes.”

Alice laughed. “There’s no use trying,” she said.
“One can’t believe impossible things.”''

“I dare say you haven’t had much practice,” said the queen.
“When I was your age, I always did it for half an hour a day.
Why, sometimes I’ve believed as many as six impossible things
before breakfast.”
\end{quote}

Sheikh Nazim has been the subject of many outlandish claims
and linked to some far\–fetched schemes.
Perhaps one of the more bizarre is “The Moon Temple Project”,
a venture aimed at erecting a Mosque on the Moon
as a “true sign for rational people.”
The project’s founder, a Russian Muslim called Asadula, said,
“There is some Divine predetermination in the fact
that the crescent moon is the symbol of Islam
and the minarets atop our mosques resemble spaceships.”
Asadula got the calling to build the Moon Temple
while praying for his brother who was in surgery:

\begin{quote}
\itshape
“And in an instant, oh wonder, I lost all sense of reality.
It was as though some super-powerful beam of light struck me in the eyes
with its radiant brightness.
I closed my eyes to shut out the blinding light, but this did not help —
the celestial radiance continued to permeate to the core of my very being.
I felt as though all the organs of my body had been pierced through
and strung out on this mysterious thread of unprecedented light.
And, as though continuing some conversation begun long ago
and causing my whole body to tremble, a Voice said unto me:

‘And such ordeals shall pass, but your destiny is invincible!
Your brother has not come to the end
of the goodly provision provided unto him.
He will recover.
And he will stand by your side as you toil together
to build a Temple on the Moon.
This will be no easy journey.
Your path, Asadula, is winding and rocky.
And you will raise a Temple on the Moon as a symbol of human faith.
And the five-fold azan (call to prayer) for salvation
will be heard throughout the universe.’”
\end{quote}

Sufism is admired in the West as it suits the trend for religion
to be moderate, liberal, and embracing a variety of lifestyles.
One Sufi Sheikh I invited to give a talk at SOAS explained
that Sufism is the pinnacle of all religions and one can find
Christian, Jewish and Hindu Sufis as well as Muslim Sufis.
Sufi Sheikhs were often regarded with the same awe in the 60s and 70s
as Hindu Gurus and other mystics of the East,
and many a hippie trail ended up following Sufism in one form or another.
But I was unconvinced by Sheikh Nazim’s circle
and the Sufi circles I \cor{susequently}{subsequently} visited.
My experiences made me very skeptical of the claims of such Sheikhs,
which seemed at best harmless eccentricity
and at worst dangerous self\–delusion.
Of course I did not blame Sufism as a whole for the short\cor{-}{}comings
of particular Sufi Sheikhs or their followers,
but it did put me off joining a Sufi order.
Nevertheless I was drawn towards the spiritual and metaphorical understanding
of Islam that Sufism taught and although I could not call myself a Sufi,
I did develop a leaning towards a less literal interpretation of the \Quran.
A belief that behind the words was a deeper meaning and significance
that was not immediately apparent.
I also enjoyed Sufi writings, poetry and parables.
Amongst these parables are the light-hearted stories of Joha
(also known as Mullah Nasrudin).
They are told throughout the Muslim world
and are part of an oral tradition going back centuries.
They are \cor{humerous}{humorous} tales that at the same time impart a wisdom.
Perhaps my favourite is one that pokes fun at Sufism itself:

\begin{quote}
\itshape
Joha was walking in the Bazaar with a large group of followers.
Whatever Joha did his followers immediately copied.
Every few steps Joha would stop and shake his hands in the air,
touch his feet and jump up yelling “Hu Hu Hu!”
So his followers would also stop and do exactly the same thing.

One of the Merchants, who knew Joha, quietly asked him;

“What are you doing my old friend?
Why are these people imitating you?”

“I have become a Sufi Sheikh.”
Replied Joha “These are my Murids (spiritual seekers),
I am helping them reach enlightenment!”

“How do you know when they reach enlightenment?”

“That’s the easy part! Every morning I count them.
The ones who have left — have reached enlightenment!”
\end{quote}


%Chapter 4:
\chapter{Growth of Activism}

\img{scale=0.7}{Assassination_of_Sadat.jpg}{}

“Hassan! Hassan! Look at this!”
shouted my father from the living room.
I hurried downstairs and found him standing
in the middle of the room staring at the TV.
The scene looked like a military parade of some sort,
but something had gone badly wrong.
It was chaotic, people were shouting in Arabic
and I could hear the crack\–crack\–crack of automatic weapons.
In the centre was a podium with chairs strewn about.
Soldiers standing at the back of an army truck
were firing towards the centre of the podium.
One soldier ran right up to it,
raised his rifle and began shooting down onto the chairs below.
It was October 1981, and this was the assassination of President Sadat
by a previously unheard of group called “Islamic Jihad” —
amongst its members was Ayman Zawahiri.
If I had ever been in doubt as to the political nature of Islam,
those doubts were soon dispelled.
Muslims were rarely off the TV screens, at the time,
from the Iran Hostage Crisis to the bombing of the US embassy in Beirut.

It had been my spiritual search for truth that had brought me to Islam.
I was attracted by the mystical passages of the \Quran\
and the promise it offered of inner knowledge and enlightenment.
But once I became a Muslim it was taken for granted that I would support
the political stance of other Muslims on issues such as Palestine,
Kashmir, Afghanistan and later on Bosnia, Chechnya and Iraq.
This assumption was made, not only by Muslims, but by non-Muslims too.
As soon as someone knew I was a Muslim, there would be remarks such as:

\begin{quote}
“Wasn’t it stupid of the Americans to send helicopters
to try and rescue the hostages in Iran?”

“I think dividing Cyprus in two states is the only solution, don’t you?”

“Did you hear what President Asad did to the Muslims in Hama?”
\end{quote}

At the time I knew very little about these issues,
but I soon realized I was expected to be an expert,
simply by virtue of being a Muslim.
Only days after I had returned from Egypt in 1979
I was sitting at my brother-in-law’s house
with my father and a young Arab student.

\begin{quote}
“Where are you from?” I asked.

“Palestine.”

“Where’s Palestine?”
\end{quote}

I had heard the word Palestine many times.
When I was very young, my father showed me a scar on his leg, saying,
“This is where I was shot by the Jews in Palestine.”
It was one of the few times he talked about his past.
I thought the scar was very cool,
but had no idea what he meant by ‘Palestine.’
Both my father and the young Palestinian were horrified at my ignorance.
After giving me a quick history lesson,
my father again proudly related the tale of how he had been wounded in 1948,
fighting with the Muslim Brotherhood, against the ‘Jews’.
I felt ashamed at my lack of knowledge of such an important issue
and made sure I read up on the subject.

Unlike Christianity, Islam sees no division between politics and religion.
There is no “Render unto Caesar the things that are Caesar’s,
and unto God the things that are God’s.”
Prophet Muhammad was a military and political leader
as well as a spiritual leader,
and he made it clear that Islam applies to every part of one’s life,
both the public and the private.
Muslims are to be regarded as one body; as he said,
“If one part of the body feels pain, then the whole body suffers.”
Therefore I felt my commitment to Islam meant a commitment
to my Muslim brothers and sisters around the world.
I began to take a keen interest in global politics
and read about conflicts and countries I had previously known nothing about.
The two major issues at the time were the Russian invasion
of Afghanistan and, of course, Palestine.

The plight of the Palestinians was highlighted in 1982
when unarmed Palestinian men, women and children
in the Sabra and Shatila refugee camps,
were massacred by Christian militia while the camps
were surrounded by the Israeli military.
I remember seeing pictures of whole families lying dead in the narrow streets,
their bodies bloated by the hot sun,
their hands still clutching the ID papers they had been desperately showing.
The images created enormous anger within me.
I also felt huge frustration that Muslim leaders were doing nothing to help.
The Russian invasion of Afghanistan triggered quite different emotions.
The struggle of the Mujahideen against the might of a superpower
was inspiring and it confirmed to me the belief
that only by returning to Islam could Muslims
ever put right injustices we had suffered.

Another factor that helped politicize Muslims in the UK at the time
was the influx of Muslims from other countries, particularly Arabs.
Some were exiles and dissidents who brought with them an international agenda,
but most were students or low-paid workers,
who came here looking for a better life.
Many were not practicing Muslims when they first arrived.
Feelings of isolation and estrangement brought them to the Mosque
and then to a local study circle where many soon became very devout.
I’ve always thought it ironic that Muslims
who were not religious in their own country
should become religious only after coming to a non-Muslim country.
There appears to be a close connection between loss of identity
as a result of being uprooted from one’s environment
and the sense of belonging and self-esteem that religion provides.

The Arabs in particular soon discovered
they were looked up to by other Muslims in the UK.
A racial hierarchy has always existed amongst Muslims with Arabs at the top,
despite the fact they make up only about 12\% of Muslims world-wide.
The main reason they are looked up to is that they hold the key
to understanding the words of God: the Arabic Language.
Muslims believe that the \Quran\ is the literal speech of God
and cannot be translated.
Once it is translated, it is no longer the speech of God,
but merely one person’s interpretation of the meaning.
A non-Arabic speaker will always be at a disadvantage in any dispute
over the meaning of the \Quran, with an Arabic speaker.
When all else fails, the Arabic speaker can simply claim
that the other cannot understand the true meaning of God’s word.
Another reason for this hierarchy is that the Prophet himself was an Arab.
This has a religious implication,
since imitating the Prophet Muhammad is an important aspect of Islam.
Muhammad reflected the Arab culture in which he lived, and as a result,
Arab customs, clothes, and dietary habits heavily influenced Islamic customs,
clothes and dietary habits.
Many traditional schools of Islamic thought also stipulate
that the Khalifah (leader of the Muslims)
must be an Arab from the tribe of Quraysh, the Prophet’s tribe.

Many Arabs who came to the UK reveled in the feeling of self\–importance
their new role as Islamic ‘experts’ gave them
and swapped their western lifestyle for a Jilbab,
white cap and the title of Sheikh.
This transformation of westernized young men
into religious leaders was exemplified by Abu Hamza.
He arrived in England in 1979 — the same year I became a practicing Muslim —
though the only thing he was practicing at the time
was his chat up lines in the Soho night club where he worked as a bouncer.
It’s hard to conjure up an image of Abu Hamza
that is different from the one-eyed,
hook handed gargoyle, so beloved of the tabloids,
but according to his English ex-wife he was a handsome,
romantic and tender young man who had many women chasing after him.
But, in 1984, after attending radical talks at a local mosque,
he began to become deeply religious,
traveling to Afghanistan to fight Jihad against the Russians,
where he lost a hand and the use of his left eye.
He then retuned to the UK and began preaching as Sheikh Abu Hamza.
Although Abu Hamza was a spectacular example of this transformation,
it was a pattern I witnessed often.
A common factor in all of them was the simplistic literal view of Islam
they adopted which allowed them to explain the \Quran\ at face value,
without reference to the centuries of traditional Islamic scholarship,
which none of them had any formal training in.

The majority of indigenous British Muslims were Asians
from South East Asia or East Africa.
Their parents had come here in the 60s or 70s
and still regarded themselves as Pakistanis or Indians rather than British.
Their children, on the other hand,
felt less connection to their parents’ countries,
having grown up here in England.
They were keen to learn more about an Islam
that was free from cultural baggage
in order to incorporate it into their daily life,
and they saw religion, rather than nationality,
as being the basis of their identity.
But they were cut off from the Islamic traditions
and communities of the Indian sub\–continent.
Their parents were unable to offer much guidance in Islam.
Having come to the West as economic migrants,
they either only paid lip\–service to Islam
or their knowledge was minimal and confused with their local culture.
As a result, the younger generation became increasingly influenced
by the more recent Arab immigrants.
This combination created the right ingredients for literalist
and militant movements to take root in the UK.
The egalitarian nature of literalism attracted the young,
as did the sense of rebellion that militant ideology offered.
In reality, these groups had more in common
with the sort of revolutionary political parties
I had experienced at university than with traditional Islamic movements.

Despite this, most Islamic meetings in the early 80s
were broad-minded and inclusive to begin with.
‘The East Finchley Da\`wa Society’, of which I became a member,
was a typical example.
It was as a meeting specifically for young Muslims
who wanted to learn the basics of Islam.
The meetings were mixed and informal
and we rotated the leadership amongst the members on a weekly basis.
A diverse array of speakers were invited, including Sheikh Darsh,
a delegation from The Federation of Student Islamic Societies,
a speaker on alternative medicine, an expert on Yoga,
the modernist Dr.\ Essawi, and brother Yusuf Islam.
On one occasion we invited some American evangelicals to share in our meeting.
They spent the whole evening trying to convert us all to Christianity.
Something everyone there took with good humour and impeccable hospitality.

\begin{quote}
“I wanna preeyyy with ya Hassaan!” said Tom,
a tall and broad\–shouldered man from Tennessee,
as he placed his hand on my thigh and squeezed it affectionately.

“OK,” I said trying not look down at his hand.

The party of Americans then bowed their heads as Tom prayed.

\emph{“In the name of Jeeezuz we thank you Lord
for the opportunity to share your love with our dear Mooozlem friends.
We ask in the mighty name of Jeeezuz,
that you open their hearts and receive your Word.
May you shine your face upon them,
and show them that your love is greater than anything that we can imagine.
In Jeeezuz’ name, we pray these things.”}

\emph{“Ay-men!”} they all said in unison.
\end{quote}

Tom then led the discussion explaining to us the ‘errors’
in the Islamic understanding of Jesus and Christianity
and urged us all to let Jesus into our lives so that we may be saved.
We then shared tea and Jammie Dodgers.

The Da\`wa society produced its own magazine called “The Clarion”
which I edited and included articles about Islam and topical issues.
They were sometimes serious, but often light\–hearted and humorous,
such as a spoof recipe by Benazir Bhutto,
then President of Pakistan, called “Political Hot Pot,”
and an Agony Maulana dispensing advice on how to stop one’s husband snoring
and a mock interview with President
\cor{Hafiz al-Asad}{Hafez al-Assad} of Syria,
father of the current president.
The interviewer asks:

\begin{quote}
\itshape
“You have been the ruler of your country for many years.
What is the secret of your success?”

“It’s quite simple.
I don’t hold elections.
They cause too many problems, like the possibility of someone else winning.
So I’ve banned elections and torture and kill all opposition.”

“Torture and killing seem to be key factors in your policies, don’t they?”

“Oh yes, I find the best way to deal with people is to kill them.
Some of my most trusted allies are dead people.”

“But your body guards are not dead?”

“No, no… of course not,
after all they wouldn’t be much good if they were dead, would they?
I’ve only had their brains removed, tongues pulled out and eyes skewered.”

“You left their hearing?”

“I’m not the heartless Capitalist, Zionist, Communist, Right wing,
Left wing, Western, Eastern and everyone who ever existed,
butcher that the press, make me out to be.
Besides, I want them to hear my orders.”
\end{quote}

We also organized sports activities, camping trips and excursions.
On one outing to Exeter, to meet local Muslims there,
we took a break in a nearby park
and the men soon began selecting teams to play football.
As we were short of numbers we did our best to encourage
some of the more senior members of our group to join in.
Amongst our party was Sheikh Hamid,
one of the Imams at \cor{Regents}{Regent’s} Park Mosque at the time.
He was a huge figure with a big round belly,
and was dressed in his full Friday Sermon robes and turban.

\begin{quote}
“Come on Sheikh!” I said;
“Playing sports is part of being a good Muslim!” I teased.

To our surprise Sheikh Hamid gamely got up and began walking towards us.

“He’s on my side!”
I said, grabbing Sheikh Hamid by the arm.

“That’s not fair, you have a Divine advantage,” chuckled Khalid.

“They can have him,” mumbled Ishfaq;
“I don’t think he’s going to be much help.”
\end{quote}

However Sheikh Hamid took the game seriously
and used his weight to barge everyone over like skittles
before thumping the ball in any direction he happened to be facing.
Everyone was laughing so much we could barely play.
After the game, as we walked back, with Sheikh Hamid carrying his turban
and robes over his arm and sweat pouring down his chubby face,
he smiled and said:

\begin{quote}
“When’s the next game?”

“Well, whenever it is, Sheikh,” said Khalid.
“Next time you’re on my side!”
\end{quote}

But by the mid 80s the Da\`wa Society had already begun to change.
A current of hard-line and narrow-minded doctrine began creeping into meetings,
rather like a hidden infection.
Often it was a casual comment that someone had heard ‘somewhere’:

\begin{quote}
“You’re not allowed to say Salam to a non-Muslim.”

“It’s forbidden to cut your beard.”

“You mustn’t draw faces.”
\end{quote}

These comments were passed on unquestioned,
because most didn’t have the knowledge or expertise to challenge them.
Meetings became more divided as those who were pushing a particular doctrine
insisted that only they had the ‘true’ Islam and everyone else was wrong.
We started seeing new members attending the Da\`wa Society,
each of them with their own agenda.

\begin{quote}
“Any announcements?” I asked at the end of one meeting.

“Yes, the Islamic Association of North London is holding
an event at the Community Hall next Sunday
to celebrate the Birthday of Prophet Muhammad.
Everyone is invited.”

“That’s Bid\`ah!” interjected a young man
who had appeared at the meeting for the first time.
“We are not allowed to celebrate the prophet’s birthday.
It’s Haram!”
\end{quote}

After a short silence we moved on to other announcements.

He called himself Abu Zubayr, although that wasn’t his real name.
He travelled regularly all the way from East London, along with his friends,
to give us lessons about the ‘correct’ Islamic beliefs.

\begin{quote}
“Islam is perfect and complete.
It cannot be changed in any way.
Anything new is ‘Bid\`ah’ — innovation — and will take you to Hell-Fire.”
\end{quote}

He spoke slowly and precisely
as though he were following a well rehearsed script.

\begin{quote}
“Prophet Muhammad — peace be upon him — said,
‘Every Bid\`ah is a going astray and every going astray is in Hell-fire.’”
He paused to sip his water, taking three sips just as the Prophet did.

“Today Muslims are indulging in all sorts of evil deviations.
They have become misguided and corrupt.
Brothers and Sisters!
We must return to pure Islam.
Return to Islam as it was practiced by the Prophet
and his righteous companions,
if we want success in this life and the next.”
\end{quote}

Over the next few weeks we learnt that many more things were forbidden —
such as, telling jokes that were not literally true,
as it was considered a form of lying.
This meant that most jokes are forbidden,
since there are few that don’t create imaginary situations.
Listening to Music was also forbidden —
apart from playing a drum made of animal skin on Eid.
Abu \cor{Zubair}{Zubayr} told one new convert that she wasn’t allowed
to attend Christmas dinner with her non-Muslim parents
because she would be committing Shirk (associating partners with God)
the greatest sin in the sight of God.
Abu Zubayr and his companions represented a minority of those at our meeting,
but the majority of us did not have the skills
or the energy to fend off his onslaught.

\begin{quote}
“We must always remain God\–conscious,”
I said during a talk I’d prepared.
“We should try and remember God is always with us. He is everywhere!”

“That is Kufr! (Infidelity),” interjected Abu Zubayr.

“What is?”

“To say Allah is everywhere!”

“Why?”

“Because Allah has told us in the \Quran\
that He is on the throne above the seven heavens;
\emph{‘The Beneficent One, Who is established on the Throne.’}”

“But it’s simply a figure of speech.
God is beyond our understanding.”

“By saying God is everywhere you are implying that God is also in the toilet —
wa \`Aoothu billah (I seek refuge with Allah)!”
\end{quote}

Abu Zubayr and his friends were Salafis — also known as Wahhabis —
who espoused a literalist and puritanical form of Islam
that seeks to cleanse the religion of what they regard
as innovations, superstitions and heresies.
Throughout the 80s, Saudi Arabia financed the spread of Salafi doctrine
% Sentence structure?? "...books flooded..." \cor{flooded}{flooding}?
and subsidized Salafi books
\cor{flooded}{flooding} Muslim bookshops up and down the country.
They also set up offices such as the Muslim World League,
in Tottenham Court Road, that gave financial help to Islamic organizations,
mosques, schools, students and individuals
who were willing to adopt their views.
Since Muslims had no other source to turn to when they needed help, most,
if not all, were willing to accept the strings attached,
thinking it was not a serious problem.
I myself applied to the Muslim World League
for a grant to study Arabic and Islam in Egypt.
I was offered a place,
not in Egypt where they considered the teaching to be deviant,
but at Medina University in Saudi Arabia.
There the curriculum was carefully prepared and taught by Salafi teachers,
to ensure that all the students would learn ‘true’ Islam.
Fortunately by the time I had to make my decision,
I had been offered a place at SOAS and so was able to turn it down,
though my brother Lutfi decided to take up a similar offer
and studied at Medina University for several years.
He returned with a very dim view of Saudi education and many shocking tales,
such as when he fell asleep in the Prophet’s Mosque in Medina,
only to be awoken by the Mutawwa (Religious Police) beating him with sticks.
The Mutawwa were appointed by the Committee for the Propagation of Virtue
and the Prevention of Vice to ensure
that everyone complied with the Islamic Law.
This involved enforcing Islamic dress, prayer times, dietary laws,
arresting boys and girls caught socializing
and seizing un-Islamic items such as Western music and films.
They also prohibit idolatry,
which is loosely defined and apparently includes falling asleep
in the Prophet’s mosque.
In 2002 they prevented schoolgirls from escaping a burning building in Mecca,
because the girls were not covered properly.
Fifteen girls died and many more injured as a result.
The Mutawwa have even enlisted the help of modern technology
to enforce Shari\`ah Law, launching a website where people can anonymously
tip off the authorities about un\–Islamic activity.

The Salafis were not the only militant group gaining ground at the time.
Hizbu\–Tahrir (literally, the Party of Liberation)
was their main competitor for the hearts and minds of young British Muslims.
One of their prominent members at the time, Farid Kasim,
became a regular visitor to Da\`wa Society meetings.
His one overriding obsession was the “Islamic State” (Khilafah.)
He listened to our talks, not to learn or contribute,
but to hijack them and talk about the Khilafah.

\begin{quote}
“Democracy is completely against Islam.
An Islamic State run by the Khalifah
is the only form of Islamic government acceptable;
\emph{‘And those who do not rule by what Allah has revealed are infidels.’}”

“But there are many principles within Democracy
that are completely Islamic,” I replied.
“The idea of Shura (consultation), for example, is the basis of democracy!”

“Democracy is totally incompatible with Islam.
Democracy is about giving humans the right to make laws.
It gives man that which is entitled exclusively for the Creator.
It is obligatory upon every Muslim to reject Democracy.”

“Muslim countries aren’t much better, are they?”

“That’s because they have abandoned the Khilafah and the Shari\`ah.”

“But you can’t just impose an Islamic State on people.”

“If you wait for the people to be ready,
then you will never have an Islamic State.
We need to change the government first, then people will change.”

“Isn’t Islam about transforming the individual?”

“Islam is about establishing God’s Law on earth.”
\end{quote}

Farid was a very confrontational young man with an incredible energy
and drive that typified the party he helped establish
in the UK with Omar Bakri and others.
In fact he and Omar Bakri were too radical even for Hizbu-Tahrir
and they left to form an even more militant group
called Al-Muhajiroun famous for, amongst other things,
“The Magnificent 19” conference in praise of the suicide bombers
responsible for the September 11\Sup{th} attacks.

At universities too, Muslim students were becoming more radicalised.
When I joined SOAS Islamic Society in 1980 it was a very scholarly society
that attracted professors as well as students.
I continued this tradition after I became president
and organized talks on the development of Sufism,
exhibitions of Islamic Art and seminars on archeological finds.
But by the end of my presidency there were a growing number
of Hizbu-Tahrir militants trying to muscle their way in.
During one meeting, a non-Muslim peace activist
who I’d invited to talk about the Palestinian situation was heckled
by Hizbu-Tahrir members who stood up and shouted slogans and abuse,
eventually bringing the meeting to a halt, in a chaotic atmosphere.
Gradually, the audience we used to attract stopped coming.
It was difficult to exclude the radicals
from either the Islamic Society or Da\`wa meetings.
By definition, they were for Muslims,
and we felt unable to exclude people who were, in every way, pious Muslims.
Although their approach was confrontational and aggressive,
they were at least trying to do something about what was happening to Muslims.
So we tolerated their presence in the hope
that they would moderate their stance.
But they became ever bolder,
taking control of Islamic societies
and setting out a much more militant agenda.
By the late 80s the influence of these groups had reached a high point
and many young British Muslims had become radicalized
and were no longer willing to passively accept
the perceived injustices or attacks on their religion.
They were ready to show their anger
and frustration at the slightest provocation.
In 1988, that provocation came in the form of a book, “The Satanic Verses.”

\img{scale=1}{Satanic_Verses_Demo.jpg}{}

The first time I heard about “The Satanic Verses”
was at a Da\`wa Society meeting.
After the usual talk and discussion, I asked,

\begin{quote}
“Any other business?”

“Yes,” said Naeem a gentle, soft spoken young Asian from Kenya.
“I have some photocopies from a book called ‘The Satanic Verses,’
by Salman Rushdie.”
He passed them around.
“This book blasphemes against our blessed prophet Muhammad,
and the mothers of the believers.”
\end{quote}

There were several derogatory quotes about someone called Mahound,
some swearing, and comments about the prophet’s wives
being whores and the Holy Ka\`ba a brothel.
None of it made much sense to me.
But it didn’t matter;
we were all in agreement that it was a vicious attack on Islam;
and there were several mutterings of “Astaghfirullah!”
(May God forgive me) and “Aoozobillah!” (I seek refuge with God)
and even the odd cry of “Allahu Akbar!” (God is Great!).
The action we decided to take was fairly mild
in comparison with what was to come later.
Naeem had the address of Penguin Books,
and we each composed individual letters demanding the immediate withdrawal
of the novel and a public apology to Muslims.

Soon, however, more militant action was demanded
as word about the book was passed around.
A momentum built up and the book took on a much greater significance,
triggering a wave of pent up anger.
Demonstrations and marches were organized.
Many countries started banning the book,
but this only seemed to increase the level of protest,
and book burnings were held in Bolton and Bradford.
Then in February of 1989 Ayatollah Khomeini issued his fatwa saying,

\begin{quote}
“It is incumbent on every Muslim to employ everything he has,
his life and his wealth, to send (Rushdie) to hell.”
\end{quote}

Although Khomeini was Shi\`a and most Muslims in the UK are Sunni,
his Fatwa met with approval from many Muslims on the street.
There were some more sane and reasoned voices
who argued that since Rushdie did not live in an Islamic State,
such a punishment simply wasn’t applicable,
but their voices were drowned out
by the hysteria now surrounding the whole issue.

I knew many Muslims who supported the death sentence on Rushdie —
not because they knew anything about Islamic Law —
but simply because they felt that was
what was expected of them by fellow Muslims.
They didn’t have the knowledge to argue against the Fatwa
and didn’t want others to think their resolve was weak.
Few — if any — had the slightest idea what the book actually said.
To this day, I have never met anyone who has read “The Satanic Verses”.
I did try reading it myself once, I read the first few pages,
several times, and then put it down, thinking I’ll read it later.
I never did.
But it no longer mattered what the book actually said,
it had become a symbol of evil and declaring one’s opposition to it
was to declare one’s allegiance to Islam.
There were few Muslims, at the time,
that weren’t caught up in the hysteria surrounding it, including Yusuf Islam.

I first met Yusuf in 1980 when I attended a circle called
“The Companions of the Mosque” which he led
in \cor{Regents}{Regent’s} Park Mosque,
but got to know him well during the fifteen years
I worked as a teacher at Islamia School.
In the early days he was very influenced by traditionalists
with dogmatic and literalist views.
In some ways his spiritual journey reflects the stages
many converts go through, ending in a reconciliation with their past.
Of course the difference for Yusuf is that he is famous.
New converts, by definition, have yet to gain a full understanding
and experience of their new faith
and must inevitably rely on the guidance of others.
With Muslims, that means lots of people telling you different things.
This is bewildering enough for an ordinary convert,
but in Yusuf’s case it was much worse as there was a huge competition
to have such a famous personality endorse their particular views and stances.
When Khomeini issued his Fatwa, the media were watching Yusuf’s every move,
eager for a sensational headline.
A few months later Yusuf was giving a talk at Kingston University,
about how he became a Muslim,
when someone asked him about Islam’s view on apostates.
Yusuf gave the straightforward and honest reply
that the punishment according to “Shari\`ah” (Islamic Law)
is the death sentence.
He couldn’t have said anything else,
because that is the plain fact of the matter —
at least according to the views he followed at the time.
The headlines the next day read, “Cat Stevens Says Kill Rushdie!”

I have never understood why the media singled Yusuf Islam out in its anger,
as though he had written Shari\`ah Law himself.
Although there were a few Muslims who argued that the death sentence
wasn’t applicable to Rushdie,
most were placed in the position of having to defend —
in principle at least —
a law that said it was morally right to execute someone
for choosing what they wished to believe.
These days there are a growing number of Sheikhs and Imams
who have made strong arguments against the death penalty for apostates —
and more crucially these arguments have become
more widely known about amongst Muslims.
But in those days they were very few and those that did
were either not known about or did not get
the media attention that those who suported the Fatwa did.
This meant that few ordinary Muslims had the courage
to publicly oppose the death penalty for apostates —
even if in their heart of hearts they did.


%Chapter 5:
\chapter{Islamia School}

“Hadrabak ya walad!” yelled Teacher Rafiqa,
the large Egyptian lady seated in front of a class of five-year-olds.
The words meant “I will hit you, child!”
They were directed at a boy climbing over the back of another child.
The boy knew she meant it and got down.
The class sat still and quiet while Teacher Rafiqa was in the room.
Even the naughtiest child dared not misbehave in her class.
She was teaching them Arabic, and “Hadrabak!” was a phrase
the children picked up quickly.
She continued chanting Arabic words,
slapping her thigh with each syllable as she did.

\begin{quote}
“Battatun.” (Duck).

“Battatun,” chanted the children.

“Tufahatun.” (Apple).

“Tufahatun,” chanted the children.
\end{quote}

When the lesson finished Teacher Rafiqa collected her sheets and marched out,
leaving Teacher Kulsum, the regular class teacher, to take over.
The children had been sitting still for a long time and were fidgety.
So she got them all to have a stretch and played a quick game of Ahmad Says,
which is like Simon Says — but it’s Ahmad who’s saying it, not Simon.
Then she sang a nursery rhyme to the tune of
“Boys and Girls come out to play;” everyone joined in.

\begin{quote}
\itshape
Boys and Girls \cor{its}{it’s} time to pray

Azaan is called five times a day

Stop your playing and leave your sleep

Ignore the noises in the street

Pack up your toys and leave your games

Remember Allah most glorious of names

Come in a hurry come clean and smart

Speak to your maker and open your heart

Praise and thank Him all day long

And say you’re sorry if you’ve done wrong

He’ll forgive you all and give you the power

To think and do good each second and hour.
\end{quote}

It was almost play time,
so Teacher \cor{Kuslum}{Kulsum} lined them up by the door,
the girls looking angelic in their white hijabs and beige pinafores.

\begin{quote}
“Walk quietly, in a straight line,”
said Teacher Kulsum as the children set off along the corridor.
“And don’t run!”
\end{quote}

But some of the boys had already started to canter past the girls at the front,
and once they turned the corner it became a full blown gallop
to get to the playground first.
Teacher \cor{Kuslum}{Kulsum} made her way to the staffroom
where Teacher Rafiqa had laid out some snacks.

\begin{quote}
“Cup of tea, brother Hassan?”
said sister Nasreen as I came in the door,
carrying a pile books to mark.

“Yes please,” I replied.
“And if there’s any of the fresh mint left,
could you pop that in it too please?”
\end{quote}

I sat down at one end of the long table
and began ‘flicking and ticking’ through the books, adding comments such as,
“Remember to use capital letters for place names.”

\begin{quote}
“Did you hear what happened to sister Sandra
when she was leaving school yesterday?” said sister Maheen.

“No?”

“That racist over the road — the one who lives in the bed and breakfast —
started swearing at her and told her to go home.”

“Stupid man.
What did she say?”

“She told him she was going home.”
\end{quote}

Teacher Abdullah appeared at the door with a lady
in a short skirt and low cut blouse.

\begin{quote}
“This is Caroline, from ‘Scholastic Books’.
She’ll be attending our staff meeting today.
She has some ideas about reading schemes to present to us.”
\end{quote}

Caroline stood nervously at the door.

\begin{quote}
“Come and sit down, habibti (darling),” said Teacher Rafiqa,
holding Caroline’s hand and leading her towards a chair
as though she were a small child.
Then without asking if Caroline wanted anything,
she placed an enormous slice of cream cake
on a plate and put it in front of her.

“Oh thank you, but it’s too much,” said Caroline.

“No,” said \cor{teacher}{Teacher} Rafiqa sternly,
“you must eat it all; you are too thin.”
\end{quote}

Islamia School was both a mad and a wonderful place.
The sincerity, commitment and genuine warmth of those involved
made one feel part of a huge family,
albeit a rather odd and dysfunctional family.
Throughout the fifteen years I spent there as a teacher
it was always much more than a job to me.
I mixed socially with the teachers and parents;
we attended prayers together, went to the same Islamic circles;
my children played with their children.
Regardless of whatever doubts and inner turmoil I had,
the community at Islamia School kept me going
and provided an anchor on which to hold.
Everyone at the school was a constant reminder of the reality of Muslims
that was a world away from the demonizing headlines of the tabloids.
It was also a reminder of how people’s humanity
always triumphs over narrow-minded dogma — eventually.

Islamia School was founded as a nursery in 1983
by Yusuf Islam with a small group of parents —
my sister and her husband being amongst them —
and expanded into an infant, junior and then secondary schools.
It was a natural consequence of the increasing Islamic awareness
amongst young Muslims in the late 70s and early 80s.
Many of them were now married and starting to have children,
and so their attention inevitably turned to education.
They wanted to provide an Islamic alternative
to the secular schools on offer in the UK.
But although everyone agreed that they wanted an “Islamic School”,
the details of what that meant to them were not clear.
To Teacher Rafiqa, it meant importing the traditional system of rote learning
and corporal punishment she had been used to as a child in Egypt.
To others, such as Teacher Nazeer, an Asian born in the UK
and educated at public school,
it meant adopting the methods and curriculum of the West,
with minor ‘Islamic’ concessions,
such as beginning the lesson with al-Fatiha (first chapter of the \Quran)
and ending it with al-Asr (103\Sup{rd} chapter of the \Quran.)
My own commitment to the cause of Islamia School was related
to the crisis of identity that I had experienced as a child.
I hoped the school would give Muslim children in the UK
a strong sense of their identity as British Muslims.
I wanted them to feel confident about who they were
and have others around them who shared their values and beliefs.
Yusuf Islam, who became the chair of the board of governors,
had a holistic vision of Islamic education as it had existed
during the golden age of Islamic history and wanted a school
that could combine the spiritual with the temporal.

Yusuf financed the school out of his own money
and was actively involved at every level,
from what was taught in the classroom to what food was served in the canteen.
He would arrive every morning as I was lining up the children
in the playground to recite the morning Du\`a (prayer)
and would always be on the look-out to see if the playground had been cleaned,
whether teachers were arriving on time
or whether policies were being implemented.
Everyone at Islamia School was well aware that Yusuf was the man in charge
and that he could be a hard task master at times, but they also loved him.
It is hard not to love Yusuf, as he is such an endearing character,
so natural, honest and childlike,
always enthusiastic and full of creative ideas
about how to make the school a brighter place.
He came up with the idea of \cor{re-painting}{repainting} the gymnasium
based on the theme of ‘Night’ and ‘Day’,
with the circle around one of the basket-ball nets
as the moon surrounded by a night sky and the circle on the opposite end
as the sun surrounded by a clear blue sky.
Yusuf wrote Nasheeds (Islamic songs) for the children,
and I took small groups of children up to the mosque to practice songs
with him for the Albert Hall performance in 2003,
commemorating twenty years of Islamia School.
Of course Yusuf has a wonderful singing voice,
and the children responded to him well,
as he taught them how to harmonise and get the key and timing right.
But he is also a perfectionist and wanted things done again and again,
when they didn’t meet his high expectations,
and at times needed reminding that they were only children
and could do with a break.
Yusuf could be very whimsical and some of the ideas
he would throw out were not to be taken seriously.

\begin{quote}
“I like the way you line up the children for the morning Du\`a,
Teacher Hassan,” he said one morning.

“Yes, it’s a good way to start the day.”

“How about doing some military drills with the children?
Marching up and down the playground in formation,
to teach them some discipline?”

“I suppose we could do something like that,” I replied.\\
“Though I’m not sure when we could do it.”

“How about in PE lessons?”

“I’ll look into it.”
\end{quote}

In the early days of Islamia School Yusuf seemed almost embarrassed
by his past as a pop star and never talked about it.
I remember sitting with him in the staff room
as we listened to a radio program about Islamia School.
The radio host presented a short biography of Cat Stevens
and started to play ‘Wild World’; Yusuf jumped up and turned the music down,
remaining there with his hand on the volume until the talking had resumed,
then sat down as we listened to the rest of the programme.
During this time when he was still a relatively new convert to Islam,
I always felt he was not being true to himself
and was obsessed with what people might think of him.
When controversial topics were raised he often asked others
what their positions were,
as if he was trying to discover what his own view should be,
rather than trusting his intuition.
However Yusuf mellowed a great deal over the years
and became more relaxed and willing to trust his instincts.
One example of this was his attitude to music.
At first he followed the strict view on the prohibition of musical instruments,
and the only music lessons in the school were the Nasheed classes
where songs were sung without any accompaniment.
Yusuf set up a record label, Mountain of Light,
to record them as well as those of other artists.
It wasn’t long before he started to follow the opinion
that using drums was acceptable.
Then he adopted a more liberal opinion that allowed electronic instruments.
He has now reached the position of allowing the full spectrum of musical
instruments and has now recorded songs that are not on religious themes.
This has drawn condemnation from some of the more
hard-line elements in the Muslim community.
Knowing how such criticism had upset him in the past, I told him,
as we sat together in the mosque one day,
not to take any notice of these narrow-minded idiots.

\begin{quote}
“No,” he replied, “I’ve gone beyond that point now.
It doesn’t bother me what they say.”
\end{quote}

Islamia School also changed and evolved over the years
as it struggled to find its identity.
It was a pioneering venture to create an Islamic school
to compete with the best schools in the UK
while at the same time providing a traditional Islamic education.
During one INSET day teachers were set the task of formulating
a Mission Statement that encapsulated the aim of Islamia School.
We eventually came up with this:

\begin{quote}
“To strive to provide the best education, in a secure Islamic environment,
through the knowledge and application of the \Quran\ \& Sunnah.”
\end{quote}

But merging the British National Curriculum with the values of
the \Quran\ and Sunnah was no easy task and required compromises.
The first problem was that the only text books we had were either
the standard texts books found in UK schools,
which therefore made no reference to the \Quran\ and Sunnah,
or they were sub\–standard books borrowed from Muslim countries.
Everyone agreed that we should use the same text books
used in \cor{State Schools}{state schools} in the UK and we did our best
to input our own knowledge of Islam into lessons.
But this was ad hoc and unevenly applied,
and Yusuf was keen to devise a standard Islamic syllabus.
He directed us to collect \Quran{}ic verses or Hadith
that related to scientific or historical areas of the National Curriculum
and collate it together into an Islamic Syllabus.

At first the school adopted many of the strict Salafi views,
such as not allowing musical instruments or the drawing of faces
and the censoring of material deemed un-Islamic
according to strict Salafi criteria.
It’s true that a fair number of parents were strongly influenced
by Salafi doctrines, but the deciding factor was Yusuf himself.
He was the real power behind Islamia School.
No decision could be made without his approval and he dictated
the sort of people he wanted in charge and the policies implemented.
Everyone in the school knew Yusuf was the real person in charge —
regardless of who was head teacher or Treasurer
(an important post at Islamia School.)
To be fair though, Yusuf’s leaning towards Salafi doctrines
may have been partly due to the need
to raise donations from wealthy Muslims — including Saudis.
It costs a great deal to run a school,
and although Yusuf put a huge amount of his own money into it,
more was always needed.
This came partly from fees, but more crucially,
from donations from wealthy individuals.
The money was raised and distributed around the school
through the Trust Office.
Most of those Yusuf employed in the Trust Office were Salafis.
Many of them were also parents of children at the school.
As parents and custodians of the purse strings,
the Trust Office felt they had the right to impose their views on the school,
and relations between them and the more liberal-minded teachers
were often very strained.
Yusuf was forced into the position of peace broker,
walking a tightrope between the different opinions.

A parent who had studied in Saudi Arabia was appointed
% TODO: Islamisization? Or Islamization?
by the Trust Office as “Islamisization Officer”
to check school books for ‘un-Islamic’ content.
I was English co-ordinator at the time
and was told to provide him with the school’s text books.
He was given a small office and a little stamp
that \cor{that}{} read “Un-Islamic Content”.
He would flick \cor{though}{through} the books
and when he saw something he didn’t like,
down came the stamp in the middle of the page.
One book that was stamped “Un-Islamic Content” was a story about a boy
who woke up one day as a girl and had to spend the day like that
before being magically turned back.
As usual with the Salafi mentality,
only the outward aspects were considered, not the deeper meaning,
which was seeing things from another’s perspective.
Since it showed a boy wearing women’s clothes
it was deemed to be against Islamic morality.
Amongst other texts that received this treatment were a set of books
that related Greek myths and fables,
a story about a boy who wrote letters to his girlfriend
and a book that explained the story behind the ‘Willow Pattern’
commonly found on Chinese pottery — it was a love story.

Pressure on the school to adopt Salafi views didn’t only come from Yusuf
and the Trust \cor{office}{Office},
but from pro-Salafi parents\cor{}{ as well}.
Parent power has always been very strong at Islamia School,
as the school was established by parents in the first place.
Although our parents reflected the huge spectrum of views within Islam,
it was the more dogmatic who were the most vocal and active.
On one occasion while teaching a class of six-year-olds
I was stopped by a mother wearing the Niqab (face veil).

\begin{quote}
“My son says that you are teaching them music!”

“You mean nursery rhymes?”

“Yes! I don’t want my children learning Kuffar (infidel) songs.
I don’t allow them to listen to such things at home,
nor do I allow them to read Kuffar books!”

“There’s nothing un-Islamic about them;
they’re an essential part of language development.”

“Music is Haram! I sent my children to a Muslim school
to be free from such Kuffar influence!”

“I’m sorry, but you’ll have to take it up with the Head Teacher.”

“I also need to talk to you about \cor{is}{} my son’s Dyslexia;
he requires special help!”
\end{quote}

It was true that her son was having difficulty reading and writing,
but I suspected his problems may have had more to do with being deprived of
hearing infidel nursery rhymes and infidel books
from an early age than with any medical condition.

Even our cleaners were telling us what to do.
I arrived at school one morning to find someone
had written a Hadith of the Prophet,
across the children’s display of Ancient Egypt, using a black marker:

\begin{quote}
“Whoever makes a picture will be punished on the Day of Judgment
and will be asked to give life to what he has created!”
\end{quote}

On other occasions faces were blotted out or displays torn.
All this would happen during the evening when no-one was around.
I remember one morning we had a visit
from the then Secretary of State for Education, John Patten.
As we walked down the corridor I saw, to my horror,
that the ‘Night Patrol’ had been busy with an Alphabet strip up ahead.
Someone had carefully gouged out the eyes from the pictures of animals.
I looked the other way and pretended not to notice.
I don’t know if John Patten did the same, but nothing was said.

During the weekends the Trust Office let out the school premises
to Islamic Sunday schools or groups holding bazaars.
During one such bazaar held by a group of Salafi brothers,
I discovered they were using my classroom as a crèche.
I had a display of World War II aeroplanes hanging from the ceiling,
including a Lancaster \cor{Bomber}{bomber} which my young son and I
had carefully constructed together.
The children were jumping on tables pulling the planes down
and throwing them across the classroom,
smashing off the propellers and breaking the wings.
Sitting in a corner on a little chair was a large black shape
wearing a pair of spectacles.
It took me a second to realise that this was a woman.
She sat silently reading \Quran\ to herself
completely ignoring the children smashing up my classroom.
I asked her what was going on but my words seemed
to startle her and she turned her face away
muttering “Aoothu billah” (I seek refuge with God.)

I realised it was pointless trying to talk to her
and so went angrily to see the organiser, a muscular,
full bearded brother wearing a grey Jilbab and black soft-leather slippers.

\begin{quote}
“Look what the children have done to my display,”
I said as I held out the broken fuselage of my Lancaster bomber.

“Astaghfirullah, brother, you are upset about models of Kafir planes,
while we are here struggling to raise money for the Mujahideen in Afghansitan.
Fear Allah!”
\end{quote}

The balance of power started to shift once Sheikh Ahmad Babakir
was appointed as the School’s Imam.
He came from a very inclusive Sufi tradition
and was a staunch opponent of the literalist Salafi teachings.
This of course angered the Salafis, who considered Sufis to be deviants,
and they boycotted his prayers, on the advice of Sohaib Hasan,
an Imam who often advised Yusuf and many of the other Salafis there.
Sohaib Hasan is now head of the Shari\`ah Council
seeking government recognition.
It was clear that the Salafis there hoped
that Sheikh \cor{Ahamd}{Ahmad} would have to be replaced.
But Sheikh Ahmad had become \cor{became }{}extremely popular
with the teachers and children.
He was a warm and immediately likable personality,
who related to children well.
He was able to communicate at their level
and used jokes and funny voices to make them laugh.
Both the adults and the children eagerly looked forward to his assemblies.
He also had a deep and encyclopaedic knowledge of Islam
which he was able to relate to everyday matters.
There was always a queue of children
and adults outside his door waiting to consult him.
Yusuf too was greatly impressed with Sheikh Ahmad
and put his support firmly behind him.
The Salafis realised they had no choice
but to accept him and were forced to end their boycott.
The final blow to the Salafi influence came
when state funding was granted by the new Labour Government in 1998.
The school no longer had to dance to the tune of outside donors,
and the Trust Office was moved out of the school.

Since that time the school has become much more confident
in its moderate and inclusive stance.
The Primary School web site that I was asked to put up around that time stated:

\begin{quote}
“We are committed to building racial and religious harmony
through respect and tolerance for others.
This is an essential part of the Islamic faith as the \Quran\ says;

‘We have made you nations and tribes so that you can get to know one another’”
\end{quote}

I was interviewed once by a post-grad student
researching denominational schools in the UK.

\begin{quote}
“Don’t you think faith schools create division
and segregation within society?” she asked.

“There’s always a danger that can happen,
but so long as the school has a clear policy
and shared ethos to combat isolationist or extremist views,
then I don’t believe it is a problem.”

“Are children at Islamia well integrated into mainstream British society?”

“Yes, definitely, and I think Islamia School is in a better position
to integrate Muslim children into British society than many state schools.”

“How can that be, when you’re segregating children according to religion?”

“Because Islamia School is able to reach the very families
who resist integration.
When we stress the importance of making
a positive contribution to British society,
we do so from a position of authority within the Islamic faith.
When Sheikh Ahmad or the teachers tell the children
they must respect other faiths and have good relations with their neighbours,
they know that this is what Islam says.
Unfortunately many Muslim families who hold extreme views
will not change their views when they hear the same thing
said in a non-Muslim school,
because it is not coming from an Islamic source.”

“Do you have any evidence that your children are well integrated?”

“You only have to look at former pupils who are well integrated
and successful members of society.
Another point worth mentioning is that the Muslim communities
who are experiencing problems integrating,
highlighted by the riots in some northern towns,
are youngsters who went to state schools, not Muslim schools.
Of course I’m not blaming State schools.
The problem is not one of which school they go to,
but what ideas they internalise,
and in this respect Muslim schools can actually be a positive influence —
so long as they have the right ethos of course.”
\end{quote}

Teachers at Islamia School had become used to being asked
such questions by TV reporters,
newspaper journalists and researchers,
so much so that most of us could probably have
completed a thesis on faith schools ourselves.
Islamia School also received a steady stream of very important visitors,
such as government ministers, celebrities and foreign dignitaries.
Our most high profile visit was from His Royal Highness, the Prince of Wales.

In the days leading up to the visit the school was cleaned and polished,
new displays put up and classrooms made to look their best.
Teacher Rafiqa rehearsed Nasheeds with children she had carefully selected
in her usual nepotistic manner,
including members of her own extended family and favourites.
The kitchen staff were given strict instructions about what the Prince could
and could not eat and drink and prepared a range of delicacies.
As my class was one of the three classes the Prince was scheduled to visit
during his tour of the school,
we set about making a class book about British Muslims to present to him.
When the big day arrived we stood in a line in the playground,
the boys with their hair neatly combed and the girls in their best hijabs,
one of them carrying a huge bunch of flowers to give to the Prince.
When his black limousines pulled slowly up to the gates
the atmosphere was electric, and one little girl found it all too much
and wet herself and had to be taken off to get changed.
The Prince was greeted warmly by Yusuf Islam,
and then they turned to listen to the choir
who had been patiently waiting for their five minutes of fame.
Teacher Rafiqa gave them the nod, and they burst into song —
one specially chosen for the occasion —
as Charles and Yusuf looked on, all smiles.

\begin{quote}
\itshape
Peaceful peaceful peaceful peace

Peaceful peaceful peaceful peace

Peaceful peaceful peaceful peace

Peaceful peaceful peaceful peace

We make peace a symbol of ours.

In the name of peace we gather here now

Oh Lord! Please make these days of ours

On this earth filled with peace.

Peaceful peaceful peaceful peace

Peaceful peaceful peaceful peace

Peaceful peaceful peaceful peace

Peaceful peaceful peaceful peace

So Lord indeed You are salaam

From You comes salaam and with You is salaam

To You belongs the command of all things

Between Your hands are the hearts of all beings.

Peaceful peaceful peaceful peace

Peaceful peaceful peaceful peace
\end{quote}

The Prince was then led away to begin his tour,
and so I rushed off to my class to calm the children down and get them ready.
After visiting Year One, Yusuf led the Prince to my class.
As he came in he greeted us with “Salam Alaykum,”
to the enormous delight of the children who responded with
“Wa-Alaykumus Salam Wa Rahmatu-llahi wa Barakatuhu,”
which was chanted in a long, drawn-out manner
that forced everyone to freeze for a few seconds.
I then welcomed the Prince and talked about what we were doing in class.
I was impressed with Prince Charles;
he listened carefully to what the children said
and responded on their level and with great humour.
He was relaxed and seemed extremely comfortable with the children,
who were over\–excited about having the Prince sitting at their table
and the media circus on all sides of them.
He sportingly allowed himself to be taken by the hand
and dragged off around the classroom by the children
and visibly enjoyed the affection they were showing to him.
We had been told \cor{before-hand}{beforehand} to address him as
“Your Royal Highness” or just simply “Sir,”
and I had passed this on to the children before the visit.
At one point they took him over to see a time-line
we had put up about the Tudors.
A little girl pointed to Queen Elizabeth and shouted,

\begin{quote}
“Look, Your High Majesty… that’s your grandma!”

“Indeed!” chuckled Charles.
\end{quote}

At the end of his visit to our class we presented him
with the book we made about British Muslims.
The book contained a brief history of Muslims in Britain
as well as information on notable and upper-class British Muslims.
The children had also written about
what being a Muslim in Britain meant to them.
At the end of the book I put an article about the Woking Mosque,
the first Mosque in England —
and how it lies at the centre of several Ley lines —
I thought Prince Charles would appreciate it.
Two children then read a short poem we had written in class:

\begin{quote}
\itshape
Oh Allah to you we pray

Oh Allah we do obey

Oh Allah we always say

Bismillahi every day

Oh Allah we worship you

Oh Allah we work for you

Oh Allah there’s one, not two

This we know is really true!
\end{quote}

Before leaving, Prince Charles leaned close to me and whispered,

\begin{quote}
“Your children are a credit to you,” as we shook hands.

“Thank you, Your Royal Highness,” I replied.
\end{quote}

After the tour we all gathered in the hall upstairs,
where Yusuf gave a short speech welcoming our royal guest,
before the Prince gave his own speech,
saying he was moved by what he had seen of the school
and the welcome everyone gave him.
He praised the children and school in general, adding,
“You are ambassadors for a sometimes much misunderstood faith.
I believe that Islam has much to teach increasingly secular societies
like ours in Britain.”

Another notable visitor to the school was the ex-boxer Muhammad Ali.
I had always admired him when I was growing up in the 60s and 70s.
I loved his anti\–establishment attitude
and his sharp and witty quips during interviews.
He broke the stereotype often applied to both black people and Muslims:
that they were not intelligent and did not excel in anything.
As a Muslim at the top of his profession,
he made me feel good about my Muslim identity,
even though at the time I wasn’t practising Islam.
As with Prince Charles, Muhammad Ali was escorted around the school
by Yusuf Islam — though without the posse of photographers
that had accompanied the Prince.

\begin{quote}
“It is a great honour to meet you, sir,” I said.
“I have always been a big fan of yours.”
\end{quote}

“Thank you.” I knew Muhammad Ali had Parkinson’s disease,
but it was only when I saw him struggling to get his words out
that it really hit me.
It made me sad to see him like that: a man once famous for his sharp mind
and powerful physique, now so slow and frail.
It gave him an air of vulnerability that made one want to hug him.
The children seemed to feel the same way and swarmed around him
as though he were a big teddy bear.
The bell for playtime went, and they insisted that Muhammad Ali
come out to the playground with them to join in their games.
His face beamed a willingness to go with them,
and so Yusuf abandoned the tour and we followed them to the playground
and watched Muhammad Ali play hide and seek with the children,
Ali pretending to be a monster, holding his arms aloft,
growling and chasing after them.
They screamed wildly and ran in all directions, only to slow down
% \cor{lofted}{lifted}?
as he got nearer so they could get caught and lofted into the air.
My seven-year-old son, Yaseen, was amongst the children there,
and I felt enormous pride watching Muhammad Ali grab him and growl,
while giving him a cuddle at the same time.


%Chapter 6:
\chapter{Sheikh Faisal}

It was 1992 and I had just arrived for my interview at Islamia School
for the job of class teacher.
As I entered the small school office I saw a slim and youngish looking
Jamaican Muslim wearing chic grey Armani jacket over a long white Jilbab.
On his head was a cotton laced cap and had a small beard on his chin.
We exchanged Salams as I entered the room,
then sat in silence until he was called in.
I discovered later that his name was Sheikh Faisal
and that he had applied for the job of Islamic Studies teacher.
Both he and I were successful in our interviews
and we started at Islamia School together —
he as the new Sheikh and I as the new class teacher.

Little was known about Sheikh Faisal at the time apart from the fact
that he had recently graduated from an Islamic Studies course in Saudi Arabia,
but he impressed everyone with his charisma and sense of humour.
He had a style that the children loved and enjoyed
wearing fashionable jackets and expensive robes.
He used to say that it was part of Islam to look good.
But his behaviour quickly began to make the teachers concerned.
He encouraged a rather over-familiar relationship with the children —
especially the girls.
He could often be seen walking around the playground or seated in the canteen,
surrounded by girls holding his hand.
His Friday sermons were also a little too ‘adult’ in their content,
raising topics such as sexual relationships,
contraception and Jihad along with derogatory remarks about the Kuffar.
I myself spoke to the head teacher and others
about his inappropriate comments and behaviour and generally avoided him.
I think the feeling was mutual and we rarely spoke to each other.
Then about 6 months later day Dr.\ Baig called me into his office
and asked if I could give the Friday Khutbah (Sermon).

\begin{quote}
“Is Sheikh Faisal ill?” I replied.

“He has decided to leave.” Replied Dr.\ Baig.
\end{quote}

I later discovered that he had been asked to resign.
For quite a few weeks after that, Dr.\ Baig and I rotated the job
of giving the Khutbah to the whole school,
until another Sheikh was found — a Palestinian named, Sheikh Abdul Salam.

Faisal became the Imam at the Brixton Mosque and,
being a convert of Jamaican origin,
was able to build up an excellent rapport
with the Afro-Caribbean population there, many of whom were also converts.
He was a magnetic speaker, often using street language to make jokes
and ridicule the rulers and religious leaders of Muslim countries.
This made him popular amongst the youth.
However the long standing and well respected Sheikhs and Imams
regarded him as a crackpot,
and he earned the title of Sheikhu-Takfeer
(The Sheikh of declaring people infidels)
for his habit of calling other Muslims Kafir.
He soon fell out of favour with the Brixton Mosque Committee,
and after an acrimonious struggle, they succeeded in removing him.
The Committee at Brixton Mosque were Salafis like Faisal,
but they were loyal to the more moderate Saudi Salafi Sheikhs.
This reflected the major split amongst the Salafis
that was occurring in the 90s.
Bin Laden was the main instigator of this when he turned against the Saudis
and their scholars in 1991 for allowing the US
to have bases in Saudi Arabia during the Gulf War.
Bin Laden came out with a series of Fatwas (religious decrees)
culminating in his 1998 Fatwa to
‘kill Americans and their Allies, civilian as well as military.’

Faisal continued preaching elsewhere,
and his fiery and controversial sermons found favour
amongst the young and disaffected.
He held a circle in Tower Hamlets and was invited to speak
at gatherings up and down the country.

I didn’t see or hear about Sheikh Faisal until a few years later
when I turned up one day at Willesden Library Centre to attend Friday prayers
and was shocked to see him standing in front of the congregation
giving the sermon.
Willesden Library Centre was a stone’s throw from my house
and when I wasn’t attending Friday Prayers at Islamia School,
I would pray there.
I knew the community in Willesden well;
many were friends or family of children I taught.
A few weeks before I saw Faisal I had been approached by one of the organisers,
who also happened to be the father of a girl in my class,
a jovial and amiable man of Turkish origin, married to an English convert.

\begin{quote}
“We are having difficulty finding a regular Khateeb (speaker),
brother Hassan,” he told me.
“Would you be able to fill in for a while?”
\end{quote}

I declined his request, but suggested the names of possible speakers.
During the following weeks we had an assortment of speakers, some good,
some not so good, until the day I saw \cor{Shiekh}{Sheikh} Faisal
standing there.
I was surprised and disappointed, but sat down and listened.
He was talking about Jihad and his rhetoric had become even more radical.
I could barely believe what I was hearing.

\begin{quote}
“It is our duty to go out and kill the Kuffar!”
\end{quote}

More worshippers packed into the upstairs function room
that had been cleared of chairs and sheets placed on the floor.

\begin{quote}
“We spread Islam by the Sword and so what? Today we are going to spread
Islam by the Kalashnikov and there is nothing you can do about it.
This is what Jihad means, it means killing the Kuffar!”
\end{quote}

His words echoed around Willesden Library Centre.

\begin{quote}
“The Munafiqs (Hypocrites) claim the greatest Jihad is to struggle
with the nafs (ego\/self), but that is a fabrication.
The true meaning of Jihad is fighting with the sword.
It is the highest act of worship a Muslim can do.
If you want to go to Paradise, it’s easy.
Just kill a Kafir!
By killing that Kafir you have purchased your ticket to paradise.”
\end{quote}

I could see library centre staff walking along the corridors outside the room
and I wondered what they must have been thinking.

\begin{quote}
“Allah ordered us to terrorise his enemies!
He the Most Glorified, Most High said;

‘Prepare, against them, your strength to the utmost of your power,
including steeds of war, to terrorise the enemies of Allah.’

So yes! We are terrorists, because Allah has commanded us
to be and we are proud of that!”
\end{quote}

Some worshippers began looking at each other,
but most sat with their heads bowed in silence.
It is true that the Muslim community must bear the responsibility
for allowing such people to preach,
but it is not easy for the ordinary individual
to confront radical preachers such as Sheikh Faisal.
Most Muslims will not question a person
who appears to be in a position of authority.
This is especially true when in public and surrounded by other Muslims.
The vast majority of Muslims are not well versed in Islam
and lack both the knowledge and confidence to raise objections.
Nevertheless the ease at which Faisal became a Sheikh
and started spreading his hate is worrying.

Born Trevor William Forrest to a Salvation Army family
of practising Christians in Jamaica,
Faisal converted to Islam as a teenager.
He travelled first to Guyana and then to Saudi Arabia
where he attended Muhammad bin Saud University in Riyadh.
There is no uniform process by which someone
can become an Imam or Sheikh in Islam,
nor any central authority that appoints clergy to positions of authority.
So with his Diploma from Saudi Arabia,
he simply put on some gold braided robes
and started calling himself a Sheikh.

After Faisal’s Khutba I saw him standing outside the entrance to the car park,
surrounded by a group of brothers.
They were talking animatedly amongst themselves,
but stopped when I approached and looked me up and down warily.

\begin{quote}
“Assalamu-Alaykum, Faisal”

“Wa-Alaykum Assalam wa rahmatullahi wa barakatuhu, Hassan,” replied Faisal.
“How are things at Islamia School?”
\end{quote}

Faisal rarely looked one in the eye when he spoke.
He shifted his eyes sideways and downwards as he talked.

\begin{quote}
“Fine, al-\cor{Hamdulilah}{Hamdulillah}.”

“I hear you have Sheikh Ahmad there now!”

“Yes, masha-Allah.
He’s an excellent Imam.”
Faisal gave a sarcastic smile to the brothers around him.

“Is Dr.\ Baig still there?”

“No, he left.
Teacher Abdullah is the Head now.
Listen, I wanted to ask you about your Khutba,
what you said about Jihad meaning killing the Kuffar.”

“Look it up in the \Quran\ and Hadith, Hassan.
That’s exactly what it means.”

“What about the hadith that the struggle to conquer one’s own ego
is the greatest Jihad?”

“Ibn Taymia said,
‘This hadith has no source and nobody in the field of Islamic knowledge
has narrated it.’
This is supported by many other scholars.”

“But there are many scholars who disagree with Ibn Taymia’s interpretations.”

“There are many wicked scholars who will say anything if you pay them enough;
Scholars for Dollars!”
Faisal’s entourage chuckled.
\end{quote}

I thought it ironic that he should accuse non-Salafi scholars of being
‘Scholars for Dollars’, when it was the Salafi scholars
who had been well financed by the Saudis in their effort
to ‘cleanse’ Islam of heresies and pour doubt on every interpretation
that conflicted with their narrow doctrine.
Several books have been written exposing ‘weak’ or ‘fabricated’ Hadith
such as ‘The ink of a scholar is holier than the blood of a martyr’
and ‘Seek knowledge, even to China’.
The Saudis have also financed ‘edited’ versions of classical texts
to purge them of ‘mistakes’.
They published a translation of Ibn Kathir’s famous \Quran{}ic commentary,
which although regarded highly by Salafis,
nevertheless contains some narrations they don’t approve of.
In a verse about Jews who broke the Sabbath being turned into monkeys and pigs,
Ibn Kathir includes an interpretation that says it is ‘allegorical’
and means only that they were ‘humiliated’ and made ‘lowly’.
Such interpretations don’t suit the literalist Salafi creed;
they prefer to believe that the Jews were literally transformed
into monkeys and pigs, and so you won’t find
the metaphorical interpretation in their ‘edited’ translation.
Similarly Yusuf Ali, in his famous translation of the \Quran,
says regarding one verse,
“Slavery is now no longer applicable in the true spirit of Islam.”
Yet in the new Salafi version, that sentence has been removed.
They still regard slavery as permissible according to Islam.
It is true that the extremist Salafis have fallen out
with the moderate Saudi Salafis,
but their beliefs are basically the same,
which is why it is hard to have any pity for moderate Salafis
when they complain that Islam has been hijacked by the terrorists,
since it is their ideology that created them.

It was clear that anyone who took Faisal’s words to heart
could be capable of virtually anything.
However when I spoke to others about the things he was saying,
most seemed to think nothing would come of it and dismissed him as a crank.
He continued to preach at Willesden Library Centre every other Friday,
his audio tapes on sale in the foyer next to perfume bottles and rosary beads.
I could see that some of those attending were not from the area,
proof that he had built up a dedicated following by then.
I considered complaining to Brent Council or even the police,
but couldn’t bring myself to do it.
My sense of Islamic brotherhood made such a thing seem traitorous.
I tried to convince myself that telling the authorities about someone
who was spreading such hatred was not being a ‘traitor’
since that person was going against Islam, but I still couldn’t do it.
I decided the least I could do was to avoid listening to him
and consoled myself with the thought that it was just hot air.


%Chapter 7:
\chapter{Tuesday Afternoon}

\img{scale=0.4}{911_Second_Plane.jpg}{}

I was teaching in class on September the 11\Sup{th}, 2001,
as the news began filtering through.
Brother Mansoor, our Head Teacher and Sheikh Omar
were standing in the corridor talking anxiously.
I knew from their expressions something was wrong.

\begin{quote}
“What’s happened?”

“Haven’t you heard?”

“An airliner has crashed into a building in New York.”

“Was it an accident?”

“The news is still coming through.”

“It’s two planes!” said sister Saeeda, hurrying along the corridor.
“And a third into the Pentagon!”

“Ya Allah, Ya Allah!” said Sheikh Omar.
\end{quote}

Mansoor, himself a New Yorker, looked distraught.

\begin{quote}
“They’re saying Osama Bin Laden is behind it.”

“Oh no! Please God, don’t let this be Muslims.” Teacher Naseema joined us.
I could hear the whistle in the playground and the children lining up.

“Shall we say anything?”

“No, not yet.”
\end{quote}

I went back to class and tried to get on with the lesson as normal,
but it’s amazing how quickly the children had got wind of something.

\begin{quote}
“Teacher?” said one little boy.
“Tariq said there’s been a big car crash in America; is that true?”

“I’ve no idea.
I haven’t seen the news.
Let’s just concentrate on our work, shall we?”
\end{quote}

A colleague popped her head in the door of my classroom.
Her eyes wide open — more news:
thousands dead, a fourth plane, Muslims blamed, suicide attack.

When school finished I just wanted to go home,
but I rushed to the office instead.
Mansoor was on the phone.

\begin{quote}
“We’ve contacted parents to collect children promptly and go straight home.”

“What’s the latest?”

“Two planes crashed into the World Trade Centre,
one plane has hit the Pentagon and a fourth plane has crashed
into fields in Pennsylvania, probably heading for the White House.”
\end{quote}

Sister Neelam came hurrying in.

\begin{quote}
“Did you hear about the phone calls Hassan?”

“What phone calls?”

“Anonymous threats.
Police are outside at the moment, keeping a low profile.”

“We better stay until all the children are safely home!”
\end{quote}

Mansoor put the phone down.

\begin{quote}
“I’m sure it was just some idiots.
But we can’t take any chances.”

“Any more news? How’s your family Mansoor? Have you been able to phone them?”

“No way to get through at the moment.”

“What shall we do tomorrow?”

“Perhaps we should think about closing the school.”

“Doesn’t that send the wrong message? Why let some idiots affect us?”

“True, but the children’s safety is our priority.
The situation is unpredictable.”
\end{quote}

Neelam and I went to stand by the gates as parents swiftly came and went.
I never saw the playground clear so fast.

\begin{quote}
“Mansoor, I better take the children home!”

“Of course, Hassan.
I’ll phone tonight.”
\end{quote}

I dashed across the playground towards the car park,
pulling my two children by the hand.

\begin{quote}
“What’s wrong, Baba?” asked Fairooza, my daughter.

“Nothing, darling.
I just want to get home quickly.”
\end{quote}

As I got to the gates I saw one of our mums walking past.
She was an English convert.

\begin{quote}
“Have you heard the good news brother?”

“What good news?”

“The Muslims have struck America!” she smiled broadly
and clenched her fist as though she had just scored a goal.

“What?”

“Yes, Mashallah! The Muslims have struck at the heart of the Infidels!”
\end{quote}

Her smiling face made me feel sick.
I felt like slapping it.
But I didn’t.
I just got in the car and drove home.
There was an eerie silence in the streets.
The moment I got home I switched the TV on.
I think I sat there the whole evening, just watching the planes
crash into the World Trade Centre over and over again.
As I stared at the twin towers crumbling and falling
I had a strange feeling of synchronicity —
the sort of feeling you get when you’re in a really foul mood
and it starts to thunder outside.
It was a scary feeling, but at the same time clarifying.
I understood why these men believed it was right
to fly those planes into the twin towers.
I understood their crude dichotomy of believers and infidels,
and the fact that I understood it frightened me.
% TODO: Sheikh Fadil? Or Sheikh Faisal?
I thought about Sheikh Fadil, Abu Hammam and the goal scoring mother.
I began to reflect on my search for meaning and truth,
how I had felt when I first became a practicing Muslim,
on the threshold to a higher understanding of God and spiritual enlightenment.
I realized that somewhere along the way I had been led astray.
Obsessed by form and ritual, worried whether my soap contained pig fat
or the food I just bought contained E\–numbers.
Without noticing, I had been diverted, inch by inch,
until I was now so far removed from my search for God,
I had completely lost Him.
The very things that had led me to Islam, my heart and mind,
now seemed to be locked away in a little box, as though I was afraid of them,
afraid to think for myself, afraid to step outside a life of imitation
and conformity to a set of rules that I hoped would bring me salvation.
Was slavish submission truly God’s ultimate concern for mankind?
Was entry to heaven simply a mechanical process
that did not involve the heart and mind,
but only blind obedience — as the \Quran\ says,
\emph{“We hear and we obey.”}

As I thought about the Muslims who had flown those planes
I knew it was a type of insanity,
an insanity brought about by the sincerest form of faith
and the purest from of devotion.

School was closed on Wednesday, but it was back to normal on Thursday —
although normal is not the right word.
Everyone was in shock.
There were reports of racial attacks:
a brick had been thrown through a window,
some of our children had been subjected to physical
and verbal abuse and the school had received threatening emails.

We decided to start the day with a whole school assembly
and address the issue.

\begin{quote}
“The attacks that took place on Tuesday in the United States
are completely against the teachings of Islam!” said Sheikh Omar.

“It goes against the clear command of Allah, who says;

\emph{‘Whoever kills one innocent person it is as though
he has killed the whole of humanity!’}”
\end{quote}

The children were more subdued than usual.

\begin{quote}
“Don’t take any notice of those who blame you for what happened,”
said Sheikh Omar, “It’s not your fault.
Remember what the \Quran\ says when foolish people say bad things.
\emph{‘Say; Peace! and walk on by.’}”
\end{quote}

\emph{“These attacks are completely against Islam!”}
I heard this many times over the next few weeks, months and years.
They were repeated every time an attack or atrocity was committed
in the name of Islam.
But I began to feel very uneasy about this statement.
If these attacks are completely against Islam why are these people doing it?
Don’t they know their own religion?
Hasn’t anyone pointed out the terrible mistake they’re making?
In the staff room the discussions invariably turned to conspiracy theories.

\begin{quote}
“The Jews did it!”

“Ask yourself who benefits from this? Bin Laden or the Jews?”

“Bin Laden had nothing to do with it”

“Bin Laden is a CIA agent.”

“It’s America’s fault for supporting Israel.”

“The CIA did it!”

“The Jews and the CIA did it!”
\end{quote}

One claim which passed around as solid fact was that 4000 Israelis
did not turn up to the World Trade Centre on the day of the bombing.
It was presented as proof that Israel was behind the attacks.
Rather like Chinese Whispers, this particular myth was modified
to 4000 ‘Jews’ rather than ‘Israelis’,
as someone along the line must have realised
it was highly unlikely that 4000 Israelis worked in the World Trade Centre.
Everyone I knew lapped it up unquestioningly.
No one wanted to look at the facts.
There were five Israeli deaths and around 400 Jewish deaths.
A total of 2,071 occupants of the World Trade Centre died.
That means 15\% of those who died were Jewish.
Since the Jewish population in New York is around 12\%
and for the US as a whole just over 1\%,
the proportion of Jews who died is, if anything,
slightly higher than one might expect.
But Muslims continue to pass around the ‘fact’
that 4000 Jews stayed away from work,
as proof that ‘they’ had prior knowledge.

The average Muslim’s love of conspiracy theories is not confined to Jews.
Wild theories about anything and anyone abound.
One theory is that April Fool’s Day is the day Christians massacred Muslims.
I was once passed a circular at a gathering in Leicester,
stating that playing April Fool pranks were Haram.
It read;

\begin{quote}
“Many of us celebrate what is known as April Fool or,
if it is translated literally, the ‘Trick of April’.
But how much do we know of the bitter secret behind this day?
When the Muslims ruled Spain, approximately one thousand years ago,
they were a force that could not be destroyed.
The Christians wished they could wipe Islam from the face of the earth…
They tried numerous times and never succeeded.
After that, the Kuffar (infidels) sent their spies to Spain
to study and find out the secret of the Muslims’ strength.
They discovered that adhering to Taqwa (Piety\/God-fearing) was the reason.
When the Christians discovered the secret of the Muslims’ strength,
they started to think of strategies to break it.
On this basis they began to send wine and cigars to Spain for free.

This tactic produced results, and the faith of the Muslims began to weaken,
especially among the young generation.
The result of that was that the western Catholic Christians subdued
the whole of Spain and put an end to the Muslim rule of that land
which had lasted for more than 800 years.
The last stronghold of the Muslims, in Grenada, fell on April 1\Sup{st},
hence they considered this to be the ‘Trick of April’.
From that year until the present,
they celebrate this day and consider the Muslims to be fools.”
\end{quote}

Clearly the temptation of free wine and cigars
was too much for the pious Muslims to bear.
I asked the brother who gave me the handout what the evidence was for this.
He said it came to him from ‘good authority,’
but did not clarify exactly what that was
and appeared angry that I should even ask.
Other conspiracies include the “The Simpsons” cartoons,
which aims to corrupt young people and poisoning the water supply
in Muslim countries with aphrodisiacs
to make young men and women sexually promiscuous.

Amongst the most hilarious conspiracy theories
is the hidden message in the Coca-Cola logo,
which when held upside down to a mirror,
spells the Arabic words “La Muhammad, La Makka”
(There is no Muhammad and no Mecca),
or a Nike sports shoe that has Allah written on the sole.
No one has yet explained to me what the Coca-Cola or Nike companies
hoped to achieve by these dastardly plots
or why they should want to suffer a huge drop in sales by insulting Muslims.
One friend offered the explanation that Coca-Cola and Nike are owned by Jews.
He didn’t feel the need to expand any further.
Perhaps they hoped that were enough Muslims to drink from bottles
with the message ‘there is no Makkah and no Muhammad’,
it might make Makkah and Muhammad disappear,
and if enough people were to wear the Nike shoes
in question and walk on Allah, it might kill Him.
(Fortunately Muslims uncovered these plots
before such terrible things could happen.)
This conspiracy mentality is so interwoven
into the very fabric of the way Muslims think,
that they cannot see how ridiculous it makes them look.
But it is no trivial matter, because it justifies prejudices and feelings
of being the victim which in turn justifies hatred and bigotry
towards the perceived perpetrators as well as diverting Muslims
from any serious introspection.

While it frustrated me that most Muslims refused to accept
that there was a problem within ourselves,
it was the extremists that concerned me most.
When I heard about the attacks in America,
I remembered how Sheikh Faisal had unashamedly encouraged Muslims
to do exactly this sort of thing.
I realised how wrong I was to dismiss him
and his kind as ineffectual lunatics.
Faisal was still giving his sermons at Willesden and elsewhere.
I felt I had to do something.
I had a responsibility to the impressionable minds
he was filling full of hatred and violence.
I decided to seek advice from Sheikh Ahmad.
I found his answer inspiring.
He quoted the saying of the Prophet;

\begin{quote}
\itshape
“Help your brother when he is wronging others or being wronged.

The people asked ‘Oh Prophet of God, we can help him if he is being wronged,
but how can we help him when he is wronging others?’

The prophet replied ‘Stop him!’”
\end{quote}

Although I did not know it at the time,
but amongst Sheikh Faisal’s followers was Jermain Lindsey,
one of the July 7\Sup{th} London bombers.
He must have sat, just as I had done, listening to Sheikh Faisal
exhorting Muslims to “Kill the Kuffar”, and that is exactly what he did.
He killed twenty-six Kuffar sitting on an underground train
between Kings Cross and Russell Square in London.
Others who sat there listening to Faisal’s sermons include Zacarias Moussaoui,
implicated in the 9/11 plot, and Richard Reid, the shoe bomber,
who had tried to blow up an American Airlines plane.
Both had faithfully attempted to follow Faisal’s advice to “Kill the Kuffar.”
I did eventually write a letter to both Brent Council
and the Police about Faisal’s speeches.
However, he was allowed to continue speaking for a year and,
if the press are to be believed,
was only picked up by a chance finding of tapesof his hate speeches
in the boot of a car belonging to a terrorist suspect.

When Faisal was finally arrested he claimed he had been sent to the UK
to preach by Sheikh Rajhi, a Saudi Sheikh and a member of Al-Rabwa Office
for Islamic Da\`wa and Guidance in Riyadh.
No doubt Sheikh Rajhi and all the moderate Saudi Salafis
would strenuously deny they are responsible for Faisal’s violent views,
but it is yet more proof — if proof were needed —
that the Salafi creed forms the basis of the Jihadi ideology.
Sheikh Faisal claimed that he was only saying what the \Quran\
and the Prophet said and that if he were put on \cor{trail}{trial}
then so too would be Islam.
The prospect of Islam being on trial in a British court intrigued me,
and I was eager to hear the arguments for and against the religion of Islam
being the source of Faisal’s hate speeches.
However the judge side-stepped that potentially explosive issue, saying;

\begin{quote}
“The defendant’s claim \cor{to be}{} to be able to justify any utterance
of his with reference to the Koran or the Hadith…
may well have a relevance to your consideration of his intention
in saying what he did, but it does not of itself afford him a defence in law…
that the use of those words in the circumstances he spoke them
are proved to be contrary to the law of this country and thus an offence,
no more or less than a similar citation from anyone else’s holy book
including the Bible would be.”
\end{quote}

The jury convicted Faisal of soliciting murder,
and he was sentenced to nine years in prison.
But the question of how far Islam itself was responsible
for the terrorist attacks wouldn’t go away,
and I was beginning to believe more and more that, at the very least,
Islam did lend itself rather too easily to extremist interpretation.
Soon after 9/11 Bin Laden gave an interview to Al-Jazeera saying;

\begin{quote}
“The scholars and people of the knowledge amongst them say
that if the disbelievers were to kill our children and women,
then we should not feel ashamed to do the same to them.
Allah says;

\emph{‘And if you punish (your enemy), then punish them with the like
of that with which you were afflicted’} (\QRef{16:126})”
\end{quote}

It would have been easier to dismiss Bin Laden and his group of fanatics
were they the only ones who propagated
such a hard-line and literalist understanding of Islam,
but there were many Muslims who held similar views
and justified the September 11\Sup{th} attacks according to Islam.
After breaking my fast during Ramadan,
a friend and I began discussing the meaning and limits of Jihad.

\begin{quote}
“Even if Bin Laden is fighting Jihad in defence of oppressed Muslims —
as he claims — the \Quran\ clearly forbids killing innocent people.
God said, ‘He who kills one innocent soul it is as though
he has killed the whole of mankind.’ Look how serious it is.
How can any true Muslim even think about doing such a thing?”

“Hassan, don’t be like those foolish Muslims who run around saying,
‘The \Quran\ says killing innocents is forbidden,
therefore these terrorists are going against Islam’ —
any scholar will tell you it’s not that simple.
Firstly, there’s a difference between the people
who worked at the World Trade Centre and the average Joe.
These people maintain the economic infrastructure
that supports American aggression against Muslims.”

“Does that mean we have the right to murder them?
And what about the people on the planes,
the firemen and other emergency services —
surely you can’t deny they’re innocent?”

“The Prophet was asked about the people in the homes of Mushrikun
(Polytheists) when they are attacked at night and their women
and children are affected, he said,
‘They are part of them.’ Also Ibn Qudamah, may Allah have mercy on him, said:
‘It is permissible to use Catapult because the Prophet
used Catapult on the people of Ta\´if.’
The catapult of course does not distinguish between soldier and civilian.
It is a sad fact of war, Hassan, that innocents are sometimes killed.
All the scholars are agreed that the killing of innocents
in such circumstances is permitted,
so long as they are not the intended target.
The truth is that the US and her allies have killed
far more innocents than the so-called terrorists.
When Muslims do it, it’s murder; when America does it,
it’s called ‘collateral damage.’”

“The actions of the US government don’t concern me;
what concerns me is the morality of what I claim to believe in.
I cannot accept that it is ever right to make innocent people pay
for the crimes of others?”

“The \Quran\ says that we can retaliate in the same manner as they attack us.
So if they kill our innocents we can do likewise.”

“It doesn’t mean that we can break the rule on killing innocents.”

“Have a look at the evidence presented by Sheikh Shu\`aybee.”
He handed me little pamphlet.
“If someone votes for a Government that is waging war on Muslims,
then they are not truly innocent are they?
And if they are paying taxes to that government to help it
buy guns and bombs that are used to kill innocent Muslims
then they are not truly innocent are they?”

“That’s ridiculous! Ordinary people are not responsible
for the actions of their government?
And we have no choice about taxes.”

“But isn’t that what Democracy is supposed to be about?
Government by the people for the people?”

“Well, yes, but that doesn’t make them responsible
for decisions they didn’t make.
Look, if you start using such a narrow definition
of who is innocent then no one is truly innocent.
Use your common sense for God’s sake.”
\end{quote}

I couldn’t understand how such a devout and pious Muslim,
who carried out every religious duty to a fault,
could be so confident in beliefs that were so obviously immoral,
so obviously wrong.
This led me to an even more troubling thought; was I like him?
Not in the sense of trying to justify killing innocent people,
but what if I was also confident in beliefs that were wrong?

It was a thought I couldn’t bear to think.
But the more I tried to push it from my mind the more it haunted me.
What if the \Quran\ is not the word of God?
What if Muhammad is not the prophet of God?
And what if Islam is not true?
But fear of burning in Hell terrified me
and would not allow me to think about the answer rationally.


%Chapter 8:
\chapter{Revelation \& Reason}

\img{scale=0.6}{Man_Reading_Koran.jpg}{}

\begin{quote}
\emph{“As for those women from whom you fear rebellion (first)
admonish them (next), refuse to share their beds, (and last) hit them.”}
(\QRef{4:34})
\end{quote}

Amongst the verses of the \Quran\ that troubled me, this one stood out.
I tried many times to explain it in a way that made sense,
but it constantly gnawed at my conscience.
I had already started to look much more critically
at many things in Islam when I heard about Rezeya’s story.
She was an Asian woman in her fifties whom I had known
for many years through her volunteer work for a Muslim charity.
I would never have imagined what was going on at home
if she hadn’t been present when I was discussing
verse 34 of Suratu\–Nisa with some friends.

She approached me privately after the discussion
and confided in me that her husband beat her at home.
She said he would do it for the slightest reason.
In other ways he was a good man who cared for his family
and provided a comfortable life for them all, so she endured the abuse,
seeking solace in the knowledge that God saw her suffering
and would one day reward her for her patience.
When they moved from Pakistan to England, things improved slightly.
Her husband was busy building up the family business,
and she was busy with their six children.
When the last of the children had moved out
and her husband began to spend more time at home,
the beatings started again.
Rezeya was a very pious Muslim and never missed an opportunity
to say extra prayers or recite \Quran.
Like many Muslims of non-Arab origin, she couldn’t understand Arabic,
and it was only after learning English and reading translations
that she eventually understood the meaning of what she had been reciting.
She said that one night she came upon the verse
giving permission for a husband to hit his wife.
She spent the night crying alone in her bedroom.
I told her that this verse didn’t mean her husband had the right to beat her.
I said it more to comfort her than anything.

Of course I knew wife beating was an evil that exists in all cultures
and all societies, but giving Divine sanction for a man to hit his wife —
as the \Quran\ does — surely only made things worse?
I read books and articles about the subject
and spoke to as many Sheikhs as I could.
I was assured that the conditions and restrictions
which the \Quran\ placed upon wife beating amounted to a virtual ban,
particularly since Muslims are obligated to follow the Prophet’s example
and he never laid a finger on his wives\footnotemark,
saying “The best of you is the best to his wife.”
These sheikhs said that the words ‘Hit them’ simply referred
to a symbolic show of displeasure to be administered
using a feather, handkerchief or the ever versatile Miswak.
There were even some, so desperate to re-write
‘awkward’ passages of the \Quran,
that they claimed the words ‘and hit them!’ (wadriboohunna)
actually means “and leave them alone!” or even “Make love to them!”
The more I thought about such explanations,
the more illogical and ridiculous they seemed.
If it is true that the conditions and restrictions do amount to a virtual ban,
then why have a verse saying ‘Hit them’ at all?
As for hitting someone with a feather, handkerchief or Miswak,
the very thought is ludicrous.
But the most dishonest of all were the ones who tried to claim
that “Hit them” meant “Leave them.”
Or “Make love to them.”
Not only does this reveal complete ignorance of Arabic,
but it makes nonsense of the restrictions the prophet
placed on the type of beating allowed.
The fundamentalists, on the other hand, unashamedly proclaimed
that hitting one’s wife meant exactly that, to hit her, plain and simple.
Although I found their view distressing, I had to respect their honesty.

One day while browsing through the books
at \cor{Regents}{Regent’s} Park Mosque,
I came across one called “Marital Discord”
by a professor of Shari\`ah in Saudi Arabia.
Although it represented the puritanical Salafi view,
it was far truer to the context of the \Quran\
than the explanation of the modernists.
It quoted classical scholars to explain that ‘rebellion’ meant behaviour
such as \emph{“Leaving the house without the permission of her husband,
preventing him from sexually enjoying her
and not beautifying her self for him.”}
To correct this behaviour one must first administer verbal admonition
and next refuse to share her bed and finally:

\begin{quote}
\emph{%
“Perhaps the solution to the problem requires some harshness and toughness.
There are some people who cannot be rectified
by good behaviour and soft advice.
Kindness and softness just makes such people more arrogant and haughty.
However if they are met with toughness,
then they respond by cooling down and ending their defiance”}

They must be \emph{‘taken by the hand’}
and given a \emph{‘light beating’}, defined as
\emph{“Beating which does not cause bleeding nor does one fear from it
injury to life or limb or tearing of the skin, breakage or disfigurement.”}
The author quotes a saying from Asma a daughter of Abu Bakr:
\emph{“I was the fourth of four wives of al\–Zubair.
Whenever he would reprimand one of us,
he would break off a branch from the wooden clothes hangers
and beat her with it until he broke it over her.”}
\end{quote}

The book also discusses what a wife should do if a man ‘rebels’ against her,
which in the man’s case is defined as, ‘hating, cursing or abusing her.’
In response, the wife must also follow three steps,
but they are very different from those prescribed for her husband:

\begin{quote}
“First, the wife should use all of her charms,
intuition and wisdom to try to discover the reason
behind her husband’s estrangement.
She should study the causes in a subtle and clever manner.”

“Second, she should admonish her husband,
reminding him of his responsibility in front of Allah towards his wife,
such as proper behaviour and good conduct.”

“Third, the woman should try to please her husband to make things right.
She should try to seek his pleasure.
She should seek the means that will affect him
and change his behaviour to a better way
and provide solutions to their problems.
This is what is alluded to in the \Quran.”

\emph{“And if a woman fears rebellion or desertion on her husband’s part
there is no sin for them both if they make terms of peace between themselves
and making peace is better.”} (\QRef{4:128})

“That is, there is no sin upon them if they come to some kind of agreement
between themselves in which she may give up some of her due rights
in order to stay in the marriage.
For example, she may give up some of her rights
to maintenance or housing with her.
Or she may give up all of either or both of those rights
in order to remain under his protection in a noble marriage.
Or she may give up part or all of her dower
in exchange for his divorcing her.”
\end{quote}

I had tried to convince myself that it was my instincts
that were wrong and that I should just submit to the Divine word,
without trying to apply my own defective reasoning to it.
God cannot be wrong?
I must be wrong, and I should simply accept
that He knows that which I do not.
But the doubts remained and I grew tired of trying
to suppress how I really felt.
I realised too that it wasn’t truly God I was questioning,
but the words that I had for so long believed were His.
In fact it was my conviction that if there was a God
he was far greater and far more wonderful
than to have such words attributed to him.
I simply didn’t believe this verse was the ‘Word of God.’
I also knew that rejecting one verse undermined the whole \Quran,
but it was either that or lie to myself about how I truly felt.
I soon found myself doubting other verses, particularly those about Hell.

\begin{quote}
\itshape
“As for those who reject Our Signs, We will roast them in a Fire.
Every time their skins are burned off,
We will replace them with new skins so that they can taste the punishment.
Allah is Almighty, All-Wise.” (\QRef{4:56})
\end{quote}

Muslim scholars are quick to point out that it is the skin
that feels pain and that this is God’s way of emphasising the relentless
and eternal nature of the agony the unbeliever experiences in Hell.
Amongst the many forms of torture described in the \Quran\
are drinks of molten brass and thorny fruit that shred
the intestines and resemble the head of Satan.
One verse graphically describes
how the unbelievers will be melted from inside:

\begin{quote}
\itshape
“As for those who disbelieve, garments of fire will be cut out for them;
boiling fluid will be poured over their heads,
melting that which is in their bellies and their skins too,
and for them are hooked rods of iron.
Whenever, in their anguish, they try to escape they are driven back therein
and (it will be said): Taste the doom of burning.”
(\QRef{22:19–22})
\end{quote}

The scholar Ibn Kathir relates a saying of the Prophet regarding this verse,
\emph{“The scorching fluid poured on their heads will pierce through the skull,
flow into the belly, and strip away the insides until it reaches the feet.
This is the melting.
Then they will be made new again.”}

\begin{quote}
“I’ve always found it difficult to square
the idea of Hell with any sort of logic, let alone compassion,”
I told Farouq, as we sat in the canteen at Regent’s Park Mosque.
“What is there to be gained from torturing someone for eternity?”

“Doesn’t that frighten you?”

“Of course it does.”

“That’s the point!
It is supposed to frighten you so much that you will be good!”

“There are lots of non-Muslims who are good!”

“Not believing in God is the worst thing you can do!”

“Many non-Muslims do believe in God.”

“Not properly.
They associate partners with God.”

“So you’re saying believe in Islam or burn horribly forever!”

“No, not exactly.”

“What are you saying then?”

“Those who reject God.”

“OK, so believe in God — properly of course — or burn horribly forever!”

“Well if you must put it like that then, yes.”

“Do you think fear is a good way to make people believe?”

“Sometimes people need a little nudge to make them do the right thing.”

“I can see how that might nudge people into doing what you want them to do.
Is that sort of believer God wants?
People who obey him because they’re scared shitless?
Frankly, I have more respect for those who refuse,
even over their own salvation.”

“Well it’s not to make you believe.
It’s describing how those who don’t believe will be punished.”

“You’re changing your argument now.
OK then, why torture someone like that? What is to be gained from it?”

“Oh Hassan, just say your prayers, and God will answer you!”
\end{quote}

The problem was that I was finding it difficult to say my prayers.
They had become mechanical,
and I began to skip them once I got home from school.
One day I was sitting in a mosque after Friday prayer
as the Imam started to make Du\`a.
I held out my hands in readiness
to chant the collective Ameen with everyone else.

\begin{quote}
“O Allah, humiliate the unbelievers!”

“Oh Allah, demolish the houses of the unbelievers!”

“O Allah, destroy… punish… debase…”
\end{quote}

As he said each line, I found I couldn’t say “Ameen” anymore.
I put my hands down, got up and walked out.

I didn’t know what I believed anymore.
I asked my fellow Muslims, whenever I had the chance,
what made them believe in Islam.
Most thought it was an odd question to ask.
The replies I usually got were ‘Because I know it’s the truth’
or ‘Because Islam has given me certainty, peace of heart,
changed my life’ or ‘Because I want to go to heaven’.
But the answers didn’t satisfy me.
Believing something is true doesn’t mean it is true,
nor is experiencing peace or a better life proof that something is true.

\begin{quote}
“How do I know that God really did send down the \Quran?”
I asked Tarek, an old friend from SOAS.

“Hassan, you’ve read the life of Muhammad.
You know he was a good man.
Do you really think he made it up himself?”

“Firstly, how good he was depends on your perspective.
Secondly, being a good person doesn’t prove the \Quran\ is from God.
Many good men claimed they received messages from God.
Why should I believe one over the other?
During Muhammad’s lifetime there was another ‘Prophet’ called Musaylama
who claimed to receive messages from God.
Of course we now call him the ‘false prophet’.
But had he been victorious and not Muhammad,
we would be praising his good character and declaring,
‘There is no God but Allah and Musaylama is the Prophet of God.’”

“Musaylama never came up with anything like the \Quran.
You cannot find a book like it.
Look at the wisdom and beauty it contains.
You of all people should be able to recognise that it is the truth from God.”

“I really don’t know any more, Tarek.
I really don’t know.
Yes, there are many wise and wonderful words in the \Quran.
But there are also things that are not so wonderful.
If my judgement is based purely on internal evidence,
a completely rational and unemotional study of the \Quran,
then I have to say that there are things
I find difficult to believe are the words of God.”

“Faith is not about rational study or reason.
You used to tell me about how the Orientalists
had studied Islam more deeply than many Muslims
and yet were unable to see the truth because their hearts were closed.”

“But if the proof is not objective analysis and instead is an inner feeling,
a personal revelation, then how do I know I’m not deluding myself?
There are people who ‘know’ that Jesus is the son of God, others ‘know’
that Brahma created the universe, that Vishnu preserves it
and that Shiva destroys it, and still others ‘know’
that space aliens abducted them.
This sort of personal experience or insight is not proof.
It is subjective and un-verifiable.”

“Un-verifiable by who and what, Hassan? By scientific tests?
Even those who only accept the evidence of modern science or logic
are following a belief, a belief that it is infallible.
But there are more things in this world
than can be grasped by science or logic.”

“Yes… perhaps, but then who’s to say one person’s belief
is any truer than another’s?”

“Perhaps no one’s beliefs are better than another’s, Hassan.
Perhaps it’s just a case of what works for you?”

“What if nothing works for me?”
\end{quote}

I felt frightened and guilty about questioning Islam —
frightened God would punish me in Hell,
and guilty about having doubts when Islam —
and more importantly Muslims — were under attack right, left and centre.
I felt like a traitor to my brothers and sisters.
I thought of a verse in the \Quran:

\begin{quote}
\emph{“And some men worship Allah on an edge, if any good reaches him,
he is content, but if trial befalls him, he turns back on his face,
losing the world of this life and the Hereafter.
This is the greatest loss.”}
\end{quote}

Was this me? The selfish fair-weather believer who believes only
when things are going well, but runs away when things get tough?
Was I about to lose it all — this world and the next?
Should I suppress all doubts and stay loyal to Muslims
in their time of hardship?
Is that what being a good Muslim is about?
Loyalty, even though it is to a lie?
I found so much in Islam that terrified me about having doubts.
The psychological pressure to suppress them was enormous.
Questioning something I had believed in utterly
all my adult life was hard enough,
but when the consequences seemed so severe
and the sense of betrayal so strong, doubt became unthinkable.

\footnotetext{Actually this is not strictly true.
I later discovered a hadith where the prophet slaps Aisha on the chest
causing her pain, for following him out of the tent during the night.
See: Sahih Muslim: Book 004, Number 2127}


%Chapter 9:
\chapter{London Bombings}

\img{scale=1.2}{London_Bombings.jpg}{}

Mansoor, our Head Teacher, was late for work one Thursday morning.

\begin{quote}
“I heard there’s an electrical fault on the Underground,
some sort of power surge,” said sister Noura, the secretary.
“He’s probably been delayed.”

“You mean more than usual,” I joked.
\end{quote}

An hour later, he still hadn’t arrived.
I tried phoning him on my mobile, but it wouldn’t connect.
At playtime we heard reports
that there had been explosions at several tube stations.
My heart sank and Mansoor’s lateness now seemed ominous.
As the morning progressed, the full horror of what had happened became clear.
Three underground trains and a London bus —
packed with commuters — had been blown up.
I had hoped against hope that something like this
wouldn’t happen here in the UK, but in all honesty I had been expecting it.
It was no secret that the Jihadis regarded the UK as a prime target.

What astonished everyone was that the bombers were British.
Their leader, Mohammad Siddique Khan,
appeared to be a well adjusted and integrated young man.
School friends pointed out that he had both white and Asian friends
and seemed friendly, polite and considerate — a nice guy.
But none of these things astonished me.
I recognised the possibility of being both a genuinely good person
and yet a person capable of shockingly bigoted views.
This is the power of some extreme forms of faith:
it can make a good person say and do things
he wouldn’t even dream of otherwise.
I also recognised the bombers’ crisis of identity,
the appearance of normality, and the storm of confusion within.
Being a devout Muslim in the West creates an inner conflict
that is hard to resolve.
This is because — for the devout Muslim —
Islam is a complete way of life, a “Divine Code of Life”.
It guides every aspect of one’s personal and public life,
from the brushing of one’s teeth to Social, Political and Economic issues
such as laws governing trade practises and punishments for adultery.
Living in the UK, or any modern Western society, presents Muslims with
the day-to-day reality of a competing world-view
that challenges traditional Islamic values.
Ideas such as Democracy, Freedom, Equality and Pluralism
and issues such as Human Rights, Sexuality and the relationship
between men and women create a dilemma for Muslims.
How should they react to these issues and still remain faithful
to the ideal of Islam as a perfect and unchanging ‘Divine Code of life?’
Do these things undermine Islamic values?
Can Muslims selectively incorporate what some may regard
as the good ideas and reject the bad?
Can Muslims apply human reasoning to find new interpretations of Islam
that embrace these ideas?
Or have all the various permutations of Islam
been comprehensively detailed by the Scholars of the past?
Should Muslims stick resolutely to the classical model
of Islamic Law and the Islamic State?
The result is a crisis and confusion amongst Muslims.

When faced with such a conflict between a perfect ‘Divine Code’
and the modern western world, Muslims have no choice
but to conclude that it is the modern world,
dominated by non-Muslim philosophies and values,
that is the source of the problem.
Islam by its very nature compels Muslims to look back to the past
for the perfect model for society and to be suspicious of anything new.
No matter what group, party or sect you ask, whether extremist,
moderate or esoteric, they are all bound to accept that the example
set by Muhammad 1400 years ago is the example they must follow.
So though almost every Muslim today feels this conflict,
few can admit that Islam needs to change because
that is tantamount to saying God got it wrong the first time!

So Muslim apologists do their best to put a modernistic gloss
on problematic issues, while claiming
that the ‘perfect’ and ‘ideal’ Islamic State would resolve everything.
Of course when one points to places where Shari\`ah has been tried,
such as Saudi Arabia, Sudan, Pakistan and Afghanistan,
almost every Muslim you meet will say,
“Oh, they are not doing it properly!” or
“That country is not really Islamic,” adding,
“But if we were to return to \emph{true} Islam everything would be fine.”
This is of course a very convenient way of avoiding
the implications of applying Islamic Law in our day and age.

It is this confusion amongst contemporary Muslims
that extremist preachers following the Jihadi ideology
exploit to gain support for their violent tactics.
They represent arguably the most obvious and simplistic response
to the conflict Muslims feel.
Back to basics! Cut away all the accumulated debris of 1400 years
of innovations and return to the ‘pure’ Islam,
precisely as it was practised at the time of the prophet.
The main vehicle for achieving this is Jihad (Struggle)
against the Kuffar and the setting up of the Khilafah (Islamic State).
They see the failure and weakness of Muslims as being due to
their having strayed far away from the ‘pure’ Islam of the prophet
and having abandoned the essential duty of Jihad.

This simplistic, black and white literalist version of Islam provides
an irresistible solution to disaffected and alienated young Muslims.
They see the Muslim leaders in the UK and even their families
as only paying ‘lip service’ to the Islamic ideals they claim to follow.
These young Muslims compare the glory and supremacy
that the Islamic State had during its ‘Golden Age’
and the abject weakness, ignorance and poverty of the Muslim people today
and conclude that the decline of Muslims is due to the fact
that they are not following the ‘true’ Islamic teachings.
So these young Muslims join the Jihad to re\–establish
the ‘true’ teachings through whatever means they see fit.

Having accepted this approach they see every setback
and failure to defeat the enemies of Islam
as a result of not following Islam ‘properly.’
They are drawn ever deeper into extremism, applying every minute detail
of \Quran\ and Sunnah ever more rigorously and harshly,
to the extent that Islam loses it\cor{’}{}s moral and spiritual dimension
entirely and becomes merely a mechanical exercise of applying a bewildering
and complicated set of rules, where human reasoning plays no part.
It is simply a matter of following the ‘evidence’ (daleel)
from the \Quran\ and Sunna ‘literally’.
As a result the extremists have no moral dilemma over the fact
that this perfect ‘Islamic State’ amounts to little more than
a brutal system of punishments or that this glorious Jihad boils down to
indiscriminate slaughter of innocent civilians on an underground train.

Of course, the vast majority of Muslims are strongly opposed \cor{}{to}
the violent actions of suicide bombers
who take the lives of innocents,
but as yet they have been unable to conclusively invalidate
the arguments of the extremists.
There are plenty of apologists who explain that ‘Islam is a religion of Peace’
and that ‘Islam is against terrorism.’
But such slogans are for the Western media and carry no authority at all
with the extremists who only accept arguments based on \Quran\ and Sunnah.
Even those Moderate scholars who do use Islamic arguments
are yet to win the battle against the extremists.

Muhammad Siddique Khan said in his Martydom Video:

\begin{quote}
“Our driving motivation doesn’t come from tangible commodities
that this world has to offer.
Our religion is Islam — obedience to the one true God, Allah,
and following the footsteps of the final prophet and messenger Muhammad.
This is how our ethical stances are dictated.”
\end{quote}

Anyone who truly wishes to understand the motivation of the Jihadis
should pay very close attention to these words:
“This is how our ethical stances are dictated.”
Regardless of how most moderate Muslims wish to interpret Islam,
the fact is that there are Muslims who find clear and compelling justification
in the religion of Islam to slaughter innocent people
travelling to work on an underground train.
The excuse that these people are twisting the peaceful religion of Islam
is simply not good enough.
The reality is that there persists to be a significant militant minority
who stubbornly refuse to see it that way.
If one Islamic Scholar produces evidence
that killing innocent people is prohibited,
another will produce evidence that in certain circumstances it is permissible.
If one scholar produces evidence that suicide bombing is Haram,
another will produce evidence that it is the highest form of Martyrdom.
The result is that the majority of Muslims are confused
and caught within a dilemma of conscience.
Either they concede that the extremists have a point
and end up trying to defend acts that are indefensible
or they do their best to quote the peaceful parts of the \Quran\ and Hadith
and hope that the person they are speaking to
doesn’t know too much about Islam.

Mansoor finally turned up in a taxi, to everyone’s great relief.
He explained he’d been re\cor{-}{}directed to Kings Cross,
just as some of the first victims were emerging,
and stayed on the scene to help.
He looked shaken and dishevelled.

\begin{quote}
“There were people coming out, coughing, covered in soot;
no-one knew what was going on; everyone tried to help.
Then the police and ambulance crews took over.
There was one man who had blood all over him;
his clothes were black and in shreds; they put him on a stretcher.”

“Did they say whether it was terrorists?”

“No, I heard someone say it was a bomb,” he said.
“I don’t know what happened, but it looks really bad, Hassan, really bad.”
\end{quote}

In the days and weeks that followed the London attacks,
there were the usual denials and conspiracy theories.

\begin{quote}
“They were tricked into it,” said Saeeda,
the lady who served the dinner in the kitchens.

“Tricked into carrying bombs and blowing themselves and the passengers up?”

“Yes, didn’t you hear that they bought return tickets.
Why would they do that?”

“Perhaps so that they wouldn’t arouse suspicion.”

“And on the morning of the attack there was a simulated attack going on
as part of a training exercise — now don’t tell me that is coincidence.”

“I’m not sure what that proves.”

“It proves it was all planned by the secret services.”
\end{quote}

The Muslims around me seemed to be so deep in denial it felt hopeless.


%Chapter 10:
\chapter{Reforming the Unreformable}

\img{scale=0.4}{Quranic_Script.jpg}{}

Despite my doubts I still clung desperately to the belief
that the \Quran\ was from God but it’s deeper message
was misunderstood by literalists.
I scoured the sayings of Islamic scholars in my efforts
to find evidence to refute the hard-liners.
But the more I searched the more I realized it was a futile exercise.
No matter how many sound arguments I put forward,
there were an equal number — perhaps more — that contradicted them.
I could never entirely invalidate the views of the literalists,
simply because the literalist tradition is every bit as valid as any other.
The fact is that the \Quran\ often contradicts itself,
providing evidence for those who wish to see freedom of religion
in verses such as \emph{“There is no compulsion in religion”}
and evidence for those who wish to see a perpetual Jihad against
the unbelievers in verses such as
\emph{“When the forbidden months are past, then fight and slay the Pagans
wherever ye find them, and seize them, beleaguer them,
and lie in wait for them in every stratagem (of war).”}
Everyone can find what he wants in the \Quran:
peace, love and tolerance or war, hatred and prejudice.

Islamic scholars sought to resolve the contradictions by using the idea
of Naskh (Abrogation) which takes its legitimacy from the \Quran:

\begin{quote}
\emph{“Nothing of our revelation do we abrogate or cause to be forgotten,
except that we bring (in place) one better or the like thereof.”}
(\QRef{2:106})
\end{quote}

Many of the conflicting verses come from the two main periods
of revelation in Mecca and Medina.
In the earlier Meccan period, when Muhammad was beginning his mission,
the verses were of a more general and spiritual nature and included
a much softer and conciliatory approach towards unbelievers.
The later Medinan period, however, was much harsher,
as it was revealed during a time of great hostility,
when the Meccan polytheists were trying to crush the Muslims in Medina.
The verses from this period are also concerned with making laws
and guidelines for the newly formed Islamic State in Medina.
The principle of abrogation actually works in favour of the extremists,
because it is the later, harsher verses
that supersede the earlier and more tolerant ones.
So for example, the earlier verse saying “There is no compulsion in religion”
is considered by many traditional scholars to have been abrogated
by the later verse commanding Muslims to fight unbelievers —
the so\–called ‘Verse of the Sword.’

However one modern scholar, Abdullahi Ahmed An\–Na\`im,
has turned this argument on its head by arguing
that today we must reverse this process
and use the earlier Meccan verses to abrogate the later Medinan ones.
The reason, he says, is that the later ones were revealed to deal
with the situation at the time and so are confined to their context,
while the earlier message of Mecca is
“The eternal and fundamental message of Islam,
emphasizing the inherent dignity of all human beings,
regardless of gender, religious belief, race…
equality between men and women and complete freedom of choice
in matters of religion and faith.”
An\–Na\`im was a student of Mahmoud Ta Ha who originally propagated this idea
in Sudan, and as a result, was executed for apostasy.
His followers fled Sudan, and Abdullahi An\–Na\`im himself
now lives in America.
An\–Na\`im’s argument sounds good, but is in fact an over\–generalization.
It simply isn’t true that all the harsh verses fall into the Medinan period
and the tolerant verses fall into the Meccan period.
There are instances of the opposite being the case.
For example Surah al\–Bayyinah is considered by many to be a Meccan Surah
yet it is extremely intolerant of Jews and Christians as well as polytheists;

\begin{quote}
\emph{“Those who disbelieve, among the People of the Book (Christians \& Jews)
and among the Polytheists, will be in Hell-Fire, to dwell therein for ever.
They are the worst of creatures.”}
(\QRef{98:6})
\end{quote}

While a verse from Surah al-Ma\´idah, that most would regard
as Medinan seems quite tolerant;

\begin{quote}
\emph{“And nearest among them in love to the believers wilt thou find those
who say ‘we are Christians’ because amongst these are men devoted to learning
and men who have renounced the world, and they are not arrogant.”}
(\QRef{5:82})
\end{quote}

More importantly, many verses and Hadith are difficult or impossible
to date with absolute certainty making such a distinction arbitrary.

I certainly agreed that many verses of the \Quran\ were not relevant
for today’s conditions, but I wasn’t going to get bogged down
with arguments over abrogation or when and why verses were revealed.
My own solution was simple:
human reason should be the deciding factor of what is
or what is not relevant to our own situation.
It was a solution that created more problems than it solved,
not least of which was to cast doubt on the Divine nature
of the holy text itself, yet it I clung to it out of desperation.
The only other alternative was to reject Islam completely
and I simply wasn’t able or ready to do that.
Unsurprisingly I found that most Muslims rejected the idea out of hand
and I was told time and time again that Muslims
cannot ‘cherry pick’ what is relevant and what is not.
I was almost attacked by an angry Algerian brother in Regent’s Park Mosque
when I suggested to him that verse 25 of Sura Nisa was limited
to it\cor{’}{}s time period and was no\cor{-}{ }longer
relevant in our day and age.
The verse directs those believers unable to marry free women
to marry their slave girls.

Instead of arguing with people I decided to write articles about the need
for reform but didn’t have much success in getting them published.
At school I went to see Sheikh Omar in his office
and discussed my thoughts with him.
I knew that as a Sufi, he didn’t hold literalist views
and so might be more receptive to my ideas.
I was encouraged to find he agreed that not all verses of the \Quran\
were relevant to our situation, but he insisted that this
was not because part of the \Quran’s message was limited to its time period,
but because it offered different solutions for different situations
and that the verses about Jihad against the Meccan polytheists
were different from today’s situation.
My problem with that argument is that it still leaves the possibility
that such verses could be relevant under the right circumstances
and that it is merely a matter of opinion what ‘the right circumstances’ are.

I had many discussions with my fellow teachers at Islamia School.
During one conversation in the staff room with a group of about six teachers,
we got on to the subject of equality,
and the fact that Shari\`ah Law gives a woman half the legal status of a man.
In the case of inheritance, a daughter will only receive half
what her brother receives, and where a woman is a witness,
her testimony is considered half that of a man’s.

\begin{quote}
“That’s because the \Quran\ says, ‘Men are protectors of women,’”
said sister Amatilah
“And the Prophet said, ‘Women are deficient in religion and mind,’”

“What does ‘deficient in mind’ mean?” asked sister Nargis,
clearly a little unsettled by the hadith.

“Firstly God made men stronger, and so they are the protectors and providers.
A woman’s deficiency in religion is because
she cannot pray all the prayers, due to menstruation.
Her deficiency in mind is because of the nature of women and their role
as mothers and the fact that women can be emotional and irrational,
which makes their testimony less reliable.”

“The world has changed, sister Amatilah,” I said.
“Many women also work and support their families financially —
just look around this room; there are six women and one man.
To deny them a full inheritance on the basis that their husbands
will look after them is simply not fair.
As for their testimony being unreliable,
men can be just as emotional and irrational.
We cannot justify such a law in this day and age.
In this instance the \Quran\ and Sunnah
are not relevant to today’s situation.”

“Astaghfirullah, Hassan, how can you say that the \Quran\ and Sunnah
are no longer relevant to our day and age?
The \Quran\ is the last revelation to mankind and will remain
till the end of time as the perfect guide for all places and at all times.”

“I’m not saying all of it is not relevant just some parts.
Prophet Muhammad was first and foremost a prophet
sent to his people with solutions for the situation at the time.
The \Quran\ says that Muhammad was sent with the message
\emph{‘So that you can warn the Mother of Cities (Makkah)
and it\cor{’}{}s surrounds.’}
Clearly the message he brought was in response
to the circumstances of 7\Sup{th} century Arabia
and we are not required to follow every detail of it today.”

“The \Quran\ says that Prophet Muhammad was sent ‘as a mercy to the world.’”
said sister Amatilah.

“That doesn’t mean the whole of his message was meant to apply literally
at all times all over the world.”
I replied,
“All the prophets God sent came with a specific message to their people
as well as a universal message of belief in God.
It is a mistake to think Muhammad’s message was different.”

“But Muhammad \emph{was} different.” insisted sister Amatilah,
“He was ‘The Seal of Prophets’; he brought the final and complete message.”

“Just because he was the last prophet doesn’t mean
the whole of his message applies for all times.
It simply means that man has now reached a stage in his development
where it is no\cor{-}{ }longer necessary to send any more prophets.”

“Exactly, no more prophets will come because the message of Islam is the final
and complete message for mankind until the Day of Judgment.”
said sister Amatilah triumphantly.
\end{quote}

I knew I wasn’t getting anywhere and was glad that it was time
to bring the children in from the playground.
Later that day two of the sisters who had been present at the discussion
came up to me, quite independently of each other and said they agreed with me.
Both said it in hushed tones and made sure no one else was listening.
This was something that I witnessed quite often.
Muslims who would remain silent and unquestioning
when in the company of other Muslims,
but would express their feelings or reservations when they knew
that you too had similar reservations — but even then only in private.
The reason they didn’t challenge people like sister Amatullah was quite simple.
They didn’t have enough knowledge about Islam, while the sister Amatilah did.
Although there were many Muslims eager to support
a more liberal interpretation of Islam,
those who had the better Islamic knowledge were resistant to change.
It was a frustrating situation and I was growing weary.
Not least because my efforts to reform Islam only served
to undermine my faith in it.


%Chapter 11:
\chapter{Religion}

\img{scale=0.6}{Neasden_Temple.jpg}{}

The Swaminarayan Hindu Temple in Neasden is a stunning building,
with its mass of domes, elaborately carved walls and huge marble staircase.
I drove past it many times, but never allowed myself to go in;
it was a place of Shirk, idol worship,
the most unforgivable crime in the sight of God.
But by now my faith had dwindled to the point
that I no longer regarded it with the same aversion.
On the contrary I was intensely curious to see
what it was like inside and how Hindus prayed.
One day while passing the temple I decided to park and take a look inside.
I cautiously walked up to the perimeter gates,
it felt like I was doing something very naughty.
I saw a security guard sitting in a little cabin
and I thought he was going to stop me,
but he just smiled and nodded, so I kept walking.
As I approached the main doors I saw there was a bookshop
near the entrance and a man standing at the counter.
I went over to him and asked if I could look around.
“Yes, of course,” he replied and showed me where to put my shoes.
The building was just as impressive inside as it was outside
and I felt dwarfed as I walked along the corridor
and up the steps to the prayer room.
I was a little surprised that the prayer room itself
was quite small in comparison with the vastness of the temple complex.
At one end there was an enclosure with ornate and brightly decorated
statues of Krishna and other sacred figures.
Several worshippers were deep in prayer.
One man was lying prostrate on the ground, hands outstretched;
others knelt or walked around the shrine, chanting prayers.
A bell rang, and monks appeared from inside the enclosure.
They were carrying large plates of curry and rice.
The monks placed the food in front of the statues and then closed the curtains.

\begin{quote}
“Why did they do that?” I asked the man next to me.

“So the gods can eat.”

“But they don’t really eat, right?”

“The statues are treated as gods,
because they have the divine spirit of the gods within them.
The monks wake them, bathe them, dress them and offer them food,
just as though they were living gods.”

“What do they do with the food afterwards?”

“The monks eat it or it’s given in charity.”
\end{quote}

I began to feel hungry and made my way to the temple canteen.

Not long after my visit to the Neasden temple I was walking out
from my house when I heard someone calling me.
“Teacher Hassan! Teacher Hassan!” shouted the voice.
I turned around to find one of the White Ladies running towards me,
holding her Sari off the ground to prevent herself from tripping.
I called the women who attended the Brahma Kumaris Centre opposite my house
the ‘White Ladies’ because they always wore white from head to toe,
usually Saris — though anything white would do:
white trousers, white jumpers, white shoes and white socks.
(I often wondered whether their houses had white furniture,
white televisions and white cutlery.)\@
They were a bit of a nuisance because they parked their cars
all along the road when they came to worship,
making it impossible for anyone else to find a parking space.
They didn’t usually say anything to you either,
though were pleasant enough if you greeted them first.
I was very surprised to see one of them running towards me,
calling me “Teacher Hassan,”
something only those who knew me at Islamia School did.

\begin{quote}
“Assalamu-Alaykum, Teacher Hassan.
Do you remember me?”
She was a short, slim Asian woman, in her late thirties or early forties.

“I’m afraid I don’t, I’m sorry.”

“I’m Seema, Wasim’s mother;
he was in your class at Islamia School a few years back.”

“Oh yes!” I gasped, feeling astonished to see someone
who had been a devout Muslim now quite obviously following a form of Hinduism.

“Yes, Wasim, a lovely boy, \emph{Mashaallah},” I hesitated,
“Do you attend the Brahma Kumaris Centre?”

“Yes, I go there regularly.”

“Aren’t you a Muslim anymore then?”
I blurted that out in a way that didn’t sound right.

“In a way I still consider myself a Muslim,
but I now appreciate the significance of Islam much better.”

“It’s odd that we should meet like this because I too have begun
to move away from the traditional understanding of Islam,” I said.

“Do you believe things happen for a reason,” she said suddenly,
“I mean that there are powers at work beyond our understanding,
but are ultimately for our good?”

“Erm… yes, I think I do.”

“You may find this strange, Teacher Hassan,
but I had a voice inside me urging me to call you.”

“That’s pretty amazing.”

“I’d like to invite you to attend some of the talks at the centre.
I feel you would really benefit from them.”

“Yes OK… \emph{Insha-allah}.”
\end{quote}

I began attending a series of one-to-one lectures
that introduced the Brahama Kumaris beliefs.
The topics included, Self\–Realisation, Reincarnation, Nirvana, and Karma.
Seema explained that time is cyclic and goes through five stages,
starting with the ‘Golden Age’, a time of peace, love, and harmony.
Each following stage marks a decline on the previous,
until we reach the last age — the one we’re in now —
a time of greed and war, disaster and calamity.
If, however, one has woken to one’s true self,
one will be re-incarnated into the new ‘Golden Age.’
She explained that one must lead a life of purity and that included celibacy.
It all sounded quite nice, apart from the celibacy.
(Not that I was in a relationship — but one always lives in hope.)

The centre had a bookshop on the ground floor,
and I began to browse through the books on meditation, vegetarian cooking,
and positive thinking.
Amongst the title\cor{’}{}s on display were \emph{“Don’t Get Mad Get Wise”}
and \emph{“The Seven AHA!s Of Highly Enlightened Souls.”}\@
I decided to choose a book that gave an overview
of the Brahma Kumaris movement and its core beliefs.

In the 1930s a wealthy Indian jeweler, Dada Lekhraj,
claimed to have received a series of revelations from God.
He was shown that the end of the present age was imminent
and that a select few would purify their souls
and achieve a place in the newly regenerated world to come.
Those who did not purify their souls would be trapped
in an eternally repeating cycle of misery and hardship.
Dada Lekhraj encouraged women to join and become spiritual teachers,
which went against the prevailing culture.
Despite opposition the movement grew
and now boasts a million followers worldwide.
When Dada Lekhraj died in 1969 the messages from God continued
to be received by others from within the movement.

As I sat in the foyer of the centre one day, waiting for my next lesson,
an elderly Indian man sat next to me.
We soon got to talking about religions and I asked him what the Brahma Kumaris
thought about religious leaders such as Muhammad or Jesus.

\begin{quote}
“They were pure souls who did much good in the world.
In fact we believe that they are reincarnated in different ages,
right up to today.”

“Really, what would they be doing now?”

“They would probably be following the spiritual path of Brahma Kumaris.”
\end{quote}

Whenever I visited the centre after that I couldn’t help playing ‘Spot Jesus’.
There were a surprising number of likely candidates.

At home, money was getting tight, and so I decided to advertise for a lodger.
A Nigerian lady came to see the room, her name was Abisoye,
or Abi for short, about thirty years old, tall, slim and very attractive.
She told me she was a practicing Christian
and that her faith was the most important thing in her life.
She was extremely eager to take the room and told me right there
and then that she wanted it.
She didn’t have any references and was new to the country,
but something about her made me trust\cor{ed}{} her.
I liked her open-hearted and joyful approach to life.
We got on extremely well and of course started discussing religion,
and I felt very comfortable opening up to her.
She invited me to come along to the Pentecostal church
she attended in Notting Hill.
I accepted, but I asked her not to tell anyone I was a Muslim,
as I just wanted to sit quietly at the back and observe.
When we got there the main hall was full,
so Abi took me upstairs to the balcony.
On stage there was a band consisting of two guitarists,
bass, drums and keyboards.
Everyone in the church was standing and there was a buzz of excitement.

\begin{quote}
“Is it OK if I sit down, Abi?”

“Of course, Hassan.”
\end{quote}

Suddenly the crowd hushed, and all eyes were fixed to the stage
as the pastor strode on to the sound of music.

\begin{quote}
“ARE YOU READY TO PRAISE GOD?”
\end{quote}

Loud cheers and cries of ‘Praise the Lord’ and ‘Hallelujah’
rang around the church as people jumped up and down and screamed.

I thought I’d better stand up.
There were cameras hanging from the roof, and they panned the congregation.
Images of cheering worshippers flashed up
onto the huge screen behind the stage.
I bowed my head as the camera came my way.

\begin{quote}
“The Spirit of the Lord is upon me to preach the good news… AMEN?”

“AMEN!” replied the crowd.
\end{quote}

Throughout the sermon people were shouting out “Amen!” and “Hallelujah!”
Even Abi was getting excited and squeezed my hand,
which almost made me shout ‘Hallelujah’ too!
After the sermon the band sang the first of their songs.
I love watching live music,
but there’s something not right about religious songs such as these —
they just make me cringe.
The lyrics were put up on the big screen,
and everyone began singing and swaying.
Some got very carried away.
One little old English lady,
who for some strange reason was dressed in an Indian Sari and pink trainers,
began dancing wildly, while a fat African man in front of me
was rotating on the spot and pointing dramatically in the air,
giving everyone a jubilant glare.
I was the only one not moving or singing
and began to feel extremely self\–conscious.

\begin{quote}
\itshape
“All to Jesus I surrender,

Humbly at His feet I bow,

Worldly pleasures all forsaken,

Take me Jesus, take me now!”
\end{quote}

The camera pointed towards me.
I quickly mumbled: \emph{“Take me Jesus, take me now!”}

Towards the end of the night’s proceedings an invitation was extended
for people to come forward to ‘receive the Lord.’
Abi looked at me, but I pretended not to notice.
A young woman went up to the stage and was quickly followed by several others.
The preacher placed his hand on her head and prayed loudly.
She stood there, hand raised high, eyes closed, beaming from cheek to cheek.
Then she fell down and was caught by a couple of others,
who seemed ready for this sort of thing.
They lowered her to the floor, where she stayed for a while, face down.
A woman in front of me, who hadn’t even gone to the stage,
suddenly fell to the floor and curled up in a foetal position.
Others closed their eyes and began praying in tongues.
Every now and then the whole congregation started clapping passionately
and I felt obliged to join in.

\begin{quote}
“Why are we clapping?” I whispered to Abi.

“It’s a ‘Clap offering’”

“What’s that?”

“It is applause for God.”
\end{quote}

After the service Abi was keen to know if the spirit of Jesus had come to me.

\begin{quote}
“I was impressed by the warmth and sincerity of everyone there.
I have great admiration and respect for your faith, Abi.”

“What did you think about the service?”

“I found the whole experience very enjoyable.”
What I really meant to say was;
‘I found the whole experience very embarrassing.’
\end{quote}

I have mixed feelings about extremely religious people — of whatever faith.
On the one hand I admire the sincerity, warmth and love many exhibit,
while on the other hand I can’t stand how gullible,
shallow and feeble\–minded some of them can be.
I often observed this amongst my fellow Muslims where they were willing
to accept the most ridiculous proof that Islam is the truth,
such as finding verses of the \Quran\ written in Arabic on the side of a fish,
on the back of a cow or within a variety of fruit or vegetables.
I was once handed a picture of an aubergine sliced in half.
The writing below the picture explained that
‘the word Allah can clearly be seen written across the middle.’
Unlike the word Allah written on the sole of a Nike shoe,
this was placed there by God Himself and so instead of being sacrilegious,
it was a miraculous affirmation of the truth of Islam,
even if one was to inadvertently eat it.

\begin{quote}
“Mashallah, ” said the person sitting next to me,
“This is what Allah meant when he said,
\emph{‘We will show them Our signs in all the regions of the earth
and in their own souls, until they clearly see that this is the truth.’}”

“Yes brother,” said a sister sitting opposite,
“there are so many signs all around, if only people open their eyes.”
\end{quote}

I had my eyes open, and it took a great deal of added imagination
to decipher the word Allah, in the picture in front of me.
Other ‘clear’ signs are a tree in Australia that is making Ruku\`
(bowing in prayer), and a tree lined road in Germany
where the tree trunks spell out the words;
\emph{“There is no God but Allah and Muhammad is the prophet of Allah”}
in Arabic.
Both these ‘clear signs’ have been photographed, mass\–produced,
framed and sold in Muslim bookshops up and down the country.
In the case of the tree\–lined road, the words have been highlighted
in white and the rest darkened, to make sure the ‘clear signs’ are clear.

One morning, during the daily teachers’ briefing,
a sister handed out pictures she had just photocopied.
The picture showed some people digging around a giant skeleton
and the writing beneath explained:

\begin{quote}
“Recent gas exploration activity in the south east region of the Arabian desert
uncovered a skeletal remains of a human of phenomenal size.
This region of the Arabian desert is called the Empty Quarter,
or in Arabic, ‘Rab-Ul-Khalee’.
The discovery was made by the Aramco Exploration team.
As God states in the \Quran\ that He had created people of phenomenal size
the like of which He has not created since.
These were the people of Aad where Prophet Hud was sent.
They were very tall, big, and very powerful,
such that they could put their arms around a tree trunk and uproot it.
Later these people, who were given all the power,
turned against God and the Prophet and transgressed
beyond all boundaries set by God.
As a result they were destroyed.
The Ulema of Saudi Arabia believe these to be the remains of the people of Aad.
Saudi Military has secured the whole area and no one is allowed
to enter except the ARAMCO personnel.
It has been kept in secrecy, but a military helicopter took some pictures
from the air and one of the pictures leaked out.”
\end{quote}

I almost laughed out loud when I saw the picture,
but managed to prevent myself and waited for someone to say something.
But the only comments were:

\begin{quote}
\emph{“Subhanallah!”}

\emph{“Mashallah.”}

“They want to hide the truth of the \Quran\ from everyone.”
\end{quote}

I leaned closer to the brother sitting next to me and whispered,

\begin{quote}
“Rashid, I really don’t think this is a genuine picture.”

“Perhaps you are just being sceptical again, Hassan;
it is well know\cor{}{n} the people of Ad were giants.”
\end{quote}

The sister who gave it to me said she had got it in an email,
so when I had a spare moment, I did a quick search on the internet and found —
as I had suspected — it was hoax created using Photoshop.

Another characteristic of many religious people
that I find irritating is the obsession with converts.
Abi had a magazine delivered to the house that seemed to be filled entirely
with stories of famous people who Jesus had saved,
while at the Brahma Kumaris centre I was often told, proudly,
that a lot of ‘English people’ had become followers of their teachings.
Muslims also have a fixation with converts, and names such as Muhammad Ali,
Malcolm X, Yusuf Islam, Jermaine Jackson and Chris Eubank
are mentioned as though to say, ‘You see, Islam is true.’
Of course, religious people are reluctant to mention
the names of converts that are known for all the wrong reasons.
Amongst the less celebrated converts to Islam are people
like John Allen Muhammad, convicted serial killer,
John Walker Lindh, Taliban fighter, Adam Gadahn,
Al\–Qa\`idah operative, Richard Reid, shoe bomber,
and Carlos the Jackal, convicted murderer and international terrorist.
There are also some people mentioned as converts to Islam
who are not actually converts at all.
People such as Neil Armstrong, Jacques Cousteau, and Will Smith
are still frequently claimed to be Muslims.
Prince Charles is also said to have converted ‘in secret’.
I have been told this quite a few times, in hushed tones of course —
so as to keep it a secret.
When I questioned one brother about this, he insisted it was true and that
it happened after Prince Charles had visited Sheikh Nazim in Cyprus.
I was told that it must remain a ‘secret’ or \emph{they}
would stop him from being King, and my friend winked and tapped his nose.

Princess Diana, it is claimed, also converted to Islam.
That’s why she was ‘murdered’;
\emph{they} didn’t want the mother of the heir to throne to be a Muslim.
What a shame! If only Prince Charles had known.

I began to feel less and less comfortable amongst devout believers
and I felt tired of humouring people who believed things
that I regarded as absurd.
I just wanted to get away from it all.
Then one day, out of the blue, my old friend John Shackelton emailed me.
He had found my email address on a page I put up that included information
about the Deeply Vale music festival we attended in 1979.
I hadn’t seen him since then and was overjoyed to get the email.
It turned out that he was no longer a practising Christian
and was also trying to re\–discover himself,
revealing an odd parallel between our lives.
We decided to meet up at the Cropredy music festival in Oxfordshire.
It was great seeing him again after all those years.

\begin{quote}
“You haven’t changed!” I said.

“I was about to say the same thing to you!”

“That’s a bit disappointing isn’t it?
After twenty-five years you’d think we’d have bloody changed!”

“Well apart from the fact we’ve both been given some old bastard’s body!”

“Where’s the wisdom that’s supposed to come with age?
I’m less sure of myself than I was back then!”

“Maybe that’s Wisdom.”

“Fuck that!”

“Would you prefer to delude yourself you have all the answers!”

“No. I’d rather really have all the answers!”
\end{quote}

It was great to see John again after all these years.
It was as if the floodgates were opened and I just let go.
John and I spent the next three days getting drunk and stoned
and collapsing face down in our tents, talking in tongues.
Of course there was also the music,
particularly the Saturday night performance of Fairport Convention.
Every year they went through their well\–thumbed catalogue of songs
culminating in the anthem “Meet on the Ledge”.
As always they were joined on stage by many
of the performers from the weekend’s music.
Everyone linked arms, as the crowd held aloft glow sticks
or torches and swayed from side to side.

\begin{quote}
“It reminds me of a revivalist meeting,” said John.

“It’s all bullshit.”

“I thought ‘Matty Groves’ was good.”

“No. I mean groups, mobs, crowds, anything you have to be a member of!”

“You’re all individuals!” said John, mimicking The Life of Brian.

“We’re just looking for people to tell us
what to do and avoid thinking for ourselves.”

“True! Now tell me, what should we do now!”

“I think a pilgrimage to the Holy Beer Tent is in order.”

“Lead on, Brother! Lead on!”
\end{quote}


%Chapter 12:
\chapter{Pandora’s Box}

\img{scale=0.5}{Pandoras_Box.jpg}{}

There was no-where I could hide.
I couldn’t walk down the street without bumping into this or that brother
or sister or child from my school or acquaintance from the mosque.
At Islamia School I found it harder and harder to carry on
being “Teacher Hassan” the role model others looked up to
and I started staying in my classroom at break times
to avoiding having to see anyone.
But making the decision to leave Islamia School was a very difficult one.
It had been a huge part of my life, the children were like my own children
and my colleagues were my friends.
It was hard to walk away from the life and social circle
I had known for so many years.
I was also concerned about the effect of taking my children
out of the environment they had grown up in and felt secure in.
But as long as I remained at the school I knew I couldn’t move on.
I had to take stock of who I was and where I was.
I needed space.
I wanted to change everything and make a fresh start,
get away from religion and get away from the expectations that surrounded me.
I handed in my resignation and left Islamia School in April 2006.

After a few months I knew I had made the right decision
and I began to feel more relaxed and confident within myself.
The greatest relief was that I didn’t have to live a double life
of publicly expressing beliefs that I privately no longer believed.
I was finally free to be myself, though I wasn’t sure who that was.
There were plenty of things on offer.
If I didn’t like Islam, how about becoming a Christian?
Or maybe a Hindu or Buddhist?
Would I like to be an Atheist or an Agnostic?
How about some sort of New Age philosophy?
Society seems to hate people it can’t label,
but after such a long and difficult process of trying to cast off one label,
I was determined not to fall for another.
The house I moved to was in a very bland Oxfordshire town.
I liked it because of its blandness.
The 1970s three-bed semis all looked the same with their manicured lawns,
regimented rows of tulips saluting you
and gleaming silver Ford Mondeos in the drive ways.
The man who walked his dog at 9am, like clockwork, always smiled
and said “Morning!” as he passed by with his copy
of the Guardian tucked under his arm.
No one knew me; I felt invisible; and that’s what I wanted to be — invisible.
As I was now a single parent of 4 children I took a part-time job
as an online teaching mentor as well as delivering eggs for my brother’s farm.

Apart from some discussions with my two brothers —
I hadn’t spoken to the rest of my family yet about what I believed.
They all thought I was still a Muslim.
I always said the usual things like Assalamu-Alaykum, Al-Hamdulilah,
Insha\´Allah and occasionally joined in prayers with family.
They knew I had many doubts and some ‘strange’ ideas,
but I had never said I wasn’t a Muslim anymore.
It was something that I hadn’t even admitted to myself yet.
Then one day in early 2007 I received a text from my eldest sister Kamelia.
It read:

\begin{quote}
\itshape
How u doing?
I heard startlingly that u r becoming an apostate!
Maybe u should try 2 get hold of american writer —
jeffry lang’s ‘help i’m losing my religion’ love xx
\end{quote}

Seeing the word apostate made my heart skip a beat.
I put the phone quickly back in my pocket.
I told myself that I shouldn’t reply as it would only upset Kamelia
if I was to confirm that I was an apostate.
But the truth was I still couldn’t admit it to myself.
Apostates are considered the lowest form of life by Muslims.
They have wilfully turned away from Islam
after having being guided to the truth.
They have abandoned God and sold their souls to Satan.
I remember reading what one Muslim had said on a discussion board
about the Somali apostate Ayaan Hirs Ali:

\begin{quote}
\itshape
“Allah says these people will receive humiliation in this life,
and a worse humiliation and torment in the next life!
But the best part is, we can rest assured
that this walking piece of FILTH will burn in Hell forever and ever!
Usually the idea of Hellfire is too much to want on anyone,
but filthy apostates deserve nothing else! Allah’s justice will prevail,
in this life and ESPECIALLY in the next life where fools and dogs
like this filthy woman will eternally regret what they did,
and will wish they didn’t make such a grave mistake!”
\end{quote}

Now, here I was having to admit to my sister that I was myself an apostate —
a Murtadd — that most vile and despised of creatures.
But I didn’t want to go on pretending, to others, or to myself, anymore.
I took the phone out of my pocket and wrote:

\begin{quote}
\emph{“I have lost my faith in religion, but not in God.”}
\end{quote}

I chose my words carefully so as to not lie and yet reassure Kamelia
that I wasn’t a completely lost cause.
I knew Kamelia would rationalise that my doubts about religion
were just disillusionment with Muslims —
something I had talked about many times in the past.
But from that moment on I realised that I wasn’t a Muslim any longer.
I finally consciously recognised what I subconsciously had known for some time.

The days and weeks seemed to flow by quickly as I slipped into the routines
of taking the kids to school, getting them home, cooking,
cleaning and doing my online teaching and deliveries.
Routines are an invaluable coping tool.
They distract us from morbid thoughts and from the silliness of the world.
Most importantly, routines are natural; they make you feel normal.
The world around me was still full of confusion and conflict,
but I could feel a subtle change deep within me.
I felt more positive and at ease with myself.
My fears and insecurities weren’t completely gone though.

There was still one question that troubled me — the question of morality.
Muslims take for granted that there are moral absolutes,
unchanging standards of good and evil, taught to us by God.
They provide us with the framework by which to live our lives
as good and decent human beings.
Without absolute moral standards Muslims believe
that man will become corrupt and sinful,
drowning in a sea of moral relativism where ‘anything goes’.
Now that I rejected Islam I no longer believed
that the morality it presented was divinely ordained.
I had lost my yardstick for what was right and wrong
and felt a sense of moral confusion.
In reality I was still the same person I had always been
and behaved in the same way.
But the thought, that without strict moral boundaries
I might be slowly corrupted, frightened me.
I thought about how the sheikhs and Imams in the mosques
had warned about the dire consequences of abandoning God’s law
and how they had constantly cited examples of immorality
in the decadent, secular West as proof.
Even in those days I was always aware of the way they exaggerated the truth.
I remember one Sheikh saying that
“Women walk naked in the streets and couples openly have sex in the parks.”
I was also aware of their double standards;
they had no problem quoting criminal cases that were not typical
of Western behaviour, such as paedophilia, rape, or serial killings
as proof of the evil morality of the West,
yet protested with a great deal of righteous indignation,
if anyone dared to quote criminal cases that were not typical
of Muslim behaviour such as examples honour killings,
wife beaters or terrorists.
I also knew through Rachida’s work with Muslim women’s groups
that paedophiles, rapists and many other types of deviants
did exist in Muslim societies but were often brushed under the carpet
by families under pressure to preserve their honour.
But perhaps most crucially, the \Quran\ itself presents
very dubious standards of morality in verses that sanction slavery,
taking concubines, hitting wives or torturing unbelievers
in the most savage manner for all eternity.
But if I no longer took my morality from Islam, where would I get it?
How do I know what is right and wrong?
Are there absolute standards?
Or is everything relative?

I was sitting in KFC in Oxford town centre one day
when I noticed a smiling Buddhist Monk on the other side of the street.
He was standing at the foot of the old Saxon tower of St Michael’s Church,
begging for money.
The crowds that flowed up and down Cornmarket Street
seemed oblivious to his presence.
Groups of Italian or Japanese tourists leaned their faces
in together in front of the camera,
big grins and waving arms,
with the old church and its little monk nothing more than a curious
backdrop to an Academy-Award winning snapshot of amazing people.
I decided to go across and see the monk at the foot of the tower.
He was only a few steps away, but crossing Cornmarket Street
through crowds of pedestrians proved difficult.
It seemed everyone was either blind or were bumping into me on purpose.
I began to wonder whether both the Monk and I were invisible
or had fallen into a parallel universe.
When I finally reached the other side the Monk,
who had been watching me all the way,
held out his little tin and smiled.
I took out a couple of pounds and dropped them in.

\begin{quote}
“Buddhists don’t believe in God, do they?” I said.

“Buddhism is about reaching a state of enlightenment.
Understanding that reality is not what it appears to be.”

“But if you don’t believe in God,
and don’t believe God tells us what is right and what is wrong,
then where does morality come from?”

“I think you know.” He smiled.
It sounded very wise and final.
I wanted to tell him that I didn’t know,
but thought it would be rude to spoil the moment.
Perhaps this was one of the \emph{“Seven AHAs of the enlightened soul!”}
\end{quote}

I’ve never been a big fan of Buddhism.
I have come across quite a few pretentious Western converts
who managed to make it seem like a trendy and self\–absorbed pastime.
There is also all this nonsense about people being born
with disabilities because they were bad in a previous life.
It seems to me that this monk should \emph{know} better
than to believe such things.
Despite that, I realised he had a point.
I do know. I may not be able to define or explain it,
but I have always had inner moral compass and I still had it —
despite my loss of faith in religion.
In fact it could be said that it was my moral compass
that led me to reject religion.

Muslims constantly mock the idea that man can come up with moral standards
by himself and point to what they see as the moral malaise
of human society when left to its own devices;
the constant shifting of the moral goal posts to suit current trends.
I had always accepted this argument in the past, but now I doubted it.
If Muslims have absolute standards of morality,
then they should agree about what’s right and wrong.
But they don’t.
Muslims differ a great deal about many moral issues
and some can’t even agree on major ones.
Secondly, although human beings, left to their own devices,
may not agree on everything,
there is a great deal of agreement on a vast number of issues
and most people don’t need the ten commandments to tell them
that one shouldn’t murder, steal or commit adultery.
If anything, religion can prevent people from acting in a moral way.
It is also no coincidence that only in recent times,
when religion has been largely ignored, have we had such broad agreement
on moral standards such as those set out
in the Universal Declaration of Human Rights,
which some fundamentalists still object to,
arguing it contradicts Islamic Law.

Maybe it is easier and more comfortable to think
that we have been given an external guide to what is right and wrong.
But that seems to defeat the whole object
of our existence as self-aware beings.
To struggle with questions of good and evil,
it seems to be precisely why we have such an ability;
it’s what makes us human.

The irony is that it is man’s struggle for truth
and understanding that has led to religions in the first place.
It is very difficult to resist the temptation
to tell others of one’s experiences,
teach them the wisdom one has learned
or cast in stone the answers one has discovered.
Of course there is nothing wrong with learning
from the great minds of the past, and it would be foolish,
not to say arrogant, to think that one can ignore centuries of human wisdom —
whether they claimed divine inspiration or not.
The grave mistake is to forget that words are merely symbols
used to point to experiences that can never be fully conveyed by words.
In the case of spiritual experiences, we are talking about something
that is beyond the realm of language.
Any words that attempt to describe such things should never be taken literally
and should be constantly challenged in the light of our own experience.

Life is often a twisting and at times very painful road.
I never imagined when I started out on that road
that I would be at the place I am now,
so I cannot presume to know where I will be in the future.
But I do know that I don’t believe in Islam anymore.
I don’t believe the \Quran\ is the literal and infallible word of God.
Nor do I believe in the Bible or any other book that claims Divine authorship.
It’s clear to me that the author of these books are men.
Men whose source of inspiration was not a Divine Being,
but themselves and the context and time they lived in.
For me the magic of Islam has vanished.
It\cor{’}{}s light, dim and it\cor{’}{}s pearls, fakes.
Like Pandora I glimpsed inside the locked chest
and the magic of Islam escaped.
I cannot put it back.


\backmatter

\chapter{Index}
TODO:

\end{document}
