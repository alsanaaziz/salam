% Common macros:

% Use Scheherazade for Arabic. Use \ar{...} for inline Arabic text.
\newfontfamily{\arabicfont}[Script=Arabic,Scale=1.7]{Scheherazade}
\newcommand{\ar}[1]{\RL{\arabicfont#1}}

% Transliterated "hamza".
\def\´{ʾ} % ˀ
\let \hamza=\´ % Needed in \footnotetext.

% Transliterated "`ayin".
\def\`{ʿ} % ˁ
\let \ayin=\` % Needed in \footnotetext.

% Aliases for subscript and superscript:
\let \Sub=\textsubscript
\let \Sup=\textsuperscript

% Corrective comments.
\newcommand{\cmt}[2]{#1} % #2 comments on #1.
\newcommand{\cor}[2]{#2} % #2 corrects #1.

% Macros for the word Qur'an.
\def \Quran{Qur\-\´ān} % Used when followed by a punctuation mark.

% Breakable forward slash.
% \def\/{\discretionary{/}{}{/}}
\def\/{\hskip0pt/\hskip0pt}

% For hyphenation: Trick TeX into thinking al\–XYZ are two separate words.
\def\–{\hskip0pt-\hskip0pt}

% Macros for custom footnote marks.
\let \oldtfn=\thefootnote
\def\fnmarksym[#1]{\def\thefootnote{#1}\footnotemark%
\addtocounter{footnote}{-1}}
\def\fntextsym#1{\footnotetext{#1}\let \thefootnote=\oldtfn}

% Prints symbols for dividing paragraphs.
\def\pardivider{\centerline{***}} % \ar{۞۞۞} ❧❦

% Qur'an reference numbers. TODO: link to web page.
\newcommand{\QRef}[1]{{\color{darkblue}#1}}

% "Nota bene" macro.
\newcommand{\NB}[1]{\emph{\small NB: #1}}
