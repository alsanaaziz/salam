%!TEX TS-program = xelatex
%!TEX encoding = UTF-8 Unicode
\documentclass[12pt]{memoir}
% \usepackage{arabxetex}
\usepackage[xetex,bookmarksnumbered=true,pdfborder=0]{hyperref}
\usepackage{xltxtra}
\usepackage{polyglossia}
\usepackage{multicol}
\usepackage{bidi}
\usepackage{xunicode}

\usepackage{fontspec}
\usepackage{color}

\setdefaultlanguage[variant=british]{english}
\setotherlanguage{arabic}
\setmainfont{DejaVu Serif}

\special{pdf: docinfo <<
  /Author   (Abbas AbdulNoor)
  /Title    (My Ordeal with the Qur’an)
  /Keywords (Arabic Qur'an Quran Koran Islam critique translation)
  /Subject  (A linguistical critique of the Qur’an)
>>}

% Common macros:

% Use Scheherazade for Arabic. Use \ar{...} for inline Arabic text.
\newfontfamily{\arabicfont}[Script=Arabic,Scale=1.7]{Scheherazade}
\newcommand{\ar}[1]{\RL{\arabicfont#1}}

% Transliterated "hamza".
\def\´{ʾ} % ˀ
\let \hamza=\´ % Needed in \footnotetext.

% Transliterated "`ayin".
\def\`{ʿ} % ˁ
\let \ayin=\` % Needed in \footnotetext.

% Aliases for subscript and superscript:
\let \Sub=\textsubscript
\let \Sup=\textsuperscript

% Corrective comments.
\newcommand{\cmt}[2]{#1} % #2 comments on #1.
\newcommand{\cor}[2]{#2} % #2 corrects #1.

% Macros for the word Qur'an.
\def \Quran{Qur\-\´ān} % Used when followed by a punctuation mark.

% Breakable forward slash.
% \def\/{\discretionary{/}{}{/}}
\def\/{\hskip0pt/\hskip0pt}

% For hyphenation: Trick TeX into thinking al\–XYZ are two separate words.
\def\–{\hskip0pt-\hskip0pt}

% Macros for custom footnote marks.
\let \oldtfn=\thefootnote
\def\fnmarksym[#1]{\def\thefootnote{#1}\footnotemark%
\addtocounter{footnote}{-1}}
\def\fntextsym#1{\footnotetext{#1}\let \thefootnote=\oldtfn}

% Prints symbols for dividing paragraphs.
\def\pardivider{\centerline{***}} % \ar{۞۞۞} ❧❦

% Qur'an reference numbers. TODO: link to web page.
\newcommand{\QRef}[1]{{\color{darkblue}#1}}

% "Nota bene" macro.
\newcommand{\NB}[1]{\emph{\small NB: #1}}


\definecolor{darkblue}{rgb}{0,0,0.5}

% Hyphenation rules:
\hyphenation{Naba-tae-an Naba-tae-ans}
\hyphenation{Shah-ras-ta-ni}
\hyphenation{Gha-za-li}
\hyphenation{Bagh-da-di}
\hyphenation{Mu\`-ta-zi-lah Mu\`-ta-zi-lite Mu\`-ta-zi-lites}
\hyphenation{Za-mal-ka-ni}
\hyphenation{Ja-za-ri}
\hyphenation{Qu-shai-ry}
\hyphenation{Ja-hiz}
\hyphenation{Taw-hi-di}
\hyphenation{Sa-rakh-si}
\hyphenation{Shu-\`araa\`}
\hyphenation{Ra-wan-di}
\hyphenation{Ha-ma-da-ni}
\hyphenation{Rafi-\`i}
\hyphenation{Rafi-dah}


\renewcommand\pardivider{\centerline{\ar{۞۞۞}}}

% Paragraph spacing:
\setlength{\parskip}{1ex plus 0.5ex minus 0.2ex}

% Hanging numbers for chapters and sections.
% \chapterstyle{hangnum}
\hangsecnum

% Default page style.
\pagestyle{plain}

\def\arabictitle{مِحْنَتِي مَعَ القُرْآن ومَعَ اللَّهِ فِي القُرْآن}
\title{My Ordeal with the \Quran\\{\Large And Allah in the \Quran}
\vskip 0.5cm
\ar{\Large \arabictitle}}

\def\abbasAr{عَبَّاس عَبْدُ النّور}
\def\hassanAr{حَسَن}
\author{Abbas Abdul Noor\\\ar{\abbasAr} \and
Hassan\\\ar{\hassanAr}\\{\small(Translator)}}

%%%%%%%%%%%%%%%%%%%%%%%%%%%%%%% Document begin %%%%%%%%%%%%%%%%%%%%%%%%%%%%%%%%
\begin{document}
\frontmatter

% Title:
\maketitle
\thispagestyle{empty}
\cleardoublepage

% Table of Contents:
\setcounter{page}{1}
\tableofcontents

% Preface:
\chapter{Preface}
TODO:

% Introduction:
\chapter{Introduction}
TODO:

\mainmatter

% Chapters:

% Chapter 1:
\chapter{My Journey from Faith to Doubt}

\section{The Belief Phase}
TODO:
\section{The Test Phase}
TODO:
\section{The Whirlwind Phase}
TODO:
\section{The Investigation Phase}
TODO:
\section{The Breaking-Off Phase}
TODO:

% Chapter 2:
\chapter{The Methodology of Examining the \Quran}

There are two methods for understanding the \Quran.
They are: The Methodology of
Transmission, that gives precedence to revelation over reason, the
unquestioning acceptance of the veracity of the text and the inability of
reason to comprehend its ultimate aims and objectives, and the Methodology of
Reason that gives precedence to reason over revelation and its ability to
comprehend the truth without need for reference to the text.
For text is the last concern of the mind that is in itself free,
and independent of belief.

For that reason I will employ, in this book, the Methodology of Reason, that
Descartes established at the beginning of the modern age even though he did not
always abide by it and in particular in understanding religious texts but
manoeuvred, twisted and distorted the neck of Reason
to block the rot that fills Revelation and its garbage that diseases minds.

See how this great man compromises for the sake of (divine) text.
Descartes was not the first to compromise, not at all,
and he will not be the last apart from those who believe
in Reason and act according to it and trust that which
Reason obligates — and they are few indeed!
For (divine) text has such influence and power, few can withstand.

The fundamental principle of the methodology of Reason is impartiality and
objectivity and to approach the research with a mind free from prejudice and
bias “Bias is Sickness” as they say. In this spirit we must resolutely proceed
in studying the \Quran\ and treat it as we would any other scientific research
and subject it to examination and analysis and skepticism and rejection and
contestation because that is what will make our examination fruitful and
profitable and make it of universal benefit.

Applying the methodology of Reason to the \Quran\ is in my view a dangerous and
massive event that will shake the earth below the feet of blind faith (Taqleed)
and inertia, and putrid decay. And it something that must be done the for the
most extreme cure is cauterization.

The \Quran\ has deep roots in our cultural composition and if these roots are
shaken then composition changes to a different composition and destiny changes
to a different destiny and people changes to a different people and as a
consequence a new generation emerges that wasn’t in the reckoning.

For that reason the first thing I will confront you with in this discourse is
that I doubt the \Quran\ and in the god of the \Quran\ and in the teachings of
the \Quran\ and in the miraculous nature of the \Quran\ and in the sublime
language of the \Quran.

I insist on doubt, I embrace it on principle.
For doubts — as al\–Ghazali says — leads one to the truth.
So one who does not doubt cannot look and one who does not look
cannot see and one who cannot see will remain in blindness and error.

This is my method in doing the work and this is how I start examining,
thinking, reading and reflect until the circumstances lead me to something
resembling certainty. That is because what we call the miraculous nature of the
\Quran\ and infallibility of the \Quran\ is really only like any human piece of
work containing error as well as correctness.

I’m aware of the consequences which I have arrived at but that won’t deter me
from proving them and broadcasting them and expressing my opinion freely. I
know in advance that it will lead to mortal dangers and grave confrontations
that I perhaps don’t need. But no! For truth deserves to be followed. I will
take refuge in a mountain that will protect me from the water
(\NB{This is a reference to Noah’s son})
as far as I am able and if not then martyrdom is better than I suffer
incapacity and weakness to declare what I believe in and what many others
besides me believe in, though they are waiting for the spark of light to set
alight after that many sparks of light and sparks of light light up the dark
tunnel that we are living in. So is there any other way to escape from a path?

As for the reasons that led me to doubt in the \Quran, they are because of its
contradictions, generalisations, pompous rhetoric, and fecetious phrases that
have no meaning. The grammatical and stylistic errors that the classical
scholars were at their wits end trying to find explanations. And others, both
scientific and historical that I consider the Lord of the worlds above making.
Just as the \Quran\ is full of rhetorical explosive charges, verbal bombs, that
create such an extreme uproar that ears almost become deaf but after deep
analysis and despite what it contains of sweetness and charm and alluring
beauty, it is pale, emaciated, little content, lacking substance, bubbles in
the air, radiating beams of light like fireworks, except that they soon
extinguish and fall to the ground spent, leaving behind it pitch dark.

\begin{quote}
It is as though it is a bolt of lightening glistening with fury — then fizzles
out and it is as though it never shone.
(a line from a poem by \emph{Ibn Sina})
\end{quote}

Many of the prose of the masters of eloquence (classical literati)
and even the doggerel of soothsayers is better — a thousand times —
than many of the \Quran{}ic verses that are of nonsensical language,
stuffed full of fairy tales, that the \Quran{}ic commentators —
and strangely, amongst them Mu\`tazilites —
became masters at dealing with and defending.

There remains another matter and it isn’t the last. It is the matter of the
indictment of the \Quran\ upon the \Quran. For the narrative of the \Quran\ is
confused — and how confused — for how abundant is the confusion of the \Quran.
“The Almighty” said: “And if this was from other than Allah you would have
found a lot of contradiction.”

The \Quran\ has passed the guilty verdict upon itself! For that which it
contains of contradictions goes beyond the limit of ‘a lot’. Nay, it is the
centre of every disparity and contradiction. The amount of disparities and
contradictions in any book in the world has never reached the level of the
\Quran.
Yet despite this they want us to believe
that there is neither disparity nor contradiction in the \Quran.
We must ignore the evidence to believe that which
does not agree with reason nor with the evidence in the manner of “Believe
Allah and disbelieve the stomach of your brother”
and if you don’t (ignore evidence and reason) then you will see and hear that
which will not please you.
(\NB{Reference to a hadith where someone came to the prophet complaining of his
brothers stomach\/bowel problem, the prophet said ‘give him honey’ —
but the guy
returned saying it has got worse, so the prophet said ‘give him honey’ this
happened twice more — finally the prophet said ‘believe Allah and disbelieve
the stomach of your brother’.})

I am not calling for the renunciation of religion, for that is a difficult
objective, in fact it is a demand that cannot be sought. Because religion for
its adherents is sweet nectar and for so long I my self savoured this sweetness
until I returned to my senses.

I say I am not calling for the renunciation of religion but am calling for the
end to resorting to religion for decisions in all matters and sticking its nose
into every tiny matter in the affairs of life. And that (can be achieved) by
applying Secularism as a principle both in thought and in life. Secularism is
not disbelief, nor is it a call to disbelief as some of its enemies portray
it, but it is merely placing a limit to the interference between religion and
the state.

Religion is not the execution of the captive, nor the stoning of the adulterer,
nor the chopping of the hand of the thief. Religion, according to secularists,
is that which lives in the heart and dwells in the conscience.
Believe what you like, but beware of imposing your beliefs on others
or making it into a system for government or life.
Religion is for God while the nation is for everyone.
That is the slogan of Secularism.

There is nothing inviolable, nothing sacred in Secularism. The only thing that
is inviolable and sacred in it is Humanity and the value of Humanity and the
freedom of Humanity and respect for the dignity of Humanity. The lack of
exploitation of man by another. The unbeliever (Kafir) is not the one who
disbelieves in religion. The only unbeliever is the one who disbelieves in
Humanity and Human Rights.

For the value of life is Reason. The value of life is Freedom. The value of
life is Progress and Development. The value of life is Innovative Vision and
expressing it according to what suits the requirements of the time and
place. As for disbelief and belief, angel and devil, it (creates) conflict that
impedes the development of innovations and the flow of progress in a world of
powers and balances of powers and centers of powers.

The thing that most frightens man is to be consigned to the debris of memory.
Ruminating over myths and delusions. In a trance over the invisible, and text,
and miraculousness and rhetoric and to follow the stories of the garden of Eden
and the Houris, and the (verse of) Light and the servant boys (of paradise) and
stories of Jinn and tales of Luqman and the like of these stories and tales
that for so long fertilized the minds and imaginations, both in the near and
distant past but then today lose the bet (i.e.\ discover it was all BS.)


% Chapter 3:
\chapter{The \Quran\ According to the Belief of Muslims}

\section{The \Quran\ is the Speech of God}
TODO:
\section{The \Quran\ is the Centre for all
Schools of Thought \& Opinion in Islam}
TODO:
\section{The Linguistic Beauty was the \Quran’s Key
to the Hearts of the Pre-Islamic Arabs}
TODO:
\section[The Work of the Exegetes of the \Quran]
{The Work of the Exegetes (Mufassirun) of the \Quran}
TODO:
\section{An Inevitable Revolution}
TODO:

% Chapter 4:
\chapter{The Miraculous Nature of the \Quran}

\section{The Belief of Muslims in the Miraculous Nature (of the \Quran)}

\begin{quote}
“Say: ‘If the whole of mankind and Jinns were to gather together
to produce the like of this \Quran, they could not produce the like thereof,
even if they backed up each other with help and support.’” (\QRef{17:88})
\end{quote}

The \Quran\ is indeed a unique book. It is prose and yet unlike prose. It is
poetry and yet unlike poetry. It is metered and rhyming and yet it is not like
the (standard) meters and rhymes. So what is it then? It is the \Quran\ and
that’s it!

Perhaps the best description of the \Quran\ is that which the late Dean of
Arabic literature, Dr.\ Ta Ha Hussain said: “The Genres of Arabic expression
are poetry, prose and \Quran.” For the \Quran\ is not poetry — no! and it is not
prose. It is a type of speech that is of a singular nature, unique of its kind.
It’s \Quran! For that reason they (the scholars) are united in the opinion that
what is called the miraculous nature of the \Quran\ is its amazing composition.

Miraculousness (al\–\`Ijaaz) in the Arabic language
comes from “To Make Unable”,
in other words it attributes the inability to another
and a miracle is called a miracle
because mankind is unable to replicate it.

The (scholarly) discipline of the Miraculous Nature (of the \Quran) was a
discipline that was an innovation in religion.
This discipline reached its full
maturity in the 4\Sup{th} century of the Hijra
when it became independent and grew
into a discipline in its own right.
Today it’s a fundamental tenet of faith
that no-one can dare throw doubt on.
Beginning in the 4\Sup{th} century of the Hijra
the (discipline) of the Miraculous Nature became indelibly written in stone.

Despite that there were those who cast doubt on this belief, going right back
to the first centuries of Islam.

Perhaps the first of these was al\–Ja\`d ibn Dirham,
tutor to Marwan ibn Muhammad the last of the Umayyad Caliphs.
For he was the first one to openly express
skepticism of the \Quran, and refutation of it and rejection of things in it.
He said that its eloquence was not a miracle and that people can do the like of
it and better than it when no-one had before him had said such as that. Marwan
— who was nicknamed the donkey — used to follow his view to the extent that he
was linked to him and called “Marwan al\–Ja\`di”\fnmark.

\fntext{See: Mustafa Sadiq al\–Rafi\`i, “The Miraculous Nature of the
\Quran\ and the Prophetic Rhetoric.” Page 160.}

During the mid Abbasid period this view (that the \Quran\ was not a miracle)
spread along with other views of a similar nature,
such as the view that the \Quran\ was created as well as its opposition
(those who believed it was not created, but eternal —
on the “Protected Tablet”).
The first to go to great lengths in that was \`Isa ibn Sabih,
known as Abu Musa al\–Mirdar, who was one of
the Mu\`tazilite scholars and amongst the leading ones.
He was called the Monk of the Mu\`tazilites
and differed from the rest of the Mu\`tazilites in all of
the issues that concern us here.
Saying about the \Quran\ that people are able
to produce the like of this \Quran\ as regards eloquence,
and composition and rhetorical beauty\fnmark.

\fntext{Al\–Baghdadi, “The Difference Between the Groups” Page 164–165;
and al\–Shahrastani, “The Book of Sects and Creeds”, 1/68–69.}

Similar to that (view, that others could produce the like of the \Quran) was
the view taken by his contemporary, Ibrahim Ibn Sayyar Ibn Hani\` al\–Nazzam,
who expounded many of the works of the philosophers and
combined their ideas with the ideas of the Mu\`tazilites\fnmark.
But he differed from his colleagues in 13 matters,
while al\–Baghdadi increased that number to 21.

\fntext{Al\–Shahrastani, 1/35–45.}

If al\–Shahrastani labels the areas al\–Nazzam differed from his colleagues,
“issues”, these “issues” become “shameful scandals”
in the view of al\–Baghdadi!
So the 9\Sup{th} issue that al\–Shahrastani reproaches al\–Nazzam about;
becomes “The 25\Sup{th} shameful scandal of his shameful scandals”
according to the wording of al\–Baghdadi:
“His view regarding the miraculous nature of the \Quran\ that it is
to do with the fact it predicts events of the past and future, and to do with
the fact it diverted the causes of opposition and prevented the Arabs — by
force and incapacitation — from being concerned with (trying to imitate) it,
because if he (Allah) let them then they would have been able to produce a Sura
(chapter) the like of it in beautiful rhetoric, eloquence and composition” For
mankind is able to produce the like of this \Quran, but Allah diverted them
from doing that and prevented them by placing hinderance and incapacity within
them to do so. This is “The View of Divertion.”\fnmark

(\NB{In other words al\–Nazzam’s view was that the miraculous nature of the
\Quran\ was NOT that it could not be imitated — in his view it could easily be
imitated — but that Allah prevented the Arabs from doing so!})

\fntext{Ibidem, 1/56–57.}

Now we ask what is the nature of the Miraculousness of the \Quran?

The scholars of Arabic — especially the scholars of language and elegant speech
— are completely united that the \Quran\ is in itself a miracle. That its
miraculousness is in its wonderful composition, in the eloquence of its
expressions, the astounding nature of its clear speech, its unique style that
is unlike any other style, its captivating verbal impact, that reveals itself
in its acoustic structure, and linguistic beauty, and its sublime artistry.

Al\–Qadi Abu Bakr (d.\ 1148) said the nature of the miraculousness
of the \Quran\ is in its composition, arrangement, and structure.
That it’s beyond all types of standard composition
in the language of the Arabs, departing from their
styles of oration and for this reason they were unable to oppose it.
The composition of the \Quran\ had no model to imitate,
nor any antecedent to emulate and it’s unreasonable (to think)
that the like of it could happen by chance.
He said: “the miraculousness of \Quran\ is much clearer in some parts
while in some parts it is more subtle, and more obscure.”\fnmark

\fntext{Ibidem, p.\ 122.}

Al\–Imam Fakhr al\–Din (d.\ 1210) said the nature of the miraculousness is the
eloquence, and unique style, free from all defects.

Al\–Zamalkani (d.\ 727h) said the nature of the miraculousness derives from the
composition that’s unique to it and is not haphazard.
In that its words are finely balanced in construction,
in meter and the reason behind the way it’s been put together, in meaning.
So that every type occurs in the best possible
place for its pronunciation and meaning.

And Ibn Atiyya said: The correct (opinion) and the one that laymen and experts
are agreed upon in regard to its miraculousness is that it is its composition
and the soundness of its meanings, and in the arrangement of the eloquence of
its wording. And that is because Allah’s knowledge surrounds all things and
surrounds all (aspects) of language. So since the organisation of the wording
in the \Quran\ is something his knowledge completely surrounds, i.e.\ each word
perfectly suits the one it follows and each meaning is elucidated after another
and that is the case from beginning to end of the \Quran\ and man is encompassed
by ignorance, bewilderment and perplexity and it is self\–evident that no human
being encompasses all that, then as a result the arrangement of the \Quran\ is
furthest epitome of eloquence and for that reason one destroys the saying of
those who claim that the Arabs were able to replicate the like of it or that
they were ‘diverted’ from doing so. The correct (opinion) is that it is not
within the ability of anyone ever!\fnmark

\fntext{All the quotes are taken from the previous reference, p.\ 133
with some slight edits in wording but not meaning.}

However the scholars disagree about the difference in the degrees of eloquence
of the verses of \Quran\ after having agreed it is in the highest forms of
eloquence, in so far as you cannot find phrasing that is more suitable nor
balanced, to convey that meaning.

Al\–Qadi (Abu Bakr d.\ 1148) takes the opinion of ‘negation’, meaning negation
of there being any difference (in degrees of eloquence). For every word in it
is depicted in its highest (form\/usage) even though some people are better at
sensing it than others.

While Abu al\–Qushairy and others take the opinion of ‘difference’ (in degrees
of eloquence), he said: “We do not claim that everything in the \Quran\ is in
the highest rank of eloquence.”

And likewise others have said:
“In the \Quran\ is (both) the Eloquent and the Most Eloquent.”
This is the opinion taken by Sheikh \`Izz al\–Din \`Abd al\–Salam
who then asked: “Why was the \Quran\ not entirely in the most eloquent form?”
Al\–Sadr Mawhoob al\–Jazari replied to the effect that
‘if the \Quran\ had come in that (most eloquent form)
it would not be in the usual style of speech from the Lord,
combining the Most Eloquent with the Eloquent and so the argument
for the miraculous (nature of the \Quran) would not be effective.
So it came in their usual style of speech
to highlight the inability to challenge it and so they can’t say, for example:
‘You have brought that which we have no ability in its like. Just as it would
not be right for a sighted person to say to a blind person: “I beat you by
virtue of my sight.” Because the blind person will say to him: “Your victory
can only be valid if I was able to see, and (then you could say) your sight was
better than mine. But if I lack the ability to see then how can I make a
challenge?”\fnmark

\fntext{Ibidem, p.\ 109.}

In any case the \Quran\ is in the eyes of Muslims the Prophet’s greatest
miracle. No falsehood in it either before or behind it. “Indeed everything in
the \Quran\ is a miracle in respect of the music of its letters, the kinship
between its wording, the synchronicity of its words with its expressions,
and well\–knit arrangement in its resonance and that which it arrives at in
regards to composition between the words and the fact that every word
intentionally fits its counterpart. As though the weave of each part perfects
its picture and completes its objective. Its meanings coalesce with its
words as though its meanings are related to its pronunciations and its
pronunciations were designed for it, and made to fit its size.”\fnmark

\fntext{Ibidem, p.\ 99.}


\section{What kind of Miraculousness is it?}

Now we say: Indeed the belief in the miraculousness of the \Quran\ is no more
than a myth amongst myths. Indeed! The \Quran\ is not amongst the secrets of the
gods. It doesn’t bear the slightest relation to divine inspiration that takes
it outside the (normal) activity of (human) history. It’s a purely human
achievement that follows the norms of humanity in strength and weakness,
correctness and error, agreement and contradiction, cohesion and disparity,
consistency and inconsistency, uniformity and disarray.

The direct result of all that is that the \Quran\ is a very ordinary book. For
that reason it is necessary to remove it from its safe refuge, outside Human
history and return it to the world of people. After that it will no longer be
storehouse for timeless wisdom nor a divine book protected from error that no
falsehood can approach it from either front or behind.

In that way, it and its time and its context become part of the historical
process of the area which has witnessed, and continues to witness every day,
comparable books that influenced these books and are influenced by it and
ignite the interaction between them.

Every star\–struck believer, regardless of whether he is from the common people
or their elite or even from the elite of the elite, relies (on the belief) that
“in the \Quran, due to beauty of the words and the splendour of the style
especially, no-one can attain the phrases, style and meanings.”\fnmark

\fntext{Muhammad Abu Zahra, “The Greatest Miracle.”}

And that challenge, that Allah announced in the \Quran\ for Man \& Jinn to bring
the like of this \Quran,

\begin{quote}
“Say: ‘If the whole of mankind and Jinns were to gather together
to produce the like of this \Quran, they could not produce the like thereof,
even if they backed up each other with help and support.’” (\QRef{17:88})
\end{quote}

is absolutely true, but it doesn’t apply only to the \Quran, it also applies to
every great work. For just as Man \& Jinn are not able to produce the like of
the \Quran, likewise they cannot produce the like of that which Plato brought,
nor al\–Jahiz, nor al\–Tawhidi, nor Dante, nor Goethe, nor Shakespear...

Great works always contain the fingerprints of their authors. It is a part of
their identity. So if it is impossible to imitate these fingerprints, then it
is also impossible to imitate these works. Each one is a unique weave that has
no match in the works of man and thus establishes its character. Despite this
each one is not free from flaws and errors and defects that the critic can be
aware of. Likewise the \Quran. In the work of al\–Jahiz and al\–Tawhidi is that
which far surpasses what is in some of the verses of the \Quran, as we shall
see, but who dares criticise the \Quran?

Indeed the Muslims of the Middle Ages during the Golden Age, had more freedom
than Muslims in this time. If not then why does no-one dare, like al\–Sarakhsi
and Ibn Rawandi and al\–Razi, to defame the most holy symbol of Muslims, the
valuable of valuables that gives meaning to their existence and bestows on them
hope and immortality.

All efforts and active forces in the Islamic world have been enlisted to repel
the “Enemies of Allah”. Criticism of Allah’s book has been met with a reception
that varies from forbearance \& irritation, between insult \& vilification and
between suppression \& temperance, and ‘dealing’ with antagonists ranged
between chit-chat \& bluster,
to finding excuses and haphazard solutions — or as
I myself call it, “Patching” (the holes) — to save the word of Allah from the
clutches of the deniers, the astray and the ones who lead others astray.
Between hitting and slapping, punching and physical eradication, seeking
closeness to Allah through the blood of that insolent fabricator of lies about
Allah, denier of his signs, so that he be a warning to his like, the forces of
the Devil, “and Satan indeed found his calculation true concerning them, for
they followed him...” (\QRef{34:20}) them and the seducers.
Then they topple into the Fire of Hell all of them together\fnmark.
They are the ones who Allah curses, and those who curse, curse them!!

\fntext{Allusion to what is related in Sura
al\–Shu\ayin{}araa\ayin{} \QRef{26:94}.}

Indeed opposing the \Quran\ was a natural process that arose with the rise of
Islam, but the new religion killed it in the cradle, or at least was able to
silence it for a while. That was after the astounding victory that it achieved
in the Arabian Peninsula and the area surrounding it. Indeed it was such a
tremendous breakthrough that it temporarily diverted attention away from that
which interplays in it of (opposing) forces and deep contradictions that don’t
appear on the surface except in moments of quiet and stability or at the times
of fitna.

For that reason it is not strange that this process started anew or returned to
the open when the Umayyad dynasty began to disintegrate and draw towards its
inevitable end. For indeed Islam injured the pride of many of the leaders of
the heretics (Zanadiqa) — and they were the Shu\`ubiyyah
(\NB{A popularist movement against the supremacy of the Arabs})
— and nationalist pride overtook them and led them to fanaticism for the
religion of their fathers such as Zoroastrianism and Manichean dualism and
hatred towards Islam that ended their glory and destroyed their dreams in
lasting and noble life.
A group of poets who belonged to “The League of the Mujjan”
(\NB{A group of libertine\/dissident intellectuals such as Bashar ibn Burd and
Abu Nuwas})
joined them, fleeing from the constraints of religion
and seeking a life of freedom with no restrictions or regulations.

Then came the Abbasid period where the Shu\`ubiyyah movement was active side by
side with the Heretical movement (Harkatu az\–Zandaqa) and the attacks on Islam
intensified and disparagement of its holy of holies — the \Quran. And at the
head of this movement were poets, satirists, and disaffected thinkers, the most
famous of them: Salih Ibn Abdul Quddus, and Abdul Karim ibn Abu al\–\`Awjaa\`,
and Abu \`Isa al\–Warraq and Bashar ibn Burd,
and his adversary Hammad Ajrad, and
Iban ibn Abdul Hamid al\–Lahiqi, and Ibn Muqaffa\`,
and (his son?) Muhammad ibn Abdullah ibn Muqaffa\`,
and Abd al\–Masih al\–Kindi who we shall say a few words
about in a bit to show the participation of non\–Muslims in the attack on the
\Quran...

But the most famous of all these without argument is:
Abu al\–Hussein Ahmad ibn Yahya ibn Ishaq al\–Rawandi,
and Abu Bakr Muhammad ibn Zakariyya al\–Razi,
under both of whom the movement of Heretics reached its climax
and extent of its maturity and we will discuss now each of them briefly,
enough to clarify what we mean.


\subsection{Ibn al-Rawandi (d.\ 298H / 910AD)}

(\NB{none of his books have survived. What exists are quotes from critics.})

At first the Dissident Movement, or the Movement of Heretics, was simply a
spontaneous individual attitude or libertine outburst or a transient
intellectual position, but then this movement began to manifest itself and
crystalize with the passing of time until it became a comprehensive school of
thought based on the pillars of reason. It acquired supporters who believed in
it and worked to publicise it and disseminate its principles.
This movement continued to grow,
consolidate and rise until it reached its apex under Ibn al\–Rawandi.
His view of Prophethood formed the cornerstone of the barrage
on the \Quran, by these Heretics, though without that extending to doubt in the
existence of Allah who revealed the \Quran.

Doubt in Prophethood was the furthest extent that this movement of Heretics
achieved in Islam. Then it came to a halt after a violent shake-up in concepts
and doctrines grew out of it,
in the 4\Sup{th} century (of Hijra) that drew towards
it the movements of the concealed ideologies that were influenced by Gnosticism
and Esoteric Knowledge and especially those associated with the Shi\`ism
and the Isma\`iliyyah Shi\`ism in particular.

Ibn Rawandi was the most famous of the dissenters
of the 3\Sup{rd} century of Hijrah,
though only a little is known about him,
even the date of his birth and death are not known for certain.
He was originally a Mu\`tazilite (A non\–orthodox Sunni School of thought)
then recanted and leaned towards Shi\`ism and became
a bitter enemy to the Mu\`tazilites.

He was a vehement believer in Reason, praising it and relying on it in all
matters and affairs. Reason, in his opinion, was: “The greatest gift bestowed
by God, glorified is he, upon his creation. Indeed it is through it that the
Lord and his blessings can be known and by virtue of it that orders \&
prohibitions, his promises \& threats become valid.”\fnmark\@
% TODO: find correct place of this footnote:
\fntext{Quoted from Dr.\ Abd al\–Rahman Badawi, from “History of
Disbelief in Islam” page 202.}
He wrote a book called “The Scandal of the Mu\`tazilah”\fnmark\
which was a critical analysis of the Mu\`tazilite School of thought
from the perspective of the Shi\`ah al\–Rafidah and
a reply to the book of al\–Jahiz “The Virtue of the Mu\`tazilah”.
But this period did not last long and we see him after that
amongst the group of those who the author of the “The Catalogue” (Kitab
al\–Fihrist, by Ibn al\–Nadim d.\ 995 AD ) gives the title “Theologians who
manifest Islam, but conceal heresy.”
He was influenced in this by Abu \`Isa al\–Warraq
who was a teacher of his and encouraged him towards heresy.

\fntext{See: Ibidem, p.\ 87, 186 and that which is
after it.}

Ibn Rawandi began his Heretical writings in the latter years of his life, and
they are the books that he owes his importance and high status to. Amongst
these books is a book where he dealt a massive blow to the \Quran. He called it
“The Crushing Blow”. It was, as its title suggests, a merciless attack on the
\Quran.

A third book is attributed to him called the book of “The Emerald” where he
refutes the concept of Prophethood in Islam and attacks the belief in the
Miraculousness of the \Quran. We said that this book is “attributed to him” due
to a reference that it’s said it’s attributable to al\–Jaba\`iy and goes on to
say: “Indeed Ibn al\–Rawandi and Abu \`Isa Muhammad Ibn Harun al\–Warraq the
Heretic, also dispute with one another over the book of “The Emerald” each one
claiming that it is amongst their compilations as both were in complete
accordance in attacking the \Quran.”\fnmark

\fntext{Ibidem, p.\ 112 and 182.}

In the first and third parts of this book, Ibn al\–Rawandi
(or Abu \`Isa al\–Warraq?) presents his opinion
on Reason and Religions that depend on Revelation
and explains the position on each.
He begins his book with Human Reason,
praising it and going to great lengths in celebrating the fact that it is the
only path to enlightenment. For that reason his opponents must agree with him
that Reason is the mightiest thing that man possesses and is the sole refuge to
solve problems, indeed! “The prophet bore witness to the high status and
majesty of Reason.”\fnmark

\fntext{Quote from the pervious reference p.\ 186–187.}

So Reason should be used to analyse Prophethood. Either the teachings of the
Prophet agree with Reason, and in that case there is no need for it because
Reason is in no need of it, or it contradicts reason in which case it is false.
For that reason it was necessary for Ibn al\–Rawandi to be surprised
at the position of Muhammad and wonder;
“Why did he bring that which negates him
if he was authentic?”\fnmark\@
For the revelation of Muhammad is in complete opposition to reason.
Then, what is the meaning of these obligatory religious
injunctions upon the Muslim, such as ritual washing, and prayer and
circumambulating around the Ka\`abah and visiting the holy sites?

\fntext{Quote from the pervious reference p.\ 84.}

Regarding that, Ibn al\–Rawandi says “Indeed the prophet brought that which
contradicts Reason, such as prayer, ritual cleansing from impurity, throwing
stones at pillars during Hajj, and walking around a house that cannot hear nor
see? Or dashing between two rocky mounds that can neither benefit nor harm. All
of this has nothing to do with Reason. So what is the difference between (the
hills of) Safa and Marwa and (the hills of) Abu Qubays and Hira? And walking
round the (holy) house is no different than walking round any other
houses.”\fnmark

\fntext{Quote from the pervious reference 101–102. Abu Qubais and
Hira are mountains in Makkah.}

Ibn al\–Rawandi used the myths of the Brahmans to express his bold views.
He used them as the means by which to attack
“Divinely revealed” religions and laws
(\NB{Since it is easier for Muslims to recognise the superiority of using
reason in relation to the claims of ‘divine inspiration’ of others,})
so he could hide beneath this veil his belief (about Islam).
He made them as
analogies for the (necessity of) Reason and Intellect so that they could be set
free of their own accord and express the views and thoughts that naturally
occur to them, while attaching it to delusional characters to soften its blows
upon the audience.

In this vein and in the name of Reason that he never ceases to praise and extol
for a moment, he goes on to attack the \Quran\ in his previously mentioned book
“The Emerald”. He reviews in this book the concept of the Miraculousness of the
\Quran\ and crticises it ruthlessly, and annihilates the view of the divine
origin of the \Quran\ and puts forward a simple, concrete, logical and reasoned
view with no ambiguity in it. Convincing the intellect of the human nature of
the \Quran, refuting those who say that it is an inspiration from Allah and a
revelation from an all-wise and all\–knowing entity.

It is also related that Ibn al\–Rawandi said — regarding refuting the belief in
the miraculousness of the \Quran:

“Indeed it is not impossible that one Arab tribe is more eloquent than all the
other tribes, and that a group of people in this tribe are more eloquent than
(others) in this tribe and one member of this group is more eloquent than this
group... and suppose that his eloquence was spread amongst the Arabs... so what
is its wisdom upon the non-Arabs who do not understand the Arabic language?
What is the proof for them?”\fnmark

\fntext{Ibidem, p.\ 87.}

And Ibn al\–Rawandi mocked the theatrical spectacle of the angels
who Allah sent down from heaven during the battle of Badr,
to help the prophet.
He said, indeed:
“They had limited effect, little power, despite their great number and
the combination of them and Muslims,
they could not kill more than 70 people...
And where were the angels during the battle of Uhud when the prophet was
skulking in fear amongst the slain?
Why didn’t Allah help him in that situation?”\fnmark

\fntext{Ibidem, p.\ 87.}

It was also related in the book of “The Emerald”, quoting from the book of
“The Victory” by al\–Khayyat, his saying: “Indeed the \Quran\ is not the speech
of a wise god. In it are contradictions and mistakes and passages that are in
the realms of the impossible.”\fnmark\@
As in the theatrical episode of the angels of Badr that we just mentioned.

% TODO: Check if \fnmark is correct:
\fntext{Ibidem, p.\ 110.}

Then indeed Ibn al\–Rawandi finds in the discourse of Aktham Ibn Sayfi better
(language) than (the \Quran\ that boasts) “Indeed we have given you the
Abundance” (108/1)\fnmark\@
\fntext{Ibidem, p.\ 111.}
As Ibn al\–Jawzi says in his brief allusion to the book of “The Emerald”: “Then
he begins with attack on the \Quran\ and claims the existence of linguistical
mistakes in it.”\fnmark

\fntext{Ibidem, p.\ 120.}

And before ibn al\–Rawandi exploits criticism of the \Quran\ in his book “The
Crushing Blow”, and Ibn Jawzi has preserved for us copies of this criticism,
for amongst the parts that he preserved for us in his book “Al\–Muntathim Fi
al\–Tarikh”, from the book of “The Crushing Blow” which has not survived, the
following piece: “When (Muhammad) described (in the \Quran) Paradise, he said:
In it are rivers of laban, whose taste has not gone off, and that is milk, yet
no one desires that apart from the hungry. And he mentioned honey that no-one
wants at all and Ginger, which is not tasty except as a drink and silk brocade
(sundus) which is used as a spread, not as clothes and likewise embroidered
brocade (Istabriq) which is a thick\/rough type of silk brocade.
He said one who
imagines himself in Paradise wearing this rough clothing and drinking milk and
ginger, will be like a bride in a Kurdish or Nabataean wedding!”\fnmark

\fntext{Ibidem, p.\ 133.}

Ibn al\–Rawandi turns his attention to the Divine challenge to bring the like of
the \Quran\ and says: “If you want the like of it in respect of superior speech,
we can bring you a thousand like it from the speech of the masters of rhetoric
and champions of eloquence and poets and is more fluent in wording and more
concisely conveys the meanings, more elegantly rendered and expressed and more
beautifully rhymed. An if you are not content with that then we demand from you
the same that you demand from us!”\fnmark

\fntext{Ibidem, p.\ 216.}

(\NB{There appears to be an opening quotation mark missing from the original in
this next bit, and I’m not sure where it should go.})

Even the Mu\`tazilah who reject all Miracles or at least attach no importance to
them, still believe in the miracle of the \Quran.\fnmark\@
\fntext{Ibidem, p.\ 119 and 153.}
But al\–Nazzam, who was the most bold and freethinking of the Mu\`tazilite
theologians, rejected the miraculous nature of the \Quran\ in regard to its
composition, he rejected what was related of miracles of our prophet, peace be
upon him, as regards splitting the moon, the pebbles in his hand glorifying
God, the gushing of water from his fingers, to arrive by way of rejection of
the miracles of our prophet, peace be upon him, to the rejection of his
prophethood.”\fnmark

\fntext{Al\–Baghdadi, “The Difference Between the Groups”, p.\ 132;
See also p.\ 149–150}


\subsection{Abd al-Masih al-Kindi (9\Sup{th} Century AD)}

This attack on Islam was not restricted to apostate Muslims.
No, indeed, non\–Muslims entered the ranks,
galvanized by the fury of the fierce offensive
being waged on the new religion.
Perhaps the most famous of these, whose quotes
have reached us, was the philosopher Abd al\–Masih ibn Ishaq al\–Kindi
(\NB{Not to be confused with the well known Muslim philosopher also called
al\–Kindi}.)\@
He was a Nestorian that it is claimed lived in the courtyard of (the Caliph)
al\–Ma\`mun who, no doubt due to his openness towards those
who differed from him in views and belief,
(\NB{Al\–Ma\`mun was famous for gathering opposing sects and religions to hear
them debate,})
tolerated the ferocious criticism of this
Christian who attacked the rituals of Islam and its beliefs, one after another
and especially the rites of Hajj.

His views that concern us here are those connected to our topic and his
explanation for the effect of the \Quran\ in that “The Nabataeans, the rabble,
non-Arabs, the gullible and the ignorant who have no understanding of the
Arabic language”, are the only ones who would be duped by the claim of the
Miraculous nature of the \Quran\ in respect of its composition.”\fnmark

\fntext{Quote from Dr.\ Badawi, from “The History of Disbelief in Islam,”
p.\ 129.}


\subsection{Abu Bakr al-Razi (d.\ 311H / 923AD)}

Al\–Razi is the second of the two who, without rival,
courageously barged their way over the red line.
Many before them hovered close but never quite hit the mark.
Either because of their fear or lack of resources.
As for al\–Razi and before him, Ibn al\–Rawandi, they are the indisputable
masters of the field.
Indeed all those who attempted to reply to them could not match them.
Not at all! They were not on their level.
They were dwarfs that cannot be compared to either of them.
No way! No way!

Each one was a revolutionary, rebellious visionary,
who revealed the concealed (thoughts),
brought out the pent-up (feelings) and freed the suppressed (minds).
They thought the thoughts that were not thought about.
No! That were not allowed to be thought about.
Each one of them would not accept anything less than making the most holy
of holies the object of their criticism,
and delving into it to uncover its flaws,
and disgrace its myths and illusions.
Exposing what it contains of threats, claims and hearsay on account
of which man is crushed, and paralyzes his abilities and enslaves him to
supernatural powers and invisible entities.
To rob and intimidate him like an unsheathed sword hanging over his head,
not allowing him any room to move to see what is beyond his nose or know
what is going on around him.
Thus he must live his life, hostage to the fears, anxieties,
whisperings and misgivings that come between him and
achieving his best potential.
Destroying all his ambitions of self realization and personal freedom.

Al\–Razi was a philosopher, doctor and alchemist of the highest order just as
he was the pillar of the dissident and heretical movement during his age and
the following centuries.

If there was a difference between him and Ibn al\–Rawandi then it was in the
degree of depth and widening of the details and (his) ability to generate
new ideas from old ones, but both believed and relied upon Reason and
both base their judgments and conclusions upon Reason.
In their opinion, Reason was to the yardstick to measure everything.

If Ibn al\–Rawandi, in his heretical and irreligious meditations,
worked within a similar atmosphere to that of the Muslim theologians, then:
“Al\–Razi attacked and criticised the shortcomings of Religion from the
perspective of Philosophy.”\fnmark

\fntext{Ibidem, p.\ 127.}

In the same way as Ibn al\–Rawandi used the Brahmins as a vehicle by which
to disguise his views, and to place on their tongues what was really in his
own mind regarding the invalidation of prophethood and virtues of Reason,
al\–Razi also did likewise, in that he attributes to it (Reason) not just the
(ability to arrive at) ethical behaviour, as Ibn al\–Rawandi did,
but attributes to it (knowledge) of divine matters also.
For he said, indeed we:
“Through it (Reason) arrive at knowledge of the Creator,
Mighty \& Glorified is He.”\fnmark

\fntext{Ibidem, p.\ 203.}

This proves that there is no justification for Prophethood as long as Reason
is able to lead us to all that is ethical and unethical.
In any case Ibn al\–Rawandi:
“Moved in Theological and Religious field of study,
where as al\–Razi moved in the Scietific field.”\fnmark

\fntext{Ibidem, p.\ 217.}

In summary, there is no doubt that Ibn al\–Rawandi blazed the trail,
and opened the way, but al\–Razi watered it and boarded it with palm trees
and beautified it with flowers and scented herbs
and raised upon it a lofty edifice.

Al\–Razi praised Reason “using language which surpasses that used by the
great rationalists of all ages, even in the modern age,”
as Abd al\–Rahman confirmed in his aforementioned book.

By virtue of Reason, man is in no need of Prophethood, nor Religion,
nor all the Divine Books and as a consequence; nor the \Quran.
By virtue of Reason and Reason alone, we can know good from bad
and truth from falsehood.
There is no authority other than the authority of Reason,
nor any belief other than belief in Reason...
and if this is its magnitude then we must never minimise its value,
nor reduce its status, and never make it a subject when it is the master.

Prophethood was al\–Razi’s overriding concern, he demolished it on the basis
that Reason has no need for it.
He said: “From whence did you make it necessary that God singled out a people
for Prophethood instead of another?
Preferring them over (other) people?
Giving them evidences and forcing others to be in need of them
(in need of these people)?
And from whence did you allow in the wisdom of the Wise that he chooses
that for them and raises some over others confirming enmity between them,
increasing wars and with that annihilate people?”\fnmark

\fntext{Ibidem, p.\ 205.}

We are not so concerned here that al\–Razi heaps criticism and abuse on
prophethood \& prophets, and elaborates in great detail on that.
What concerns us is his criticism of Religions, so we can arrive,
in that way, at his opinion on the \Quran.
For that reason we see him turning his attention to “Revealed” Religions
and the books they brought which they ascribe divinity to.
He analyses them without bias, favouritism, or discrimination.
For all of them are of equal importance.\fnmark

\fntext{Ibidem, p.\ 208–211.}

For the disbelief of al\–Razi was not aimed at a specific religion
without another, in other words it was not aimed at Islam alone.
That highlights the objectivity of al\–Razi and the soundness of his opinion.
For all religions were subject to attack and abuse.
For they do not say the same things.
They contradict one another despite the fact they claim to come from the same
source and (claim) they are free from defect and lies.
But how can that be the case when they contain absurdities and contradictions.

Here the adversary poses the question: If religions are as you say,
then how can we explain the adherence of the masses to them?

Al\–Razi responds to this objection by (saying) that the followers of
(the various) religions have taken the religion from their
(religious) leaders by way of imitation.
They are prevented from questioning or scrutinizing the foundations,
and tales are related to them that discourage them
from questioning these foundations.
Whoever contravenes that is accused of Kufr (disbelief).
If the (religious) leaders are asked to prove the truth of what they say
they fly into a rage and spill the blood of one who demands that of them.

Then came (a period of) long familiarity, the passing of time,
acquaintance and deception of the people by the goat-bearded (clergy)
who stand at the front of religious gatherings and shriek out
lies and gibberish while around them the weak-minded men, women and children
(listen) until it all roots itself deeply within the people and
becomes a predisposition and habitual.\fnmark

\fntext{Ibidem, p.\ 211–212.}

Then al\–Razi returns to his charge of contradictions in the “Holy” books
as proof of their falsity.
For the contradiction of religions leads to contradiction of
revealed books that brought them.
He begins with the Torah and the \Quran\ and the prophetic Hadith and
what they contain of anthropomorphic and human\–like qualities (of God).
He mentions what is in the Torah of putting the fat on the fire
so that the Lord can smell its scent.
Also how it depicts an image of an old man with white hair and beard.
This human\–like and anthropomorphic description contradicts impassive and
impervious nature of God to things like smells etc...
All this announces that God is constructed, fabricated,
reacting to things like the rest of creation.

Likewise al\–Razi attacks Christianity and its claim of the existence of
an uncreated ancient being by the side of God;
the Messiah his son, which leads to associating a partner with God.
Furthermore how can we reconcile his saying that he came to fulfill the Torah
with his abolishing its laws and changing its rulings?
Strangely, during his criticisms of Christianity,
he did not mention — in the texts we have —
the passages in the \Quran\ about corruption of the Gospels.\fnmark

\fntext{Ibidem, p.\ 213–214.}

Anthropomorphism and contradictions are not only limited to Judaism and
Christianity but also envelop the sayings of the prophet and the \Quran...
and that is exemplified by what is related from the prophet when he said
“I saw my Lord in the best of forms.
He put his hand on my shoulders until I felt
the cold of his fingertips on my chest.”\fnmark\
\fntext{Ibidem, p.\ 214.
(\ar{الثَنْدَوَة} is the flesh between the nipples.)}
and his saying “Beside the throne by the shoulder of Israfeel,
and he will be groaning the groan of a young camel being saddled.”%
\fnmark\@
(\NB{Israfeel is the angel who blows the trumpet twice on the Day of Judgment.
Once to destroy everything and a second time to bring humans back to life and
summon them for judgment.})

\fntext{Ibidem, p.\ 214.}

It’s also obvious that many of the verses of the \Quran\ demonstrate
anthropomorphism and no-one can deny that apart from the arrogant.
For example His saying, mighty and glorified is he:
“The Compassionate One is firmly established on the Throne.” (\QRef{20:5})
and He also said: “And eight (angels) will carry the Throne of your Lord
above them on that Day” (\QRef{69:17}) and His saying:
“Those who carry the Throne and those around him...” (\QRef{40:7})
So how can this his make sense, be sound, correct in light of the fact that
God is completely and utterly free from all the attributes of the profane as
made clear in His — Most High — saying:
“There is nothing whatever like unto Him...” (\QRef{42:11})

Likewise how can we reconcile verses about predestination
with others about free-will?
And perhaps al\–Razi borrowed these questions from the books of
Theological Discourse as Abd al\–Rahman Badawi noted.\fnmark

\fntext{Ibidem, p.\ 218.}

As for the view that these verses require “Esoteric Interpretation” (Ta\`wil)
in other words taking them to have a hidden meaning that is not the plain
meaning of the words, that was of no interest to al\–Razi, he rejected it
utterly and paid no regard to Ta\`wil, not taking it seriously at all.
Because Ta\`wil in his opinion and the opinion of his like, was just
interpolation and deceitful pretense — or in my own expression:
“patching up” — the intent of which was to rescue the text, however one can,
and give it an acceptable meaning.
For al\–Razi and his like approached religions as it appeared plainly
in its texts and not as is it is (claimed to be)
wrapped up in hidden meanings.\fnmark

\fntext{Ibidem, p.\ 214–215.}

Al\–Razi criticised the \Quran\ also on the basis of what it said that
contradicted Christianity and Judaism.
He said: “Indeed the \Quran\ contradicts that which the Jews and Christians
believe regarding the death of the Messiah — upon him be peace.
Since the Jews and Christians say the Messiah was killed and crucified,
but the \Quran\ says he was not killed and not crucified and that God
raised him up to himself.”\fnmark

\fntext{Ibidem, p.\ 215.}

Thus does al\–Razi use religions and divine books to undermine each other
to arrive at the result that they are all false!
Because the contradictions between them declares their falsehood in total
as long as they claim that they come from the same divine source.

After this attack on all religions al\–Razi comments also, saying
“Indeed, by God, we are amazed at what you say that the \Quran\ is a miracle
when it is full of contradictions.
It is the narration of ancient myths,
it has no benefit nor is it proof of anything.”\fnmark

\fntext{Ibidem, p.\ 216 and 218
in two different versions.}

And this is a view that is completely sound,
for in the \Quran\ are conundrums and riddles — ambiguities and mysteries,
that the greatest scholars of Tafseer until today,
haven’t been able to arrive at any conclusive conclusions on.
Despite all the ink they have spilled,
and the efforts they spent in meaningless summations, tedious disputations,
and nonsensical prattle with the sole obsession of rescuing a text
that cannot be rescued except through sophistry, interpolation,
prevarication, nonsense and legends.\fnmark

\fntext{Whoever wishes to compose an approximate picture —
even if it is not precise — of these prattlers and nonsense\–talkers,
then let him listen to the recordings of Sheikh Mutawali Sha\ayin{}rawi,
who’s voice reverberates all over Arab radio.
He explains the \Quran\ with a sharp tongue that erupts like a flood
that he uses to delight the masses and ignorant amongst the scholars,
while the idiots sitting around him roar out the words: “Allah! Allah!” or
“Allah is great! Allah is great!” and grow in zeal and impulsiveness.
If they weren’t in the mosque in a solemn religious gathering they would fill
the world with shouts and clapping as they do at public rallies and
I have complete confidence that they don’t understand a thing that’s going on.
This is an example that is emulated by the ignorant amongst the scholars and
the religious teachers and preachers and
Imams of mosques and the rest of this type.
He (Sha\ayin{}rawi) is regarded by his followers and admirers,
to be amongst the greatest (scholars) of Tafseer in this day and age —
even a unique phenomenon amongst the phenomena of this age.
He is even considered by his pupils to be amongst those who
the prophet alluded to in the famous hadith:
“Indeed God will send to this Ummah (Nation) at the beginning of
every 100 years he who will renew\/reestablish their religion for them!”}

Just as the \Quran\ challenged Mankind and Jinn to bring the like of it,
likewise al\–Razi challenged the Arab Scholars of eloquence to bring
the like of that which is in the book of “Elements” (by Euclid)
or Almagest (by Ptolemy) and others.
Al\–Razi says: “Indeed we demand from you the like of that which you claim we
are not able to do,”\fnmark\
and with this he threw the burden of proof back to the adversary.
In other words with this challenge he showed that the proof itself must lie
with the adversary (making the claim), since it is not within the ability of
man to bring the same that another man has brought, no matter how great
is his ability in copying and perfecting the art of imitation.

\fntext{Ibidem, p.\ 218.}

Furthermore indeed these books and their like are more useful and of greater
benefit than the \Quran\ and all the divine books, because they contain
knowledge that benefits people in their livelihoods
and situations in the real world,
while the Torah, Gospels and \Quran\ benefit nothing.
And if one must discuss Miraculousness and Proof then these useful books
are more deserving of having such things ascribed to them.
In this respect, al\–Razi says: “By Allah, if he wanted a book to be a
Proof the books like the ‘Elements’ or ‘Almagest’ that lead to understanding
of the movement of the stars and planets, or the books of logic,
or the books of medicine that is of benefit to the body,
would be more deserving of being (called) Proof than those (divine books)
that are of no benefit or harm.”\fnmark\@
Meaning the \Quran\ and its like.

\fntext{Ibidem, p.\ 219.}

In any case I am not the first to present criticism of the \Quran.
I cannot claim that honour.
No! Nor will I be the last, for indeed my work here has precedence,
but it differs from that which preceded it in
respect of the method of treatment,
and in respect of the level and terms and fields of knowledge.
But it is the duty of the pioneer to always acknowledge those
who blazed the trail and opened the way before them.
As for the right of the one who went before upon the one who comes after,
it is that no-one will deny him other than the arrogant fool.
For if the one who comes after had not found assistance and
clarification from the one who was before,
things would not go right for him and he could not complete his intent,
and his efforts would be futile and his aim confounded and
thus is the blade blunted and the mind become dull and aspiration fails.
“And those who went before are the foremost.
They are the ones who will be drawn near.” (\QRef{56:10–11})
(\NB{There is of course irony in that quote from \Quran.})

\section{The Eloquence of the \Quran}

We must now ask: Is the \Quran\ really miraculous?

The belief in the Miraculous nature of the \Quran\
does not withstand scrutiny in any way.
There are numerous fallacies that surround this belief.
We have seen some clear examples of that from Ibn al\–Rawandi and
Abu Bakr al\–Razi and in a little while we shall see many other examples
that refute this belief so long as we look impartially and objectively
at the issues so we are not swayed by the majority or the prevailing views,
for scientific facts are not discovered by voting as in Parliaments
no matter how large the number of votes support it.

Miraculousness is of two types in my opinion: Language and Meaning.

As for the Miraculous Language, its conditions are clarity of expression,
fluency of wording, being free from complexity,
weak composition and disharmony.
The speech must have a uniform level of quality, excellence, and perfection.

But miraculous language has no value
if it is not accompanied by miraculous meaning.
If not then it’s just an arrangement of words cobbled together,
good\–looking gibberish, meaningless padding.
For that reason eloquent speech must have consistency and symmetry
in its ideas and packed with meaning.
It must be free from error and contradiction.

But the verses of the \Quran\ are uneven in quality
in both language and meaning and this was noticed
by the classical scholars as confirmed by al\–Suyuti.

Although a large portion of verses are the height of excellence and beauty,
another portion of verses fall far below that,
while others are weak and flawed.

In the same way ambiguity and riddles envelope a significant number of
the verses to the extent that one is confused when trying to understand
the intended meaning of this or that verse.
While some appear to have no meaning at all,
despite the fact that the exegetes (Mufassirun) and
scholars of Eloquence “discovered” a thousand and one meanings.

Indeed the books of (the scholars) of Eloquence are full of chapters
that have no meaning and have been contrived simply to provide an escape
route and justification for the babble in some of the verses
that confront the reader.
Using the pretext of delving deeply for the secrets
and sublime miraculousness of the \Quran.

In my opinion the whole science of Balagha (Eloquence)
was contrived in order to defend the \Quran.
In other words, only for ideological reasons, not to find the truth.
Indeed, Ideology is the governing factor in all the treaties of our scholars
in this field at the expense of objectivity and scientific methodology.

Finally, in addition to what we see in the \Quran\ of fragmentation
and disarray, not to mention blatant scientific mistakes.

So does all that correspond with the belief in the miraculous
(nature of the \Quran) in any way?
Or are there locks on hearts?
(Reference to \Quran\ \QRef{47:24}).
This is what we shall investigate now.

The majority of those who studied the \Quran{}ic text are not Westerners —
if not all of them.
They treat it on the basis that it is a holy text.
That it cannot be criticised.
Since no falsehood comes to it from the front nor the back.
The presumption of its authenticity and infallibility is a prerequisite
that places a barrier that comes between us and it.
It deprives us of much of the wealth that may accrue from (their study of) it.
In that way we close all the doors that were open
in front of us before we begin.
And nothing remains for us to do in this case,
but pour everything we possess of effort into embellishing,
and polishing the text and imposing upon it
that which is unlikely and defend it — right or wrong —
and to “discover” what is in it of hidden treasures and secrets and
wisdoms and meanings that boggle the mind and astound the intellect and
thus begins the journey of searching for the pearls.

The text may not be more than a collection of bombastic speech
that does not mean anything, but the exegete (Mufassir) —
with his believing background and generous expectations —
presupposes there is the wisdom of the ages in the text,
because it is from the a wise all\–knowing one.
“The Trustworthy Spirit has descended with it,
to thy heart so that you may be of the warners.” (\QRef{26:193–194}).
I say, if the text didn’t mean anything,
then indeed the exegetes ended up seeing everything in it!
It became the protected pearl and the hidden jewel.
But this is a profoundly bankrupt method of dealing with the \Quran{}ic text.
It doesn’t reap anything other than hot air and does not result
in anything other than waffle, double-talk, concoction,
and falsely attributing to the text that
which never occurred to its original author at all!

Indeed! The \Quran\ is not amongst the secrets of the gods,
it is not connected in any way to divine inspiration
that would take it outside historical trends.
It is purely a human achievement that complies with human principles.
Like all human efforts it is subject to strength \& weakness,
accuracy \& error, agreement \& contradiction, cohesion \& disarray,
consistency \& inconsistency, originality \& imitation,
depth \& superficiality, lucidity \& brittleness...

The direct result of all that is that the \Quran\ is a very ordinary book.
For that reason it is necessary to remove it from its safe and secure refuge,
outside Human history and return it to the world of people.
After that it will no longer be storehouse for timeless wisdom,
nor a divine book protected from error that no falsehood
can approach it from either front or behind.
In that way, it and its time and its context become
part of the historical process and unfolding events.
(\NB{This paragraph and part of the last is repeated
from the beginning of this chapter.})

If you read the \Quran\ you will find ample evidence of the Divine Being,
acts of worship, exhortations, morals, legislation, injunctions, wisdoms,
parables, stories and legends...\ but you will hardly come across one page
where ideas correlate or flow in a connected sequence
or follow on from one another,
unless the text is recounting the narrative of a story,
or establishing a rule, which requires a certain amount of elaboration.
But as soon as it finishes, it jumps to another subject
that has no connection to it.
That is then interspersed with digressions that
interrupt the narrative flow leaving it without a point.
So our waffling exegetes (Mufassirun) are forced
to come up with a point for it.
If they find a point then it is only stumbled across after
strenuous excavation that the wafflers attribute to profound wisdom.

There are complete pages in the \Quran\ that are full confusion,
as well as offensive words and weak expressions.
It contains hollowness, affectation, artifice, fabrication, and ambiguity.
Words that have meanings that conflict with one another,
making it hard for one to decide
which of the two conflicting aspects is the intended one.
If that was simply confined to insignificant secondary issues
it would be less important,
but it extends also to issues of belief and legislation.

Not forgetting, that in addition to these the errors and flaws,
the \Quran\ contains contradictions that the eye cannot miss.
How much effort the wafflers spent in trying to conceal them
and give them strange meanings that they don’t have,
to make them the epitome of wisdom and sobriety!

In addition to this series of drawbacks that the \Quran\ is packed with
and which we shall see detailed for ourselves,
is the mixing of the speech of Allah with
the speech of man within a single verse.
So while the first half of the verse starts off
in the words of the prophet or one of the pious,
we find the second half ending in speech that cannot be a human speaking,
but must be attributed to Allah.
So either this portion has been inserted
into the text or the verse is incomplete,
half of it being lost, so the scribe completed it —
and most of them didn’t understand what they were transcribing —
according to whatever words came to their minds,
repairing the verse and filling its gap.
This is despite all that is commonly disseminated
about the authentication of the text and close attention
to detail during its recording process.

Last but not least, the scholars find very great difficulty in
accepting many verses from the “Wise Reminder” (a name of the \Quran)
due to its complete opposition to scientific facts in the present time.
These verses are true as long as science, philosophy,
and myths are all approximately one and the same thing.
But today the situation has changed and the position has become clear
as to the extent of the naïvety of the \Quran\ when we see
that it accepted all and sundry of handed\–down knowledge of ancient times
and then attributed it to the “treasure” of Divine knowledge
about the secrets of the universe, life and destiny.

Despite all this they want us to believe that the \Quran;
“Had it been from other than Allah, they would surely
have found therein Much discrepancy” (\QRef{4:82})
but the patching\–up of the Wafflers is a guarantor by which every conflict
is reconciled and the reply to every objection
and bestows on the \Quran\ a fluent,
harmonious unity free from defects so they can produce in front of them:
“An Arabic \Quran\ without any crookedness” (\QRef{39:28}).

We shall discuss all that in the widest scope possible as well as detailing,
clarifying and illustrating as the situation permits so we can open
covered hearts and deaf ears and remove the veil over eyes
(\NB{Ironic references to the \Quran})
that cannot see other than what they want to see, and let loose the tongues
so that they do not say anything about the truth
except the truth and express nothing but the truth.

In this regard, and whatever our verdict on the \Quran,
it does contain bouquets of masterpieces and marvels that the fair\–minded —
regardless of where their loyalties lie or
what their beliefs and convictions are —
cannot fail to be moved by them and bow in prostration.
But is the whole \Quran\ like that?
No and a thousand times no!
For indeed these verses and those that surround them
are spectrums of light and rings of radiance
that captivate the mind, heart and emotions.
But because of the ink they caused to be shed, pens they aroused,
energies they let loose and passions they stirred —
I say because of spotlights they were put under —
these verses hid another portion of verses from sight
and cast them into the dark.
As a result we only see that which catches the sight
and are blind to anything else.
But if we remain in this state — whether we realise it or not —
we will pass the same verdict on them both and how foolish is that!
Thus we would put the dull verses in the same category as the glittering verses
and be oblivious to the huge gap between them
simply because they share the same name; \Quran.
Just like one who puts mud (\ar{الثرى}) in the same category
as the Pleiades stars cluster (\ar{الثريا})
because they share he same root (\ar{ث ر ي}).

So never think that the whole \Quran\ is of the same quality,
cast in the mold of these outstanding verses
that we presented in the previous pages — certainly not.
These are instances of pearls and gems being found amongst earth and pebbles.
Like neighbouring pieces of land with a sprinkling of grape vines here
and there while in other places grow poisonous shrubs, gum trees,
flowers and date palms between sand dunes that are scattered
with weeds, cane stalks, and harmful herbs.
Are these the same, for example?

This is what the \Quran\ is like.
It is — as we mentioned before and as we shall see in more detail —
not on one level of quality, brilliance or splendour.
But contains the poor as well as the rich and all that lies in\–between that.
Such a mixture of things that it is very difficult
for the mind to see how to reconcile them.
But they are reconciled by force and coercion
and when concoction and waffling (of the Mufassirun)
gets involved in sewing together the tears,
mending the cracks and plugging the holes,
some of them easy to accomplish and some so intractable
they require huge effort and resources and some are enigmatic mysteries
as though the mind was fettered by them.
We shall remove from you your covering, oh reader,
so that your vision tomorrow will be sharp!
(\NB{Ref to \Quran\ \QRef{50:22}})
And tomorrow is near for he who envisages it!
(\NB{Ref to a line of poetry that has become a saying.})

1. Look at this wonderful pearl where the \Quran\ describes uncovering
the secrets of the wrong\–doers and exposing their affair in front of God
who makes their body-parts speak on the day of Judgment.
So that they bear witness against them about what they have committed of sins
that they thought have been brushed under the carpet,
never to return but they were recorded and able to articulate the truth:

\begin{quote}
“On the Day that the enemies of Allah will be gathered together to the Fire,
they will be marched in ranks.
At length, when they reach the (Fire),
their ears and their eyes and their skins testify against them
as to what they used to do.
They will say to their skins: “Why are you bearing witness against us?”
They will say: “Allah who makes everything speak has made us speak:
He created you for the first time, and to Him you are returned.
You did not hide yourselves lest your ears and your eyes and
your skins should bear witness against you,
but you thought that Allah did not know much of what you did.
But this thought of yours which you did entertain concerning your Lord,
has brought you to destruction,
and (now) have you become of those utterly lost!” (\QRef{41:19–23})
\end{quote}

So if this is a “divine” masterpiece and is of inimitable style,
the like of which cannot be achieved —
and that is true, then is it possible to achieve
the same as this “human” masterpiece by al\–Jahiz? (781–868)
which he states in his unique and delightful style, in his book;
“Squaring the Circle”, which overflows with style,
eloquence, clarity and illumination:

\begin{quote}
“Nay why do their sayings concern you or their dispute weigh upon you?
Those of understanding and who speak from knowledge,
know that the abundance of your width detracts from the height of your stature
and what shows of your width absorbs what shows of your height.
Although they differ about your height, they agree about your width,
and since they spitefully concede to you a part and
unjustly deny from you a part, you have gained what they conceded,
while you stand by your claim regarding what they didn’t concede.
I swear that the eyes make mistakes and the senses lie and
there is no conclusive verdict other than that given by intellect and
no true enlightenment except by way of the mind since it is the rein
for the limbs and the measure for the senses.”\fnmark
\end{quote}

\fntext{“Squaring the Circle”, edited by Charles Pellat, p.\ 5.}

One cannot mention the princes of speech without mentioning
Abu Hayyan al\–Tawhidi (923–1023).
For he wrote comprehensive works, and on his tongue wisdoms gushed forth
and deep meanings swarmed, yet his age deprived him the acclaim he deserved.
I present to you here this text which is from the beginning of (his book)
“Enjoyment and Conviviality” in which he describes the world,
in the most briefest of ways, so full of meaning and in concise expressions
as though he is describing his burning soul and faltering fortune:

\begin{quote}
“Indeed this fleeting (world) is beloved, its luxury sought after,
and a place amongst those of high council is solicited by any means and manner.
For this world is sweet and verdant, delectable and lush.
He who is timid, his task will be arduous,
while he who’s pressing is passionate,
his coming and going will advance continually,
while he who is held captive by his expectations,
his hardship will be long and his misfortune great,
while he who’s greed and desire are inflamed,
his impotence and deficiency will be exposed.”\fnmark
\end{quote}

\fntext{“Enjoyment and Conviviality”, edited by Ahmad Amin
and Ahmad al\–Zayn, Cairo, p.\ 13.}

Badi\` al\–Zaman (al\–Hamadani) (967–1007) was intricate
just as al\–Jahiz and al\–Tawhidi were
(\NB{Their work had layers of meaning}),
he was a master at delighting (the reader),
words in his hands were obedient and compliant,
redolent with fragrance and aroma, diffusing the scent of perfume.
A great deal of his work has reached us that one never finishes contemplating.
They are no less excellent and eloquent than many of the verses
of the “Reminder of the Wise” (The \Quran).
But many of the readers take it (the writings of al\–Hamadani) for granted.
Let us read this beautiful artistic piece where
he describes his hunger during a year of famine in Baghdad,
and how all his hopes of obtaining food evaporated,
and he ended up with nothing but pain and grief.
He uses the (fictional character) \`Isa ibn Hisham to relate it:

\begin{quote}
“\`Isa ibn Hisham related to us and said: I was in Baghdad the year of
famine year, and so I approached a group,
huddled like the stars of Pleiades, in order to ask something of them.
Amongst them was a youth with a lisp in his tongue.
He asked: ‘What do you want?’
(\NB{\Quran{}ic ref to \QRef{20:95}.})
I replied: ‘There are two conditions in which a man prospers not;
that of a beggar wearied by hunger,
and that of an exile to whom return is impossible.’
The boy then said: ‘Which of the two gaps would you like me to fill first?’
I answered: ‘Hunger, for it has become extreme with me.’
He said: ‘What would you say to a loaf of bread on a clean table,
picked herbs with sour vinegar, fine almonds with strong mustard,
roast meat ranged on a skewer with a little salt,
brought to you now by one who will not procrastinate with promises
nor torture you with waiting, and who will afterwards follow it up
with golden goblets of grape? Is that preferable to you, or a large company,
full cups, variety of dessert, spread carpets, brilliant lights,
and a skilful minstrel with the eye and neck of a gazelle?
‘If you don’t want this or that, then what do you say about some fresh meat,
river fish, fried aubergines, the wine of Qutrubbul,
freshly harvested apples, a soft bed on a high apartment,
opposite a flowing river, a bubbling fountain,
and a garden with streams in it?’
\`Isa ibn Hisham related: So I said: ‘I am the slave of all three
(options you have given me).’
The boy said: ‘And so am I their servant, if only we had them!!’
I said: ‘May God not bless you! You have revived desires which
despair had killed, then you snatched away the object of its relish?!’”
\end{quote}

Can you see this captivating beauty that
the \Quran\ does not have a monopoly over?
Al\–Jahiz, al\–Tawhidi, Badi\` al\–Zaman
and many other greats of prose and poetry
such as Ibn Muqaffa\`, Abu Nuwas, Abu al\–\`Ala al\–Ma\`arri
from the classical literati and al\–Mazini, al\–Rafi\`i, al\–Aqqad,
and Ta Ha Hussein from the moderns —
they and their like have left us masterpieces that are as good as —
if not better, at times, than some of the verses of the \Quran.
They left us a massive legacy full of profound wisdoms and clear signs
(\NB{Ref to the words for ‘clear verses’ in the \Quran}).
But which one of them claimed that he is speaking under heavenly inspiration
or that he encompasses the secrets of the Divine?

So the \Quran\ as we have mentioned previously is not on one level of quality.
On the contrary it is characterised by mediocrity, banality, weakness,
disarray, fragmentation, confusion, ambiguity,
and disclaimers alongside excellent verses that exhibit
majesty, greatness, eloquence, cohesion, clarity and complete accountability.
The exegetes were at their wits end in trying to explain this phenomena
and so embarked on desperate efforts to ignore them and push them out of
the spotlight so that we wouldn’t come upon any of them during
the discussions about eloquence, rhetoric, beauty, marvels, linguistic
artistry of the others that adorn the \Quran.

They focussed on the masterpieces in books about miraculousness of the \Quran\
and used them to illustrate it in every chapter, section and page —
almost in every line — of their forced books,
whether appropriately or inappropriately,
until ears were sick of them and the mind was bored of them.
Indeed the degree that the spotlight was concentrated on some verses
was equalled by the degree that another portion of verses
were swept under the carpet.
They imposed an invisible barrier around them,
so that attention skips over them swiftly and lightly
leaving no time for pondering or contemplation.

All our readings of the \Quran\ are readings done as acts of worship
that only increase blindness upon blindness each time more
is committed to memory and the tongue perfects recitation.
It is not a recitation that involves analysis, critique,
understanding or penetrating appraisal.

Yes, indeed the exegetes were hard pushed to explain these verses
and create ways out for them.
They ignored them whenever citing examples (of eloquence)
and resorted to their “contortion” every time they came across them
in their writings and forcing them to encompass meanings
they didn’t encompass in order to preserve its (the \Quran’s) integrity.

They were the knights in shining armour, ever present,
never tiring of a challenge,
never finding it too burdensome to come up with a reply,
never letting any objective defeat them,
never letting weariness weigh them down.
They were standing by the door answering every visitor.
The students of Hermeneutics can find fertile ground and expansive pasture,
amongst them, to support their analytical theories.
You know them by their signs (\Quran{}ic reference).
They are people of waffle, carriers of the incense.
Some of them went to such absurd extremes that they became a laughing stock.
They “discovered” amongst the confusing, bewildering, fluctuating, disordered,
disturbed, incomprehensible, contradictory verses, eloquent nuances
and sublime connotations that are too subtle for the ordinary mind,
that escape people’s understanding and challenge the intellect to the extent
that no-one can fully perceive them other than those ‘firm in knowledge’
(ref to \Quran) — even if they can perceive them!!

Give me a lunatic and I will unearth pearls, jewels
and timeless gems of wisdom from his speech.

They were able to extract meaning from that which had no meaning,
they never found it hard to make the barren fertile,
the mute articulate, incoherence eloquent
and every old man into the prime of his youth.
In their hands everything is brilliant and fluid,
excellent and magnificent and hence comes from Heaven.
Even if it is a thorn, bitter gourd,
deadly poison or the like of such scourges.

For Heaven cannot stand up on its own except with the help of the one-eyed,
the lame, the scrawny and every decrepit moronic dim-wit.
Repent to the dim-wits, for indeed they hold the keys to Heaven!

The judgment of the critic becomes defective the stronger is his faith
(in that which he is analysing\/critiquing),
until he only sees in the \Quran\ what he wants to see
and is blind to what he doesn’t want to see.
If you expose to him the extent of the flaws in the \Quran\
and the abundance of contradictions and his handling of them,
he will fume and seeth with rage, and insult and curse.
He will block his ears to you as he has blocked his mind
and make the most vile accusations against you.
Woe to you, for you have come to create Fitna
and turn him away from his religion unless Allah makes him
steadfast and blesses him with the blessing of strong faith.

Watch how he will block his ears as though saying “This is a manifest lie”
(\QRef{24:12}) as the people of Noah did when he said — speaking to his lord;

\begin{quote}
“And every time I have called to them, that Thou mightest forgive them,
they have (only) thrust their fingers into their ears,
covered themselves up with their garments, grown obstinate,
and given themselves up to arrogance.” (\QRef{71:7})
\end{quote}

And this is what the Polytheists of Mecca did, so the \Quran\ said to them:

\begin{quote}
“Had we sent down unto thee (Muhammad) (actual) writing upon parchment,
so that they could feel it with their hands,
those who disbelieve would have said:
This is naught else than mere magic” (\QRef{6:7})
\end{quote}

Woe upon woe to he who utters a single word of criticism against
the truth of Islam and what a catastrophe and calamity of calamities
if this criticism harms a single word of the \Quran.
So I wonder what is the difference between us and what we see today
and between the people of Noah and the polytheists of Mecca?\fnmark

\fntext{Perhaps you have heard of the ministerial crisis in Kuwait
and the demand for the dismissal of the Minister of Awqaf.
Why? Because of the issue of a new edition of the \Quran\
containing some unintentional mistakes,
which will send the minister to Hell on a day when no intercession
will be had except one who has taken a promise from Allah.
The printing mistakes came to light while he was minister —
‘Perish his hands!’ — and appeared in a number of copies —
May Allah humiliate him, he has brought a terrible thing,
the heavens almost shatter, the earth split,
the mountains collapse that he allowed the book of God
to have flaws enter it and didn’t prevent or avoid it — may God destroy him.
He thought it was a trivial matter.
He didn’t see it — Woe to him! — as a grave, serious, obligation.
So return him to Allah — him and his like — for that is more pure and fitting.
If he desists not then he and his like will burn long in the Fire of Hell.
And none shall come to the Merciful, except as a slave,
and each one will come to him on the Day of Judgment alone!
(NB: \url{http://news.bbc.co.uk/1/hi/world/middle_east/335404.stm})}

In summary, those who prattle on about the \Quran, heaping praises on it,
raving about its eloquence and sublime beauty,
filling the world with clamour about the miraculousness of the \Quran,
and that it is the ‘Greatest Miracle’\fnmark\ of the \Quran\
only cite the amazing and excellent (verses) that grace the \Quran\
and which form the basis of the magic of the \Quran.
Their attention was poured on selected verses where there is no doubt
about their eloquence and height of excellence and beauty.

\fntext{The name of a book by Muhammad Abu Zahra that is extolled
by the masses — nay, by the select and the select of the select.}

But how many of them didn’t turn their attention to the weak
and poor (verses) of the \Quran\ of which we shall give examples of shortly,
and if they did turn their attention to them,
they undertook the task of polishing, refinement and improvement
in order to repair their defects and cover their flaws
so that they leave their hands as protected gold nuggets
or hidden pearls befitting the Lord of Might and Majesty —
the Cleaver of daybreak (ref to \Quran) until the Day of Judgment!

\pardivider

Eloquence is composing words in such a way that they correspond
to the meanings and embellish them with radiance.
Eloquence is not simply to address people according to what they understand.
Eloquence is to raise them up to reach your objectives
by clarifying them in ways that make (people) comprehend
everything you want to enlighten them with.

Addressing people according to their mentality and understanding,
sacrifices meaning, depth and comprehension.
It gives precedence to a loose meaning over a precise and complete meaning.
It distances what is said from its objectives.
Thus it is the job of the literary master to aspire for something higher
through his artistic expression and not to resort to facile speech.

But we find many verses of the \Quran\ are vague and based on ambiguous,
imprecise concepts that do not deliver clarity of underlying meaning,
because the words in them lack precision and accuracy
and some more like puzzles and quizzes.

For precise language helps mold precise ideas,
while ambiguous language confuses the mind and muddles one’s thinking.
For that reason, if we want speech to be eloquent,
then we must fulfill the condition of clarity and transparency
and effectiveness to reach the ear in the most beautiful language
and most lucid speech not to mention soundness of meaning,
being error free and having no contradictions.
It does not befit the writer of eloquent speech
that his meanings should be disordered or contradictory
or to make mistakes in wording and meaning.

Amongst the things that contribute to clarity are candour, conciseness,
soundness, using moving words without being too abstract.
Using short sentences without overstatement
and having preference for subtlety over coarseness.
Avoiding padding, obscureness, concoction,
and using words with multiple meanings and in particular words
that have contradictory meanings.

Clear, eloquent speech must also have all its parts linked
with each other and flow harmoniously and in succession
and sequence with each other part.
It should not jump from one sentence to another
before fully dealing with and completing its components.
This means each sentence should carry the same seed as the following sentence
and the following sentence concludes and completes the previous sentence.
In this way each part flows naturally from each other part,
aesthetically unified, coalescing and integrated
as if they were a solid cemented structure (ref to \Quran.)

In short: Eloquence (al\–Balagha) is from al\–Buloogh which means
to reach and in relation to our subject here it is for the meaning
to reach those for who it was intended.
The crux of the matter is reaching the meaning and arriving at it.
However, the meaning only becomes apparent if it has been demonstrated clearly
and likewise it is not clear when it has not been demonstrated well.
So whenever it (the meaning) is hidden or abstruse
then the speech has lost its purpose and becomes just babble
with no use or benefit behind it.

\pardivider

Now, after this short tour round the subject of Eloquence and its conditions
and the difference between eloquent speech and speech
that is not eloquent we are justified in asking:
What is the position of the \Quran\ in all this?
What is its degree of eloquence in it?
Is it on one level of eloquence or is there a disparity between its verses?
What is the degree of this disparity?
This is what we shall discuss in the following section.

\section{Where is the Eloquence of the \Quran?}

There are red lines that all Muslims who study the \Quran\
do not allow themselves to cross.
None of them start from zero.
On the contrary they start off with absolute faith in His words —
Exalted is He!

\begin{quote}
“Indeed it is a Book of exalted power.
No falsehood can approach it from before or behind it:
It is sent down by One full of wisdom, worthy of all praise.”
(\QRef{41:41–42})
\end{quote}

And His saying:

\begin{quote}
“Do they not then ponder on the \Quran?
Had it been from any other than Allah,
they would surely have found therein much discrepancy.” (\QRef{4:82})
\end{quote}

So falsehood cannot creep into the \Quran\ in any way
just as it is free from discrepancy.
These two unassailable principles are not open to discussion.
We can also add a third verse that emphasises the infallibility
and inviolability of the \Quran:

\begin{quote}
“Say: ‘If the whole of mankind and Jinns were to gather together
to produce the like of this \Quran, they could not produce the like thereof,
even if they backed up each other with help and support.’” (\QRef{17:88})
\end{quote}

Would that I knew how a person can analyse the \Quran\ in an objective,
dispassionate, independent way when his hands are tied by these three verses?
Remove these shackles and you will immediately see that falsehood
has indeed found its way into the \Quran,
just like any human accomplishment.
That it resounds with contradictions and all types of disparity.
That it is possible to bring the like of it and even better than it also.
Remove the covering from your eyes and free yourselves from restrictions.
‘But to whom are you singing your Psalms to Oh David?’
(\NB{Arabic proverb, i.e.\ ‘Will anyone listen?’})
No one wants to risk playing with fire,
it simply doesn’t enter the mind and if it does,
then it can’t entertain it, and if it does entertain it
then it cannot act upon it.
Nay, even those who are beset by doubts about the soundness of the \Quran\
do not dare declare their true opinion.
If they do then they do it apologetically and behind a veil (\Quran{}ic ref.) —
a thousand and one veils.

Thus he who wants to know their views about this,
must use a level of ingenuity and intelligence
that would enable him to bring out the suppressed in their writings
and reveal the repressed by reading between the lines.
As I said previously, they don’t want to play with fire,
choosing health and desiring peace, instead.
As for me I love playing with fire and there will be many more after me.
It is fire that burns out the blemishes on gold,
and destroys all spots and stains.
If you aspire for something higher, you must live dangerously!!

\pardivider

The first thing that strikes one in the \Quran\ is its disjointed nature.
Yet this disjointedness is not felt by the believer,
firstly because of his long familiarity with the text and secondly
because his faith is a protective armour,
shielding him from paying attention to the flaws this text contains.
As for the non\–Muslim, and especially if he is an Orientalist
studying the \Quran\ for the first time,
he will be stunned when he sees this strange cocktail in a single chapter —
in fact, in a single page — of the word of the Lord of the Worlds.
He may have been taken aback by many things,
but not a cocktail like the \Quran.

1. Continuity is rare in the \Quran.
In fact it is non\–existent apart from Sura Yusuf
and some of the short stories,
then it reverts to its original style of interruption and disjunction.
Even Sura Yusuf which contains one hundred and eleven verses,
has the last nine verses disjointed from those before it,
not to mention that these nine verses between them are a strange cocktail
with no connection between the elements that they are made of.
However the waffling exegetes had no problem in uniting this untidy hem
into one piece and creating all kinds of links and ties between its elements.
And no wonder! For each one of them — like Allah — is ‘able to do all things’!
That is when they turn their attention to any disjointedness
or disarray in the \Quran\ — or at least — when they admit to it!!

2. Look at these jumping verses and show me what links them together?
(\NB{Verses \QRef{17:70–88} of Sura al\–Isra\`})

\begin{quote}
70. Indeed we have honoured the children of Adam.
We carry them on the land and the sea,
and have made provision of good things for them,
and have greatly preferred them above many of those whom we created.

71. On the day when We will call every people with their Imam;
then whoever is given his book in his right hand, these shall read their book;
and they shall not be wronged in the least.

72. Whoever is blind here will be blind in the Hereafter,
and most astray from the Path.

73. And they indeed strove hard to beguile you
(\NB{Singular, i.e.\ addressing Muhammad})
away from that which We have revealed to you,
that you should invent other than it against Us;
and then would they have taken you as a friend.

74. And if We had not made you firm
you might almost have inclined to them a little.

75. In that case We would certainly have made you to taste
double (punishment) in this life and double (punishment) after death,
then you would not have found any helper against Us.

76. And surely they purposed to unsettle you from the land
that they might expel you from it,
and in that case they will not tarry behind you but a little.

77. (Such was Our) way in the case of those whom We sent before you,
you will find no change in Our ways.

78. Establish regular prayers,
at the sun’s decline till the darkness of the night,
and the morning recitation; surely the morning recitation is witnessed.

79. And some part of the night awake for it, an extra one for you,
maybe your Lord will raise you to a praised position.

80. And say: My Lord! make me to enter a goodly entering,
and cause me to go forth a goodly going forth,
and grant me from Thy Presence an authority to aid (me).

81. And say: Truth has come and Falsehood has vanished;
surely falsehood is ever bound to perish.

82. And We reveal of the \Quran\ that which is a healing
and a mercy for believers though it increase the evil\–doers
in nothing but ruin.

83. And when We bestow Our favours on man he turns away and behaves proudly,
and when evil afflicts him, he is despairing.

84. Everyone acts according to his own disposition,
but your Lord best knows who is best guided in the path.

85. They ask you concerning the soul, say:
The soul is one of the commands of my Lord,
and you are not given anything of knowledge but a little.

86. And if We wanted, We could certainly take away
that which We have revealed to you,
then you would not find anyone to plead your case against Us.

87. Except for Mercy from your Lord: Indeed his bounty is to you great.

88. Say: If men and jinn should combine together
to bring the like of this \Quran,
they could not bring the like of it,
though some of them were aiders of others.

(\QRef{17:70–88})
\end{quote}

Indeed the whole of Sura al\–Isra\` is like this.
Jumps that the \Quran\ makes from one place to another,
without traversing the wide roads or intersections between them
or covering the vast spaces that lead to them.
Does this have the slightest connection with eloquence,
oh princes of eloquence?
Answer me, oh champions of twisting, turning and apologetics?
I cannot see in all that other than an insult to the mind
and lulling it into unhealthy consequences and a terrible end!
(ref to \Quran.)\@
What is the difference between you and the journalists of the third world
that sell themselves to the ruler and promote his decree
in every place without conscience or integrity?

Disjointedness and imbalance in verses of the \Quran\ are the rule,
while cohesion, continuity and consistency are the exception.

3. What do you say, please, about the following verses?
Give me your opinion on the matter oh masters of clear speech and eloquence,
oh guardians of logic and evidence.
Allah — Exalted is He — said during the story of Yunus (Jonah)
when the whale swallowed him up:

\begin{quote}
143. But had it not been that he was of those who glorify,

144. He would certainly have remained in its belly
till the day when they are raised.

145. Then We cast him on to the bare shore in a state of sickness.

146. And We caused to grow up for him a gourd\–plant.

147. And We sent him to a hundred thousand, or more.

148. And they believed, so We gave them comfort for a while.

149. Now ask them their opinion:
Is it that your Lord has daughters while they have sons?

150. Or did We create the angels females while they were witnesses?

(\QRef{37:143–150})
\end{quote}

What do angels and their gender have to do with the story of Yunus, here?
How about adding a new section to the subdivisions of eloquence
called the Section of Dissonance or the Section of Incongruity
or such headings that signify the upside\–down standards in the \Quran?

4. Perhaps the cocktail quality here won’t show too much
after a bit of patching\–up,
making it possible to link these disparate verses
in the usual style of the people (the Mufassirun).
But what sort of patching\–up can link the elements of this cocktail
which the eye cannot miss?
A verse from the East and a verse from the West,
and ‘from every valley a stick’, as they say.
(\NB{Arabic saying meaning ‘hotchpotch’; a confused mixture.})

\begin{quote}
52. On the Day when the excuse of the wrongdoers will not benefit them
and they will be cursed and have an evil abode.

53. And We indeed gave Moses the guidance,
and We made the children of Israel inherit the Book,

54. A guide and a reminder for men of understanding.

55. Then have patience, surely the promise of Allah is true,
and ask forgiveness for your sin and sing the praise of your Lord
in the evening and the morning.

56. Lo! those who wrangle concerning the revelations of Allah
without a warrant having come unto them,
there is naught else in their breasts save pride which they will never attain.
So take thou refuge in Allah.
Lo! He, only He, is the Hearer, the Seer.

57. Indeed the creation of the heavens and the earth is greater
than the creation of the men, but most people do not know.

(\QRef{40:52–57})
\end{quote}

Indeed it appears that the disjointedness in the verses of the \Quran\
is part of the necessary requirements of the Wise Revelation!
Turn the pages of the \Quran\ as you like,
you will not find one page free of disjointedness.
They jump out at you without any effort to search and hunt for them.
So is there some profound wisdom in that which eludes our inadequate minds?
That only the ‘firm in knowledge’ can comprehend, and ‘how few are they!’
(\Quran{}ic references.)

5. Continuity is almost never maintained except in the stories
and some of the legislative verses.
Beyond that you see the verses scattered to the four winds
(\Quran, Sura \QRef{18}, verses \QRef{18:46}{46–51}):

\begin{quote}
Wealth and sons are adornment of the life of this world,
but the things that endure, good deeds, are best in the sight of your Lord,
as reward, and better in respect of hope.

On the Day We shall remove the mountains,
and you will see the earth protruding and We shall gather them,
all together, and We shall not leave out any one of them.

And they shall be brought before your Lord, standing in ranks:
Now certainly you have come to Us as We created you at first.
Nay, you thought that We had not appointed to you
a time of the fulfillment of the promise.

And the Book shall be placed,
then you will see the guilty fearing from what is in it, and they will say:
Ah! woe to us! what a book is this!
it does not omit a small one nor a great one, but numbers them (all);
and what they had done they shall find present (there);
and your Lord does not deal unjustly with anyone.

And when We said to the angels: Make obeisance to Adam;
they made obeisance but Iblis (did it not).
He was of the jinn, so he transgressed the commandment of his Lord.
What! would you then take him and his offspring for friends rather than Me,
and they are your enemies?
Evil is (this) change for the unjust.

I did not make them witnesses of the creation of the heavens and the earth,
nor of the creation of their own souls;
nor could I take those who lead (others) astray for aiders.

One Day He will say: Call on those whom you considered to be My associates.
So they shall call on them, but they shall not answer them,
and We will cause a separation between them.

(\QRef{18:46–51})
\end{quote}

6. Strangely this disjointedness is not confined to imbalance
in the sequence of the verses on a single page making it
into an amazing assemblage of disparate verses,
but this imbalance intrudes upon a single verse separating its two ends
and resulting in the last part being at odds with the first part

\begin{quote}
“To Him is referred the knowledge of the hour,
and there come not forth any of the fruits from their sheaths,
nor does a female bear, nor does she give birth, except with His knowledge.
And on the day when He calls out to them: Where are My partners?
They will say, we confess to you, not one of us is a witness (for them).”
(\QRef{41:47})
\end{quote}

What has the end of this verse got to do with the first part?
Why is it that those who harp on about the eloquence
and miraculous nature of the \Quran\ ignore this verse
and its like and limit themselves to the excellent verses which no-one —
regardless of his position on the \Quran\ —
can avoid warming to willingly or not?
As for the other verses — the shaky, unstable and disordered verses
that don’t withstand scrutiny, they pass over them,
oblivious and feigning to be oblivious.
While if they do deal with them, they repair them
and weave threads like a spider’s web to camouflage them
and conceal their flaws.
The masses fall for that and even the select do,
but it’s inconceivable that a rare,
selective shortcoming could make the truly critical eye fall for it:
Even though this shortcoming obscures the truth
and makes one turn a blind eye, in order to play safe.

For the believer — even if he is amongst the experts, or experts of experts —
he will see things through his desire and not his senses,
with his heart and not his mind.
Only the unbiased inquiring eye — and few are they! —
is able to penetrate deeply into matters and probe into the truth of things,
until in a few blinks of an eye, the full radiance of the sun is revealed,
or the true essence of things.
A spider’s web is indeed a spider’s web.
A building cannot stand upright with it,
nor can it keep the suppressed in check.
There is no substance to it, nor can it withstand scrutiny.
However our silence dignifies it.
So who is with me in removing the silence from it.
Indeed the weakest of houses is the house of the spider! (\Quran{}ic ref)

7. Now I present to you these verses (\QRef{4:2–3}),
please help me to understand them — May Allah help you:

\begin{quote}
“And give to the orphans their property, and do not substitute
something worthless for something good,
and do not devour their property into your own property,
that is indeed a great sin.”

“And if you fear you cannot deal fairly with the orphans,
then marry such women as seem good to you two and three and four,
but if you fear that you will not do justice,
then only one or what your right hands possess; this is more proper,
that you may not deviate from the right course.”

(\QRef{4:2–3})
\end{quote}

This last verse is amongst the strangest of things,
for it combines within it two matters that it is not possible
to combine unless it’s possible to combine oil and water.
Despite all that I have read in the books of Tafsir,
and what they contain of reasonable and atrocious and empty waffle
and forced meanings, until now, I am still unable to understand
the connection between justice to orphans and marriage.

It is most likely that between the opening clause “If you fear...”
and the concluding clause “Then marry...” in the second verse,
there is a missing third verse, or a deleted (verse),
omitted either unintentionally or intentionally.
As long as there isn’t some ‘profound wisdom’ or ‘eloquent significance’
that the waffling exegetes have got us accustomed to expecting!!
If not then everything in their bag of tricks
to rescue the verse is of no use.

For the verse as it stands and the way it is, makes no sense!
Indeed the rigidity (of the exegetes) was unable to shed light
upon this verse and could only leaves it as it is — as it was revealed —
fearing alteration or saying something about the word of God
that it doesn’t contain.

8. There is a significant stylistic error
that I used to consider the \Quran\ should be above falling into.
After the \Quran\ describes the comforts of paradise
and things the believer can look forward to,
‘that no eye has seen, nor ear heard, nor mind encompassed’ (from Hadith) —
and this proceeds from the premise of creating the world into a new creation —
it then turns to back to the premise (of creating a new creation)
instead of starting with the premise and ending with its consequence,
or rather, one of its consequences!
This is a back-to-front way of doing things
that the \Quran\ should not slip into (\QRef{21:101–104}):

\begin{quote}
Those for whom the good from Us has gone before,
will be removed far from it (Hell).

They will not hear its faintest sound,
and they shall abide forever in that which their souls long for.

The Supreme Horror will not grieve them,
and the angels will meet them (saying):
‘This is your day which you were promised.’

The Day that We roll up the heavens like a scroll rolled up for books,
as We originated the first creation, (so) We shall produce a new one,
a promise (binding on Us); surely We will bring it about.

(\QRef{21:101–104})
\end{quote}

Shouldn’t it have started with the rolling up of the heavens
then mentioned what follows on from that,
of rewards and punishments people will get?
Oh masters of eloquence, which sub\–division of eloquence
does this upside\–down sequencing fall under?
Is breaking up continuity with a conflicting verse
that has no link with what went before it,
then resuming the narrative after that, an aberration,
anomaly and disharmony or is it amongst the sublime signs of miraculousness?
Please only speak the truth about miraculousness.
Miraculousness is precision in narrative, continuity and harmony,
each part adhering to and supporting the other so as to arrive exactly
at what the author intends and desires, without interruption,
aberration or anomaly in the eloquent, miraculous speech.

9. After relating the story of the People of the Cave\fnmarksym[*]{}\
and how God raised them from their sleep,
the \Quran\ turned to the question of their number
and the people’s dispute about it.
But instead of telling us this number —
this mystery, this unusual curiosity, this hidden secret —
it withheld it from us, so that it would make our hearts sad.

\fntextsym{\hyperref[apdx:cave]
{See Appendix \ref*{apdx:cave} for Ibn Ishaq’s account of the Asbabu\–Nuzul
(Reasons for Revelation) of this story.}}

\begin{quote}
“(Some) will say: three, the fourth being their dog; (some) say:
Five, the sixth being their dog, making conjectures at what is unknown;
and (some) say: Seven, and the eighth their dog.
Say: My Lord best knows their number, none knows them but a few;
enter not, therefore, into controversies concerning them,
except on a matter that is clear,
nor consult any of them about them (the Sleepers).”
(\QRef{18:22})
\end{quote}

If only He had completed the final part of the story
and bestowed upon us the knowledge of how long they stayed in the cave —
them and their beloved dog.
But He preferred — Glorified is He — for the sake of a wisdom
no-one knows except Him —
to dash our hopes to know the truth of this curious affair.
I cannot see, and I am only a poor servant, any reason for that,
even if our masters, the exegetes (Mufassirun),
can see a thousand and one reasons.

Then it says, straightaway after the previous verse:

\begin{quote}
“And don’t say about anything that I will do that tomorrow!
Unless (saying) if Allah wills it,
and mention your Lord if you forget
and say perhaps my Lord will guide me
to a nearer course to the right than this.”
(\QRef{18:23–24})
\end{quote}

And now you have the pleasing item
and the happy surprise after this long wait;

\begin{quote}
“And they remained in their cave three hundred years
and (some) add (another) nine.”
(\QRef{18:25})
\end{quote}

If only He — May He be Glorified — would settle on this number,
but He insists on it remaining wrapped\–up
in the unseen things of the Heavens and the Earth;

\begin{quote}
“Say God knows best how long they remained;
to Him are (known) the unseen things of the heavens and the earth;
how clear His sight and how clear His hearing!
They have no protector other than Him;
nor does He share His Command with any person whatsoever.”
(\QRef{18:26})
\end{quote}

Who knows? Maybe He — May He be Glorified — doesn’t know their number —
them and their auspicious dog — nor how long they remained in the cave.
Instead we have extravagant polemical conjectures
and broad idiosyncratic flip\–flopping and loose,
unrestrained linguistic flapping\–about.
Would that He had never mentioned this story at all,
for it is a story that is cut-off,
I don’t know what the masters of the art of story telling
think about it.

10. Amongst the strangest verses of the \Quran\ and most disordered,
confused and furthest from fluency, soundness, and harmony,
are those that are that way because of the numerous
parenthetical sentences in them, of which there is no end to.
So much so that their edges become entangled with other verses so that
one finds it hard to come across the conclusion of the original verse —
if there is a conclusion — and distinguish it from the rest of the verses.
This was a matter that weighed heavily on the poor exegetes
and forced them to approximate a conclusion for them,
so as to at least preserve the integrity! (of the \Quran)
Indeed amongst the strangest of these and furthest from unity
and cohesion are these verses —
a sprawling long cocktail that is talking about the Jews:

\begin{quote}
“Then because of their breaking of their covenant,
and their disbelief in the revelations of Allah,
and their slaying of the prophets wrongfully, and their saying:
Our hearts are hardened, nay, Allah set a seal upon them for their disbelief,
so they believe not save a few and because of their disbelief
and of their saying against Mary a tremendous lie,
and their saying, we slew the Messiah, Jesus son of Mary, Allah’s messenger.
They slew him not nor crucified him, but it was made to appear so to them,
and indeed those who disagree concerning it are in doubt thereof;
they have no knowledge thereof save pursuit of a conjecture;
they slew him not, for certain, but Allah took him up unto Himself,
Allah was ever Mighty, Wise.
And There is not one of the People of the Scripture
but will believe in him before his death,
and on the Day of Resurrection he will be a witness against them
so because of wrongdoing of the Jews We forbade them good things
which were lawful to them,
and because of their hindering many from Allah’s way,
And of their taking usury and they were forbidden it,
and of their devouring people’s wealth by false pretences,
We have prepared for those of them who disbelieve a painful punishment.”
(\QRef{4:155–161})
\end{quote}

Is this mish-mash part of the marvelous eloquence?
Why doesn’t anyone cite these verses when talking
about the beauty of the \Quran,
the precision of the \Quran\ and sublime melody of the \Quran?
No, they limit themselves to the excellent verses.
Then again perhaps mixing things into a hodgepodge
is part of the miraculous nature of the \Quran.

11. Finally I present to you these two verses — without comment —
for you to decide for yourself what comment they deserve:

\begin{quote}
“And when We said to you: Surely your Lord encompasses men;
and We did not make the vision which We showed you
but as a trial for men and the cursed tree in the \Quran\ as well;
and We put terror into them, but it only adds to their great inordinacy.
And when We said unto the angels:
prostrate before Adam and they prostrated except Iblis, he said:
Shall I prostrate before that which you created of clay?”
(\QRef{17:60–61})
\end{quote}

\pardivider


\section{Disorder in the Distribution of Topics}

This feature of glaring fragmentation in the \Quran\
resulted in considerable anarchy in the distribution of verses
and inability to pursue and explore topics properly.
The \Quran\ is not an academic book divided into chapters
that deal with a specific issue in each one.
Just as the names of the suras do not signify anything important.
The chapter of the Cow (Al\–Baqara) for example does not talk about cows.
It was only named thus because it contains a short story about it
and it could have been named any other name.
Likewise the chapter of the Bee (Al-Nahl) and the Ant (Al-Naml) etc...

Since the \Quran\ is not divided into topics or sections or chapters,
you will find one topic sprinkled over multiple suras
and a variety of verses, inserted here and there.
I don’t know the reason for that other than this must be
amongst the requirements of eloquence and miraculousness.
Who knows, maybe behind this bizarre design
is a mighty wisdom that minds cannot comprehend.

1. Here’s the chapter of Women (Sura An\–Nisa\´) for example.
Chapter number 4 with 176 verses.
It only deals with the subject of women in 32 verses.
What remains of the sura is a fragmented varied assortment
that meanders around individual religious issues,
such as prayer, zakat, kindness to parents, family ties, inheritence,
forgiveness, accepting the decree of God, Jews, Christians,
worshipping Jesus as God, rejection of Polytheism.
Plus long narratives about fighting and jihad and making migration
in the way of God, which in my view should be attached to Sura Al\–Tawba
or Sura Al\–Ahzab, since there is no place for it in this sura,
in fact, it is totally out of place in it.

It’s odd that after talking about women in the first twenty five verses,
the \Quran\ then jumps suddenly to talking about repentance
and family ties from verse 26 to 33,
then it returns to speaking about women from verse 34 to 35.

Then it talks about a variety of other topics
which are not connected to each other by any single theme,
then it stops at verse 126 to resume talking about women,
and that is from verse 127 until 130.

Then it moves on to other topics and matters until the penultimate verse
of the sura, i.e.\ until verse 175.
Then it remembers that in the bow is one last arrow
so saves it to talk about another subject that has nothing to do with women,
but is shared between women and men and that is inheritance
which it didn’t complete in the previous verses
and I’m referring to Al\–Kalala (someone with no heirs,)
which it left off talking about,
until the very last verse of the sura who’s number is 176.

2. There are many other suras in the \Quran\ that talk about women,
such as Sura al\–Ahzab for example, chapter 33, containing 73 verses.
This sura begins with a general preamble from verse 1 to 3
and then from 4 to 6 it speaks about marriage and adoption.
Then comes an interjected seventh verse that has no connection
to what precedes it nor what comes after.
From verse 8 to 27 it talks about fighting and jihad.
Then it returns to talking about women and marriage
and adoption from verse 28 until 38.
Then it jumps to an interjected verse, which is verse 39.
From verse 40 until 48 is some beautiful speech about Muhammad
which in my opinion is amongst the occasional outstanding pieces
that we find in the \Quran\ —
In my opinion these verses should be in Sura Muhammad,
which is chapter 47 in the \Quran, but God’s wisdom demands
that it should be here.
From verse 49 to 59 it returns to talking about women,
marriage and adoption, and wives of the prophet with some interjections
that the \Quran\ has got us accustomed to expecting.
From verse 60 until the end of the sura is an assorted cocktail
that hardly one page of the \Quran\ is free from.

Regarding the presence of the passage about Muhammad in this sura,
in verses which I said are amongst the outstanding verses,
indeed its presence in this place detracts from its excellence
and takes away much of its beauty.
Perhaps this is part of eloquence and signs of miraculousness!
This can be applied to a large number of the \Quran’s excellent verses,
for many excellent verses have had their radiance hidden
through poor choice of positioning —
lost under a huge pile of incongruous material
that has no theme, substance, shape, nor purpose.
Like a beautiful woman from a bad origin.\fnmarksym[*]

\fntextsym{Reference to the hadith: “Beware of the green manure.”
The Companions asked: “What is the green manure?”
He said: “A beautiful woman of bad origin (i.e.\ upbringing).”
(Al\–Daraqutni)}

Likewise we see that the arrangement of the verses in the \Quran\
is very primitive and we can find the explanation for this strange phenomena
in the Abrogator and the Abrogated in the \Quran.
God Almighty said:

\begin{quote}
“None of Our revelations do We abrogate or cause to be forgotten,
but We substitute something better or similar.
Do you not know that Allah has power over all things?” (\QRef{2:106})
\end{quote}

For indeed a great deal of the \Quran\ has gone.\fnmark\@
In fact al\–Suyuti praised Abrogation saying that it is amongst the wisdoms
that God favoured this Ummah with, to make things easier.

\fntext{Jalal al-Din al-Suyuti,
The Perfect Guide to the Sciences of the \Quran, 2/25.}

Suyuti relates many examples of what Uthman left out
when he was collecting the \Quran, on the basis that it was abrogated.
Regarding this is the Hadith of Aysha who said:
“When Sura al\–Ahzab was recited at the time of the prophet
it had two hundred verses,”\fnmark%
\fntext{Ibidem.}
though now it has 73 verses only.
Just as al\–Suyuti also mentioned that a whole sura
was revealed and then removed.\fnmark
\fntext{Ibidem.}

This Abrogation has distorted the \Quran\ and left it fractured,
making it impossible to stitch together or coalesce its parts.
These shreds are what constitutes the \Quran\ that has reached us today.

The disarray and glaring fragmentation we see in the \Quran,
could be the inevitable result of multiple suras in one sura.
Or the remainder of deleted suras of which only these fragments remain.
Or perhaps they are drafts of verses
that should have been revised and reviewed,
but the sudden death of the prophet,
afflicted by the poison that the Jewish woman slipped into his food,
didn’t allow him to complete the required revisions.

My view is that this disarray in the \Quran\ must be faced by firm,
brave action to return order to the disordered verses that have no link
between them and the ones scattered here and there in hundreds of pages
that the Mushaf (\NB{name for the \Quran}) contains between its covers.
There must be an initiative to sort the muddle of these wildly disparate verses
and re\–unify them into a new, rational layout of order, composition
and arrangement of chapters that responds to the demands of the age
and create unity between this huge quantity of disparate muddle
and sweep away the staleness between its parts
which have no beginning nor end, nor head nor foot.

Throughout fourteen centuries not one voice was raised to rectify this defect,
just as in India not one voice was raised in protest
over bathing in the holy river at religious festivals or seeking healing,
even though it’s a filthy river that increases the sickness of the sick.
Likewise no voice was raised in complaint against the cows who are let free
to come and go as they please, grazing in the streets and public places,
wandering between houses and shops without anyone being allowed to touch them,
in a country where the starving see his livestock assets
destroyed in front of him but silently does nothing.
This despite that my comparison with Hindus is not a precise one.

Is this disorder in the \Quran\ from the All\–Wise, All\–Knowing One?
Oh people use your minds and don’t get left behind in the race.
Is this amongst the signs of miraculousness?
Is there not a rational one amongst you?

How much we are in need of a new \Quran\ that will do away
with the old \Quran\ and pull it up by the roots!
Yes indeed we are in need of a new \Quran\ that will keep pace with the age
and progression and evolution of events after Nietzsche declared the death
of the old God and the defeat of his dominion and sovereignty.
Forget the old \Quran, for there is no use in trying to patch-up
the decrepit if we can bring about something new.

Indeed the \Quran\ was once a breakthrough, but now it has become burnt-out.
It was the revolution of revolutions in a time that lacked revolutions.
The \Quran, in its time, was amongst the most important factors for progress.
But today it has become an obstacle to all progress.
This astonishing, baffling and peculiar hopping about that our Arab ancestors
transmitted to us from the margin of history to the dawn of history,
and once inspired them to become innovators of the age, masters of their time.
If it wasn’t for the \Quran\ they would have remained groping about
aimlessly in their stagnant plight till the day of Resurrection.
It is as though the \Quran\ propelled them to engage with the events
of the time and threw them into the vast ocean of world affairs
and helped them conquer new horizons.

Yes the \Quran\ was once a revolution, but, like all revolutions,
it is a revolution for a limited time only.
Then it must make its way to the museum.
Like all revolutions, it eventually becomes reactionary.
The revolution has been replaced by an anti\–revolution.
Yet we stubbornly insist on deluding ourselves
that the revolution is still taking place.
We are now sitting with our \Quran\ in the darkness of the museum,
brooding over memories of our life when we existed outside the museum.
Every time we raise our heads and try to get out of the museum
we are thrown back in it.
It has been centuries since we lived in the time of revolution.
We will never be able to see the truth unless
we believe in truth and embrace it,
for that alone will enable us to see the true nature of things
without the pretense and self\–deception.

The problems of the present generations of this nation cannot be solved
in the same way as they were for the first generations.
This time is a different time and the people are a different people
and the needs and expectations are not the needs and expectations of the past.
But the regressive ones amongst us insist on living with ghosts
and flirting with the spirits of the past,
and refuse to believe that the ghosts are ghosts.
That is the power of ghosts to those who believe in ghosts!

\pardivider


\section{Ambiguity in the \Quran}

Clarity of speech comes from clarity of vision
and clear vision is formulated by lucid thought and expression.
But ambiguous expression only leads to ambiguous meaning.
Many verses in the \Quran\ are constructed of ambiguous material
and so it doesn’t appeal to the mind or become clear to the intellect.
Enigmas that strut about in front of you
without you knowing what they are about.
Words transformed into unintelligible cryptic messages that baffle the mind.
They opened the door wide to folk tales, mythical fantasies, Isra\´iliyat
(\NB{Tales originating from Judeo\–Christian traditions}),
the study of secret knowledge and all manner of weird meanings,
and strange accounts.
Every commentator who dived in to discover their meaning
came out with a precious pearl of wisdom!!

1. The first of these puzzles are the Abbreviated Letters (al\–Muqatta\`at)
at the beginning of some of the suras.

\begin{quote}
Alif, Lam, Mim.
(The Cow, The Family of \´Imran, The Spider,
The Romans, Luqman, The Prostration.)

Alif, Lam, Mim, Sad. (The Heights.)

Alif, Lam, Ra\`.
(Jonah, Hud, Joseph, Abraham, The Rocky Tract.)

Alif, Lam, Mim, Ra\`. (Thunder)

Kaf, Ha\`, Ya\`, \`Ain, Sad. (Mary)

Ta\`, Ha\`. (Ta Ha.)

Ta\`, Sin, Mim. (The Poets, The Stories.)

Ta\`, Sin. (The Ant.)

Ya\`, Sin. (Yasin.)

Sad. (Sad.)

Ha\`, Mim, \`Ain, Sin, Qaf. (The Consultation.)

Qaf. (Qaf.)

Ha\`, Mim.
(Forgiver, Expounded, The Ornaments, The Smoke,
The Crouching, The Winding Sand-Tracts.)

Nun. (The Pen.)
\end{quote}

What are these puzzles?
Is this part of the \Quran\
‘whose verses have been expounded in a clear Arabic tongue’?
(\Quran{}ic ref).
Where is the clarity, Oh people?
Is it in the conundrums?
Has eloquence in the \Quran\ been transformed into a collection
of letters that don’t mean anything to us,
or perhaps He got confused — May He be Glorified —
and thought that we are like Him and encompass all things with knowledge
as if we are Him and He is us? Is miraculousness to baffle people?
One of the most important conditions of eloquence is
that people must understand what you are saying.
Perhaps the one who revealed this has an opposite opinion to that?
Enlighten me, please, if you can?

2. The matter doesn’t stop there. For if the ambiguity here surrounds letters,
we shall see, shortly, that it surrounds the “clear” verses also.
Indeed I have tried to read certain verses and just pure reading
is quite pleasurable, but it is also a burden.
For the words come in a constant stream that don’t relate to each other,
but instead hop over one another and clash with each other.
Converging and diverging, corrsponding and contrasting and contradicting,
halting and then resuming.

Narratives that end abruptly, and then look!
Here they are suddenly returning!
Wonders of expression and manipulation of words that draws before your eyes
what looks like an elaborate embroidery covered in ambiguity.
The words are able to create from letters something
that more closely resembles an intangible vision
and visions have no clear boundaries.
For the rhetorical art has the power to turn the narrative
into an ambigious melody that has no precise significance
but is able to take you out of reality and its burdens
and horrors and transports you to the garden of Eden.

This is the power of words.
For words can be insidious, devious and multi\–faceted.
They thrill with their interplay, interaction, and clashing...
They are an overflowing flood, either you drown in them
or either you swim like a skilled swimmer who saves himself
by detaching himself from the dominating power of words.

In my opinion, this is what explains the strange effect of the \Quran\
on the mind and soul of the general populace.
Nay, even the minds of the elite and the elite of the elite,
and upon the scholars and literati and poets
and philosophers and their like who cannot swim well.
But instead come out with — on a daily basis — scientific discoveries that
the \Quran\ was the first to discover fourteen centuries ago on the tongue
of an illiterate man who cannot read or write and grew up in a remote desert
far from the centres of learning and civilisation.
This is what seduces the general populace and increases
their faith in the miraculousness of the \Quran.

3. It is strange that the \Quran\ often launches
into unnecessary details that have no meaning,
while it lacks details in other places
where they should be clarified without hesitation.
Take this verse as an example:

\begin{quote}
And mention in the Book, Moses, he was one purified,
and he was a messenger, a prophet.
And we called him from the right side of the Mount,
and drew him near to Us, in communion.
(\QRef{19:51–52})
\end{quote}

I can’t understand any meaning for the word “right” in connection
to an expansive mountain terrain that has no distinguishing marks
and everything in it could be described as on the right or the left
of something else.
For direction is subjective,
it has no objective meaning but is relative.
Its meaning is defined in relation to something else.

4. Likewise when the \Quran\ presents the story
of the People of the Cave and their faithful dog,
we see it listing details to a ridiculous extent,
despite not settling on a specific number for them.
So it says — just as we human beings would when unable
to specify something precisely —
“Some say seven and some say eight”
even though God is the knower of the unseen!

5. In this respect it hasn’t escaped me to mention also these verses —
the puzzles related from Moses after he descended
from the Mount and found his people worshiping the calf.
He lost his temper and grabbed his poor brother, Harun’s neck:

\begin{quote}
So Moses returned to his people in a state of anger and sorrow.
He said: O my people! did not your Lord promise you a goodly promise:
did then the time seem long to you,
or did you wish that displeasure from your Lord should be due to you,
so that you broke (your) promise to me?

They said: We did not break (our) promise to you of our own accord,
but we were made to bear the burdens of the ornaments of the people,
then we threw them, and thus did the Samiri suggest.

So he brought forth for them a calf, a (mere) body,
which had a mooing sound, so they said:
This is your god and the god of Musa, but he forgot.

Could they not see that it did not return to them a reply,
and (that) it did not control any harm or benefit for them?

And certainly Haroun had said to them before:
O my people! You are only being tested by it.
Surely your Lord is the Merciful One,
therefore follow me and obey my order.

They said: We will by no means cease to keep to its worship
until Musa returns to us.

He (Musa) said: O Haroun! what prevented you,
when you saw them going astray,

So that you did not follow me? Did you then disobey my order?

He said: O son of my mother!
Seize me not by my beard nor by my head;
surely I was afraid lest you should say:
You have caused a division among the children of Israel
and not waited for my word.

He said: And what do you have to say, O Samiri?

He said: I saw something they did not see,
so I took a handful from the footsteps of the messenger,
and so I threw it; thus did my soul commend to me.

(\QRef{20:86–96})
\end{quote}

These verses are a collection of puzzles.
As in the case of the abbreviated letters,
the exegetes were forced to bring out all their reserves of myths
and legends and waffle according to their desires to decipher
these mysterious inscriptions and sweep away the ambiguity
which surrounds them.
Yet it is obvious that in the field of eloquence,
brevity in the wrong place impairs the meaning,
just as over\–elaboration spoils the meaning.

What is the meaning of:

\begin{quote}
“But we were made to bear the burdens of the ornaments of the people,
then we threw them.” (\QRef{20:87})
\end{quote}

Where did they throw them?
The exegetes say that they threw them in the fire.
How do they know that if it wasn’t for the stories
from the Torah which the \Quran\ says is corrupt?
What would be the harm in mentioning the word “fire”?
Why make us resort to a “corrupt” book
to understand one that is not corrupt?

But the big puzzle is the one that stands out in the last verse
where the disorder reaches its height:

\begin{quote}
“I saw something they did not see, so I took a handful
from the footsteps of the messenger, and so I threw it.”
(\QRef{20:96})
\end{quote}

What is this handful? And which prophet is it talking about?
What fertile ground is this to resurrect the Isra\´iliyat
(\NB{Narrations from Christians and Jews})
and pile myths in layers upon layers, and as a result,
the myth of those who believe in an Arabic \Quran\
“That has no crookedness in it, that they may fear God.”
(\QRef{39:28})

6. If you would like more of these puzzles in the verses of the \Quran,
then here is this verse:

\begin{quote}
And certainly We tried Sulaiman, and placed on his throne a body.
Then did he repent. (\QRef{38:34})
\end{quote}

There’s nothing like a good old fable to bestow meaning on this verse.
Oh joy! Oh joy!
At these verses that nothing can be compared to in respect
of feeding the minds of Muslims with myths, and crippling their intellect,
and diverting them from the world that is turning around them.
So that they swim along in the world of the invisible far away from
the world of the visible!!
Do you know what is this body which God placed on Sulayman’s throne?
He was a Jinn who appears to be an Arab because his name is “Sakhr”
and sat on the throne of Sulayman who had married a woman
he desired but worshipped idols.
His kingdom was contained in his famous ring.
One time he removed it when he wanted to go to the toilet
and gave it to his wife to hold.
Then that Jinn came in the form of Sulayman and took it from her
and sat on this throne.
Then Sulayman came out (of the toilet) but in an appearance
that was different from his real form which the Jinn had stolen from him,
and he saw the Jinn on his throne and said to the people
“I am Sulayman so reject him (the Jinn).”
Then he repented to God and his kingdom was returned after a few days!!

7. It is as though this huge seam of ambiguity that surrounds the \Quran,
and which puts the concept of its miraculousness in serious doubt,
is not enough and so it adds another handicap.
For amongst that which burdens the \Quran\ with ambiguity
and adds more ambiguity on top of its ambiguity,
is its frequent usage of conflicting words.
Words that contain two opposing meanings at the same time,
even in doctrinal matters and legislative verses,
even though this should be amongst those things that are totally
off\–limits in a book that is supposed to be inimitable.

Take the verb (\ar{غَبَرَ} “Ghabara”) for example.
It has two conflicting meanings: “to go” and “to stay.”
Yet this word appears seven times in seven verses
that talk about the wife of Lot:

\begin{quote}
And when Our messengers came to Ibrahim with the good news, they said:
Surely we are going to destroy the people of this town,
for its people are unjust.
He said: Surely in it is Lut.
They said: We know well who is in it; we shall certainly deliver him
and his followers, except his wife; she shall be of the Ghaabireen.
(\QRef{29:31–32})
\end{quote}

“Thus the angels of punishment did indeed take Lot
and his family out from the village and made his wife to remain,
for she was of the Ghaabireen, meaning, ‘those who remained’
in the village to earn her portion of the punishment.”

8. Perhaps using this word that reflects two conflicting meanings
is unimportant here because it doesn’t concern a matter of faith.
But the situation is quite different regarding another word
that also has two completely contradictory meanings
and in this case it pertains to a fundamental matter of faith.
I am referring to (\ar{ظَنّ} “Thanna”.)
This verb can give a meaning of doubt
and also give a meaning of certainty.
Despite this, the \Quran\ has no problem using it:

\begin{quote}
“And seek help through patience \& prayer,
and indeed it is a hard task except upon the humble,
who know (\ar{يظنون}) they will certainly meet their Lord
and to him they shall return.” (\QRef{2:45–46})
\end{quote}

Is it right to use the word (Thanna) in this case?
Perhaps the meaning here is that it is not necessary
for one to have complete conviction in the Day of Judgment?
Perhaps Allah is content for the servant in this case
to have doubt and weak faith?
So what is to stop the meaning of this verse being like that,
for the text doesn’t exclude that.

9. There is another word in the \Quran\ that has two
contradictory meanings and it is related to a fundamental legal ruling
in religion and I’m referring to the word (\ar{قرء} “Qur\`”)
which is amongst those conflicting (words),
since its meaning is both the menstruating of a woman
as well as becoming free from menstruating,
i.e.\ coming out of menstruation — at the same time!
So since that is the case then how are we to interpret His saying —
Most High is He:

\begin{quote}
“Divorced women shall wait, keeping themselves apart,
for three menstruations, (or three periods of being free from menstruation.)”
(\QRef{2:228})
\end{quote}

So which of the conflicting meanings is the intended one here?
The matter has two possibilities!

10. In this respect also is the word (\ar{إحصان} “Ihsaan”)
and its derivatives.
For it means chastity, as in not being married:

\begin{quote}
“And Maryam the daughter of \´Imran who kept chase her privates”
(\QRef{66:12})
\end{quote}

and it means marriage:

\begin{quote}
“...And when they are taken in marriage...” (\QRef{4:25})
\end{quote}

As it also means emancipation and freedom:

\begin{quote}
“...then if they are guilty of indecency,
their punishment is half that for free women...” (\QRef{4:25})
\end{quote}

Indeed this word has been used here (in 4:25)
in two different meanings in a single verse.
Who knows, maybe this is the height of miraculousness!

Tell me by your Lord: Who is responsible for this ambiguity?
What can the exegetes do in the face of these verses — the Puzzles?
I wonder was it in their ability to do anything other than what they did?
And who forced them to resort to that? Had the \Quran\ been clear,
would ambiguity have been glorified in this way by the books of tafseer?
Or are baffling puzzles amongst the aspects of eloquence
and signs of miraculousness?

If the \Quran\ was truly clear, and if the people related
what they understand and not what they don’t understand,
and if it was more sober and rational,
it would have bestowed upon the exegetes a sturdy and sound mentality
with which to deal with the \Quran\ much more earnestly
and Muslims wouldn’t have drowned in mythological fantasies
that never leaves them for a day.
On the contrary they have grown and become evermore grandiose
the more we become distant from the moment of the first inspiration.
Until we have arrived at where we have arrived of ignorance and backwardness
with no hope of escaping — in the near future at least!


\section{Obscurities of the \Quran}

In (the Science of) the Miraculousness of the \Quran,
the branch (known as) “Obscurities”
has contributed a great deal to the ambiguity of the \Quran.
It is more a source of incapacitation (\ar{تعجيز})
than it is a source of miraculousness (\ar{إعجاز}).
This branch (of science of the \Quran) is called
“The Obscurities of the \Quran.”\fnmarksym[*]

\fntextsym{“Gharib al-\Quran” was translated as “Obscurities of the \Quran.”
However it could also be translated as: “Strange”, “Rare”, “Foreign” etc...
“Gharib al-\Quran” is an established branch of \Quran{}ic Sciences
(“Uloom al-\Quran”) and about which much
was written by the classical Islamic scholars.
It came about due to the number of odd — and often —
incomprehensible words that Muslims found themselves faced with
when reading the \Quran.
These words were not part of ordinary Arabic language.
Some were rare or unusual Arabic words or spellings.
Some were from closely linked languages like Hebrew
and some from more distant foreign languages.}

The term “Obscurities of the \Quran” refers to words,
phrases and constructions in the \Quran\ that are obscure
and appear in a manner that was not used in the Arabic language before.
They are (words) that are not used in the regular meaning
that the original forms of the word convey.
As al\–Rafi\` says they are: “Perplexing to interpret
because they are not understood in the same way by those
who use them as well as the majority of people.
Altogether their number in the \Quran\
amounts to 700 words or slightly more.”\fnmark\@
\fntext{Mustafa Sadiq al-Rafi\`i,
“The Miraculousness of the \Quran”, p.\ 34.}
Just as al\–Suyuti says when emphasising the obscureness of these words,
that the Arabs who “Were the masters of pure language,
(both) those who were present at the revelation of the \Quran\
or who it reached, were at a loss with the words
they didn’t know the meaning of.”\fnmark

\fntext{Jalal al-Din al-Suyuti
“Perfection in the Sciences of the \Quran”, 1/119.}

The obscurities in the \Quran\ usually occur as strange words.
Some of these words are not from the language of the Quraysh,
and some are not even from the language of the Arabs at all.
They also occur in other things which al-Suyuti mentions,
though it’s not the place to cover them here.
They appear in the way pronouns, declensions, correlations,
and constructions are used in a manner unheard of in the speech of the Arabs.

Though the obscure words in the \Quran\ number into the hundreds,
I will suffice here with mentioning only a few examples.

Abu \`Ubayda related from Ibrahim al-Taymi that Abu Bakr al\–Siddique
was asked about (the meaning of) His saying — Exalted is He:

\begin{quote}
(\QRef{80:31}) \ar{ وَفَاكِهَةً وَأَبًّا}\\
“And fruits and fodder/straw/grasses/weeds/herbage”
\end{quote}

He replied: “Which sky will shield me and which ground will bear me,
if I was to say something about the book of God of which I know not?”\fnmark

\fntext{Ibidem 1/119.}

Al-Gharyabi related from Ibn Abbas who said:
“I understand the whole of the \Quran\ except four (things/words):\fnmark

\fntext{Ibidem 1/119. \NB{FixMe: Place of footnote mark 53?}}

\begin{quote}
(\QRef{69:36}) \ar{غِسْلِينٍ}\\
“washing of wounds/filth/refuse”

(\QRef{19:13}) \ar{وَحَنَانًا}\\
“And piety/compassion/tenderness”

(\QRef{9:114}) \ar{ لَأَوَّاهٌ}\\
“Tender/Soft-hearted/Pious/Given to prayer”

(\QRef{18:9}) \ar{وَالرَّقِيمِ}\\
“Valley called Raqeem/Tablet/Letter/Inscription.”
\end{quote}

Amongst the other obscure words also are:

\begin{quote}
\begin{Arabic}
\begin{multicols}{3}\raggedleft
قلوبنا غُلف،\\
ما ننسخ،\\
مثابة،\\
جِنَفًا،\\
بهتانًا،\\
غير متجانفٍ،\\
مدرارًا،\\
يضاهئون،\\
صنوان،\\
جُذاذًا،\\
كَطيِّ السجلِّ،\\
ثاني عِطْفه،\\
هيهات هيهات،\\
الأجداث،\\
زخرفًا،\\
برزخ،\\
رواكد،\\
يوبقهن،\\
ذي المعارج،\\
سبلًا،\\
جَدُّ ربنا،\\
فلا يخاف بخْسًا،\\
ولا رهقًا،\\
كثيبًا مهيلًا،\\
وبيلًا،\\
شواظ،\\
يطمثهن،\\
نضّاختان،\\
رفرفٍ خضر،\\
مترفين،\\
فَرَوْح ورَيحان،\\
نبرأها،\\
لا تجعلنا فتنة للذين كفروا،\\
انفقوا،\\
ومن يتّق الله يجعل له مخرجًا،\\
عتت،\\
فسحقًا،\\
لو تُدهن فيدهنون،\\
زنيم،\\
يوم يُكشف عن ساق،\\
مكظوم،\\
مذموم،\\
ليزلقونك،\\
طغى الماء،\\
يوم عسير،\\
أمشاج،\\
مستطيرًا،\\
قَمْطريرًا،\\
رواسي،\\
ألفافًا،\\
جزاء وفاقًا،\\
فُراتًا،\\
المعصرات،\\
كواعب،\\
الرادفة،\\
سَفَرة،\\
قَضْبًا،\\
عسعس،\\
عِلِّيِّين،\\
ضريع،\\
حسير،\\
يتمطّى،\\
أترابًا،\\
مرساها،\\
ممنون،\\
أرائك،\\
معاذيره\fnmark
\end{multicols}
\end{Arabic}
\end{quote}

\fntext{Ibidem 1/119–142.}

All of these are Arabic words that appear in the \Quran,
the dialect of the Quraysh mixed with the dialects of other Arab tribes,
but there are also foreign non\–Arabic words that exceed in number 100
that appear in the \Quran, for example:

\begin{quote}
\begin{Arabic}
\begin{multicols}{3}\raggedleft
سندس،\\
إستبرق،\\
أباريق،\\
أبْ،\\
الأرائك،\\
الأسباط،\\
أكواب،\\
الأوّاه،\\
ربّانيّون،\\
الرَّقيم،\\
زنجبيل،\\
سجِّيل،\\
سرادق،\\
غسَّاق،\\
القسطاس،\\
مشكاة،\\
صراط
\end{multicols}
\end{Arabic}
\end{quote}

\pardivider

Now are these obscure words — whether they are Arabic or Foreign —
signs of the miraculousness of the \Quran?
How is it right for the \Quran\ to make the challenge of bringing
the like of it when it contains jargon that is unknown?
Is this inimitability or just inability?

Where is the clarity in this? In the very terminology of the \Quran,
where is the ‘making things clear’ in this?

\begin{quote}
“Alif Lam Ra\` these are the verses of the book that makes thing clear.”
(\QRef{12:1})
\end{quote}

How can the \Quran\ be described as making things clear when it is not clear?
Or is lack of clarity, making things clear
whether we like it or not in the fashion of
“Believe Allah and disbelieve the stomach of your brother.”?\fnmarksym[*]

\fntextsym{A Hadith previously referred to.}

The strange thing is that instead of being assailed by doubts
about these anomalies, the first Muslims carried the beacon to every corner
they reached and strove gallantly to defend them.
Here the patching\–up and the “Waffling” reached its furthest extent
without them being aware of it,
while naturally they thought they were doing good.
But with some of them the matter didn’t stop at defence and they started
to extol the virtues of every obscure verse,
in fact they made this obscureness amongst the signs of miraculousness!

One of the most astonishing things said about this miraculousness is what
Ibn Jarir related through an authentic chain from Abu Maysara
the renowned Tabi\´i (follower), who said:
“The \Quran\ contains every language.”\fnmark

\fntext{Ibidem 1/142.}

A similar thing is related from Sa\`id ibn Jubayr and Wahb ibn Munabbih:
“This is an indication of the wisdom behind the appearance of these words
in the \Quran\ because it encompasses the knowledge of the first
and the last and relates everything and so it must have in it a reference
to all the various dialects and languages so as to complete the fact
that it encompasses all things. So the most sweet of them,
the most sublime of them and most well-used by the Arabs,
were selected for it.”\fnmark

\fntext{Ibidem.}

Al-Suyuti adds that Ibn al-Naqib expressed that (view) and said:
“Amongst the special qualities of the \Quran\ over
and above all the other revealed books of Allah Most High
which were only revealed in the language of the people who it was revealed to,
and contained nothing of the language of others,
is that the \Quran\ encompasses all the dialects of the Arabs
and much was also revealed in the languages of others
such as the Romans and Persians and Abyssinians.”\fnmark

\fntext{Ibidem 1/143–142.}

Al-Suyuti reiterates this by saying that:

\begin{quote}
“The prophet (pbuh) was sent to every nation and He Most High said:

‘We have not sent a prophet except in the language of his people.’
(\QRef{14:4})

So the book sent (to all nations) had to contain
the languages of every nation.”\fnmark
\end{quote}

\fntext{Ibidem 1/143.}

Can you see this clowning about, this logic which, by my life,
is more strange than the spurious obscure words of the \Quran?
Can you see this unjust disarming of the people of the clear Arabic language
by the spurious speech that they do not know, from every language,
and if they know its meaning they cannot savour it properly
for it is not from the foundations of their clear language?


\section{Weakness of the \Quran}

The final nail in the coffin regarding the inadequacy of the \Quran\
is its (linguistic) weakness.
Yes, weakness! Indeed we find it very hard to accept this
and you will accuse me of bias against the book of Allah.
For the \Quran\ is the epitome of eloquence, pure speech and elucidation
to the extent that millions upon millions believe
that its language is of super\–human nature.
So how can it be weak when the enemies of the \Quran\ did not notice
that when they were all hankering after flaws?
This is unreasonable, this is unreasonable!

However, these enemies either died in the battles that broke out
between the Muslims and the Polytheists and so their objections were lost
or destroyed and have not reached us.
Or they entered Islam amongst those who entered and were absorbed
into the general pious atmosphere with its enormous defensive imperatives
and apologetic mechanisms and they resorted to quoting evidence from
Jahiliyyah (Pre\–Islamic) poetry to attest to the soundness of the weak text.
They even celebrated that it contains eloquent nuances and great wisdoms
that our minds cannot comprehend.

By itself faith is able to do wonders, but what if it is aided
by a mind well versed in study and speculation.
On top of that, (what if) this weak (speech) was repeated continuously
to the extent that daily usage polished it and constant repetition sanctified
it with sacredness and smoothed out its unevenness
and covered its flaws with embellishment.
After this, it then reaches the status of the inherited legacy,
the intimately familiar, and the cultural conventions.
Thus, despite me or you or despite even the greatest scholars of language
and masters of eloquence and experts in their field,
it gains access to the very heart of the Arabic language
and its innermost sanctity, with no-one having any say or choice about it.
It then becomes part of linguistic good-taste and is produced as an example
and yardstick for comparison. So take heed oh those of insight!!

1. He Most High said in his detailing His bounty towards man
and the ungratefulness of man to this bounty:

\begin{quote}
“He it is Who makes you travel by land and sea;
until when you are in the ships, and they sail on with them
in a pleasant breeze, and they rejoice, a violent wind overtakes them
and the waves come on them from everywhere,
and they become certain (or think) that they will be overwhelmed,
they call to Allah, making their faith pure for Him.
If you deliver us from this, we will most certainly be of the grateful ones.
But when He delivers them, lo! behold! they rebel in the earth wrongfully.”
(\QRef{10:22–23})
\end{quote}

The aspect of weakness — nay feebleness — in the previous verses
is the poor use of pronouns which if had come from you
or I would be attributed to our ignorance,
and they would accuse us of lack of linguistic knowledge
and advise us to study the science of grammar afresh.
But if it comes from the \Quran\ then it is eloquence —
nay, they dedicate to it, its very own branch, from the branches of eloquence.

The branch that concerns us here is the branch of Iltifaat
(Sudden Transition\/Change)!!
Here is the previous verse again so you can see where the error is,
that is if you haven’t already noticed it by yourself,
because it is a screaming error that it is not possible
for anyone to hear without it jarring in his ears:

“He it is Who makes you travel by land and sea;
until when you are in the ships, and they sail with them” —
Instead of, “and they sail with you”,
and “you rejoice” instead of “they rejoice”.

Believe it or not this jarring is part of the eloquence of the \Quran.
If it wasn’t for these two lame (words) the eloquence of the \Quran\
would not be be apparent.
It’s not jarring, except in our twisted minds.
It is “Iltifaat” (Sudden Transition/Change) and “Iltifaat”
is a branch of the branches of eloquence that was invented
so as to provide an escape route for this verse and its like.

2. There is another branch called “The Style of the All-Wise.”
The prophet was asked about the phases of the moon,
i.e.\ the changing appearance of the moon from one day to another.
Instead of explaining it to them in a way they could understand —
and if he did that, then that would have been a real miracle —
he evaded the answer they hoped to hear from the One who created
the phases of the moon and instead received from him a disappointing reply
that both the old and young already know:

\begin{quote}
“They ask you concerning the phases of the moon, say:
They are but signs to mark fixed periods of time in (the affairs of) men,
and for Pilgrimage.”
(\QRef{2:189})\fnmark
\end{quote}

\fntext{Note this verse does not fall under the chapter of weak speech
but is cited here to highlight the sophistry, waffling and patching\–up.}

What an extraordinary astonishing answer!
God created the phases of the moon so that people can count time for
(their affairs such as) agriculture, trade, menstrual periods of their women,
when to fast and break fast and for their pilgrimage to the sacred house —
as the exegetes say!
Fine. If that is the case,
then I wonder how we can we explain the changing appearance of the moon —
nay the moons — around Mars, Jupiter, Saturn and the other planets?
Are there humans like us on these planets who perform Hajj to the Holy Ka\`ba
and have concerns and affairs as we do, such as our women
who have menstruation cycles and (count the time when)
they are free from menstruation, ready for prayer and fasting?

The truth is that the worst method of counting time is to count it
by Lunar phases which we have suffered from and which has caused a division
which there is no hope of repairing.
Not to mention that this reply is a glaring affirmation
of a geocentric universe, and a single sun and a single moon,
and single form of worship and religious rites.
Thus the \Quran\ diverted them from what they were actually asking about,
towards that which they were not asking about and (diverted them)
from gaining knowledge about things they didn’t know
to things they already knew.

The knowledgeable in eloquence were truly shocked by this reply
and yet they couldn’t be shocked.
How could they be shocked when it has come from the presence
of an All-Wise and All-Knowing?
So they retreated (like sheep) to the pen and exchanged the duty to be critical
so that truth and correct understanding may prevail in exchange for idiocy
and stupidity. Indeed Allah avoided the answer, on the pretext of disciplining,
guiding and teaching them the right way to ask a question,
never mind that this reply displays contempt and scorn towards the questioner.
In my view it is a reply that has no meaning other than an attempt
to stifle questioning.
It’s as though asking questions is an unforgivable sin.
It is a blatant disregard for mankind’s longing to understand
the reasons behind what we see.
Allah is the All-Wise who knows the needs of his servants
and he makes clear to us the style by which we must address Him.
This is the “Style of the All-Wise” and it is also a branch
from amongst the branches of Eloquence.

This poor (art of) eloquence!
How many have spoken falsely in its name!!
How many have told lies and fabrications about it!!

It seems al\–Mutanabbi was well aware of this game
(played by the scholars of language)
for his poetry was criticised by some of the grammarians
when he made a grammatical mistake.
Al-Mutanabbi fumed with rage and replied to the grammarian,
saying with audacious self\–confidence:
“It is up to me to say (what I like) and up to you to cite
(evidence of it being correct)”, with the implication:
“Isn’t that what you do with the \Quran?
Rules are for the little people as for the masters,
they can get away with what the little people can’t.
Get lost! Go back to your tribe and people of little ones!”

In my opinion the most important reason for the flourishing of the science
of eloquence in Islam is the defence of the \Quran\ in any way possible
and to provide solutions to the flaws in it —
not for the pure sake of knowledge or truth or elucidation.
For they came across so much in it (the \Quran) that bewildered them
and troubled their minds.
If it was in any other book it would have caused them to doubt it,
and it would have been publicly defamed to an extreme extent.
But what can they do, for it has been revealed by an Almighty, All\–Knowing One.
“An Arabic \Quran\ without crookedness” (\QRef{39:28}.)
This is the incontestable of incontestables that no Muslim can forsake.

Every Muslim of sincere faith will doubt himself rather than doubt his \Quran,
no matter what thoughts come to his mind about the \Quran,
it is not possible to defame the \Quran\ or even pause to question it.
So here comes the science of eloquence, and elucidation and marvels...
to repair what has ruptured, mend what has broken, fill in the holes,
and fix what is broken, come apart and disordered.
So that there is no rupture nor gaps, nor cracks, nor holes in the \Quran.
It is only the shortcomings of our human minds.
The science of Eloquence and Elucidation can vouch
for complete verification in this matter.

With nonsense and sophistry and waffling you can discover what you want
and hide what you want. You can do what you want and explain what you want,
and convey what you want and smooth out every crookedness you want.

I always say: Give me a madman and I can bring out the wisdoms of the first
and last, from his speech.
But it appears that the exegetes (Mufassirun) who have been raised
on more than one school of the schools of pure speech and eloquence
and bore the responsibilities of the embellishment of eloquence
and marvels and meanings...\ beat me to it by a long way!

3. “Whoso disbelieves in Allah after his belief save him who is forced thereto
and whose heart is content with the Faith but whoso opens (his) breast
to disbelief on them is wrath from Allah and for them a mighty punishment.”
(\QRef{16:106})

I implore you by all you hold dear: Did you understand anything of that?
I said to myself perhaps this verse has a mistake in transcription,
or perhaps there is a word missing or corrupted.
So I went to examine many different copies of manuscripts that were written
at different periods to see if I could find some difference between them.
But it was in vain.
For there is complete accord between all the manuscripts
in all the times and places.
Is this really the speech of the Lord of the worlds who challenged man and Jinn
to bring the like of it?
May God help the exegetes who have to chisel through rock
with their fingernails to get a tiny bit of water!

All Muslims from the East to the West recite this verse every day morning and
evening in their prayers and worship and they hear it in recitations
of the Noble \Quran, without any of them sensing any weakness
in it or confusion or jarring.

Indeed the blades of arrows are breaking
the blades of arrows\fnmarksym[*]\ (i.e.\ he has become ‘punch-drunk’),
so that the believer no-longer cares on which side falls his mortal blow.
His linguistic sensitivity has become dulled,
his taste slipshod, his instinct feeble.
Indeed his awareness of the grating disharmony
has died where verses of the \Quran\ are concerned,
but remain intact and healthy regarding everything else.
Everything in him is still in its original natural disposition.
Nay, it has become more refined and accomplished.
He has acquired skills, expertise and gifts in all things except here in this.
For when faith reigns supreme reason diminishes
and faith can do what reason can’t!!

\fntextsym{From a line of poetry by al-Mutanabbi —
it means he has been struck by so many arrows (of misfortune and sorrow)
that the ones that strike him now have no place to pierce
and are just hitting the ones already there.
In other words he can no-longer defend himself, nor cares.}

I confess in all honesty that I wasn’t fully attentive
to this verse and many like it until now.
If it hadn’t been for the fact that I — in a pragmatic way —
studied the \Quran\ critically and analytically, examining it verse by verse.
If I hadn’t divided them into categories and indices for this purpose,
the veil would have remained over my eyes.
So what do you say about the faithful who pay no attention to this!?
Don’t you see that large number of Muslim thinkers and university professors
who are no less ardent in their conviction in the fairytale
of the Miraculous Nature of the \Quran\ than any ordinary person?
They are not in a position to dissect the \Quran\
and tear open the shrouds that envelope it.
No, they are not capable of doing that.

Reading (the \Quran) is of two types:
Firstly there is devotional reading that is blind to flaws
that the eye almost pops out at,
due to their contradicting what is reasonable and acceptable.
If there is any deep thinking in this type of reading,
then it is the deep thinking of defense and justification
that sees in the verse the wisdom of the ages.
Secondly there is investigative, critical and analytical reading
that exposes flaws and puts our hands on that which the faithful
do not want to see or acknowledge and for that reason they try
to dodge and maneuver to conceal its defects with all kinds of excuses,
pretexts and justifications! Perhaps this book can bring about within them —
or within some of them at least — a painful jolt.
For there is a new form of therapy, which is therapy through shocks!

4. Here is another verse for you that is similar to the previous verse
in weakness and feebleness even though understanding it is not difficult.
So let your eyes peruse it in the hope that you may be more skilled
linguistically and more eloquently capable than I.
But distance yourself from the blessed exegetes
who do not find any flaw nor fault in it.
There’s no harm in consulting the books of Tafseer to a degree.
In fact you should consult them, so long as it’s with extreme caution:

\begin{quote}
“It is He Who sends down water from the skies,
then We bring forth with it buds of every kind,
then We bring forth from it green (foliage)
from which We bring forth grain piled up.”
(\QRef{6:99})
\end{quote}

Would that I knew if you notice anything odd when you hear this verse?
This verse contains two divine secrets —
or two examples of exemplary eloquence, if you will.
The eloquence of “Iltifaat” (sudden transition of pronoun)
“It is He Who sends down water from the skies, then We bring forth...”,
that is the first, while the second is repetition
of the verb “bring forth” three times.
A repetition that tears at the ear and makes it feel uneasy and uncomfortable —
unless uneasiness and discomfort are amongst the signs of Miraculousness!
Had Ibn Muqaffa\` or al-Jahiz or others from amongst the princes of eloquence
fallen to such depths of feebleness they would have ripped them apart
and heaped upon them criticism and vitriol.
But what can one do when polishing,
repetition and devotional recitation has bequeathed
to the faithful dullness of taste and inability to sense the jarring.

5. “And when your Lord called out to Musa:
Go to the unjust people – the people of Pharaoh.
Will they not guard (against evil)?” (\QRef{26:10–11})

And in his dialogue with Pharaoh, he asked him this:

\begin{quote}
“(Pharaoh) said: Did we not bring you up as a child among us,
and you tarried among us for (many) years of your life?
And you did your deed which you did...
He (Moses) said: I did it...
So I fled from you when I feared you,
then my Lord granted me wisdom and made me of the messengers.
And that is a favour of which you bestowed upon me\fnmarksym[*]\
that you have enslaved the children of Israel.” (\QRef{26:18–22})
\end{quote}

% Quick fix: a number appears in fntext, instead of the '*';
% fnmark is in a quote environment, that's why this command is needed.
\renewcommand{\thefootnote}{*}

\fntextsym{The verb “Manna” (\ar{مَنَّ}) means;
“To bestow favours, blessings, be kind, gracious etc...”
The verse says: “And that is a favour which you bestowed upon me”
(\ar{وتلك نعمةٌ تمنُّها عليَّ}).
Obviously this makes no sense and so some translators have translated it as
“And that is a favour which you <reproach> me.”
Apart from being a strange translation — it too makes no sense.
How can one reproach someone for something they didn’t do?
Not to mention that the person doing the reproaching
is the one who did the act himself!!??}

The puzzle here is in the last verse, the verses that precede it lead to it.
Read it and then read it again, two, three,
four and ten times and keep on reading it as long as you want.
Then tell me honestly and sincerely if you understand anything —
and I will be most grateful.

I don’t understand how enslavement can be a blessing
bestowed by Pharaoh upon Moses?
If I want to give this verse meaning
then it should \cor{}{be} something like the following:
“And that is a favour of which God has bestowed upon me,” meaning
“That He made me one of the messengers is a blessing that He bestowed upon me.”

As for the last bit of the verse:
“...that you have enslaved the children of Israel.”
It is a distortion that has no meaning to it.
Or it may be the remainder of an abrogated verse or something similar.
Yet the scribes, and reciters and readers have accepted it
in the manner it appears in the \Quran, just as the deaf,
dumb and blind would accept what is presented to them
without objection or opposition.
Indeed they say, “All of it is from our Lord” and the exegetes follow
in their footsteps and do not dare to make any change in it.
They become masters at inventing all sorts of meanings for it.
Not one of them says;
don’t sweat over it for the verse as it is phrased has no meaning!!

6. Likewise read the following bewildering verse,
and read it again and again and tell me if you understand anything

\begin{quote}
“Say: He who is in the heavens and the earth does not know the unseen,
except Allah, and they don’t know when they shall be raised.
Nay, but does their knowledge reach to the Hereafter?
Nay, they are in doubt about it. Nay, they are blind to it.”
(\QRef{27:65–66})
\end{quote}

Is there even an atom’s weight of eloquence in that verse?
Can speech get any more confused, convoluted, weak, and garbled than this?
By my life it is Miraculousness due to lack of Miraculousness!!

I don’t deny that this verse and its like amongst the baffling verses,
must have a meaning, but its meaning remains hidden in the mind of its author.
For the words as they stand are not able to uncover it,
due to its weakness, confusion, and convolutedness.
The consequence is its inability to clearly express what was intended.
This is what has left the door open for the waffle,
nonsense and fabrications of the exegetes.

This is not eloquence nor is it miraculousness.
What we have in front of us is shameful weakness.
Where is the wondrous fluency that we find with al-Jahiz?
Where is the eloquent, fluid command of the Arabic language
that is present in a non-Arab writer like Ibn Muqaffa\`
who never once claimed that it was revealed
from the presence of an All-Wise, All-Knowing?
To the extent that the meaning remains obscure, can we measure deficiency,
while by the extent of its lucidity can we measure its eloquence.

7. “And when Musa said to his servant:
I will not cease until I reach the point where the two seas meet
or I will go on for ages. So when they reached the point where the two met,
they forgot their fish, and it took its way through the sea freely.
So when they had gone farther, he said to his servant:
Bring to us our morning meal,
certainly we have met with fatigue from this our journey.
He said: Did you see when we took refuge on the rock,
and no-one made me forget it except Shaytan, to mention it.
And it took its way into the sea wonderously.” (\QRef{18:60–63})

They say that the speech of Allah does not contain anything superfluous.
That the words are in exact proportion to the meanings —
without surplus nor deficiency!
OK, yet this verse contains a surplus that creates a glaring imperfection.
In my opinion this is not just surplus but gross redundancy,
as occurs in many verses of the \Quran. The sentence;
“no-one made me forget it except Shaytan,”
is quite sufficient to lead one to the sought after meaning.
So what is the ‘profound’ wisdom in adding “to mention it”?
And if the \Quran\ is so eager to include;
“to mention it” what is the point of the pronoun in; “made me forget it”?
It should have said; “and no-one made me forget it except Shaytan,” or;
“and no-one made me forget to mention it, except Shaytan.”
However combining the two together creates a discord
that the tongue has to polish out, for sensitivity to it has died.

8. “And He has made subservient to you what is in the heavens
and what is in the earth altogether, from him.
Indeed in that are signs for a people who reflect.”
(\QRef{45:13})

I can’t see that the words;
“from him” have any meaning or serve any purpose.
It is redundancy on top of redundancy
and there is nothing more for the masters of eloquence to do other than
to make “Redundancy” a chapter amongst the chapters of eloquence.
Perhaps it is the tail end of another verse that has been abrogated
but the scribe left it there by mistake and so it has crept into the text
without it occurring to anyone to question it.
It may be due to a wisdom that only Allah knows!
And here enters the well known sophistry and chicanery to bring it out
from the wilderness of meaninglessness and bestowing upon it —
falsly and erroneously — full meaning.
Rescuing it from its affliction even though
this meaning is completely meaningless.

So they say: “He has made subservient to you... altogether, from him” means;
“He has made it subservient... in a state of being, from himself Most High!”
So then it’s denotive of state.
Yet no-one asks himself;
what is the reason for this circumstantial clause?
Is there a deception more devious than this sophistry (claiming it means):
“in a state of being, from himself,” oh professors of sophism,
instead of simply removing and deleting it from the text?
But who dares to do that?

9. “And those who disbelieve are driven to Hell in troops,
till when they reach it, its gates open, and its warders say to them:
Did there not come to you messengers from amongst you,
reciting to you the revelations of your Lord and warning you of the meeting
of this your Day? they say: Yea, verily.
But the word of doom of disbelievers is fulfilled.
It is said: Enter ye the gates of Hell, to dwell forever in it.
How terrible is the journey’s end of the arrogant.”

\begin{quote}
“And those who keep their duty to their Lord are driven
unto the Garden in troops, till when they reach it,
and its gates open, and its warders say to them:
Peace be unto you! You have done well, so enter it forever.
They say: Praise be to Allah, Who hath fulfilled to us His promise
and has made us inherit the land, so we may dwell in the Garden where we will!
How excellent is the reward of the workers.”

“And you will see the angels thronging round the Throne,
hymning the praises of their Lord.
And judgment shall be given between them in truth and it will be said:
Praise be to Allah, the Lord of the Worlds!”
(\QRef{39:71–75})
\end{quote}

These verses should be — in my opinion — amongst the splendid verses
if it wasn’t for two flaws that spoil their beauty,
like a young woman of dazzling beauty with hair growth on her lip and chin.
But again the constant repetition of these verses on the tongues
has hidden these flaws as make-up hides the blemishes of a beautiful woman.

There is an imbalance between the verses that describe the unbelievers
entering Hell and the God\–fearing’s entrance (to paradise.)
For as the unbelievers are driven to Hell and arrive at it,
the gates open for them. Arriving leads to the gates opening.
In other words there is a prepositive clause — arriving —
which is immediately followed by the consequential clause.
But that doesn’t happen in the parallel verse of those who are God\–fearing.
For the verses that describe their arrival are — at least on the surface —
a collection of prepositive clauses without a following consequence clause,
even though the consequence is understood by deduction.
The consequential clause in the first verses is understood both in wording
and deduction but as for the remaining verses the consequential clause
is only understood by deduction.

In other words the verse of the ‘God\–fearing’ contains
an extra Conjunctive “And” (\ar{واو العطف}) that spoils the whole scene
to the extent that one might think this verse
has no conditional response to it.
In the first verse the conditional response comes instantly:
“till when they reach it, its gates open,”
while there is no conditional response in the second verse
due to the insertion of the Conjunctive Particle:
“till when they reach it, and its gates open.”
So how did this cumbersome “And” sneak in here?
They say it is redundant, but it is a redundancy placed upon the people
of Paradise who are eager to know their destiny!
If you or I did such a thing it would be considered to be our inadequacy.
But if the \Quran\ does it, then it is Miraculousness.
Poor me and you!!

The second flaw in these verses is the verb “driven” which is used
for herding cattle and cannot be applied to humans.\fnmarksym[*]\@
But just as donkeys, mules, cattle and such like are driven,
so are humans in the \Quran.
But would that the matter just stopped there,
however this incorrect term is applied equally to “those who disbelieve”
and “the God\–fearing.”
It is an equal treatment that is of extreme injustice
and contains utter contempt for the “God\–fearing.”
Isn’t the reward for Goodness, Goodness? (\NB{ref to verse of \Quran})
Or is there in this affair a wisdom that is hidden from reason and the mind?
It is as though the waffling exegetes sensed the ugliness
of this equal treatment and what it contains of faultiness
and injustice to the right of the God\–fearing and so they patched up
the second verse by adding the interpretation “gently”.
So they said (it means):
“Those who disbelieve are driven ‘violently’ to Hell, in troops,” while
“Those who keep their duty to their Lord
are driven ‘gently’ to Paradise, in troops,”
forgetting that ‘herd driving’ is ‘herd driving’,
no matter whether it is done with violence or gently!

\fntextsym{Had the verb “To Drive/Herd” not been used for both the believers
and unbelievers, one might have been able to argue Allah used poetic license
to highlight the rough handling of the unbelievers.
However, the use of the same verb for the believers also,
completely rules this out.}

10. “Say: Do you indeed disbelieve in Him Who created the earth in two days,
and set up equals with Him? That is the Lord of the Worlds.”

\begin{quote}
“And He made in it mountains above it, and blessed it and measured therein
its foods in four days, alike for those who ask.”

“Then turned He to the heaven when it was smoke,
and said to it and to the earth: Come both of you, willingly or unwillingly.
They said: We come, willingly.”

“Then He ordained them seven heavens in two days,
and inspired in each heaven its mandate;
and We adorned the lower heaven with lamps to guard.
That is the measuring of the Mighty, the Knowing.”

(\QRef{41:9–12})
\end{quote}

These verses like its predecessors combine ambiguity with weakness.
More precisely its ambiguity stems from its weakness
and contradicting other verses in the \Quran.
While the reverse could also be true
(its weakness and contradictions stem from its ambiguity.)
For the lack of a clear image in the mind of the author creates confusion
and distortion when trying to fathom its meaning,
as it stumbles haphazardly this way and that.
The meanings are scattered far from the words.
The numbers (of days) conflict with the calculations (in other verses).
Indeed the text has lost its way,
to the extent that one could expect anything next.
One cannot see anything other than jumps breaking up the flow
and halting its course towards its goals.

The like of this happens in this verse we are discussing
and in other similar previous verses
that suffer from fragmentation and dislocation.

Everything the \Quran\ relates in relation to the number of days
in which Allah created the world, specifies the number as six days,
apart from this verse.
Just as all the verses connected to the number of days of creation
in the \Quran\ deal directly with the topic without superfluous
or detracting material, except here.
Ignoring the dislocation of this verse
and its lack of connection to what is before and after it,
as the \Quran\ has got us accustomed to doing,
it starts with a strange beginning.
“Say: Do you...” Is this a question?
Or a rhetorical expression of astonishment?
Or is it an assertion of fact?
Or what? Please enlighten me and I will be very grateful!

Likewise these verses are an incongruity that combines disparate elements.
An allusion to the polytheists who disbelieve
that God created the Earth in two days,
and who don’t suffice with that but also set up partners to him.
Then after that comes clarification that the one who created all of this
is the Lord of the worlds.
He then followed that by strengthening the Earth with the mountains
and apportioning its sustenance in four days.

Thus the Earth alone required 6 days continuous work from  Him —
Glorified is He — and it deserves such effort from Him — Most High is He —
due to its fundamental importance in the universe,
as was the understanding of people in ancient times.
Why not? Since it is the centre of the universe and it is its beating heart.
As for the remainder of creation — they are trivial things:
a sun, a moon and seven heavens decorated with a number of stars
so that people can use them when traveling by land or sea.
Exactly two days is quite enough for all that.

Believe it or not the creation of the heavens only took two days.
So long as it wasn’t made of cardboard but from thin paper
which is more than sufficient for the angels who don’t have heavy,
stomping feet like humans.
They have gentle ethereal feet that they don’t use for walking.
Instead they have delicate wings that makes walking unnecessary,
which reminds me of the verse of a French poet when describing his beloved:

\begin{quote}
“By God how gentle are her feet —
she walks upon the pasture without it knowing!”
\end{quote}

In conclusion, after completing the creation of the Earth in six days,
Allah created the Seven Heavens in two days, then sprinkled the stars
here and there in the lower heaven to make it look beautiful,
without bothering with the other heavens it would appear,
so they have to remain dark, since the heavens are the abode of the angels
and they don’t need lights because the angels have bodies made of light.
Perhaps the stars of the lower heaven are made of wax
and an indication of that is the short time it took to create the heavens.

The verse then concludes all that with;
“That is the measuring of the Mighty, the Knowing.”
So blessed be Allah the best of creators.

The exegetes (Mufassirun) were confused when it came to understanding
these verses that extended the number of days for the creation into eight days,
and how to reconcile them with all the other verses
that specified six days only.
So they said indeed the four days in which Allah completed
the creation of the Earth includes the first two days
in which Allah created the Earth.
A neat solution no doubt.
But if that’s correct then doesn’t it show clearly
and plainly the weakness of the \Quran\ which could surely use wording
that would be much clearer and eloquent,
yet fell short of that and into weakness and ambiguity —
especially since clarity is supposed to be an inseparable attribute
of the \Quran\ repeated over and over again on almost every page,
“in a clear Arabic language”?!

11.

\begin{quote}
“And We certainly sent Noah and Abraham,
and gave to their offspring Prophethood and the Book
and some of them were on right guidance, but many of them are transgressors.”

“Then in their footsteps We followed them up with Our messengers
and We caused Jesus the son of Mary to follow on, and gave him the Gospel;
and We placed in the hearts of those who followed him compassion and mercy.
But monasticism they invented it.
We didn’t ordained it for them, except so as to seek Allah’s pleasure.
But they didn’t observe it correctly.
So We gave those who believe, their reward,
but many of them are transgressors.”

(\QRef{57:26–27})
\end{quote}

It is not possible for anyone who delves into
the confusing verses of the \Quran, to pass by the last verse safely.
For one does not know whether monasticism is an invention of the Christians
or whether Allah decreed it for them and ordered them to do it?
Strangely the \Quran\ combines both contradictory positions
and confirms the two opposing views.
So how can its meaning make any sense?
How can they have both invented it and Allah decreed it for them?

Since the exegetes (Mufassirun) are able to turn things on their head
and do anything they like without anyone speaking a single word of criticism.
They doubt themselves without ever daring to doubt the verse;
“Its knowledge is with my Lord. My Lord never errs nor is He unmindful.”

To give it a modicum of logic they said in explanation:
“We didn’t ordain it for them, except so as to seek Allah’s pleasure.”
One must add an inferred sentence like this:
“We didn’t ordain it for them,
\emph{but they did it} except so as to seek Allah’s pleasure.”
They gave it meaning when it didn’t have meaning.
But would that they hadn’t because no-one is convinced by this meaning,
for “Can the perfume seller be of any use to
that which time has made go off?”\fnmarksym[*]
And since when was confusion amongst the signs of miraculousness?

\fntextsym{“Can the perfume seller be of any use to
that which time has made go off?”
(\ar{هل يُصلِح العطار ما افسد الدهر})
is an Arabic saying applied to things
we cannot change no matter how hard we try.}

12. It’s as though this confusion is not enough.
As though weakness is an important requirement of eloquence.
For that reason Divine Wisdom demanded — as a fitna to the disbelievers —
that this weakness should be followed by more weakness to increase
the confusion of the \Quran\ only a single verse after the previous ones:

\begin{quote}
“Oh you who believe! Fear Allah, and believe in His Messenger,
and He will give you two portions of His mercy,
and make for you a light with which you can walk,
and forgive you, and Allah is Forgiving, Merciful.”

“So that the People of the Book won’t know that they have no power
over anything of the bounty of Allah and that the bounty of Allah
is in the hand of Allah — he gives it to who he wills —
and Allah is the possessor of abundant grace.”

(\QRef{57:28–29})
\end{quote}

This verse contains two perplexing riddles and I don’t know
which one is greater than the other.
Placing the Mufassirun in an unenviable position.
It appears that the \Quran\ takes great pleasure in driving
these poor creatures to despair,
leaving them unable to do anything other than waffle.

The first riddle is this bewildering \ar{لئلّا} (so that not...)
which here has taken on the quality of quicksilver,
leaving you unable to find any meaning or purpose to it.
What makes this riddle even worse for the Mufassirun is
that it hardly empties its load in their minds, seizing them by the collars,
when it is followed a second puzzle even more perplexing
(The second error is the incorrect case ending
on the verb \ar{يقدرون} it should be \ar{يقدروا}.)
As though it is the earthquake followed by another,
such the \Quran\ speaks of in Surah al\–Nazi\`at,
(“The day on which the quaking shall quake, followed by another.”
\QRef{79:6–7})
“hearts that day shall palpitate”
and all of it is amongst the signs of the Last Hour
and I seek refuge with God.
May God save us from its horrors!

How wretched are these patient Mufassirun and how arduous are their burdens
and tasks that have been flung onto their shoulders!
Never did one word of complaint issue from them.
Never did they grumble or object.
They fearlessly stepped forward and dived into the depths of the sea
to gather the word of God and comprehend —
according to the limit of human capacity —
the dimensions and the objectives which it contains
and each diver came back with new pearls better than their colleagues.

Indeed the meaning of the last verse is plain on condition
that you don’t take any notice of the phrasing that burdens it
and takes away its meaning.
For the negative particle \ar{لئلّا} (so that not...) is a superfluous particle
that has no meaning to it here.
Nay, it is misleading and harms the verse enormously
and turns it into puzzles and riddles,
even though the intended meaning is very simple.

Just as having the “Nun” (\ar{نون}) (denoting the nominative case ending)
on the present tense verb \ar{يقدرون} (they are able/have power)
despite it being in the accusative
(because of \ar{ألّا} which is \ar{أنْ} plus \ar{لا}
which means the verb that follows must be in the accusative,
i.e.\ it should be \ar{يقدروا} not \ar{يقدرون}) is another misleading error.

The \Quran\ is simply saying:
“So that the People of the Book know
that they have no power at all over the bounty of God.”
But superfluous phrasing has burdened it heavily to the extent
that it has robbed it of what remains of meaning.
Who knows, maybe superfluous phrasing is amongst the signs of miraculousness.
So whenever you over-do superfluous phrasing
you are increasing the miraculousness.
But only the few can be so skilled in superfluous phrasing.

13.

\begin{quote}
Nun. I swear by the pen and what they write,

You are not, by the grace of your Lord, mad.

Verily for you is a reward unfailing.

Indeed you are of a tremendous nature.

Soon you will see, and they will see,

With which of you is the mad one.

(\QRef{68:1–6})
\end{quote}

These verses contain a simple straightforward meaning,
the narrative of which, flows gently through it beautifully,
but is disgracefully disturbed in the final verse,
for a wisdom that only Allah knows.
The \Quran\ insists — as usual in similar instances,
leaving me standing totally perplexed in front of it —
on destroying what it has built up and undermining what it has accomplished,
according to the principle of
“God only raises something to bring it down”\fnmarksym[*]{}
That is what this inauspicious preposition \ar{بَ} (“Ba” — by/with)
has done here (in the sentence) “By/With which of you is the mad one.”
(\ar{بأيّكم المفتون}) even though the deaf, dumb and blind deny
there is any redundancy in the speech of Allah.
For indeed this preposition is redundant, like it or not.
That is if the meaning of the verse is; soon you will see,
and they will see, “Which of you is the mad one.”

If the use of the preposition \ar{بَ} (‘Ba’) here, is not redundant,
then we have another problem and that is the word “Mad” (\ar{مفتون})
since it is a word that has no meaning here.
The correct word should be (the verbal noun) “Madness” (\ar{فتون})
(i.e.\ you can say: “Which of you is the mad one”
but if you use the preposition \ar{بَ} (‘by/with’)
then you have to say: Which of you is by/with madness.”)
Is the madness with you, Oh Muhammad or with them?
The fact is that the Mufassirun who put forward this view have corrected
“The speech of Allah,” even though they think they are merely explaining it.
If not then it has no meaning.

Regardless of whether we adopt this explanation or that,
i.e.\ whether we consider the preposition “by/with” redundant
or consider ‘mad’ to mean ‘madness’, indeed the verse in its original form
is defective and weak, lacking proper meaning.
That is as long as there is not some hidden purpose to it.

\fntextsym{Reference to the hadith:
\ar{كانت ناقة لرسول الله صلى الله عليه وسلم تسمى العضباء%
وكانت لا تسبق فجاء أعرابي على قعود له فسبقها فاشتد ذلك على%
المسلمين وقالوا سبقت العضباء فقال رسول الله صلى الله عليه وسلم إن%
 حقا على الله أن لا يرفع شيئا من الدنيا إلا وضعه}

“The Prophet had a she camel called al-Adba
which could not be beaten (in a race.)
Once a bedouin came on a young camel of his and beat it.
The Muslims became very distressed about that and said: “It beat al-Adba!”
So the Prophet said, “It is Allah’s right/law
that He only raises something in the world to bring it down.”}

14. Here’s another correction to the speech of God
that the waffling prattlers undertook,
while thinking that they are merely explaining it, (\QRef{70:40–41}):

\begin{quote}
“But nay! I swear by the Lord of the Easts and the Wests
that We are certainly able,”

“to substitute better than they and We are not to be beaten.”
(\ar{وما نحن بمسبوقين})

(The Mufassirun say:) It means; “and We are not unable to do that.”
(\ar{وما نحن بعاجزين})
\end{quote}

So if the \Quran\ meant that, then why did it miss the mark
and choose another phrase that means something different,
one that’s inappropriate and has no relation to it in any way?
Why didn’t it say “and We are not unable to do that”?
Isn’t that more eloquent and clear oh masters of eloquence?
The truth is that the Mufassirun had no choice other than
to use this word to rescue the verse.
What a predicament!
And how plentiful are these predicaments that the \Quran\ places them in.
That’s if there isn’t some ‘profound wisdom’ behind it,
hidden from the Ancients and the Moderns,
that the Lord of the Worlds has kept to himself.

Is this truly the word of God?
Is this what Man and Jinn were challenged to bring the like of?!
If the \Quran\ was composed entirely of masterpieces it wouldn’t be so bad,
but the masterpieces are like broken links strewn in open space
or oases dotted about a vast desert that has no beginning nor end.

Furthermore, even if the \Quran\ had been composed entirely of masterpieces
the challenge still has no meaning, since masterpieces cannot be imitated.
One can only bring literature that is better, worse or of a similar level,
but it is impossible to bring the same as it.
So what about if these masterpieces are like these that adorn the \Quran?
Indeed the work of Ibn Muqaffa\` and al\–Jahiz
and Abu Hayyan al\–Tawhidi\fnmark{}
are on a higher level of quality and excellence —
so can anyone bring the same as it?
Especially if we remember that we don’t find in the speech
of any of these the level of confusion, dislocation,
weakness and ambiguity that we find in the \Quran?

\fntext{I was going to mention “al\–Ma\`arri”
if he hadn’t been ambiguous like the \Quran,
though he remains on a unique level of excellence and flawlessness.}

\pardivider

\section{Contradiction is a Prominent Feature of the \Quran}

Would that the issues with the \Quran\ went no further
than the maladies we’ve mentioned,
but there are other maladies even more serious.
Perhaps the most important of these are the blatant contradictions.
Yes, indeed the \Quran\ is full of a variety of contradictions
that it is not possible to keep quiet about it.
Indeed contradiction is a prominent feature of the \Quran.

Below are some verses that combine ambiguity with contraction:

1. “The month of Ramadan, in which the Quran was revealed.” (\QRef{2:185})
But it is well known that the \Quran\ came down piecemeal,
spread out in installments and at different times and not as a single entity.
So what is the meaning of the \Quran\ being revealed in Ramadan, then?
There is no solution to this contradiction except through a fairy tale.
So the \Quran\ was first on the “Preserved Tablet”
(which came down as a whole in Ramadan)
and from the “Preserved Tablet” it came down piecemeal to the lower heaven.
Thus the problem is solved with a stroke of the pen.

2. But on which day in Ramadan did the \Quran\ come down?
“Indeed we revealed it on the Night of Power.” (\QRef{1:97})
As though the initial ambiguity is not enough
and so it is followed with more ambiguity,
intensifying the ambiguity and mystery.
So it specified the coming down of the \Quran\
as being in the “Night of Power” which is itself a collection of fairy tales:

\begin{quote}
“And what will tell you what is the Night of Power?
The Night of Power is better than a thousand nights.
In it come down the angels and the spirit
with the permission of their Lord with every decree.
Peace it is until the break of day.” (\QRef{97:5–6})
\end{quote}

Did you understand anything?
So the ambiguity of the \Quran\ cannot be understood by the believer
except with more ambiguity!
Can you blame the Mufassirun after that if they don’t find any way
to remove this ambiguity except through fairy tales?
For in them is the escape from every ambiguity!!
And how many stories have been related about the “Night of Power“
and how many victories has God accomplished
for his beloved servants during the “Night of Power”!!

3.

\begin{quote}
Wherever you may be, death will find you out,
even if you are in lofty towers.
If some good befalls them, they say, ‘this is from Allah,’
but if evil befalls them, they say, ‘this is from you.’
Say: All things are from Allah.
But what is the matter with these people that they hardly understand speech?

Whatever good befalls you is from Allah,
but whatever evil happens to you is from yourself
and We have sent you as a messenger to the people.
And enough is Allah for a witness.

(\QRef{4:78–79})
\end{quote}

The contradictory verses of the \Quran\ are usually spread far apart.
Dotted here and there with a great deal of distance separating them.
Except in a few cases as in the two previous verses,
where the second verse comes straight away to contradict the first,
before its echo has faded from the ear.
Hardly has the first verse finished confirming that good and evil
are both from Allah when the second verse immediately follows it
to confirm the opposite, that only good comes from Allah,
while evil comes from man!!

4. The following two verses are in a similar vein to the previous two:

\begin{quote}
The idolaters will say:
Had Allah willed, we would not have ascribed partners to him,
nor would our fathers have, nor would we have forbidden anything.
Likewise did those before them tell such lies till they tasted our punishment.
Say: Have you any information that you can bring forth for us?
Lo! You only follow conjecture and you only tell lies.

Say: With Allah’s is the conclusive argument.
If it had been His will, He could have certainly guided you all.

(\QRef{6:148–149})
\end{quote}

Yes, we have a thousand and one pieces of information
and all of them are based on many verses such as the last two,
and those before it and many others,
for they are all a jumble of contradictions, embracing all that’s been said,
is said and will be said regarding sayings about Predestination and Freewill
until the Day of Judgment.
Furthermore what is the meaning of accusing them of
‘only following conjecture,’ in fact worse than that, accusing them of lying?

So is relying on the four previous verses and many others, conjecture,
nay and lying? Is this reasonable?
And amazingly the verse concludes by confirming
that which it negated at the beginning:
“Had Allah willed, we would not have ascribed partners to him...
Likewise did those before them tell such lies...”
and this is what it is blaming them for!!

5. And the idolaters say; Had Allah willed,
we would not have worshipped anything beside Him,
we nor our fathers, nor would we have forbidden anything without Him.
Thus did those before them do...

(\QRef{16:35})

So is their saying;
”Had Allah willed, we would not have ascribed partners to him,” (and)
“Had Allah willed, we would not have worshipped anything beside Him,”
conjecture? Nay, lying?
In fact what they are saying is the truth, its perfectly sound and reasonable.
More than that it is supported by the \Quran\ itself,
whose utterances on this topic are no more than a cocktail of contradictions
that don’t settle on one single viewpoint and which has wearied
and sapped the energy of the Mufassirun on nonsense that contains no benefit.

6. The Jews are the chosen people of God according to the text of the \Quran:

\begin{quote}
“Oh Children of Israel! Remember my favour which I bestowed upon you,
and that I favoured you over all other peoples.”
(\QRef{2:47},\QRef{2:122}{122})
\end{quote}

On the contrary! The Jews are not Gods chosen people,
but are just like the rest of humans beings:

\begin{quote}
“Say: Oh you who are Jews!
If you claim that you are favoured of Allah to the exclusion of mankind,
then long for death if you are truthful.” (\QRef{62:6})

“The Jews and the Christians say we are the children of God
and his beloved ones.
Say: Then why does he punish you for your sins? Nay!
You are ordinary human beings like those he created.
He forgives whoever he wants and punishes whoever he wants
and to God belongs the dominion of the heavens and earth
and all that is between them and to him is the end.” (\QRef{5:18})
\end{quote}

And God will give his servants mastery over them until the Day of Judgment:

\begin{quote}
“And when your Lord announced that He would certainly send against them
until the Day of Judgment, those who would subject them to severe torment.
Your Lord is swift in retribution,
and is indeed Oft\–Forgiving, Most Merciful.” (\QRef{7:167})

Despite this they will be elevated over the earth
after spreading corruption in it twice!
I don’t understand why that was limited to twice
only when their lives have all been corruption and spreading mischief!

“And We decreed for the Children of Israel in the Book:
You will indeed spread corruption in the earth twice,
and you will certainly be elevated with mighty elevation.”

(\QRef{17:4})
\end{quote}

7. “Eternity” (\ar{الخلود}) in the \Quran\ is of three types —
each one contradicting the other:

Eternity that is unlimited and unending.
Eternity that is limited to last as long as the Heavens and Earth do.
Eternity that is limited to whatever Allah wants it to be.
So which of these types should we take seriously?

As for Eternity that is unlimited, he said:

\begin{quote}
“Allah will say:
This is the day when their truth shall benefit the truthful ones;
they shall have gardens beneath which rivers flow to abide in them for ever:
Allah is well pleased with them and they are well pleased with Allah;
this is the mighty achievement.” (\QRef{5:119})
\end{quote}

Then there is the type of eternity that is limited to last
as long as the Heavens and Earth do,
which is the strangest one because there will not be any heavens or earth
(at the time of Judgment) as they will be rolled up
with advent of the Day of Judgment and they will disappear and not return.
“The day we shall roll up the Heavens like a scroll rolled up for books...”
(\QRef{21:104})
(i.e.\ So how can Allah send people on the Day of Judgment,
to stay in a place ‘for as long as the Heavens and Earth exist,’
when they already don’t exist??!!)

Following it is the Eternity that is restricted to the Will of Allah.
By using this phrase, Allah has not restricted himself to anything.
One could say he has completely undermined the whole concept of eternity
and washed his hands of it as he did with his ‘chosen people’:

\begin{quote}
“So as for those who will be wretched they will be in the Fire;
sighing and wailing will be their portion therein,
forever in it as long as the heavens and the earth endure
except as your Lord pleases,
surely your Lord is the mighty doer of what He intends.”
(\QRef{11:106–107})
\end{quote}

Oddly the 2nd and 3rd types are mentioned together in a single verse.
If it is correct (the 2\Sup{nd} \& 3\Sup{rd} definitions of eternity,)
then this is to the benefit of “the wretched”,
since it sets a limit to their suffering.

\begin{quote}
“But as for those those who are blessed, they shall be in the Garden —
forever in it, as long as the heavens and the earth endure
except as your Lord pleases, a gift without break.” (\QRef{11:108})
\end{quote}

If this is correct then it is not to the benefit of “those who are blessed”
because it makes “those who are wretched” better than them,
since refraining from giving the wretched what was decreed
and removing the punishment is more sweet than a period of pleasure
that had been ordained as being forever,
but is then cut unexpectedly because it is tied to an arbitrary will
that has no consistency nor continuity.
But He is not to be questioned about what he does!
Indeed this — by my life —
more intensely stings the soul and causes it more pain
than anything the wretched may suffer in the punishment of Jahannam.
Where is the fairness in that?

8. “Indeed those who don’t believe in the signs of Allah,
Allah will not guide them and they shall have a painful torment.”
(\QRef{16:104})

Is this true? Nay! Is this reasonable?
What is this astounding generalization?
What is this absolute judgment that cannot be justified by logic nor reality?
What about those who believed in the signs of Allah
after being amongst those who didn’t believe?
Who guided them? Shaytan?
Did they emerge from their mothers wombs as believers?
Doesn’t this verse contradict the many other verses that cannot be counted,
where Allah guides people to belief?\fnmarksym[*]

\fntextsym{i.e.\ \QRef{2:198}: “and remember Him for He guided you
when before you were of the misguided”}

9. “They make it a favour on you that they have accepted Islam.
Say: Do not bestow on me your acceptance of Islam.
Nay! It is Allah who has bestowed upon you the favour,
that He has guided you to Faith, if you are truthful” (\QRef{49:17})

\begin{quote}
“And hold fast to the rope of Allah, all together,
and do not be divided and remember Allah’s favour on you
when you were enemies and He joined your hearts,
so that by His Grace, you became brothers;
and you were on the brink of the pit of Fire, and He saved you from it.
Thus doth Allah make His Signs clear to you that you may be guided.”
(\QRef{3:103})
\end{quote}

How truly amazing are these verses (like that mentioned in point 8 above,)
that declare that those who disbelieve, and the polytheists,
and the transgressors, and the astray and the misguiders at the time
that Islam appeared, will not be guided, even though the majority of those
who entered it were disbelievers before, or transgressors, or astray.
So who guided them then after they were not of the guided?
Didn’t Allah say repeatedly that He is the one who bestowed upon them faith
and guided them to the straight path?

The strange thing is that these verses are repeated many times in the \Quran\
until one imagines that it is the result of outburst and emotion rather
than the result of deep thought and reflection.

10. “Whoever God guides, then he will be guided while whoever He misguides,
then he will never find helpers other than him.
And we shall gather them on the day of judgment upon their faces,
blind, dumb and deaf.
Their abode will be Jahannam.
Every time it abates We shall increase the fierceness of the Fire.”
(\QRef{17:97})

If that is true (that they will be made deaf, dumb and blind)
then what is the meaning of other verses
where the inhabitants of Hell talk to each other,
blaming and reproaching one another for following the other:

\begin{quote}
“When those who were followed disown those who followed (them),
and they behold the doom, and all their aims collapse with them.”

“And those who were but followers will say: If we had another chance,
we would disown them even as they have disowned us.
Thus will Allah show them their own deeds as anguish for them.
And they will never get out from the Fire.”%
\fnmarksym[*]\ (\QRef{2:166–167})
\end{quote}

Would that I knew how the previous verse can call them blind, dumb and deaf?
Yet they have sharper vision, a more eloquent tongue
and keener hearing than you or I.
Indeed despite what they are experiencing of the punishment of Hell
and the terrors of the blazing fire,
they are able to see the people of Paradise
and what they are experiencing of bliss and ask them —
in clear Arabic language — to give them some of the water
or food God has provided them with:

\begin{quote}
“And the people of Hell will call out to the people of Paradise to
‘Pour down to us water or anything
that Allah has provided for your sustenance.’
They will say: ‘Allah has forbidden them to the disbelievers.’”
(\QRef{7:50})
\end{quote}

They will also confess to their sins and call upon God to return them
to life on earth so that they can be good — but in vain:

\begin{quote}
The Fire will burn their faces so that they are grimacing in agony.

“Were not My revelations recited unto you, but you denied them?”

They will say:
“Oh Lord! Our misfortune overwhelmed us, and we became a people astray!”

“Oh Lord! bring us out of this: if ever we go back (to disbelief),
then surely we are wrong-doers!”

He will say: “Get back in it, and don’t talk to me!”

(\QRef{23:104–108})
\end{quote}

This is just a few of the many verses that show they are
no less able to see, speak and hear as we are.
You can see that according to the \Quran\ itself,
they remain in Hell with all their senses and consciousness,
not losing any of it at all.
So where does this leave the claim
that they will be made blind, dumb and deaf?

% Workaround. See previous instance for explanation.
\renewcommand{\thefootnote}{*}

\fntextsym{“And they will never get out from the Fire” —
It’s worth pointing out that this also contradicts the verse
quoted earlier suggesting they will get out from the Fire!}

11. Believe it or not Allah took the Children of Israel out of Egypt
and also made them inherit Egypt and it’s bounties and treasures:

\begin{quote}
“We sent revelation to Moses, saying:
Take away My slaves by night, for ye will be pursued.”

“So Pharaoh sent heralds to the Cities,”

“(Saying): These (Israelites) are but a small band,”

“And most surely they have angered us.”

“But we are a multitude amply fore-warned.”

“Thus did We take them out from gardens and watersprings,”

“And Treasures, and every kind of honourable position,”

“Thus it was and We made the Children of Israel inherit it.”

(\QRef{26:52–59})
\end{quote}

No comment. For not commenting is more eloquent in this case.

12. “Surely We have sent you with the truth as a bearer of good news
and a warner and there is no people/nation (\ar{أمة})
but a warner has gone among them.” (\QRef{35:24})

But this verse contradicts:

\begin{quote}
“If We willed, We could have sent a warner to every village (\ar{قرية}).”
(\QRef{25:51})
\end{quote}

The words: People/Nation (\ar{أمة}), Town (\ar{مدينة})
and Village (\ar{قرية}) have the same meaning in the \Quran.
They mean a sedentary group that resides in a particular place
where it seeks its living and needs.
Nay, they are also applied to transient groups that are non\–sedentary:
“And when he came to the water of Midian he found there a group (\ar{أمة})
of men getting some water.” (\QRef{28:23})
They also have other meanings in the \Quran\ that don’t concern us here.

13. Would you like more contradictions of the the \Quran?
Below is a contradiction related to the story of Yunus:
Did Allah cast him onto the shore or did he not cast him?
The \Quran\ has two contradictory views on the matter
one confirming it and one negating it:

\begin{quote}
“Indeed Jonah was one of the messengers,”

“When he ran away to a ship completely laden,”

“And then drew lots and was of those who are rejected.”

“And the fish swallowed him while he was blameworthy,”

“Had he not been one of those who glorify (Allah)”

“He would certainly have remained inside the Fish
till the Day of Resurrection.”

“So We cast him onto the shore, while he was sick.”

(\QRef{37:139–145})
\end{quote}

So Allah cast him onto the shore, then.

No! He didn’t cast him onto it!

\begin{quote}
“So wait with patience for the Command of thy Lord,
and be not like the Companion of the Fish — when he cried out in despair.”

“Had it not been that favour from his Lord had reached him he would indeed
have been cast off onto the shore, in disgrace.”

(\QRef{68:48–49})
\end{quote}

So there you have it!
Allah’s favour reached him, and if it had not
then he would have cast him onto the shore!

14. When Allah chose Moses for his revelation after he left Madyan
and with him his people he was called while at the blessed valley of Tuwa
where he saw a fire burning but not getting burnt.
So Allah ordered him to go to Pharaoh with his signs in the hope
that he may be warned and fear God.
So Moses had no choice but to respond to the order of his Lord.
But he complained that his tongue was tied and he couldn’t speak clearly
and he asked Allah to cure him of it and to open his breast
and make his affairs easy, so Allah responded to his prayer:

\begin{quote}
“Go thou to Pharaoh, for he has indeed transgressed all bounds.”

“(Moses) said: ‘O my Lord! expand me my breast,”

“Ease my task for me,”

“And remove the impediment from my speech,”

“So they may understand what I say...’”

“...(Allah) said: ‘Granted is thy prayer, O Moses!’”

(\QRef{20:24–28},\QRef{20:36}{36})
\end{quote}

Did Allah really grant his prayer or was he in the same state as before?

It appears from the text that Allah granted his prayer immediately,
since he said straightaway: “(Allah) said: Granted is thy prayer, O Moses!”

But this verse contradicts another verse that shows that Moses —
despite the granting of his request —
was still suffering difficulty in speaking that prevented him from clarity
and Pharaoh found it difficult to understand what he was saying:

\begin{quote}
“And Pharaoh proclaimed among his people, saying:
‘O my people! Does not the dominion of Egypt belong to me,
(witness) these streams flowing underneath my (palace)?
What! see ye not then?”

“Am I not better than this (Moses), who is a contemptible wretch
and can scarcely express himself clearly?’”

(\QRef{43:52})
\end{quote}

So he was still unable to express himself with clarity, then.
He could not clearly or properly articulate in a way that is essential
to make plain his intention and purpose of his mission to Pharaoh.
So was Moses prayer really granted or not?

% TODO: format...
15. The Day of Judgement is the day of the greatest panic.
It is the day of unbelievable anguish and terror!!

On that day:

\begin{quote}
“The guilty will be known by their markings and will be seized
by their forelocks and their feet.” (\QRef{55:41})
\end{quote}

Let’s ignore the fact that “seized” should be in the plural,
(it should be \ar{يؤخذون} and not \ar{يؤخذ}) since it refers
to “the guilty” (\ar{المجرمون})  a plural noun, but instead ask:
Were they seized like that without being asked any questions?
Is the fact that these people are known by their marks enough
to pronounce judgment upon them?
The matter is a little confusing for me,
because in the \Quran\ are verses that stipulate they will be asked questions,
while others contradict that.
For that reason I am confused and am unable to come
to an unequivocal view on the matter:

\begin{quote}
“So, by your Lord we will most certainly question them all about that
which they used to do!” (\QRef{15:92})

“By Allah they will most certainly be questioned about that
which they fabricated.” (\QRef{16:56})

“If Allah had willed he could have made you as one nation,
but he misguides who he wants and guides who he wants,
and you will most certainly be questioned about what you used to do.”
(\QRef{16:93})

“And indeed it is a reminder for you and your people,
and soon shall they be questioned.” (\QRef{43:44})
\end{quote}

But this categorical confirmation that they will be questioned
turns into negation in other verses where those in question
are shoved into Hell without questions or trial,
relying instead on the fact that they are known by their markings.
This knowledge — so it appears — is not in need of questions and answers
or in other words; a fair trial!
Such a thing would not enter to our weak human minds,
but it appears that the angels are experts,
entrusted and well-versed in knowing the signs of people,
deserving of complete trust in such things.
If not, then Allah would not have given them free reign
to act independently as they wish.
So there is no need to hold a trial with all their complications
that never end and if Allah thought that there was any injustice in that
towards his servants he wouldn’t have allowed it.
Have you forgotten His saying, Most High is He:

\begin{quote}
“...and your Lord will not deal unjustly with anyone.” (\QRef{18:49})
\end{quote}

Allah is far above doing that!

\begin{quote}
\begin{center}
{\itshape
Remember my beautiful when I created you as a drop

And don’t forget my fashioning your figure within

So entrust to me your affair and know that

I make my own rules and do what I like.}

(Pious poetry)
\end{center}
\end{quote}

For that reason don’t be afraid of the verses
that negate asking people questions about what they did:

\begin{quote}
“And the sinners will not be asked about their sins.” (\QRef{28:78})

“When the heavens are split and become rosy red...
On that day neither man nor jinni will be questioned of his sin.”
(\QRef{55:37–39})
\end{quote}

...

\section{The \Quran\ and Science}
TODO:
\section{Everything in the \Quran\ is from God}
TODO:
\section{Verses that Have no Meaning}
TODO:
\section{The Rhymed Prose of the \Quran\ and
the Rhymed Prose of the Soothsayers}
TODO:
\section{The \Quran\ and the Belief in the Unseen}
TODO:
\section{Barbarism of the \Quran}
TODO:

% Chapter 5:
\chapter{God in the \Quran}

% This section doesn't have a number in the original.
\phantomsection
\def \IntroSectionTitle{Introduction — The Existence of God and
the non-Existence of a Likeness}
\addcontentsline{toc}{section}{\IntroSectionTitle}
\section*{\IntroSectionTitle}
TODO:
\section{Attributes of God in the \Quran}
TODO:
\section{God and the Devil}
TODO:
\section{God the Compassionate the Merciful}
TODO:
\section{God is Near and Answers}
TODO:
\section{God is the Best of Sustainers}
TODO:
\section{There is no Help Except from God}
TODO:
\section{God Crams Himself into Everything}
TODO:
\section{God is the Conqueror over his Slaves}
TODO:
\section{With God. Man Must Impose his Law}
TODO:
\section{God. An Ineffective God}
TODO:

% Start of appendices:
\clearpage
\appendix
\addappheadtotoc

\chapter{Asbab al\–Nuzul of “The People of The Cave”}
\label{apdx:cave}

“When Al\–Nadr said that to them, they sent him and \`Uqba b.\ Abu Mu\`ayt
to the Jewish rabbis in Medina and said to them, ‘Ask them about Muhammad;
describe him to them and tell them what he says,
for they are the first people of the scriptures
and have knowledge which we do not possess about the prophets.’

They carried out their instructions, and said to the rabbis,
‘You are the people of the Taurat, and we have come to you
so that you can tell us how to deal with this tribesman of ours.’

The rabbis said, ‘Ask him about three things of which we will instruct you;
if he gives you the right answer then he is an authentic prophet,
but if he does not, then the man is a rogue,
so form your own opinion about him.

Ask him what happened to the young men who disappeared in ancient days,
for they have a marvellous story.
Ask him about the mighty traveller
who reached the confines of both East and West.
Ask him what the spirit is.
If he can give you the answer, then follow him, for he is a prophet.
If he cannot, then he is a forger and treat him as you will.’

The two men returned to Quraysh at Mecca and told them
that they had a decisive way of dealing with Muhammad,
and they told them about the three questions.
They came to the apostle and called upon him to answer these questions.

He said to them, ‘I will give you your answer tomorrow,’
but he did not say, ‘if God will.’
So they went away; and the apostle, so they say,
waited for fifteen days without a revelation from God on the matter,
nor did Gabriel come to him, so that the people of Mecca
began to spread evil reports, saying,
‘Muhammad promised us an answer on the morrow,
and today is the fifteenth day we have remained without an answer.’

This delay caused the apostle great sorrow,
until Gabriel brought him the \QRef{18}{Chapter of The Cave},
in which he reproaches him for his sadness,
and told him the answers of their questions, the youths,
the mighty traveller, and the spirit.”

\emph{Ibn Ishaq’s account of the Asbab al\–Nuzul
(Reasons for Revelation) for the story of the “People of the Cave”.}


\backmatter

\chapter{Epilogue}
TODO:

\chapter{Index}
TODO:

\end{document}
