%!TEX TS-program = xelatex
%!TEX encoding = UTF-8 Unicode
\documentclass[12pt]{memoir}
% \usepackage{arabxetex}
\usepackage[xetex,bookmarksnumbered=true,pdfborder=0]{hyperref}
\usepackage{xltxtra}
\usepackage{bidi}
\usepackage{xunicode}

\usepackage{fontspec}
\usepackage[UKenglish]{babel}

\setmainfont[Mapping=tex-text]{DejaVu Serif}

\special{pdf: docinfo <<
  /Author   (Abbas AbdulNoor)
  /Title    (My Ordeal with the Qur’an)
  /Keywords (Arabic Qur'an Quran Koran Islam critique translation)
  /Subject  (A linguistical critique of the Qur’an)
>>}

\newfontfamily{\arabicfont}[Script=Arabic,Scale=1.7]{Scheherazade}
\newcommand{\ar}[1]{\RL{\arabicfont#1}}

% Macros for the word Qur'an.
\def \Quran{Qur\-ʾān} % Used when followed by a punctuation mark.
\let \Qrn=\Quran      % Used when followed by whitespace.

% Breakable forward slash.
\def\/{\discretionary{/}{}{/}}

% "Nota bene" macro.
\newcommand{\NB}[1]{\emph{\small NB: #1}}

% Paragraph spacing:
\setlength{\parskip}{1ex plus 0.5ex minus 0.2ex}

\pagestyle{plain}

\def\arabictitle{مِحْنَتِي مَعَ القُرْآن ومَعَ اللَّهِ فِي القُرْآن}
\title{My Ordeal with the \Quran\\{\Large And the God of the \Quran}\\
\ar{\Large \arabictitle}}

\def\abbasInAr{عَبَّاس عَبْدُ النّور}
\author{Abbas Abdul Noor\\\ar{\abbasInAr} \and Hassan\\(Translator)}

%%%%%%%%%%%%%%%%%%%%%%%%%%%%%%% Document begin %%%%%%%%%%%%%%%%%%%%%%%%%%%%%%%%
\begin{document}
\frontmatter

% Title:
\maketitle
\thispagestyle{empty}
\cleardoublepage

% Table of Contents:
\setcounter{page}{1}
\tableofcontents

% Preface:
\chapter{Preface}
TODO:

% Introduction:
\chapter{Introduction}
TODO:

\mainmatter

% Chapters:

% Chapter 1:
\chapter{My Journey from Doubt to Belief}

\section{The Belief Phase}
TODO:
\section{The Test Phase}
TODO:
\section{The Whirlwind Phase}
TODO:
\section{The Investigation Phase}
TODO:
\section{The Breaking-Off Phase}
TODO:

% Chapter 2:
\chapter{The Methodology of Examining the \Quran}

There are two methods to understand the \Quran. They are: The Methodology of
Transmission, that gives precedence to revelation over reason, the
unquestioning acceptance of the veracity of the text and the inability of
reason to comprehend its ultimate aims and objectives, and the Methodology of
Reason that gives precedence to reason over revelation and its ability to
comprehend the truth without need for reference to the text. For text is the
last concern of the mind that is in itself free, independent and believing.

For that reason I will employ, in this book, the Methodology of Reason, that
Descartes established at the beginning of the modern age even though he did not
always abide by it and in particular in understanding religious texts but
manoeuvred, twsited and distorted the neck of Reason to stop the rot that fills
Revelation and that which Revelation contains of garbage that diseases minds.

See how this great man compromises for the sake of (divine) text. Descartes
wasn’t the first to compromise, not at all, and he will not be the last apart
from those who believe in Reason and act according to it and trust that which
Reason obligates — and they are few indeed! For (divine) text has such
influence and power, few can withstand.

The fundamental principle of the methodology of Reason is impartiality and
objectivity and to approach the research with a mind free from prejudice and
bias “Bias is Sickness” as they say. In this spirit we must resolutely proceed
in studying the \Qrn{} and treat it as we would any other scientific research
and subject it to examination and analysis and skepticism and rejection and
contestation because that is what will make our examination fruitful and
profitable and make it of universal benefit.

Applying the methodology of Reason to the \Qrn{} is in my view a dangerous and
massive event that will shake the earth below the feet of blind faith (Taqleed)
and inertia, and putrid decay. And it something that must be done the for the
most extreme cure is cauterization.

The \Qrn{} has deep roots in our cultural composition and if these roots are
shaken then composition changes to a different composition and destiny changes
to a different destiny and people changes to a different people and as a
consequence a new generation emerges that wasn’t in the reckoning.

For that reason the first thing I will confront you with in this discourse is
that I doubt the \Qrn{} and in the god of the \Qrn{} and in the teachings of
the \Qrn{} and in the miraculous nature of the \Qrn{} and in the sublime
language of the \Quran.

I insist on doubt, I embrace it on principle. For doubts — as al-Ghazali says —
leads one to the truth. So one who does not doubt cannot look and one who does
not look cannot see and one who cannot see will remain in blindness and error.

This is my method in doing the work and this is how I start examining,
thinking, reading and reflect until the circumstances lead me to something
resembling certainty. That is because what we call the miraculous nature of the
\Qrn{} and infallibility of the \Qrn{} is really only like any human piece of
work containing error as well as correctness.

I’m aware of the consequences which I have arrived at but that won’t deter me
from proving them and broadcasting them and expressing my opinion freely. I
know in advance that it will lead to mortal dangers and grave confrontations
that I perhaps don’t need. But no! For truth deserves to be followed. I will
take refuge in a mountain that will protect me from the water
(\NB{This is a reference to Noah’s son})
as far as I am able and if not then martyrdom is better than I suffer
incapacity and weakness to declare what I believe in and what many others
besides me believe in, though they are waiting for the spark of light to set
alight after that many sparks of light and sparks of light light up the dark
tunnel that we are living in. So is there any other way to escape from a path?

As for the reasons that led me to doubt in the \Quran, they are because of its
contradictions, generalisations, pompous rhetoric, and fecetious phrases that
have no meaning. The grammatical and stylistic errors that the classical
scholars were at their wits end trying to find explanations. And others, both
scientific and historical that I consider the Lord of the worlds above making.
Just as the \Qrn{} is full of rhetorical explosive charges, verbal bombs, that
create such an extreme uproar that ears almost become deaf but after deep
analysis and despite what it contains of sweetness and charm and alluring
beauty, it is pale, emaciated, little content, lacking substance, bubbles in
the air, radiating beams of light like fireworks, except that they soon
extinguish and fall to the ground spent, leaving behind it pitch dark.

\begin{quote}
It is as though it is a bolt of lightening glistening with fury — then fizzles
out and it is as though it never shone.
(a line from a poem by \emph{Ibn Sina})
\end{quote}

Many of the prose of the masters of eloquence (classical literati) and even the
doggrel of soothsayers is better — a thousand times — than many of the \Qrn{}ic
verses that are of nonsensical language, stuffed full of fairy tales, that the
\Quran{}ic commentators — and strangely, amongst them Mu’tazilites — became
masters at dealing with and defending.

There remains another matter and it isn’t the last. It is the matter of the
indictment of the \Qrn{} upon the \Quran. For the narrative of the \Qrn{} is
confused — and how confused — for how abundant is the confusion of the \Quran.
“The Almighty” said: “And if this was from other than Allah you would have
found a lot of contradiction.”

The \Qrn{} has passed the guilty verdict upon itself! For that which it
contains of contradictions goes beyond the limit of ‘a lot’. Nay, it is the
centre of every disparity and contradiction. The amount of disparities and
contradictions in any book in the world has never reached the level of the
\Quran. Yet despite this they want us to believe that there is no disparity nor
contradiction in the \Quran. We must ignore the evidence to believe that which
does not agree with reason nor with the evidence in the manner of “Believe
Allah and disbelieve the stomach of your brother”
and if you don’t (ignore evidence and reason) then you will see and hear that
which will not please you.
(\NB{Reference to a hadith where someone came to the prophet complaining of his
brothers stomach\/bowel problem, the prophet said ‘give him honey’ —
but the guy
returned saying it has got worse, so the prophet said ‘give him honey’ this
happened twice more — finally the prophet said ‘believe Allah and disbelieve
the stomach of your brother’.})

I am not calling for the renunciation of religion, for that is a difficult
objective, in fact it is a demand that cannot be sought. Because religion for
its adherents is sweet nectar and for so long I my self savoured this sweetness
until I returned to my senses.

I say I am not calling for the renunciation of religion but am calling for the
end to resorting to religion for decisions in all matters and sticking its nose
into every tiny matter in the affairs of life. And that (can be achieved) by
applying Secularism as a principle both in thought and in life. Secularism is
not disbelief, nor is it a call to disbelief as some of its enemies portray
it, but it is merely placing a limit to the interference between religion and
the state.

Religion is not the execution of the captive, nor the stoning of the adulterer,
nor the chopping of the hand of the thief. Religion, according to secularists,
is that which lives in the heart and dwells in the conscience. Believe what you
like, but beware of imposing your beliefs on others nor make make it into a
system for government or life. Religion is for God while the nation is for
everyone. That is the slogan of Secularism.

There is nothing inviolable, nothing sacred in Secularism. The only thing that
is inviolable and sacred in it is Humanity and the value of Humanity and the
freedom of Humanity and respect for the dignity of Humanity. The lack of
exploitation of man by another. The unbeliever (Kafir) is not the one who
disbelieves in religion. The only unbeliever is the one who disbelieves in
Humanity and Human Rights.

For the value of life is Reason. The value of life is Freedom. The value of
life is Progress and Development. The value of life is Innovative Vision and
expressing it according to what suits the requirements of the the time and
place. As for disbelief and belief, angel and devil, it (creates) conflict that
impedes the development of innovations and the flow of progress in a world of
powers and balances of powers and centers of powers.

The thing that most frightens man is to be consigned to the debris of memory.
Ruminating over myths and delusions. In a trance over the invisible, and text,
and miraculousness and rhetoric and to follow the stories of the garden of Eden
and the Houris, and the (verse of) Light and the servant boys (of paradise) and
stories of Jinn and tales of Luqman and the like of these stories and tales
that for so long fertilized the minds and imaginations, both in the near and
distant past but then today lose the bet. (i.e.\ discover it was all BS)


% Chapter 3:
\chapter{The \Quran\ According to the Belief of Muslims}

\section{The \Quran\ is the Speech of God}
TODO:
\section{The \Quran\ is the Centre for all
Schools of Thought \& Opinion in Islam}
TODO:
\section{The Linguistic Beauty was the \Quran’s Key
to the Hearts of the Pre-Islamic Arabs}
TODO:
\section[The Work of the Exegetes of the \Quran]
{The Work of the Exegetes (Mufassirin) of the \Quran}
TODO:
\section{An Inevitable Revolution}
TODO:

% Chapter 4:
\chapter{The Miraculous Nature of the \Quran}

\section{The Belief of Muslims in the Miraculous Nature (of the \Quran)}

\begin{quote}
“Say: If the whole of mankind and Jinns were to gather together to produce the
like of this \Quran, they could not produce the like thereof, even if they
backed up each other with help and support.” (17:88)
\end{quote}

The \Qrn{} is indeed a unique book. It is prose and yet unlike prose. It is
poetry and yet unlike poetry. It is metered and rhyming and yet it is not like
the (standard) meters and rhymes. So what is it then? It is the \Qrn{} and
that’s it!

Perhaps the best description of the \Qrn{} is that which the late Dean of
Arabic literature, Dr.\ Ta Ha Hussain said: “The Genres of Arabic expression
are poetry, prose and \Quran.” For the \Qrn{} is not poetry — no! and it is not
prose. It is a type of speech that is of a singular nature, unique of its kind.
It’s \Quran! For that reason they (the scholars) are united in the opinion that
what is called the miraculous nature of the \Qrn{} is its amazing composition.

Miraculousness (al-’Ijaaz) in the Arabic language comes from “To Make Unable”,
in other words it attributes the inability to another and a miracle is called a
miracle because mankind is unable to replicate it.

The (scholarly) discipline of the Miraculous Nature (of the \Quran) was a
discipline that was an innovation in religion. This discipline reached its full
maturity in the 4th century of the Hijra when it became independent and grew
into a discipline in its own right. Today it’s a fundamental tenet of faith
that no-one can dare throw doubt on. Beginning in the 4th century of the Hijra
the (discipline) of the Miraculous Nature became indelibly written in stone.

Despite that there were those who cast doubt on this belief, going right back
to the first centuries of Islam.

Perhaps the first of these was al-Ja’d ibn Dirham, tutor to Marwan ibn Muhammad
the last of the Umayyad Caliphs. For he was the first one to openly express
skepticism of the \Quran, and refutation of it and rejection of things in it.
He said that its eloquence was not a miracle and that people can do the like of
it and better than it when no-one had before him had said such as that. Marwan
— who was nicknamed the donkey — used to follow his view to the extent that he
was linked to him and called “Marwan al-Ja’di”\footnotemark.

\footnotetext{See: Mustafa Sadiq al-Rafi’i, “The Miraculous Nature of the
\Qrn{} and the Prophetic Rhetoric.” Page 160.}

During the mid Abbasid period this view (that the \Qrn{} was not a miracle)
spread along with other views of a similar nature, such as the view that the
\Qrn{} was created as well as its opposition (those who believed it was not
created, but eternal — on the “Protected Tablet”). The first to go to great
lengths in that was ’Isa ibn Sabih, known as Abu Musa al-Mirdar, who was one of
the Mu’tazilite scholars and amongst the leading ones. He was called the Monk
of the Mu’tazilites and differed from the rest of the Mu’tazilites in all of
the issues that concern us here. Saying about the \Qrn{} that people are able
to produce the like of this \Qrn{} as regards eloquence, and composition and
rhetorical beauty\footnotemark.

\footnotetext{Al-Baghdadi, “The Difference Between the Groups” Page 164–165;
and al-Shahrastani, “The Book of Sects and Creeds”, 1 / 68–69.}

Similar to that (view, that others could produce the like of the \Quran) was
the view taken by his contemporary, Ibrahim Ibn Sayyar Ibn Hani’ al-Nazzam, who
expounded many of the works of the philosophers and combined their ideas with
the ideas of the Mu’tazilites\footnotemark.
But he differed from his colleagues in 13 matters,
while al-Baghdadi increased that number to 21.

\footnotetext{Al-Shahrastani, 1 / 35–45.}

If al-Shahrastani labels the areas al-Nazzam differed from his colleagues,
“issues”, these “issues” become “shameful scandals” in the view of al-Baghdadi!
So the 9th issue that al-Shahrastani reproaches al-Nazzam about; becomes “The
25th shameful scandal of his shameful scandals” according to the wording of
al-Baghdadi: “His view regarding the miraculous nature of the \Qrn{} that it is
to do with the fact it predicts events of the past and future, and to do with
the the fact it diverted the causes of opposition and prevented the Arabs — by
force and incapacitation — from being concerned with (trying to imitate) it,
because if he (Allah) let them then they would have been able to produce a Sura
(chapter) the like of it in beautiful rhetoric, eloquence and composition” For
mankind is able to produce the like of this \Quran, but Allah diverted them
from doing that and prevented them by placing hinderance and incapacity within
them to do so. This is “The View of Divertion.”\footnotemark

(\NB{In other words al-Nazzam’s view was that the miraculous nature of the
\Qrn{} was NOT that it could not be imitated — in his view it could easily be
imitated — but that Allah prevented the Arabs from doing so!})

\footnotetext{Previous Reference 1 / 56–57.}

Now we ask what is the nature of the Miraculousness of the \Quran?

The scholars of Arabic — especially the scholars of language and elegant speech
— are completely united that the \Qrn{} is in itself a miracle. That its
miraculousness is in its wonderful composition, in the eloquence of its
expressions, the astounding nature of its clear speech, its unique style that
is unlike any other style, its captivating verbal impact, that reveals itself
in its acoustic structure, and linguistic beauty, and its sublime artistry.

Al-Qadi Abu Bakr (d.\ 1148) said the nature of the miraculousness of the \Quran
is in its composition, arrangement, and structure. That it’s beyond all types
of standard composition in the language of the Arabs, departing from their
styles of oration and for this reason they were unable to oppose it. The
composition of the \Qrn{} had no model to imitate, nor any antecedent to
emulate and it’s unreasonable (to think) that the like of it could happen by
chance. He said: “the miraculousness of \Qrn{} is much clearer in some parts
while in some parts it is more subtle, and more obscure.”\footnotemark

\footnotetext{Quoted from the previous reference, p.\ 122.}

Al-Imam Fakhr al-Din (d.\ 1210) said the nature of the miraculousness is the
eloquence, and unique style, free from all defects.

Al-Zamalkani (d.\ 727h) said the nature of the miraculousness derives from the
composition that’s unique to it and is not haphazard. In that its words are
finely balanced in construction, in meter and the reason behind the way it’s
been put together, in meaning. So that every type occurs in the best possible
place for its pronunciation and meaning.

And Ibn Atiyya said: The correct (opinion) and the one that laymen and experts
are agreed upon in regard to its miraculousness is that it is its composition
and the soundness of its meanings, and in the arrangement of the eloquence of
its wording. And that is because Allah’s knowledge surrounds all things and
surrounds all (aspects) of language. So since the organisation of the wording
in the \Qrn{} is something his knowledge completely surrounds, i.e.\ each word
perfectly suits the one it follows and each meaning is elucidated after another
and that is the case from beginning to end of the \Qrn{} and man is encompassed
by ignorance, bewilderment and perplexity and it is self-evident that no human
being encompasses all that, then as a result the arrangement of the \Qrn{} is
furthest epitome of eloquence and for that reason one destroys the saying of
those who claim that the Arabs were able to replicate the like of it or that
they were ‘diverted’ from doing so. The correct (opinion) is that it is not
within the ability of anyone ever!\footnotemark

\footnotetext{All the quotes are taken from the previous reference, p.\ 133
with some slight edits in wording but not meaning.}

However the scholars disagree about the difference in the degrees of eloquence
of the verses of \Qrn{} after having agreed it is in the highest forms of
eloquence, in so far as you cannot find phrasing that is more suitable nor
balanced, to convey that meaning.

Al-Qadi (Abu Bakr d.\ 1148) takes the opinion of ‘negation’, meaning negation
of there being any difference (in degrees of eloquence). For every word in it
is depicted in its highest (form\/usage) even though some people are better at
sensing it than others.

While Abu al-Qushairy and others take the opinion of ‘difference’ (in degrees
of eloquence), he said: “We do not claim that everything in the \Qrn{} is in
the highest rank of eloquence.”

And likewise others have said: “In the \Qrn{} is (both) the Eloquent and the
Most Eloquent.” This is the opinion taken by Sheikh ’Izz al-Din ’Abd al-Salam
who then asked: “Why was the \Qrn{} not entirely in the most eloquent form?”
Al-Sadr Mawhoob al-Jazari replied to the effect that ‘if the \Qrn{} had come in
that (most eloquent form) it would not be in the usual style of speech from the
Lord, combining the Most Eloquent with the Eloquent and so the argument for the
miraculous (nature of the \Quran) would not be effective.
So it came in their usual style of speech
to highlight the inability to challenge it and so they can’t say, for example:
‘You have brought that which we have no ability in its like. Just as it would
not be right for a sighted person to say to a blind person: “I beat you by
virtue of my sight.” Because the blind person will say to him: “Your victory
can only be valid if I was able to see, and (then you could say) your sight was
better than mine. But if I lack the ability to see then how can I make a
challenge?”\footnotemark

\footnotetext{The previous reference p.\ 109.}

In any case the \Qrn{} is in the eyes of Muslims the Prophet’s greatest
miracle. No falsehood in it either before or behind it. “Indeed everything in
the \Qrn{} is a miracle in respect of the music of its letters, the kinship
between its wording, the synchronicity of its words with its expressions,
and well-knit arrangement in its resonance and that which it arrives at in
regards to composition between the words and the fact that every word
intentionally fits its counterpart. As though the weave of each part perfects
its picture and completes its objective. Its meanings coalesce with its
words as though its meanings are related to its pronunciations and its
pronunciations were designed for it, and made to fit its size.”\footnotemark

\footnotetext{The previous reference p.\ 99.}


\section{What kind of Miraculousness is it?}

Now we say: Indeed the belief in the miraculousness of the \Qrn{} is no more
than a myth amongst myths. Indeed! The \Qrn{} is not amongst the secrets of the
gods. It doesn’t bear the slightest relation to divine inspiration that takes
it outside the (normal) activity of (human) history. It’s a purely human
achievement that follows the norms of humanity in strength and weakness,
correctness and error, agreement and contradiction, cohesion and disparity,
consistency and inconsistency, uniformity and disarray.

The direct result of all that is that the \Qrn{} is a very ordinary book. For
that reason it is necessary to remove it from its safe refuge, outside Human
history and return it to the world of people. After that it will no longer be
storehouse for timeless wisdom nor a divine book protected from error that no
falsehood can approach it from either front or behind.

In that way, it and its time and its context become part of the historical
process of the area which has witnessed, and continues to witness every day,
comparable books that influenced these books and are influenced by it and
ignite the interaction between them.

Every star-struck believer, regardless of whether he is from the common people
or their elite or even from the elite of the elite, relies (on the belief) that
“in the \Quran, due to beauty of the words and the splendour of the style
especially, no-one can attain the phrases, style and meanings.”\footnotemark

\footnotetext{Muhammad Abu Zahra, “The Greatest Miracle.”}

And that challenge, that Allah announced in the \Qrn{} for Man \& Jinn to bring
the like of this \Quran,

\begin{quote}
“Say: ‘If the whole of mankind and Jinns were to gather together to produce the
like of this \Quran, they could not produce the like thereof, even if they
backed up each other with help and support.’” (17:88)
\end{quote}

is absolutely true, but it doesn’t apply only to the \Quran, it also applies to
every great work. For just as Man \& Jinn are not able to produce the like of
the \Quran, likewise they cannot produce the like of that which Plato brought,
nor al-Jahiz, nor al-Tawhidi, nor Dante, nor Goethe, nor Shakespear...

Great works always contain the fingerprints of their authors. It is a part of
their identity. So if it is impossible to imitate these fingerprints, then it
is also impossible to imitate these works. Each one is a unique weave that has
no match in the works of man and thus establishes its character. Despite this
each one is not free from flaws and errors and defects that the critic can be
aware of. Likewise the \Quran. In the work of al-Jahiz and al-Tawhidi is that
which far surpasses what is in some of the verses of the \Quran, as we shall
see, but who dares criticise the \Quran?

Indeed the Muslims of the Middle Ages during the Golden Age, had more freedom
than Muslims in this time. If not then why does no-one dare, like al-Sarakhsi
and Ibn Rawandi and al-Razi, to defame the most holy symbol of Muslims, the
valuable of valuables that gives meaning to their existence and bestows on them
hope and immortality.

All efforts and active forces in the Islamic world have been enlisted to repel
the “Enemies of Allah”. Criticism of Allah’s book has been met with a reception
that varies from forbearance \& irritation, between insult \& vilification and
between suppression \& temperance, and ‘dealing’ with antagonists ranged
between chit-chat \& bluster,
to finding excuses and haphazard solutions — or as
I myself call it, “Patching” (the holes) — to save the word of Allah from the
clutches of the deniers, the astray and the ones who lead others astray.
Between hitting and slapping, punching and physical eradication, seeking
closeness to Allah through the blood of that insolent fabricator of lies about
Allah, denier of his signs, so that he be a warning to his like, the forces of
the Devil, “and Satan indeed found his calculation true concerning them, for
they followed him...” (34:20) them and the seducers. Then they topple into the
Fire of Hell all of them together\footnotemark. They are the ones who Allah
curses, and those who curse, curse them!!

\footnotetext{Allusion to what is related in Sura al-Shu’araa’ 26/94.}

Indeed opposing the \Qrn{} was a natural process that arose with the rise of
Islam, but the new religion killed it in the cradle, or at least was able to
silence it for a while. That was after the astounding victory that it achieved
in the Arabian Peninsula and the area surrounding it. Indeed it was such a
tremendous breakthrough that it temporarily diverted attention away from that
which interplays in it of (opposing) forces and deep contradictions that don’t
appear on the surface except in moments of quiet and stability or at the times
of fitna.

For that reason it is not strange that this process started anew or returned to
the open when the Umayyad dynasty began to disintegrate and draw towards its
inevitable end. For indeed Islam injured the pride of many of the leaders of
the heretics (Zanadiqa) — and they were the Shu’ubiyyah
(\NB{A popularist movement against the the supremacy of the Arabs})
— and nationalist pride overtook them and led them to fanaticism for the
religion of their fathers such as Zoroastrianism and Manichean dualism and
hatred towards Islam that ended their glory and destroyed their dreams in
lasting and noble life. A group of poets who belonged to “The League of the
Mujjan”
(\NB{A group of libertine\/dissident intellectuals such as Bashar ibn Burd and
Abu Nuwas})
joined them, fleeing from the constraints of religion
and seeking a life of freedom with no restrictions or regulations.

Then came the Abbasid period where the Shu’ubiyya movement was active side by
side with the Heretical movement (Harkatu z-Zandaqa) and the attacks on Islam
intensified and disparagement of its holy of holies — the \Quran. And at the
head of this movement were poets, satirists, and disaffected thinkers, the most
famous of them: Salih Ibn Abdul Quddus, and Abdul Karim ibn Abu al-‘Awjaa’, and
Abu ’Isa al-Warraq and Bashar ibn Burd, and his adversary Hammad Ajrad, and
Iban ibn Abdul Hamid al-Lahiqi, and Ibn Muqaffa’, and (his son?) Muhammad ibn
Abdullah ibn Muqaffa’, and Abd al-Masih al-Kindi who we shall say a few words
about in a bit to show the participation of non-Muslims in the attack on the
\Quran...

But the most famous of all these without argument is: Abu al-Hussein Ahmad ibn
Yahya ibn Ishaq al-Rawandi, and Abu Bakr Muhammad ibn Zakariyya al-Razi, under
both of whom the movement of Heretics reached its climax and extent of its
maturity and we will discuss now each of them briefly, enough to clarify what
we mean.


\subsection{Ibn al-Rawandi (d.\ 298H / 910AD)}

(\NB{none of his books have survived. What exists are quotes from critics.})

At first the Dissident Movement, or the Movement of Heretics, was simply a
spontaneous individual attitude or libertine outburst or a transient
intellectual position, but then this movement began to manifest itself and
crystalize with the passing of time until it became a comprehensive school of
thought based on the pillars of reason. It acquired supporters who believed in
it and worked to publicise it and disseminate its principles. This movement
continued to grow, consolidate and rise until it reached its apex under Ibn
al-Rawandi. His view of Prophethood formed the cornerstone of the the barrage
on the \Quran, by these Heretics, though without that extending to doubt in the
existence of Allah who revealed the \Quran.

Doubt in Prophethood was the furthest extent that this movement of Heretics
achieved in Islam. Then it came to a halt after a violent shake-up in concepts
and doctrines grew out of it, in the 4th century (of Hijra) that drew towards
it the movements of the concealed ideologies that were influenced by Gnosticism
and Esoteric Knowledge and especially those associated with the Shi’ism and the
Isma’iliyyah Shi’ism in particular.

Ibn Rawandi was the most famous of the dissenters of the 3rd century of Hijrah,
though only a little is known about him, even the date of his birth and death
are not known for certain. He was originally a Mu’tazilite (A non-orthodox
Sunni School of thought) then recanted and leaned towards Shi’ism and became
a bitter enemy to the Mu’tazilites.

He was a vehement believer in Reason, praising it and relying on it in all
matters and affairs. Reason, in his opinion, was: “The greatest gift bestowed
by God, glorified is he, upon his creation. Indeed it is through it that the
Lord and his blessings can be known and by virtue of it that orders \&
prohibitions, his promises \& threats become valid.”\footnotemark\@
% TODO: find correct place of this footnote:
\footnotetext{Quoted from Dr.\ Abd al-Rahman Badawi, from “History of
Disbelief in Islam” page 202.}
He wrote a book called “The Scandal of the Mu’tazilah”\footnotemark\
which was a critical analysis of the Mu’tazilite School of thought
from the perspective of the Shi’ah al-Rafidah and
a reply to the book of al-Jahiz “The Virtue of the Mu’tazilah”.
But this period did not last long and we see him after that
amongst the group of those who the author of the “The Catalogue” (Kitab
al-Fihrist, by Ibn al-Nadim d.\ 995 AD ) gives the title “Theologians who
manifest Islam, but conceal heresy.” He was influenced in this by Abu ’Isa al
Warraq who was a teacher of his and encouraged him towards Heresy.

\footnotetext{See: The previous reference, page 87, 186 and that which is
after it.}

Ibn Rawandi began his Heretical writings in the latter years of his life, and
they are the books that he owes his importance and high status to. Amongst
these books is a book where he dealt a massive blow to the \Quran. He called it
“The Crushing Blow”. It was, as its title suggests, a merciless attack on the
\Quran.

A third book is attributed to him called the book of “The Emerald” where he
refutes the concept of Prophethood in Islam and attacks the belief in the
Miraculousness of the \Quran. We said that this book is “attributed to him” due
to a reference that it’s said it’s attributable to al-Jaba’iy and goes on to
say: “Indeed Ibn al-Rawandi and Abu ’Isa Muhammad Ibn Harun al-Warraq the
Heretic, also dispute with one another over the book of “The Emerald” each one
claiming that it is amongst their compilations as both were in complete
accordance in attacking the \Quran.”\footnotemark

\footnotetext{Quote from the previous reference p.\ 112 and 182.}

In the the first and third parts of this book, Ibn al-Rawandi (or Abu ’Isa al
Warraq?) presents his opinion on Reason and Religions that depend on Revelation
and explains the position on each. He begins his book with Human Reason,
praising it and going to great lengths in celebrating the fact that it is the
only path to enlightenment. For that reason his opponents must agree with him
that Reason is the mightiest thing that man possesses and is the sole refuge to
solve problems, indeed! “The prophet bore witness to the high status and
majesty of Reason.”\footnotemark

\footnotetext{Quote from the pervious reference p.\ 186–187.}

So Reason should be used to analyse Prophethood. Either the teachings of the
Prophet agree with Reason, and in that case there is no need for it because
Reason is in no need of it, or it contradicts reason in which case it is false.
For that reason it was necessary for Ibn al-Rawandi to be surprised at the
position of Muhammad and wonder; “Why did he bring that which negates him if he
was authentic?”\footnotemark For the revelation of Muhammad is in complete
opposition to reason. Then, what is the meaning of these obligatory religious
injunctions upon the Muslim, such as ritual washing, and prayer and
circumambulating around the Ka’abah and visiting the holy sites?

\footnotetext{Quote from the pervious reference p.\ 84.}

Regarding that, Ibn al-Rawandi says “Indeed the prophet brought that which
contradicts Reason, such as prayer, ritual cleansing from impurity, throwing
stones at pillars during Hajj, and walking around a house that cannot hear nor
see? Or dashing between two rocky mounds that can neither benefit nor harm. All
of this has nothing to do with Reason. So what is the difference between (the
hills of) Safa and Marwa and (the hills of) Abu Qubays and Hira? And walking
round the (holy) house is no different than walking round any other
houses.”\footnotemark

\footnotetext{Quote from the pervious reference 101–102. Abu Qubais and
Hira are mountains in Makkah.}

Ibn al-Rawandi used the myths of the Brahmans to express his bold views.
He used them as the means by which to attack
“Divinely revealed” religions and laws
(\NB{Since it is easier for Muslims to recognise the superiority of using
reason in relation to the claims of ‘divine inspiration’ of others,})
so he could hide beneath this veil his belief (about Islam).
He made them as
analogies for the (necessity of) Reason and Intellect so that they could be set
free of their own accord and express the views and thoughts that naturally
occur to them, while attaching it to delusional characters to soften its blows
upon the audience.

In this vein and in the name of Reason that he never ceases to praise and extol
for a moment, he goes on to attack the \Qrn{} in his previously mentioned book
“The Emerald”. He reviews in this book the concept of the Miraculousness of the
\Qrn{} and crticises it ruthlessly, and annihilates the view of the divine
origin of the \Qrn{} and puts forward a simple, concrete, logical and reasoned
view with no ambiguity in it. Convincing the intellect of the human nature of
the \Quran, refuting those who say that it is an inspiration from Allah and a
revelation from an all-wise and all-knowing entity.

It is also related that Ibn al-Rawandi said — regarding refuting the belief in
the miraculousness of the \Quran:

“Indeed it is not impossible that one Arab tribe is more eloquent than all the
other tribes, and that a group of people in this tribe are more eloquent than
(others) in this tribe and one member of this group is more eloquent than this
group... and suppose that his eloquence was spread amongst the Arabs... so what
is its wisdom upon the non-Arabs who do not understand the Arabic language?
What is the proof for them?”\footnotemark

\footnotetext{Quote from the previous reference p.\ 87.}

And Ibn al-Rawandi mocked the theatrical spectacle of the angels who Allah sent
down from heaven during the battle of Badr, to help the prophet. He said,
indeed: “They had limited effect, little power, despite their great number and
the combination of them and Muslims, they could not kill more than 70 people...
And where were the angels during the battle of Uhud when the prophet was
skulking in fear amongst the slain? Why didn’t Allah help him in that
situation?”\footnotemark

\footnotetext{Quote from the previous reference p.\ 87.}

It was also related in the book of “The Emerald”, quoting from the book of
“The Victory” by al-Khayyat, his saying: “Indeed the \Qrn{} is not the speech
of a wise god. In it are contradictions and mistakes and passages that are in
the realms of the impossible.”\footnotemark\@ As in the theatrical episode of
the angels of Badr that we just mentioned.

% TODO: Check if \footnotemark is correct:
\footnotetext{Quote from the previous reference p.\ 110.}

Then indeed Ibn al-Rawandi finds in the discourse of Aktham Ibn Sayfi better
(language) than (the \Qrn{} that boasts) “Indeed we have given you the
Abundance” (108/1)\footnotemark\@
\footnotetext{Quote from the previous reference p.\ 111.}
As Ibn al-Jawzi says in his brief allusion to the book of “The Emerald”: “Then
he begins with attack on the \Qrn{} and claims the existence of linguistical
mistakes in it.”\footnotemark

\footnotetext{Quote from the previous reference p.\ 120.}

And before ibn al-Rawandi exploits criticism of the \Qrn{} in his book “The
Crushing Blow”, and Ibn Jawzi has preserved for us copies of this criticism,
for amongst the parts that he preserved for us in his book “Al-Muntathim Fi
al-Tarikh”, from the book of “The Crushing Blow” which has not survived, the
following piece: “When (Muhammad) described (in the \Quran) Paradise, he said:
In it are rivers of laban, whose taste has not gone off, and that is milk, yet
no one desires that apart from the hungry. And he mentioned honey that no-one
wants at all and Ginger, which is not tasty except as a drink and silk brocade
(sundus) which is used as a spread, not as clothes and likewise embroidered
brocade (Istabriq) which is a thick\/rough type of silk brocade.
He said one who
imagines himself in Paradise wearing this rough clothing and drinking milk and
ginger, will be like a bride in a Kurdish or Nabataean wedding!”\footnotemark

\footnotetext{Quote from the previous reference p.\ 133.}

Ibn al-Rawandi turns his attention to the Divine challenge to bring the like of
the \Qrn{} and says: “If you want the like of it in respect of superior speech,
we can bring you a thousand like it from the speech of the masters of rhetoric
and champions of eloquence and poets and is more fluent in wording and more
concisely conveys the meanings, more elegantly rendered and expressed and more
beautifully rhymed. An if you are not content with that then we demand from you
the same that you demand from us!”\footnotemark

\footnotetext{Quote from the previous reference p.\ 216.}

(\NB{There appears to be a opening quotation mark missing from the original in
this next bit, and I’m not sure where it should go.})

Even the Mu’tazilah who reject all Miracles or at least attach no importance to
them, still believe in the miracle of the \Quran.\footnotemark\@
\footnotetext{Quote from the previous reference p.\ 119 and 153.}
But al-Nazzam, who was the most bold and freethinking of the Mu’tazilite
theologians, rejected the miraculous nature of the \Qrn{} in regard to its
composition, he rejected what was related of miracles of our prophet, peace be
upon him, as regards splitting the moon, the pebbles in his hand glorifying
God, the gushing of water from his fingers, to arrive by way of rejection of
the miracles of our prophet, peace be upon him, to the rejection of his
prophethood.”\footnotemark

\footnotetext{Al-Baghdadi, “The Difference Between the Groups”, p.\ 132;
See also p.\ 149–150}


\subsection{Abd al-Masih al-Kindi (9th Century AD)}

This attack on Islam was not restricted to apostate Muslims. No, indeed,
non-Muslims entered the ranks, galvanized by the fury of the fierce offensive
being waged on the new religion. Perhaps the most famous of these, whose quotes
have reached us, was the philosopher Abd al-Masih ibn Ishaq al-Kindi
(\NB{Not to be confused with the well known Muslim philosopher also called
al-Kindi}.)
He was a Nestorian that it is claimed lived in the courtyard of (the Caliph)
al-Ma’mun who, no doubt due to his openness towards those who differed from him
in views and belief,
(\NB{Al-Ma’mun was famous for gathering opposing sects and religions to hear
them debate,})
tolerated the ferocious criticism of this
Christian who attacked the rituals of Islam and its beliefs, one after another
and especially the rites of Hajj.

His views that concern us here are those connected to our topic and his
explanation for the effect of the \Qrn{} in that “The Nabataeans, the rabble,
non-Arabs, the gullible and the ignorant who have no understanding of the
Arabic language”, are the only ones who would be duped by the claim of the
Miraculous nature of the \Qrn{} in respect of its composition.”\footnotemark

\footnotetext{Quote from Dr.\ Badawi, from “The History of Disbelief in Islam,”
p.\ 129.}


\subsection{Abu Bakr al-Razi (d.\ 311H / 923AD)}

Al-Razi is the second of the two who, without rival,
courageously barged their way over the red line.
Many before them hovered close but never quite hit the mark.
Either because of their fear or lack of resources.
As for al-Razi and before him, Ibn al-Rawandi, they are the indisputable
masters of the field.
Indeed all those who attempted to reply to them could not match them.
Not at all! They were not on their level.
They were dwarfs that cannot be compared to either of them.
No way! No way!

Each one was a revolutionary, rebellious visionary,
who revealed the concealed (thoughts),
brought out the pent-up (feelings) and freed the suppressed (minds).
They thought the thoughts that were not thought about.
No! That were not allowed to be thought about.
Each one of them would not accept anything less than making the most holy
of holies the object of their criticism,
and delving into it to uncover its flaws,
and disgrace its myths and illusions.
Exposing what it contains of threats, claims and hearsay on account
of which man is crushed, and paralyzes his abilities and enslaves him to
supernatural powers and invisible entities.
To rob and intimidate him like an unsheathed sword hanging over his head,
not allowing him any room to move to see what is beyond his nose or know
what is going on around him.
Thus he must live his life, hostage to the fears, anxieties,
whisperings and misgivings that come between him and
achieving his best potential.
Destroying all his ambitions of self realization and personal freedom.

Al-Razi was a philosopher, doctor and alchemist of the highest order just as
he was the pillar of the dissident and heretical movement during his age and
the following centuries.

If there was a difference between him and Ibn al-Rawandi then it was in the
degree of depth and widening of the details and (his) ability to generate
new ideas from old ones, but both believed and relied upon Reason and
both base their judgments and conclusions upon Reason.
In their opinion, Reason was to the yardstick to measure everything.

If Ibn al-Rawandi, in his heretical and irreligious meditations,
worked within a similar atmosphere to that of the Muslim theologians, then:
“Al-Razi attacked and criticised the shortcomings of Religion from the
perspective of Philosophy.”\footnotemark

\footnotetext{Quote from the previous reference p.\ 127.}

In the same way as Ibn al-Rawandi used the Brahmins as a vehicle by which
to disguise his views, and to place on their tongues what was really in his
own mind regarding the invalidation of prophethood and virtues of Reason,
al-Razi also did likewise, in that he attributes to it (Reason) not just the
(ability to arrive at) ethical behaviour, as Ibn al-Rawandi did,
but attributes to it (knowledge) of divine matters also.
For he said, indeed we:
“Through it (Reason) arrive at knowledge of the Creator,
Mighty \& Glorified is He.”\footnotemark

\footnotetext{Quote from the previous reference p.\ 203.}

This proves that there is no justification for Prophethood as long as Reason
is able to lead us to all that is ethical and unethical.
In any case Ibn al-Rawandi:
“Moved in Theological and Religious field of study,
where as al-Razi moved in the Scietific field.”\footnotemark

\footnotetext{Quote from the previous reference p.\ 217.}

In summary, there is no doubt that Ibn al-Rawandi blazed the trail,
and opened the way, but al-Razi watered it and boarded it with palm trees
and beautified it with flowers and scented herbs
and raised upon it a lofty edifice.

Al-Razi praised Reason “using language which surpasses that used by the
great rationalists of all ages, even in the modern age,”
as Abd al-Rahman confirmed in his aforementioned book.

By virtue of Reason, man is in no need of Prophethood, nor Religion,
nor all the Divine Books and as a consequence; nor the \Quran.
By virtue of Reason and Reason alone, we can know good from bad
and truth from falsehood.
There is no authority other than the authority of Reason,
nor any belief other than belief in Reason...
and if this is its magnitude then we must never minimise its value,
nor reduce its status, and never make it a subject when it is the master.

Prophethood was al-Razi’s overriding concern, he demolished it on the basis
that Reason has no need for it.
He said: “From whence did you make it necessary that God singled out a people
for Prophethood instead of another?
Preferring them over (other) people?
Giving them evidences and forcing others to be in need of them
(in need of these people)?
And from whence did you allow in the wisdom of the Wise that he chooses
that for them and raises some over others confirming enmity between them,
increasing wars and with that annihilate people?”\footnotemark

\footnotetext{Quote from the previous reference p.\ 205.}

We are not so concerned here that al-Razi heaps criticism and abuse on
prophethood \& prophets, and elaborates in great detail on that.
What concerns us is his criticism of Religions, so we can arrive,
in that way, at his opinion on the \Quran.
For that reason we see him turning his attention to “Revealed” Religions
and the books they brought which they ascribe divinity to.
He analyses them without bias, favouritism, or discrimination.
For all of them are of equal importance.\footnotemark

\footnotetext{Quote from the previous reference p.\ 208–211.}

For the disbelief of al-Razi was not aimed at a specific religion
without another, in other words it was not aimed at Islam alone.
That highlights the objectivity of al-Razi and the soundness of his opinion.
For all religions were subject to attack and abuse.
For they do not say the same things.
They contradict one another despite the fact they claim to come from the same
source and (claim) they are free from defect and lies.
But how can that be the case when they contain absurdities and contradictions.

Here the adversary poses the question: If religions are as you say,
then how can we explain the adherence of the masses to them?

Al-Razi responds to this objection by (saying) that the followers of
(the various) religions have taken the religion from their
(religious) leaders by way of imitation.
They are prevented from questioning or scrutinizing the foundations,
and tales are related to them that discourage them
from questioning these foundations.
Whoever contravenes that is accused of Kufr (disbelief).
If the (religious) leaders are asked to prove the truth of what they say
they fly into a rage and spill the blood of one who demands that of them.

Then came (a period of) long familiarity, the passing of time,
acquaintance and deception of the people by the goat-bearded (clergy)
who stand at the front of religious gatherings and shriek out
lies and gibberish while around them the weak-minded men, women and children
(listen) until it all roots itself deeply within the people and
becomes a predisposition and habitual.\footnotemark

\footnotetext{Quote from the previous reference p.\ 211–212.}

Then al-Razi returns to his charge of contradictions in the “Holy” books
as proof of their falsity.
For the contradiction of religions leads to contradiction of
revealed books that brought them.
He begins with the Torah and the \Qrn{} and the prophetic Hadith and
what they contain of anthropomorphic and human-like qualities (of God).
He mentions what is in the Torah of putting the fat on the fire
so that the Lord can smell its scent.
Also how it depicts an image of an old man with white hair and beard.
This human-like and anthropomorphic description contradicts impassive and
impervious nature of God to things like smells etc...
All this announces that God is constructed, fabricated,
reacting to things like the rest of creation.

Likewise al-Razi attacks Christianity and its claim of the existence of
an uncreated ancient being by the side of God;
the Messiah his son, which leads to associating a partner with God.
Furthermore how can we reconcile his saying that he came to fulfill the Torah
with his abolishing its laws and changing its rulings?
Strangely, during his criticisms of Christianity,
he did not mention — in the texts we have —
the passages in the \Qrn{} about corruption of the Gospels.\footnotemark

\footnotetext{Quote from the previous reference p.\ 213–214.}

Anthropomorphism and contradictions are not only limited to Judaism and
Christianity but also envelop the the sayings of the prophet and the \Quran...
and that is exemplified by what is related from the prophet when he said
“I saw my Lord in the best of forms.
He put his hand on my shoulders until I felt
the cold of his fingertips on my chest.”\footnotemark
\footnotetext{Quote from the previous reference p.\ 214.
(\ar{الثَنْدَوَة} is the flesh between the nipples.)}
and his saying “Beside the throne by the shoulder of Israfeel,
and he will be groaning the groan of a young camel being saddled.”\footnotemark
(\NB{Israfeel is the angel who blows the trumpet twice on the day of judgment.
Once to destroy everything and a second time to bring humans back to life and
summon them for judgment.})

\footnotetext{Quote from the previous reference p.\ 214.}

It’s also obvious that many of the verses of the \Qrn{} demonstrate
anthropomorphism and no-one can deny that apart from the arrogant.
For example His saying, mighty and glorified is he:
“The Compassionate One is firmly established on the Throne.” (20:5)
and He also said: “And eight (angels) will carry the Throne of your Lord
above them on that Day” (69:17) and His saying:
“Those who carry the Throne and those around him...” (40:7)
So how can this his make sense, be sound, correct in light of the fact that
God is completely and utterly free from all the attributes of the profane as
made clear in His — Most High — saying:
“There is nothing whatever like unto Him...” (42:11)

Likewise how can we reconcile verses about predestination
with others about free-will?
And perhaps al-Razi borrowed these questions from the books of
Theological Discourse as Abd al-Rahman Badawi noted.\footnotemark

\footnotetext{Quote from the previous reference p.\ 218.}

As for the view that these verses require “Esoteric Interpretation” (Ta’wil)
in other words taking them to have a hidden meaning that is not the plain
meaning of the words, that was of no interest to al-Razi, he rejected it
utterly and paid no regard to Ta’wil, not taking it seriously at all.
Because Ta’wil in his opinion and the opinion of his like, was just
interpolation and deceitful pretense — or in my own expression:
“patching up” — the intent of which was to rescue the text, however one can,
and give it an acceptable meaning.
For al-Razi and his like approached religions as it appeared plainly
in its texts and not as is it is (claimed to be)
wrapped up in hidden meanings.\footnotemark

\footnotetext{Quote from the previous reference p.\ 214–215.}

Al-Razi criticised the \Qrn{} also on the basis of what it said that
contradicted Christianity and Judaism.
He said: “Indeed the \Qrn{} contradicts that which the Jews and Christians
believe regarding the death of the Messiah — upon him be peace.
Since the Jews and Christians say the Messiah was killed and crucified,
but the \Qrn{} says he was not killed and not crucified and that God
raised him up to himself.”\footnotemark

\footnotetext{Quote from the previous reference p.\ 215.}

Thus does al-Razi use religions and divine books to undermine each other
to arrive at the result that they are all false!
Because the contradictions between them declares their falsehood in total
as long as they claim that they come from the same divine source.

After this attack on all religions al-Razi comments also, saying
“Indeed, by God, we are amazed at what you say that the \Qrn{} is a miracle
when it is full of contradictions.
It is the narration of ancient myths,
it has no benefit nor is it proof of anything.”\footnotemark

\footnotetext{Quote from the previous reference p.\ 216 and 218
in two different versions.}

And this is a view that is completely sound,
for in the \Qrn{} are conundrums and riddles — ambiguities and mysteries,
that the greatest scholars of Tafseer until today,
haven’t been able to arrive at any conclusive conclusions on.
Despite all the ink they have spilled,
and the efforts they spent in meaningless summations, tedious disputations,
and nonsensical prattle with the sole obsession of rescuing a text
that cannot be rescued except through sophistry, interpolation,
prevarication, nonsense and legends.\footnotemark

\footnotetext{Whoever wishes to compose an approximate picture —
even if it is not precise — of these prattlers and nonsense-talkers,
then let him listen to the recordings of Sheikh Mutawali Sha’rawi,
who’s voice reverberates all over Arab radio.
He explains the \Qrn{} with a sharp tongue that erupts like a flood
that he uses to delight the masses and ignorant amongst the scholars,
while the idiots sitting around him roar out the words: “Allah! Allah!” or
“Allah is great! Allah is great!” and grow in zeal and impulsiveness.
If they weren’t in the mosque in a solemn religious gathering they would fill
the world with shouts and clapping as they do at public rallies and
I have complete confidence that they don’t understand a thing that’s going on.
This is an example that is emulated by the ignorant amongst the scholars and
the religious teachers and preachers and
Imams of mosques and the rest of this type.
He (Sha’rawi) is regarded by his followers and admirers,
to be amongst the greatest (scholars) of Tafseer in this day and age —
even a unique phenomenon amongst the phenomena of this age.
He is even considered by his pupils to be amongst those who
the prophet alluded to in the famous hadith:
“Indeed God will send to this Ummah (Nation) at the beginning of
every 100 years he who will renew\/reestablish their religion for them!”}

Just as the \Qrn{} challenged Mankind and Jinn to bring the like of it,
likewise al-Razi challenged the Arab Scholars of eloquence to bring
the like of that which is in the book of “Elements” (by Euclid)
or Almagest (by Ptolemy) and others.
Al-Razi says: “Indeed we demand from you the like of that which you claim we
are not able to do,”\footnotemark
and with this he threw the burden of proof back to the adversary.
In other words with this challenge he showed that the proof itself must lie
with the adversary (making the claim), since it is not within the ability of
man to bring the same that another man has brought, no matter how great
is his ability in copying and perfecting the art of imitation.

\footnotetext{Quote from the previous reference p.\ 218.}

Furthermore indeed these books and their like are more useful and of greater
benefit than the \Qrn{} and all the divine books, because they contain
knowledge that benefits people in their livelihoods
and situations in the real world,
while the Torah, Gospels and \Qrn{} benefit nothing.
And if one must discuss Miraculousness and Proof then these useful books
are more deserving of having such things ascribed to them.
In this respect, al-Razi says: “By Allah, if he wanted a book to be a
Proof the books like the ‘Elements’ or ‘Almagest’ that lead to understanding
of the movement of the stars and planets, or the books of logic,
or the books of medicine that is of benefit to the body,
would be more deserving of being (called) Proof than those (divine books)
that are of no benefit or harm.”\footnotemark\@
Meaning the \Qrn{} and its like.

\footnotetext{Quote from the previous reference p.\ 219.}

In any case I am not the first to present criticism of the \Quran.
I cannot claim that honour.
No! Nor will I be the last, for indeed my work here has precedence,
but it differs from that which preceded it in
respect of the method of treatment,
and in respect of the level and terms and fields of knowledge.
But it is the duty of the pioneer to always acknowledge those
who blazed the trail and opened the way before them.
As for the right of the one who went before upon the one who comes after,
it is that no-one will deny him other than the arrogant fool.
For if the one who comes after had not found assistance and
clarification from the one who was before,
things would not go right for him and he could not complete his intent,
and his efforts would be futile and his aim confounded and
thus is the blade blunted and the mind become dull and aspiration fails.
“And those who went before are the foremost.
They are the ones who will be drawn near.” (56:10–11)
(\NB{There is of course irony in that quote from \Quran.})

\section{The Eloquence of the \Quran}

We must now ask: Is the \Quran\ really miraculous?

The belief in the Miraculous nature of the \Quran\
does not withstand scrutiny in any way.
There are numerous fallacies that surround this belief.
We have seen some clear examples of that from Ibn al-Rawandi and
Abu Bakr al-Razi and in a little while we shall see many other examples
that refute this belief so long as we look impartially and objectively
at the issues so we are not swayed by the majority or the prevailing views,
for scientific facts are not discovered by voting as in Parliaments
no matter how large the number of votes support it.

Miraculousness is of two types in my opinion: Language and Meaning.

As for the Miraculous Language, its conditions are clarity of expression,
fluency of wording, being free from complexity,
weak composition and disharmony.
The speech must have a uniform level of quality, excellence, and perfection.

But miraculous language has no value
if it is not accompanied by miraculous meaning.
If not then it's just an arrangement of words cobbled together,
good-looking gibberish, meaningless padding.
For that reason eloquent speech must have consistency and symmetry
in its ideas and packed with meaning.
It must be free from error and contradiction.

But the verses of the \Quran\ are uneven in quality
in both language and meaning and this was noticed
by the classical scholars as confirmed by al-Suyuti.

Although a large portion of verses are the height of excellence and beauty,
another portion of verses fall far below that,
while others are weak and flawed.

In the same way ambiguity and riddles envelope a significant number of
the verses to the extent that one is confused when trying to understand
the intended meaning of this or that verse.
While some appear to have no meaning at all,
despite the fact that the exegetes (Mufassirun) and
scholars of Eloquence “discovered” a thousand and one meanings.

Indeed the books of (the scholars) of Eloquence are full of chapters
that have no meaning and have been contrived simply to provide an escape
route and justification for the babble in some of the verses
that confront the reader.
Using the pretext of delving deeply for the secrets
and sublime miraculousness of the \Quran.

In my opinion the whole science of Balagha (Eloquence)
was contrived in order to defend the \Quran.
In other words, only for ideological reasons, not to find the truth.
Indeed, Ideology is the governing factor in all the treaties of our scholars
in this field at the expense of objectivity and scientific methodology.

Finally, in addition to what we see in the \Quran\ of fragmentation
and disarray, not to mention blatant scientific mistakes.

So does all that correspond with the belief in the miraculous
(nature of the \Quran) in any way?
Or are there locks on hearts?
(Reference to \Quran\ 47:24).
This is what we shall investigate now.

The majority of those who studied the \Quran{}ic text are not Westerners —
if not all of them.
They treat it on the basis that it is a holy text.
That it cannot be criticised.
Since no falsehood comes to it from the front nor the back.
The presumption of its authenticity and infallibility is a prerequisite
that places a barrier that comes between us and it.
It deprives us of much of the wealth that may accrue from (their study of) it.
In that way we close all the doors that were open
in front of us before we begin.
And nothing remains for us to do in this case,
but pour everything we possess of effort into embellishing,
and polishing the text and imposing upon it
that which is unlikely and defend it — right or wrong —
and to “discover” what is in it of hidden treasures and secrets and
wisdoms and meanings that boggle the mind and astound the intellect and
thus begins the journey of searching for the pearls.

The text may not be more than a collection of bombastic speech
that does not mean anything, but the exegete (Mufassir) —
with his believing background and generous expectations —
presupposes there is the wisdom of the ages in the text,
because it is from the a wise all-knowing one.
“The Trustworthy Spirit has descended with it,
to thy heart so that you may be of the warners.” (26:193–194).
I say, if the text didn’t mean anything,
then indeed the exegetes ended up seeing everything in it!
It became the protected pearl and the hidden jewel.
But this is a profoundly bankrupt method of dealing with the \Quran{}ic text.
It doesn’t reap anything other than hot air and does not result
in anything other than waffle, double-talk, concoction,
and falsely attributing to the text that
which never occurred to its original author at all!

Indeed! The \Quran\ is not amongst the secrets of the gods,
it is not connected in any way to divine inspiration
that would take it outside historical trends.
It is purely a human achievement that complies with human principles.
Like all human efforts it is subject to strength \& weakness,
accuracy \& error, agreement \& contradiction, cohesion \& disarray,
consistency \& inconsistency, originality \& imitation,
depth \& superficiality, lucidity \& brittleness...

The direct result of all that is that the \Quran\ is a very ordinary book.
For that reason it is necessary to remove it from its safe and secure refuge,
outside Human history and return it to the world of people.
After that it will no longer be storehouse for timeless wisdom,
nor a divine book protected from error that no falsehood
can approach it from either front or behind.
In that way, it and its time and its context become
part of the historical process and unfolding events.
(\NB{This paragraph and part of the last is repeated
from the beginning of this chapter.})

If you read the \Quran\ you will find ample evidence of the Divine Being,
acts of worship, exhortations, morals, legislation, injunctions, wisdoms,
parables, stories and legends...\ but you will hardly come across one page
where ideas correlate or flow in a connected sequence
or follow on from one another,
unless the text is recounting the narrative of a story,
or establishing a rule, which requires a certain amount of elaboration.
But as soon as it finishes, it jumps to another subject
that has no connection to it.
That is then interspersed with digressions that
interrupt the narrative flow leaving it without a point.
So our waffling exegetes (Mufassirun) are forced
to come up with a point for it.
If they find a point then it is only stumbled across after
strenuous excavation that the wafflers attribute to profound wisdom.

There are complete pages in the \Quran\ that are full confusion,
as well as offensive words and weak expressions.
It contains hollowness, affectation, artifice, fabrication, and ambiguity.
Words that have meanings that conflict with one another,
making it hard for one to decide
which of the two conflicting aspects is the intended one.
If that was simply confined to insignificant secondary issues
it would be less important,
but it extends also to issues of belief and legislation.

Not forgetting, that in addition to these the errors and flaws,
the \Quran\ contains contradictions that the eye cannot miss.
How much effort the wafflers spent in trying to conceal them
and give them strange meanings that they don’t have,
to make them the epitome of wisdom and sobriety!

In addition to this series of drawbacks that the \Quran\ is packed with
and which we shall see detailed for ourselves,
is the mixing of the speech of Allah with
the speech of man within a single verse.
So while the first half of the verse starts off
in the words of the prophet or one of the pious,
we find the second half ending in speech that cannot be a human speaking,
but must be attributed to Allah.
So either this portion has been inserted
into the text or the verse is incomplete,
half of it being lost, so the scribe completed it —
and most of them didn’t understand what they were transcribing —
according to whatever words came to their minds,
repairing the verse and filling its gap.
This is despite all that is commonly disseminated
about the authentication of the text and close attention
to detail during its recording process.

Last but not least, the scholars find very great difficulty in
accepting many verses from the “Wise Reminder” (a name of the \Quran)
due to its complete opposition to scientific facts in the present time.
These verses are true as long as science, philosophy,
and myths are all approximately one and the same thing.
But today the situation has changed and the position has become clear
as to the extent of the naïvety of the \Quran\ when we see
that it accepted all and sundry of handed-down knowledge of ancient times
and then attributed it to the “treasure” of Divine knowledge
about the secrets of the universe, life and destiny.

\section{Where is the Eloquence of the \Quran?}
TODO:
\section{Disorder in the Distribution of Topics}
TODO:
\section{Ambiguity in the \Quran}
TODO:
\section{Obscurities of the \Quran}
TODO:
\section{Weakness of the \Quran}
TODO:
\section{Contradiction is the Distinguishing Feature of the \Quran}
TODO:
\section{The \Quran\ and Science}
TODO:
\section{Everything in the \Quran\ is from God}
TODO:
\section{Verses that Have no Meaning}
TODO:
\section{The Rhymed Prose of the \Quran\ and
the Rhymed Prose of the Soothsayers}
TODO:
\section{The \Quran\ and the Belief in the Unseen}
TODO:
\section{Barbarism of the \Quran}
TODO:

% Chapter 5:
\chapter{God in the \Quran}

% This section doesn't have a number in the original.
\phantomsection
\def \IntroSectionTitle{Introduction — The Existence of God and
the non-Existence of a Likeness}
\addcontentsline{toc}{section}{\IntroSectionTitle}
\section*{\IntroSectionTitle}
TODO:
\section{Attributes of God in the \Quran}
TODO:
\section{God and the Devil}
TODO:
\section{God the Compassionate the Merciful}
TODO:
\section{God is Near and Answers}
TODO:
\section{God is the Best of Sustainers}
TODO:
\section{There is no Help Except from God}
TODO:
\section{God Crams Himself into Everything}
TODO:
\section{God is the Conqueror over his Slaves}
TODO:
\section{With God. Man Must Impose his Law}
TODO:
\section{God. An Ineffective God}
TODO:

\backmatter

\chapter{Epilogue}
TODO:

\chapter{Index}
TODO:

\end{document}
